\documentclass[thesis.tex]{subfiles}

\graphicspath{{stropt/fig/}}
\DeclareGraphicsExtensions{.pdf}

\newcommand{\eg}{\textit{e.g.}}
\newcommand{\etc}{\textit{etc.}}
\newcommand{\etal}{\textit{et~al.}}
\newcommand{\ie}{\textit{i.e.}}

\newcommand{\naturalset}{\ensuremath\mathbb{N}}
\newcommand{\realset}{\ensuremath\mathbb{R}}
\newcommand{\floatset}{\ensuremath\mathbb{F}}
\newcommand{\powersetof}[1]{\ensuremath\mathcal{P}\left({#1}\right)}
\newcommand{\intervalset}{\ensuremath\mathbf{Interval}}
\newcommand{\floatintervalset}{\ensuremath\intervalset_\floatset}
\newcommand{\errorset}{\ensuremath\mathbb{E}^\sharp}
\newcommand{\labelset}{\ensuremath\mathbf{Label}}
\newcommand{\exprset}{\ensuremath\mathbf{Expr}}
\newcommand{\varset}{\ensuremath\mathbf{Var}}
\newcommand{\env}[1]{\ensuremath\mathbf{Env}_{#1}}
\newcommand{\eqrel}{\ensuremath\mathbin{\rhd}}
\DeclareMathOperator{\roundup}{\uparrow^\sharp_\circ}
\DeclareMathOperator{\rounddown}{\downarrow^\sharp_\circ}
\DeclareMathOperator{\fresh}{\mathit{fresh}}
\DeclareMathOperator{\eqstep}{\blacktriangleright}
\DeclareMathOperator{\dom}{\mathrm{Dom}}
\DeclareMathOperator{\area}{\mathrm{Area}}
\DeclareMathOperator{\error}{\mathrm{Error}}
\DeclareMathOperator{\abserr}{\mathrm{AbsError}}
\DeclareMathOperator{\abs}{\mathrm{abs}}
\DeclareMathOperator{\frontier}{\textsc{Frontier}}

\newcommand{\varx}{\ensuremath\texttt{x}}
\newcommand{\stmtsep}{\ensuremath;~}
\newcommand{\ifstmt}[3]{
    \ensuremath\texttt{if~} {#1} \texttt{~then~} ({#2}) \texttt{~else~} ({#3})
}
\newcommand{\whilestmt}[2]{
    \ensuremath\texttt{while~} {#1} \texttt{~do~} ({#2})
}

\newcommand{\enter}[1]{\ensuremath{A({#1})}}

\newcommand{\interval}[2]{\ensuremath\left[{#1}, {#2}\right]}

\newcommand{\lattice}[2]{\ensuremath\left<{#1}, {#2}\right>}
\newcommand{\join}{\ensuremath\sqcup}
\newcommand{\meet}{\ensuremath\sqcap}

\newcommand{\marteltrace}{\texttt{martel\_trace}}
\newcommand{\frontiertrace}{\texttt{frontier\_trace}}
\newcommand{\greedytrace}{\texttt{greedy\_trace}}

\begin{document}

% Front matter
\pagenumbering{roman}           % roman page numbing (invisible for empty page style)
\pagestyle{empty}               % no header or footers
% !TEX root = ../thesis-example.tex
%
% ------------------------------------  --> cover title page
\begin{titlepage}
	\pdfbookmark[0]{Cover}{Cover}
	\flushright
	\hfill
	\vfill
	{\LARGE\thesisTitle \par}
	\rule[5pt]{\textwidth}{.4pt} \par
	{\Large\thesisName}
	\vfill
	\textit{\large\thesisDate} \\
	Version: \thesisVersion
\end{titlepage}


% ------------------------------------  --> main title page
\begin{titlepage}
	\pdfbookmark[0]{Titlepage}{Titlepage}
	\tgherosfont
	\centering

	\includegraphics[width=6cm]{imperial_logo} \\[4mm]
    % {\Large \thesisUniversity} \\[4mm]
	\textsf{\thesisUniversityDepartment} \\
	% \textsf{\thesisUniversityInstitute} \\
	\textsf{\thesisUniversityGroup} \\

	\vfill
	{\LARGE \color{ctcolortitle}\textbf{\thesisTitle} \\[10mm]}
	{\Large \thesisName} \\

	\vfill
	% \begin{minipage}[t]{.27\textwidth}
		% \raggedleft
		% \textit{1. Reviewer}
	% \end{minipage}
	% \hspace*{15pt}
	% \begin{minipage}[t]{.65\textwidth}
		% {\Large \thesisFirstReviewer} \\
		  % {\small \thesisFirstReviewerDepartment} \\[-1mm]
		% {\small \thesisFirstReviewerUniversity}
	% \end{minipage} \\[5mm]
	% \begin{minipage}[t]{.27\textwidth}
		% \raggedleft
		% \textit{2. Reviewer}
	% \end{minipage}
	% \hspace*{15pt}
	% \begin{minipage}[t]{.65\textwidth}
		% {\Large \thesisSecondReviewer} \\
		  % {\small \thesisSecondReviewerDepartment} \\[-1mm]
		% {\small \thesisSecondReviewerUniversity}
	% \end{minipage} \\[10mm]
	\begin{minipage}[t]{.27\textwidth}
		\raggedleft
		\textit{Supervisor}
	\end{minipage}
	\hspace*{15pt}
	\begin{minipage}[t]{.65\textwidth}
		\thesisFirstSupervisor%\ and \thesisSecondSupervisor
	\end{minipage} \\[10mm]
    \begin{minipage}[t]{.8\textwidth}
        {\small
            Submitted in part fulfilment of the requirements for the degree of
            Doctor of Philosophy of Imperial College London and the Diploma of
            Imperial College London
        }
    \end{minipage} \\[10mm]

	\thesisDate \\

\end{titlepage}


% ------------------------------------  --> lower title back for single page layout
\hfill
\vfill
{
	\small
	\textbf{\thesisName} \\
	\textit{\thesisTitle} \\
	\thesisDate \\
	% Reviewers: \thesisFirstReviewer\ and \thesisSecondReviewer \\
	Supervisor: \thesisFirstSupervisor%\ and \thesisSecondSupervisor
	\\[1.5em]
    \textbf{\thesisUniversity} \\
	\textit{\thesisUniversityGroup} \\
	% \thesisUniversityInstitute \\
	\thesisUniversityDepartment \\
	\thesisUniversityStreetAddress \\
	\thesisUniversityPostalCode\ and \thesisUniversityCity
}
                   % INCLUDE: all titlepages
\cleardoublepage

\pagestyle{plain}               % display just page numbers
\pdfbookmark[0]{Abstract}{Abstract}
\chapter*{Abstract}
\label{sec:abstract}
\vspace*{-10mm}

This thesis introduces a new technique, and its associated tool \soap, to
automatically perform source-to-source optimization of numerical programs,
specifically targeting the trade-off among numerical accuracy, latency, and
resource usage as a \acrlong{hls} flow for \acrshort{fpga} implementations.  A
new intermediate representation, \acrshort{mir}, is introduced to carry out
the abstraction and optimization of numerical programs.  Equivalent structures
in \acrshortpl{mir} are efficiently discovered using methods based on formal
semantics by taking into account axiomatic rules from real arithmetic, such as
associativity, distributivity and others, in tandem with program equivalence
rules that enable control-flow restructuring and eliminate redundant array
accesses.  For the first time, we bring rigorous approaches from software
static analysis, specifically formal semantics and \acrlong{ai}, to bear
on program transformation for \acrlong{hls}.  New abstract semantics are
developed to generate a computable subset of equivalent \acrshortpl{mir}
from an original \acrshort{mir}.  Using formal semantics, three objectives
are calculated for each \acrshort{mir} representing a pipelined numerical
program: the accuracy of computation and an estimate of resource utilization
in \acrshort{fpga} and the latency of program execution.  The optimization of
these objectives produces a Pareto frontier consisting of a set of equivalent
\acrshortpl{mir}.  We thus go beyond existing literature by not only optimizing
the precision requirements of an implementation, but changing the structure
of the implementation itself.  Using \soap{} to optimize the structure of a
variety of real world and artificially generated arithmetic expressions in
single precision, we improve either their accuracy or the resource utilization
by up to 60\%.  When applied to a suite of computational intensive numerical
programs from PolyBench and Livermore Loops benchmarks, \soap{} has generated
circuits that enjoy up to a 12$\times$ speedup, with a simultaneous 7$\times$
increase in accuracy, at a cost of up to 4$\times$ more \acrshortpl{lut}.
                % INCLUDE: the abstracts (english and german)
\cleardoublepage

% !TEX root = ../thesis-example.tex
%
\pdfbookmark[0]{Acknowledgement}{Acknowledgement}
\chapter*{Acknowledgement}
\label{sec:acknowledgement}
\vspace*{-10mm}

\todo{Acknowledgement goes here\textellipsis}
         % INCLUDE: acknowledgement
\cleardoublepage

\setcounter{tocdepth}{2}        % define depth of toc
\tableofcontents                % display table of contents
\cleardoublepage

% Body matter
\pagenumbering{arabic}          % arabic page numbering
\setcounter{page}{1}            % set page counter
\pagestyle{maincontentstyle}    % fancy header and footer

\doublespacing

% Main text
\chapter{Introduction}

\chapter{Background}

\chapter{Structural Optimization of Arithmetic Expressions}
\label{chp:stropt}

% \begin{abstract}
    % This paper introduces SOAP, a new tool to automatically optimize the
% structure of arithmetic expressions for FPGA implementation as part of a
% high level synthesis flow, taking into account axiomatic rules derived
% from real arithmetic, such as distributivity, associativity and others. We
% explicitly target an optimized area/accuracy trade-off, allowing arithmetic
% expressions to be automatically re-written for this purpose. For the first
% time, we bring rigorous approaches from software static analysis, specifically
% formal semantics and abstract interpretation, to bear on source-to-source
% transformation for high-level synthesis. New abstract semantics are developed
% to generate a computable subset of equivalent expressions from an original
% expression. Using formal semantics, we calculate two objectives, the accuracy
% of computation and an estimate of resource utilization in FPGA\@. The
% optimization of these objectives produces a Pareto frontier consisting of a
% set of expressions. This gives the synthesis tool the flexibility to choose
% an implementation satisfying constraints on both accuracy and resource
% usage. We thus go beyond existing literature by not only optimizing the
% precision requirements of an implementation, but changing the structure of the
% implementation itself. Using our tool to optimize the structure of a variety of
% real world and artificially generated examples in single precision, we improve
% either their accuracy or the resource utilization by up to 60\%.
% \end{abstract}

\section{Introduction}
\label{sec:introduction}

The IEEE 754 standard~\cite{ieee754} for floating-point computation is
ubiquitous in computing machines. In practice, it is often neglected that
floating-point computations almost always have roundoff errors. In fact,
associativity and distributivity properties which we consider to be fundamental
laws of real numbers no longer hold under floating-point arithmetic. This opens
the possibility of using these rules to generate an expression equivalent to
the original expression in real arithmetic, which could have better quality
than the original when evaluated in floating point computation.

By exploiting rules of equivalence in arithmetic, such as associativity $(a
+ b) + c \equiv a + (b + c)$ and distributivity $(a + b) \times c \equiv a
\times c + b \times c$, it is possible to automatically generate different
implementations of the same arithmetic expression. We optimize the structures
of arithmetic expressions in terms of the following two quality metrics
relevant to FPGA implementation: the resource usage when synthesized into
circuits, and a bound on roundoff errors when evaluated. Our goal is the joint
minimization of these two quality metrics. This optimization process provides
a Pareto optimal set of implementations. For example, our tool discovered that
with single precision floating-point representation, if $a \in [0.1, 0.2]$,
then the expression ${(a + 1)}^2$ uses fewest resources when implemented in the
form $(a + 1) \times (a + 1)$ but most accurate when expanded into $(((a \times
a) + a) + a) + 1$. However it turns out that a third alternative, $((1 + a)
+ a) + (a \times a)$, is never desirable because it is neither more accurate
nor uses fewer resources than the other two possible structures. Our aim is to
automatically detect and utilize such information to optimize the structure of
expressions.

A na{\"\i}ve implementation of equivalent expression finding would be to
explore all possible equivalent expressions to find optimal choices, However
this would result in combinatorial explosion~\cite{ioualalen}. For instance,
in worst case, the parsing of a simple summation of $n$ variables could
result in $(2n - 1)!! = 1\times3\times5\times\cdots\times(2n - 1)$ distinct
expressions~\cite{ioualalen, mouilleron}. This is further complicated by
distributivity as ${(a + b)}^k$ could expand into an expression with a
summation of $2^k$ terms each with $k - 1$ multiplications. Therefore, usually
it would be infeasible to generate a complete set of equivalent expressions
using the rules of equivalence, since an expression with a moderate number of
terms will have a very large number of equivalent expressions. The methodology
explained in this paper makes use of formal semantics as well as abstract
interpretation~\cite{cousot77} to significantly reduce the space and time
requirements and produce a subset of the Pareto frontier.

In order to further increase the options available in the Pareto frontier,
we introduce freedom in choosing mantissa widths for the evaluation of the
expressions. Generally as the precision of the evaluation increases, the
utilization of resources increases for the same expression. This gives
flexibility in the trade-off between resource usage and precision. Our
approach and its associated tool, SOAP, allow high-level synthesis flows to
automatically determine whether it is a better choice to rewrite an expression,
or change its precision in order to meet optimization goals.

The three contributions of this paper are:
\begin{enumerate}
    \item Efficient methods for discovering equivalent structures of
    arithmetic expressions.
    \item A semantics-based program analysis that allows joint reasoning about
    the resource usage and safe ranges of values and errors in floating-point
    computation of arithmetic expressions.
    \item A tool which produces RTL implementations on the area-accuracy
    trade-off curve derived from structural optimization.
\end{enumerate}

This paper is structured as follows. Section~\ref{sec:related_work}
discusses related existing work in high-level synthesis and the
optimization of arithmetic expressions. We explain the basic
concepts of semantics with abstract interpretation used in this
paper in Section~\ref{sec:abstract_interpretation}. Using this,
Section~\ref{sec:semantics} explains the concrete and abstract semantics
for finding equivalent structure in arithmetic expressions, as well as
the analysis of their resource usage estimates and bounds of errors.
Section~\ref{sec:implementation} gives an overview of the implementation
details in our tool. Then we discuss the results of optimized example
expressions in Section~\ref{sec:results} and end with concluding remarks in
Section~\ref{sec:conclusion}.

\section{Related Work}
\label{sec:related_work}

High-level synthesis (HLS) is the process of compiling a high-level
representation of an application (usually in C, C++ or MATLAB) into
register-transfer-level (RTL) implementation for FPGA~\cite{coussy, gajski}.
HLS tools enable us to work in a high-level language, as opposed to facing
labor-intensive tasks such as optimizing timing, designing control logic in
the RTL implementation. This allows application designers to instead focus on
the algorithmic and functional aspects of their implementation~\cite{coussy}.
Another advantage of using HLS over traditional RTL tools is that a C
description is smaller than a traditional RTL description by a factor
of 10~\cite{coussy, bdti}, which means HLS tools are in general more
productive and less error-prone to work with. HLS tools benefit us in their
ability to automatically search the design space with a reasonable design
cost~\cite{bdti}, explore a large number of trade-offs between performance,
cost and power~\cite{mcfarland}, which is generally much more difficult to
achieve in RTL tools. HLS has received a resurgence of interest recently,
particularly in the FPGA community. Xilinx now incorporates a sophisticated
HLS flow into its Vivado design suite~\cite{vivado_hls} and the open-source
HLS tool, LegUp~\cite{legup}, is gaining significant traction in the research
community.

However, in both commercial and academic HLS tools, there is very little
support for static analysis of numerical algorithms. LLVM-based HLS
tools such as Vivado HLS and LegUp usually have some traditional static
analysis-based optimization passes such as constant propagation, alias
analysis, bitwidth reduction or even expression tree balancing to reduce
latency for numerical algorithms. There are also academic tools that
perform precision-performance trade-off by optimizing word-lengths of data
paths~\cite{constantinides}. However there are currently no HLS tools that
perform the trade-off optimization between accuracy and resource usage by
varying the \emph{structure} of arithmetic expressions.

Even in the software community, there are only a few existing techniques
for optimizing expressions by transformation, none of which consider
accuracy/run-time trade-offs. Darulova~\etal~\cite{darulova} employ a
metaheuristic technique. They use genetic programming to evolve the structure
of arithmetic expressions into more accurate forms. However there are several
disadvantages with metaheuristics, such as convergence can only be proved
empirically and scalability is difficult to control because there is no
definitive method to decide how long the algorithm must run until it reaches a
satisfactory goal. Hosangadi~\etal~\cite{hosangadi} propose an algorithm for
the factorization of polynomials to reduce addition and multiplication counts,
but this method is only suitable for factorization and it is not possible to
choose different optimization levels. Peymandoust~\etal~\cite{peymandoust}
present an approach that only deals with the factorization of polynomials
in HLS using Gr\"obner bases. The shortcomings of this are its dependence
on a set of library expressions~\cite{hosangadi} and the high computational
complexity of Gr\"obner bases. The method proposed by Martel~\cite{martel07}
is based on operational semantics with abstract interpretation, but even
their depth limited strategy is, in practice, at least exponentially complex.
Finally Ioualalen~\etal~\cite{ioualalen} introduce the abstract interpretation
of equivalent expressions, and creates a polynomially sized structure to
represent an exponential number of equivalent expressions related by rules of
equivalence. However it restricts itself to only a handful of these rules to
avoid combinatorial explosion of the structure and there are no options for
tuning its optimization level.

Since none of these above captures the optimization of both accuracy and
performance by restructuring arithmetic expressions, we base ourselves on the
software work of Martel~\cite{martel07}, but extend this work in the following
ways. Firstly, we develop new hardware-appropriate semantics to analyze not
only accuracy but also resource usage, seamlessly taking into account common
subexpression elimination. Secondly, because we consider both resource usage
and accuracy, we develop a novel multi-objective optimization approach to
scalably construct the Pareto frontier in a hierarchical manner, allowing fast
design exploration. Thirdly, equivalence finding is guided by prior knowledge
on the bounds of the expression variables, as well as local Pareto frontiers of
subexpressions while it is optimizing expression trees in a bottom-up approach,
which allows us to reduce the complexity of finding equivalent expressions
without sacrificing our ability to optimize expressions.

We begin with an introduction to formal semantics in the following section,
later in Section~\ref{sec:semantics}, we explain our approach by extending the
semantics to reason about errors, resource usage and equivalent expressions.

\section{Abstract Interpretation}
\label{bg:sec:abstract_interpretation}

As our way of living is becoming increasingly dependent on programs, errors
in safety-critical system can incur huge expenses, and even costs of lives.
For example, the maiden flight of Ariane 5 resulted in a failure, because of
a software instruction failed to convert a 64-bit floating-point number into
16-bit signed integer, as the result is too large to be represented by a 16-bit
signed integer~\cite{dowson97}.  The Patriot defense system failed to intercept
an incoming missile because of an accumulated round-off error in the system's
internal clock, resulted in the deaths of 28 people in 1991~\cite{patriot}.
\emph{Static analysis}, a process of analyzing a piece of program written in
an HLL without executing it, is therefore a research topic of great importance
to prevent similar catastrophic errors and mitigate costs of failure in the
future.

It is unfortunate that because of the halting problem~\cite{turing37} and
a direct consequence of it, Rice's theorem~\cite{rice53}, any nontrivial
property on the outcome of a program is in general undecidable.  It means
that an interesting property---a yes or no question---which is never
always true or always false for all programs, is \emph{undecidable}, or
in other words, cannot be answered.  Even a question as simple as ``does
this program return zero'' is difficult to answer.  A static analyzer can
therefore produce an answer which is either a definite ``yes'' or ``I don't
know''~\cite{mine04}, and designing one which answers a ``yes''.  Producing a
meaningful ``yes'' in an efficient manner poses a challenging task to static
analyzers.  Static analyzers rely heavily on formal techniques to perform
well.  Typical techniques employed include symbolic execution, model checking,
satisfiability modulo theories~\cite{demoura08}, data-flow analysis based on
lattices~\cite{nielson99}, and abstract interpretation~\cite{cousot77}.

This section starts by introducing the data-flow analysis framework to analyze
a simple program, abstract interpretation is then applied to this example, and
the properties of the resulting analysis are further discussed.

% This is then extended to define a scalable analysis capturing accuracy.
% Later in Chapter~\ref{chp:progopt} we accommodate sequential statements,
% \iflit~branches and \whilelit~loops in the accuracy analysis, and in
% Chapter~\ref{chp:latopt}, we further improve our analysis by supporting
% multi-dimensional arrays.

\subsection{Data-Flow Analysis Framework}
\label{bg:sub:data_flow}

\begin{figure}[ht]
    \centering
    \begin{minipage}{0.5\textwidth}
    \begin{lstlisting}
    float simple(float x)
    {
        while (x > 1.0)
            x *= 0.9;
        return x;
    }
    \end{lstlisting}
    \end{minipage}
    \caption{%
        A simple program example to be statically analyzed.
    }\label{bg:lst:simple}
\end{figure}
In this section, we use the \emph{data-flow analysis} (DFA)
framework~\cite{nielson99} to statically analyze a program named \verb|simple|
in Figure~\ref{bg:lst:simple}, which consists of only one variable \verb|x|.
We further assume an initial set $\iota \subseteq \realset$ of values of
\verb|x|, and the property that concerns us is answering whether a particular
value $x_\mathrm{invalid}$ is not in the set of all reachable final values
of \verb|x|.  A sensible definition for the set of values can be reached
by \verb|x| is a subset of all real numbers $\realset$, \ie~an element of
$\powersetof\realset$, where $\powersetof\realset$ denotes the \emph{power set}
of $\realset$, also known as the set of all subsets of $\realset$.

The first step of DFA is to translate the body of \verb|simple| into a
control/data-flow graph (CDFG), as shown in Figure~\ref{bg:fig:cdfg} where
each block consists of a single statement or conditional, and the edges in
the graph model the data- and control-flows.  The \textbf{tt} and \textbf{ff}
respectively highlight the control-flow branch taken when the conditional
\mbox{``\texttt{x < 1}''} evaluates to either true or false.
\begin{figure}[ht]
    \centering
    \tikzstyle{block} = [
        draw,
        fill=white,
        rectangle,
        minimum height=2em,
        minimum width=4em
    ]
    \begin{tikzpicture}[node distance=4em]
        \node(entr) {\textbf{entry}};
        \node(cond) [block, below of=entr] {\texttt{x > 1}};
        \node(stmt) [block, below of=cond] {\texttt{x = 0.9 * x;}};
        \node(midl) [coordinate, left of=stmt, xshift=-1.5em] {};
        \node(midr) [coordinate, right of=stmt, xshift=1.5em] {};
        \node(rtrn) [coordinate, below of=stmt, yshift=1em]
            {\texttt{return x;}};
        \node(exit) [below of=rtrn, yshift=1em] {\textbf{exit}};
        \draw[->] (entr) -- node[auto]{0} (cond);
        \draw[->] (cond) -- node[right]{\textbf{tt}} node[left]{1} (stmt);
        \draw[- ] (stmt) -- (midl);
        \draw[->] (midl) |- node[auto]{2} (cond);
        % \draw[->] (stmt) to[out=180, in=180] (cond);
        \draw[- ] (cond) -| node[auto, near start]{\textbf{ff}} (midr);
        \draw[- ] (midr) |- node[auto]{3} (rtrn);
        % \draw[->] (cond) to[out=0, in=0] (rtrn);
        \draw[->] (rtrn) -- (exit);
    \end{tikzpicture}
    \caption{%
        The CDFG of \texttt{simple} in Figure~\ref{bg:lst:simple}.
    }\label{bg:fig:cdfg}
\end{figure}

The individual blocks in the CDFG can therefore be defined as functions $f:
\powersetof\realset \to \powersetof\realset$ that admits an $S$ and produces
$T$, where both $S$ and $T$ are elements of $\powersetof\realset$.  For
instance, for the statement ``\texttt{x *= 0.9;}'', a function $f_1$ can be
defined as follows:
\begin{equation}
    f_1(S) = \{ 0.9 v \mid v \in S \}.
\end{equation}
Here, the definition of $f_1$ indicates that for all possible input values
$v$ of \verb|x| in the set $S$, we multiply it by $0.9$ and collect the
multiplied results into a new set as the output of $f_1$.  Similarly, because
\mbox{``\texttt{x > 1}''} has two conditional branches, two functions,
$f_{2,\truelit}$ and $f_{2, \falselit}$, respectively for both true- and
false-branches of it can be defined:
\begin{equation}
    \begin{aligned}
        f_{2, \truelit}(S) &= S \cap \{ v \in \realset \mid v > 1 \}, \\
        f_{2, \falselit}(S) &= S \cap \{ v \in \realset \mid v \leq 1 \}.
    \end{aligned}
\end{equation}
where $X \cap Y$ computes the intersection of the two sets $X$ and $Y$.

In the next step, the edges of the CDFG are labelled with numbers 0, 1, 2
and 3 to signify different locations of the program.  For each edge labelled
$i$, it is now possible to compute an $A(i)$, a set of values that could be
reached by \verb|x| in a program execution at each location $i$, by wiring up
the functions $f_1$, $f_{2, \truelit}$ and $f_{2, \falselit}$ that correspond
to program statements.  This gives rise to the following system of data-flow
equations:
\begin{align}
    A(0) &= \iota,
        \label{bg:eq:dfa0} \\
    A(1) &= f_{2, \truelit}(A(0) \cup A(2)),
        \label{bg:eq:dfa1} \\
    A(2) &= f_1(A(1)),
        \label{bg:eq:dfa2} \\
    A(3) &= f_{2, \falselit}(A(0) \cup A(2)),
        \label{bg:eq:dfa3}
\end{align}
where $A(0) \cup A(1)$ is the union of $A(0)$ and $A(1)$.

Unfortunately, computationally solving this system of equations is not an
easy task.  In Sections~\ref{bg:sub:most_precise} and~\ref{bg:sub:intervals},
the two significant impediments are explained, and subsequently, theories are
introduced to unravel them.


\subsection{The Most Precise Solution to a Data-Flow Equation}
\label{bg:sub:most_precise}

There are multiple solutions to this system.  For example, we can solve
it manually by substituting $A(0)$ and $A(2)$ in~\eqref{bg:eq:dfa1}
with~\eqref{bg:eq:dfa0} and~\eqref{bg:eq:dfa2}.  We arrive at:
\begin{equation}
    A(1) = \left(
        \iota \cup \left\{ 0.9 v \mid v \in A(1) \right\}
    \right) \cap \{ v \in \realset \mid v > 1 \}.
    \label{bg:eq:dfa_a1}
\end{equation}
It turns out that the set of all real numbers greater than $1$, or:
\begin{equation}
    A(1) = \{ v \in \realset \mid v > 1 \}
    \label{bg:eq:a11}
\end{equation}
is a solution to~\eqref{bg:eq:dfa_a1}.  Substituting $A(1)$ in the right-hand
side of~\eqref{bg:eq:dfa_a1} with this value proves that it is indeed the
solution for this equation, assuming all sets below are subsets of $\realset$
to simplify the derivation:
\begin{equation}
    \begin{aligned}
        A(1)
        &= \bigg( \iota \cup \Big\{ 0.9 v \mid v \in
                \left\{ v^\prime \mid v^\prime > 1 \right\}
           \Big\} \bigg) \cap \{ v \mid v > 1 \} \\
        &= \bigg( \iota \cup \left\{ 0.9 v \mid v > 1 \right\} \bigg) \cap
           \{ v \mid v > 1 \} \\
        &= \bigg( \iota \cup \left\{ v \mid v > 0.9 \right\} \bigg) \cap
           \{ v \mid v > 1 \} \\
        &= \bigg( \iota \cap \{ v \mid v > 1 \} \bigg) \cup
           \bigg(
               \left\{ v \mid v > 0.9 \right\} \cap \{ v  \mid v > 1 \}
           \bigg) \\
        &= \bigg( \iota \cap \{ v \mid v > 1 \} \bigg) \cup
           \{ v \mid v > 1 \} \\
        &= \{ v \mid v > 1 \}.
    \end{aligned}
\end{equation}

Intuitively, a manual inspection of \verb|simple| finds that \verb|x| can reach
values $v$, $0.9 v$, $0.9^2 v$, and \textellipsis, such that all values in this
sequence are greater than $1$, for each $v \in \iota$; or more succinctly, an
alternative solution to $A(1)$ should be:
\begin{equation}
    A(1) = \{ v \in \iota \mid 0.9^k v > 1 \wedge k \in \naturalset \}.
    \label{bg:eq:a12}
\end{equation}
Here $k \in \naturalset$ denotes $k$ is one of $0, 1, 2, \mathellipsis$, \ie~a
natural number.

It is evident to us the latter solution~\eqref{bg:eq:a12} is more precise,
hence more desirable, than the former~\eqref{bg:eq:a11}.  Not only does it
contain information the former has, \ie~all values reachable by $A(1)$ is
greater than 1, it also expresses the fact that it only consists of values of
the form $0.9^k v$, where $v \in \realset$ and $k \in \naturalset$.  A useful
definition of preciseness is therefore the subset relation ``$\subseteq$''.
If it is known that $X \subseteq X^\prime$, and $X$ and $X^\prime$ are both
solution to a system of data-flow equations, then $X$ is clearly more appealing
than $X^\prime$.

The set $\powersetof\realset$, with a preciseness ordering ``$\subseteq$'', is
a \emph{partially ordered set}.  It has three following properties for any $X,
Y, Z \in \powersetof\realset$: it is \emph{reflexive}, $X \subseteq X$; it has
the \emph{antisymmetry} property, \ie~if $X \subseteq Y$ and $Y \subseteq X$,
then $X = Y$; and finally it is transitive, if $X \subseteq Y$ and $Y \subseteq
Z$, then $X \subseteq Y$.  In contrast to a total ordered set such as the set
of reals $\realset$, not every two elements in $\powersetof\realset$ can be
compared, \eg~neither of the sets $\{1, 2, 3\}$ and $\{2, 3, 4\}$ is a subset
of each other.

For the purpose of computing the solution to $A(1)$'s
equation~\eqref{bg:eq:dfa_a1}, a function $f: \powersetof\realset \to
\powersetof\realset$ can be defined:
\begin{equation}
    f(X) = \left(
        \iota \cup \left\{ 0.9 v \mid v \in X \right\}
    \right) \cap \{ v \in \realset \mid v > 1 \},
    \label{bg:eq:transfer}
\end{equation}
so that all solutions of the original equation~\eqref{bg:eq:dfa_a1} are now in
this following set, which are known as the \emph{fixpoints} of $f$:
\begin{equation}
    \mathrm{Fix}(f) = \left\{
        X \in \powersetof\realset \mid
        f(X) = X
    \right\}.
\end{equation}

By using this particular definition of preciseness, two important questions
however arise:
\begin{enumerate}

    \item Is the most precise solution unique?  A unique most precise solution
    is defined as the only one which is the most precise among all possible
    solutions to the systems of data-flow equations.  In other words, if
    it exists, then it is defined as the \emph{least fixpoint} (LFP) of
    $f$ which is a subset of all other fixpoints, \ie~$\lfp (f) \subseteq
    \mathrm{Fix}(f)$.  As we have discussed earlier, multiple fixpoints exist,
    and it is possible that these fixpoint solutions are not comparable.

    \item If a unique solution exists and it is unique, how do we find it?
    This is equivalent to finding a way to compute the LFP $\lfp(f)$ using $f$.

\end{enumerate}

As it turns out, the first question can be answered by
Theorem~\ref{bg:thr:tarski}~\cite{tarski55, nielson99}, which proves that $\lfp
(f)$ is indeed unique.
\begin{theorem}
    \textup{[Tarski's fixpoint theorem]}
    If $\mathsf{L}$ is a complete lattice, and a function $g:
    \mathsf{L} \to \mathsf{L}$ is a monotone function, then $\lfp(g)$,
    the LFP of $g$ is the greatest lower bound of all fixpoints
    $\mathrm{Fix}(g)$.\label{bg:thr:tarski}
\end{theorem}

In our case, $f$ is a \emph{monotone} function, because by definition a
\emph{monotone} function satisfies the condition that if $X \subseteq Y$,
then $g(X) \subseteq g(Y)$.  In the DFA of \verb|simple|, $\mathsf{L} =
\powersetof\realset$, which is a \emph{complete lattice}\footnote{%
    Exact definitions of complete lattice and complete partial order are not
    required in this section.  Both of them can be found in~\cite{nielson99}.
}, since all power sets are complete lattices~\cite{nielson99}.  The LFP of
$f$, that is the intersection of all elements in $\mathrm{Fix}(f)$, or the
greatest lower bound of all fixpoints, can therefore be written concisely as:
\begin{equation}
    \lfp (f) = \bigcap \mathrm{Fix}(f).
\end{equation}

Secondly, another theorem~\cite{kleene52}, which is closely
related to Theorem~\ref{bg:thr:tarski}, states:
\begin{theorem}
    \textup{[Kleene's fixpoint theorem]}
    If $\mathsf{L}$ is a complete partial order (CPO), and $g: \mathsf{L} \to
    \mathsf{L}$ is a Scott-continuous function, then the $\lfp (g)$ can be
    computed as the least upper bound of all values in the sequence $\bot$,
    $g(\bot)$, $g^2(\bot)$, $g^3(\bot)$, \textellipsis{}\label{bg:thr:kleene}
\end{theorem}
Here, $\bot$ denotes the least element in $\mathsf{L}$.  A function of the form
$h^n(x)$, where $h: \mathsf{M} \to \mathsf{M}$ for any domain $\mathsf{M}$ and
$n \in \naturalset$, is recursively defined as:
\begin{equation}
    h^n(x) = \left\{
        \begin{aligned}
            & h(h^{n-1}(x)) \quad && \text{if~} n > 0, \\
            & x && \text{if~} n = 0.
        \end{aligned}
    \right.
\end{equation}

Our function $f$ is \emph{Scott-continuous}: it is monotone; and for any chain
of $X_0 \subseteq X_1 \subseteq X_2 \subseteq X_3 \subseteq \mathellipsis$,
where $X_i \in \powersetof\realset$:
\begin{equation}
    \bigcup_{i \in \naturalset} f(X_i) = f \left(
        \bigcup_{i \in \naturalset} X_i
    \right).
\end{equation}
As a CPO is more general that a complete lattice, and the least
element in $\powersetof\realset$ is the empty set $\varnothing$, using
Theorem~\ref{bg:thr:kleene} in our example analysis, the most precise solution
of $A(1)$ can therefore be computed using:
\begin{equation}
    \lfp (f) = \bigcup_{k \in \naturalset} f^k (\varnothing).
\end{equation}
The functions $f^k(\varnothing)$ for the first $k+1$ iterations can be
evaluated as follows:
\begin{equation}
    \begin{aligned}
        f^0(\varnothing) &= \varnothing, \quad\quad
        f^1(\varnothing) = \iota \cap \{ v \mid v > 1 \}, \\
        f^2(\varnothing) &= f(f^1(\varnothing))
               = \left(
                     \iota \cup
                     \{ 0.9v \mid v \in \iota \}
                 \right) \cap \{ v \mid v > 1 \}, \\
        f^3(\varnothing) &= \left(
                     \iota \cup
                     \{ 0.9v \mid v \in \iota \} \cup
                     \{ 0.9^2 v \mid v \in \iota \}
                 \right) \cap \{ v \mid v > 1 \}, \mathellipsis, \\
        f^k(\varnothing) &= \left(
                     \iota \cup
                     \{ 0.9v \mid v \in \iota \} \cup
                     \mathellipsis \cup
                     \{ 0.9^{k-1} v \mid v \in \iota \}
                 \right) \cap \{ v \mid v > 1 \}.
    \end{aligned}
\end{equation}
Finally, the most precise solution to~\eqref{bg:eq:dfa_a1} can be computed
using the LFP formula for $f$, which is exactly the same as the alternative
solution that was manually computed in~\eqref{bg:eq:a12}:
\begin{equation}
    \begin{aligned}
        \lfp (f)
            &= \bigcup_{k \in \naturalset} f^k (\varnothing) \\
            &= \{ v \mid v > 1 \} \cap
               \bigcup_{k \in \naturalset} \{ 0.9^k v \mid v \in \iota \} \\
            &= \{ v \in \iota \mid 0.9^k v > 1 \wedge k \in \naturalset \}.
    \end{aligned}
\end{equation}

Even though we have derive a method to statically analyze a program,
significant obstacles still prevent us from using it efficiently.  Firstly
in the \verb|simple| case study, because the LFP is evaluated as the union
of $f^k(\varnothing)$ in a sequence, this sequence is likely to be infinite,
and thus cannot be computed fully.  Secondly, the set of input values,
$\iota$, not only determines the number of iterations necessary in order to
calculate the LFP, but also impacts the amount of computation required in
each iteration.  For instance if $\iota = {4}$ then it is only necessary to
track the computation for a single input value $4$, whereas when $\iota =
\{ v \mid 0 \leq v \leq 1000 \}$, there are infinitely many values in the
set.  As a result, in general, the LFP of an arbitrary self-map function $f:
\mathsf{L} \to \mathsf{L}$, where $\mathsf{L}$ is a complete lattice, is thus
not computable in finite amount of time.  In Section~\ref{bg:sub:intervals},
a method known as abstract interpretation is introduced to overcome the
computability problem.


\subsection{Abstract Interpretation with Intervals}
\label{bg:sub:intervals}

A framework of methods, known as \emph{abstract interpretation} (AI), is
proposed by Cousot~\etal~\cite{cousot77} to formally mitigate the problem of
computability in program analysis.  Instead of finding the LFP, which may not
be computable, it is much more efficient to work out an \emph{approximation} of
the LFP\@.  Despite the outcome of an AI-based static analysis not as precise
as the LFP, the significant benefits of AI is two-fold.  Firstly, the program
analysis framework can now produce a ``yes'' or ``I don't know'' answer to a
query of program property in a finite amount of time.  Secondly, it provides
the means to prove the correctness of an answer produced by the static analyzer
using AI in formal mathematics.

We illustrate these concepts by putting the familiar idea of \emph{interval
arithmetic}~\cite{moore} in the framework of abstract interpretation. As
an illustration, consider the following expression and its DFG in
Figure~\ref{bg:fig:sample_tree}\@:
\begin{equation}
    (a + b) \times (a + b)
    \label{bg:eq:absint_sample}
\end{equation}
\begin{figure}[ht]
    \centering
    \includegraphics[scale=0.6]{sample_tree}
    \caption{The DFG for the sample expression.}\label{bg:fig:sample_tree}
\end{figure}

We may wish to ask: if initially $a$ and $b$ are real numbers in the range of
$[0.2, 0.3]$ and $[2, 3]$ respectively, what would be the outcome of evaluating
this expression with real arithmetic? A straightforward approach is simulation.
Evaluating the expression for a large quantity of inputs will produce a set
of possible outputs of the expression. However the simulation approach is
unsafe, since there are infinite number of real-valued inputs possible and it
is infeasible to simulate for all.

A better method might be to represent the possible values of $a$ and $b$ using
ranges. To compute the ranges of its output values, we could operate on ranges
rather than values (note that the superscript $\sharp$ denotes ranges). Assume
that $a^\sharp_{init} = [0.2, 0.3]$, $b^\sharp_{init} = [2, 3]$, which are the
input ranges of $a$ and $b$, and $\enter{l}$ where $l \in \{1, 2, 3, 4\}$ are
the intervals of the outputs of the boxes labelled with $l$ in the DFG\@. We
extract the data flow from the DFG to produce the following set of equations:
\begin{equation}
    \begin{aligned}
        \enter{1} &= a^\sharp_{init} \\
        \enter{2} &= b^\sharp_{init} \\
        \enter{3} &= \enter{1} + \enter{2} \\
        \enter{4} &= \enter{3} \times \enter{3}
    \end{aligned}
    \label{bg:eq:absint_sample_analysis}
\end{equation}
For the equations above to make sense, addition and multiplication need to be
defined on intervals. We may define the following interval operations:
\begin{equation}
    \begin{aligned}
        \interval{a}{b} + \interval{c}{d} &= \interval{a + c}{b + d} \\
        \interval{a}{b} - \interval{c}{d} &=  \interval{a - d}{b - c} \\
        \interval{a}{b} \times \interval{c}{d} &=
            \interval{\min(s)}{\max(s)} \\
        \text{where~} s &= \{ a \times c, a \times d, b \times c, b \times d \}
    \end{aligned}
    \label{bg:eq:interval_operations}
\end{equation}
The solution to the set of~\eqref{bg:eq:absint_sample_analysis} for $\enter{4}$
is $[4.84, 10.89]$, which represents a safe bound on the output at the end
of program execution. Note that in actual execution of the program, the
semantics represent the values of intermediate variables, which are real
values. In our case, a set of real values forms the set of all possible
values produced by our code. However computing this set precisely is not,
in general, a possible task. Instead, we use abstract interpretation based
on intervals, which gives the abstract semantics of this program. Here, we
have achieved a classical interval analysis by \emph{defining} the meaning of
addition and multiplication on abstract mathematical structures (in this case
intervals) which capture a safe approximation of the original semantics of the
program.

Later in Sections~\ref{so:sec:resource}~and~\ref{so:sec:equivalent} of
Chapter~\ref{chp:stropt}, we further generalize the idea by defining the
meaning of these operations on more complex abstract structures which allow
us to scalably reason about the area of FPGA implementations and equivalent
program structures respectively.


\subsection{Accuracy Analysis}
\label{bg:sub:accuracy}

Because we optimize numerical programs in a way that may have significant
impact on accuracy, and one of our objectives is to minimize round-off error
in the process, it is necessary to perform accuracy analysis on optimized
candidates.

Since our numerical programs make use of floating-point arithmetic, we first
introduce the concepts of the floating-point representation~\cite{ieee754}. Any
values $v$ representable in floating-point with standard exponent offset can be
expressed with the format given by the following equation:
\begin{equation}
    v = s \times 2^{e + 2^{k - 1} - 1} \times 1.{m_1 m_2 m_3 \ldots m_p}
    \label{bg:eq:floating_point}
\end{equation}
In~\eqref{bg:eq:floating_point}, the bit $s$ is the sign bit, the $k$-bit
unsigned integer $e$ is known as the exponent bits, and the $p$-bits $m_1 m_2
m_3 \ldots m_p$ are the mantissa bits, here we use $1.{m_1 m_2 m_3 \ldots m_p}$
to indicate a fixed-point number represented in unsigned binary format.

Because of the finite characteristic of IEEE 754 floating-point format, it
is not always possible to represent exact values with it. Computations in
floating-point arithmetic often induces roundoff errors. Therefore, following
Martel~\cite{martel07}, we bound with ranges the values of floating-point
calculations, as well as their roundoff errors. Our accuracy analysis
determines the bounds of all possible outputs and their associated range of
roundoff errors for expressions. For example, assume that real variables $a
\in [0.2, 0.3]$, $b \in [2.3, 2.4]$, it is possible to derive that in single
precision floating-point computation with rounding to the nearest, ${(a + b)}^2
\in [6.24999857, 7.29000187]$ and the error caused by this computation is
bounded by $[-1.60634534\times10^{-6}, 1.60634534\times10^{-6}]$.

We employ abstract error semantics for the calculation of errors described
in~\cite{ioualalen, martel07}. First we define the domain $\errorset
= \floatintervalset\times\intervalset$, where $\intervalset$ and
$\floatintervalset$ respectively represent the set of real intervals, and
the set of floating-point intervals (intervals exactly representable in
floating-point arithmetic). The value $(x^\sharp, \mu^\sharp) \in \errorset$
represents a safe bound on floating-point values and the accumulated error
represented as a range of real values. Then addition and multiplication can be
defined for the semantics as in~\eqref{bg:eq:error_semantics}:
\begin{equation}
    \begin{aligned}
        \left( x^\sharp_1, \mu^\sharp_1 \right) +
        \left( x^\sharp_2, \mu^\sharp_2 \right)
    &=  \left(
            \roundup{x^\sharp_1 + x^\sharp_2},
            \mu^\sharp_1 + \mu^\sharp_2 +
            \rounddown{x^\sharp_1 + x^\sharp_2}
        \right) \\
        \left( x^\sharp_1, \mu^\sharp_1 \right) -
        \left( x^\sharp_2, \mu^\sharp_2 \right)
    &=  \left(
            \roundup{x^\sharp_1 - x^\sharp_2},
            \mu^\sharp_1 - \mu^\sharp_2 +
            \rounddown{x^\sharp_1 - x^\sharp_2}
        \right) \\
        \left( x^\sharp_1, \mu^\sharp_1 \right) \times
        \left( x^\sharp_2, \mu^\sharp_2 \right)
    &=  \left(
            \roundup{x^\sharp_1 \times x^\sharp_2},
            x^\sharp_1 \times \mu^\sharp_2 + x^\sharp_2 \times \mu^\sharp_1 +
            \mu^\sharp_1 \times \mu^\sharp_2 +
            \rounddown{x^\sharp_1 \times x^\sharp_2}
        \right) \\
    &\qquad\qquad\qquad\qquad\qquad\qquad\text{~for~}
        \left( x^\sharp_1, \mu^\sharp_1 \right) \in \errorset,
        \left( x^\sharp_2, \mu^\sharp_2 \right) \in \errorset
    \end{aligned}
    \label{bg:eq:error_semantics}
\end{equation}

The addition, subtraction and multiplication of intervals follow
the standard rules of interval arithmetic defined earlier
in~\eqref{bg:eq:interval_operations}.  In~\eqref{bg:eq:error_semantics}, the
function $\rounddownop: \intervalset \to \intervalset$ determines the range
of roundoff error due to the floating-point computation under one of the
\emph{rounding modes} $\circ \in \{ -\infty, \infty, 0, \neg0, \sim \}$ which
are round towards negative infinity, towards infinity, towards zero, away from
zero and towards nearest floating-point value respectively. It is defined as:
\begin{equation}
    \begin{aligned}
        & \downarrow^\sharp_\circ([a, b]) = \left\{
            \begin{aligned}
                & \left[ -\frac{z}{2}, \frac{z}{2}\right]
                    & \quad \circ & \text{~is~}\sim \\
                & \left[ -z, z\right]
                    & \quad \circ & \in \{ -\infty, \infty, 0, \neg0 \}
            \end{aligned}
        \right. \\
        & \qquad\qquad\qquad\qquad \text{where~} z = \max(ulp(a), ulp(b))
    \end{aligned}
\end{equation}
Here $z$ denotes the maximum rounding error that can occur for values
within the range $[a, b]$, and the unit of the last place (ulp) function
$ulp(x)$~\cite{muller} characterizes the distance between two adjacent
floating-point values $f_1$ and $f_2$ satisfying $f_1 \leq x \leq
f_2$~\cite{goldberg}. In our analysis, the function $ulp$ is defined as:
\begin{equation}
    ulp(x) = 2^{e(x) + 2^{k - 1} - 1} \times 2^{-p}
\end{equation}
where $e(x)$ is the exponent of $x$, $k$ and $p$ are the parameters of the
floating-point format as defined in~\eqref{bg:eq:floating_point}. The function
$\roundupop: \intervalset \to \floatintervalset$ computes the floating-point
bound from a real bound, by rounding the infimum $a$ and supremum $b$ of the
input interval $[a, b]$:
\begin{equation}
    \roundupop\left(\left[a, b\right]\right)
    = {\left[
        \uparrow_\circ{\left(a\right)},
        \uparrow_\circ{\left(b\right)}
    \right]}_\floatset
\end{equation}
where the subscript $\floatset$ indicates the interval is a floating-point
interval, and we define $\uparrow_\circ: \realset \to \floatset$ to be the
function that rounds a real number to a floating-point value, under the
rounding mode $\circ$.

Expressions can be evaluated for their accuracy by the method as follows.
Initially the expression is parsed into a data flow graph (DFG). By way of
illustration, the sample expression ${(a + b)}^2$ has the tree structure
in Figure~\ref{bg:fig:sample_tree}. Then the exact ranges of values of $a$ and
$b$ are converted into the abstract semantics using a cast operation as in
\eqref{bg:eq:cast}:
\begin{equation}
    \mathrm{cast}\left(x^\sharp\right) = \left(
        \roundup{x^\sharp}, \rounddown{x^\sharp}
    \right)
    \label{bg:eq:cast}
\end{equation}
For example, for the real variable $a \in [0.2, 0.3]$ under single precision
with rounding to nearest,
\begin{equation}
    \mathrm{cast}\left([0.2, 0.3]\right) = \left(
        {\left[0.200000003, 0.300000012\right]}_\floatset,
        \left[-1/67108864, 1/67108864\right]
    \right)
\end{equation}
After this, the propagation of bounds in the data flow graph is carried out as
described in Section~\ref{bg:sub:intervals}, where the difference is the abstract
error semantics defined in~\eqref{bg:eq:error_semantics} is used in lieu of the
interval semantics. At the root of the tree (\ie~the exit of the DFG) we find
the value of the accuracy analysis result for the expression.

\section{Novel Semantics}
\label{so:sec:semantics}

\subsection{Accuracy Analysis}

In Section~\ref{bg:sub:accuracy} of Chapter~\ref{chp:background}, we described
a technique to analyze the round-off error of evaluating an expression tree.
Throughout this chapter, we use the function $\error: \exprset\to\errorset$ to
represent the above-mentioned analysis of evaluation accuracy, where $\exprset$
denotes the set of all expressions.


\subsection{Resource Usage Analysis}

Here we define similar formal semantics which calculate an approximation to the
FPGA resource usage of an expression, taking into account common subexpression
elimination. This is important as, for example, rewriting $a \times b + a
\times c$ as $a \times (b + c)$ in the larger expression $(a \times b + a
\times c) + {(a \times b)}^2$ causes the common subexpression $a \times b$ to
be no longer present in both terms. Our analysis must capture this.

The analysis proceeds by labelling subexpressions. Intuitively, the set of
labels $\labelset$, is used to assign unique labels to unique expressions,
so it is possible to easily identify and reuse them. For convenience, let
the function $\fresh: \exprset\to\labelset$ assign a distinct label to each
expression or variable, where $\exprset$ is the set of all expressions. Before
we introduce the labeling semantics, we define the environment $\lambda:
\labelset\to\exprset\cup\{\bot\}$, which is a function that maps labels to
expressions, and $\env{}$ denotes the set of such environments. A label $l$ in
the domain of $\lambda\in\env{}$ that maps to $\bot$ indicates that $l$ does
not map to an expression. An element $(l, \lambda)\in\labelset\times\env{}$
stands for the labeling scheme of an expression. Initially, we map all labels
to $\bot$, then in the mapping $\lambda$, each leaf of an expression is
assigned a unique label, and the unique label $l$ is used to identify the leaf.
That is for the leaf variable or constant $x$:
\begin{equation}
    (l, \lambda) = (\fresh(x), [\fresh(x)\mapsto{x}])
\end{equation}
This equation uses $[\fresh(x)\mapsto{x}]$ to indicate an environment that
maps the label $\fresh(x)$ to the expression $x$ and all other labels map
to $\bot$, in other words, if $l = \fresh(x)$ and $l^\prime \neq l$, then
$\lambda(l) = x$ and $\lambda(l^\prime) = \bot$.

\begin{figure}[ht]
    \centering
    \includegraphics[scale=0.6]{sample_tree}
    \caption{The DFG for the sample expression.}\label{so:fig:sample_tree}
\end{figure}
For example, for the DFG in Figure~\ref{so:fig:sample_tree}, taken from
Section~\ref{bg:sec:abstract_interpretation} of Chapter~\ref{chp:background},
we have for the variables $a$ and $b$:
\begin{equation}
    \begin{aligned}
        (l_a, \lambda_a) &= (\fresh(a), [\fresh(a)\mapsto{a}])
                   = (l_1, [l_1 \mapsto a]) \\
        (l_b, \lambda_b) &= (l_2, [l_2 \mapsto b])
    \end{aligned}
    \label{so:eq:variable_env}
\end{equation}
Then the environments are propagated in the flow direction of the DFG, using
the following formulation of the labeling semantics:
\begin{equation}
    \begin{aligned}
        (l_x, \lambda_x) \circ (l_y, \lambda_y)
            &= (l, (\lambda_x\odot\lambda_y)
                      [l\mapsto{l_x \circ l_y}]) \\
            \text{where~} l &= \fresh(l_x \circ l_y),
                          \circ\in\{+, -, \times\}
    \end{aligned}
    \label{so:eq:labeling_semantics}
\end{equation}
Specifically, $\lambda=\lambda_x\odot\lambda_y$ signifies that $\lambda_y$
is used to update the mapping in $\lambda_x$, if the mapping does not
exist in $\lambda_x$, and result in a new environment $\lambda$; and
$\lambda[l\mapsto{x}]$ is a shorthand for $\lambda\odot[l\mapsto{x}]$.  As
an example, with the expression in Figure~\ref{so:fig:sample_tree}, using
\eqref{so:eq:variable_env}, recall to mind that $l_1 = l_a$, $l_2 = l_b$, we
derive for the subexpression $a + b$:
\begin{equation}
    \begin{aligned}
        (l_{a + b}, \lambda_{a + b})
            &= (l_a, \lambda_a) + (l_b, \lambda_b) \\
            &= (l_3, (\lambda_a \odot \lambda_b) [l_3\mapsto{l_a + l_b}]) \\
            &  \text{where~} l_3 = \fresh(l_a + l_b) \\
            &= (l_3, [l_1\mapsto{a}]\odot
                     [l_2\mapsto{b}]\odot
                     [l_3\mapsto{l_1 + l_2}]) \\
            &= (l_3, [l_1\mapsto{a}, l_2\mapsto{b}, l_3\mapsto{l_1 + l_2}])
    \end{aligned}
\end{equation}
\todo{George: I am a little confused by addition here.  What is the definition
of ``+'' on labels?  If $l_a + l_b$ should be read purely as a syntactic
construct, why does it need a distinct representation as $l_3$?}
Finally, for the full expression $(a + b) \times (a + b)$:
\begin{equation}
    \begin{aligned}
        (l, \lambda)
            &= (l_{a + b}, \lambda_{a + b}) \times
               (l_{a + b}, \lambda_{a + b}) \\
            &= (l_4, [l_1\mapsto{a}, l_2\mapsto{b},
                      l_3\mapsto{l_1 + l_2}, l_4\mapsto{l_3 \times l_3}])
    \end{aligned}
\end{equation}
From the above derivation, it is clear that the semantics capture the reuse
of subexpressions. The estimation of area is performed by counting, for an
expression, the numbers of additions, subtractions and multiplications in
the final labeling environment, then calculating the number of LUTs used to
synthesize the expression. If the number of operators is $n_\circ$ where
$\circ\in\{+,-,\times\}$, then the number of LUTs in total for the expressions
is estimated as $\sum_{\circ\in\{+,-,\times\}} A_\circ n_\circ$, where the
value $A_\circ$ denotes the number of LUTs per $\circ$ operator, which is
dependent on the type of the operator and the floating-point format used to
generate the operator.

In the following sections, we use the function $\area: \exprset\to\naturalset$
to denote our resource usage analysis.

\subsection{Equivalent Expressions Analysis}
\label{so:sub:equivalent_expressions_analysis}

In earlier sections, we introduce semantics that define additions and
multiplications on intervals, then gradually transition to error semantics that
compute bounds of values and errors, as well as labelling environments that
allow common subexpression elimination, by defining arithmetic operations on
these structures. In this section, we now take the leap from not only analyzing
an expression for its quality, to defining arithmetic operations on sets of
equivalent expressions, and use these rules to discover equivalent expressions.
Before this, it is necessary to formally define equivalent expressions and
functions to discover them.

\subsubsection{Discovering Equivalent Expressions}

From an expression, a set of equivalent expressions can be discovered by our
\emph{equivalence relation} $\eqrel$ on the set of all expressions $\exprset$,
and $\eqrel \subset \exprset\times\exprset$.  It is noteworthy that a relation
is said to be an equivalence relation when it is reflexive, symmetric and
transitive, \ie~for all $e_1, e_2, e_3 \in \exprset$, we have the following
rules in our inference system:
\begin{equation}
    \begin{aligned}
        \text{Reflexivity}
            &: e_1 \eqrel e_1 \\
        \text{Symmetry}
            &: \text{~if~} e_1 \eqrel e_2,
            \text{~then~} e_2 \eqrel e_1 \\
        \text{Transitivity}
            &: \text{~if~} e_1 \eqrel e_2 \text{~and~} e_2 \eqrel e_3,
            \text{~then~} e_1 \eqrel e_3.
    \end{aligned}
    \label{so:eq:equivalence_relation}
\end{equation}

We extend our inference system with additional rules that relate equivalent
expressions.  Let's define $e_1, e_2, e_3 \in \exprset$, $v_1, v_2, v_3 \in
\realset$, and $\circ \in \{+, \times\}$.  First, the arithmetic rules are:
\begin{equation}
    \begin{aligned}
        \text{Associativity}(\circ)
            &: (e_1 \circ e_2) \circ e_3 \eqrel e_1 \circ (e_2 \circ e_3) \\
        \text{Commutativity}(\circ)
            &: e_1 \circ e_2 \eqrel e_2 \circ e_1 \\
        \text{Distributivity}
            &: e_1 \times (e_2 + e_3) \eqrel e_1 \times e_2 + e_1 \times e_3
    \end{aligned}
    \label{so:eq:equivalence_arithmetic}
\end{equation}
Secondly, the reduction rules are:
\begin{equation}
    \begin{aligned}
        \text{Identity}(\times)
            &: e_1 \times 1 \eqrel e_1 \quad &
        \text{Zero Propagation}
            &: e_1 \times 0 \eqrel 0 \\
        \text{Identity}(+)
            &: e_1 + 0 \eqrel e_1 &
        \text{Constant Propagation}(\circ)
            &: \inference{v_3 = v_1 \circ v_2}{v_1 \circ v_2 \eqrel v_3}
    \end{aligned}
    \label{so:eq:equivalence_reduction}
\end{equation}
The Constant Propagation rule states that if an expression is a
summation/multiplication of two values, then it can be simply evaluated to
produce the result. Finally, the following two allow structural induction on
expression trees, \eg~it is possible to derive that $a + (b + c) \eqrel a + (c
+ b)$ from $b + c \eqrel c + b$:
\begin{equation}
    \begin{aligned}
        \text{Tree}(\circ)
            : \inference{e_1 \eqrel e_2}{e_3 \circ e_1 \eqrel e_3 \circ e_2}
        \quad &
        \text{Tree}^\prime(\circ)
            : \inference{e_1 \eqrel e_2}{e_1 \circ e_3 \eqrel e_2 \circ e_3}
    \end{aligned}
    \label{so:eq:equivalence_tree}
\end{equation}

We say that $e_1$ is equivalent to $e_2$ if and only if $e_1 \eqrel e_2$ For
some expressions $e_1$ and $e_2$.  Although for simplicity, we have not defined
rules for subtraction and division, these rules can be easily added and are
present in our framework.


\subsubsection{Scalable Methods}

The above rules of equivalence relates an expression with all of its equivalent
expressions.  In general because of combinatorial explosion, the set of all
equivalent expressions is so large to be derived, which motivates us to develop
scalable methods that execute fast enough even with large expressions.

Instead of deriving the full set of equivalent expressions, we can define
a new relation $\eqgenrel$, a subset of $\eqrel$, which is identical to
our equivalent relation $\eqrel$ except that we do not have transitivity
in~\eqref{so:eq:equivalence_relation} from $\eqrel$, to generate equivalent
expressions in a series of steps.

We define the function $\eqstep: \powersetof\exprset\to\powersetof\exprset$,
where $\powersetof\exprset$ denotes the power set of $\exprset$, which
generates a (possibly larger) set of equivalent expressions from an initial set
of equivalent expressions by one step of $\eqgenrel$, that is:
\begin{equation}
    \eqstep(\epsilon) = \left\{
        e^\prime\in\exprset \mid
        e \eqgenrel e^\prime \wedge e\in\epsilon\right\}
    \label{so:eq:eqstep}
\end{equation}
where $\epsilon$ is a set of equivalent expressions.
\begin{corollary}
    By the definition of $\eqstep$ in~\eqref{so:eq:eqstep}, $\eqstep(\epsilon_a
    \cup \epsilon_b) = \eqstep(\epsilon_a) \cup \eqstep(\epsilon_b)$.
    \label{so:cor:union}
\end{corollary}

From this, we may note that we can define a function
$\eqstep^{\star:N}(\epsilon)$ to generate a set of equivalent expressions,
by taking the union of $N$ steps of $\eqstep$ of $\epsilon$, as given by the
following formula:
\begin{equation}
    \eqstep^{\star:N}(\epsilon) = \bigcup_{i = 0}^N \eqstep^i(\epsilon)
    \label{so:eq:transitive_generator}
\end{equation}
Here we define:
\begin{equation}
    \begin{aligned}
        \eqstep^0(\epsilon) &= \epsilon \quad \text{and~} \\
        \eqstep^i(\epsilon) &= \eqstep\left(
            \eqstep^{i - 1}\left(\epsilon\right)
        \right) \quad \text{for~} i \in \{ 0, 1, 2, \cdots \}
    \end{aligned}
\end{equation}
By allowing $N$ to approach $\infty$, we obtain the full set of equivalent
expressions of $\epsilon$, \ie~the transitive closure:
\begin{equation}
    \eqstep^\star(\epsilon)
    = \eqstep^{\star:\infty}(\epsilon)
    = \bigcup_{i = 0}^\infty \eqstep^i(\epsilon)
    \label{so:eq:transitive_closure}
\end{equation}

\begin{lemma}
    $\eqstep^{\star:N}(\epsilon) = \epsilon \cup
    \eqstep\left(\eqstep^{\star:N-1}(\epsilon)\right)$.
    \label{so:lem:transitive}
\end{lemma}
\begin{proof}
    Following~\eqref{so:eq:transitive_generator}, $\eqstep^{\star:N}(\epsilon)
    = \eqstep^0(\epsilon) \cup \eqstep^1(\epsilon) \cup \cdots \cup
    \eqstep^N(\epsilon)$.  We then apply Corollary~\eqref{so:cor:union} to the
    right-hand side to get $\epsilon \cup \eqstep\left( \eqstep^0(\epsilon)
    \cup \eqstep^1(\epsilon) \cup \cdots \cup \eqstep^{N-1}(\epsilon)\right)$,
    which equals to $\epsilon \cup \eqstep\left( \eqstep^{\star:N-1}(\epsilon)
    \right)$ by definition.
\end{proof}

In practice, it is often infeasible to generate the full transitive closure of
a given expression, we therefore impose further constraints on how we discover
equivalent expressions.

First, instead of exploring the full transitive closure, that is, by allowing
the number of steps $N$ in~\eqref{so:eq:transitive_generator} to be infinite,
we may restrict $N$ to be a small finite value to allow a smaller set of
equivalent expressions to be computed.

Second, the complexity of equivalent expression finding is reduced by fixing
the structure of subexpressions at a certain depth $k$ in the original
expression.  The definition of depth is given as follows: first the root
of the parse tree of an expression is assigned depth $d = 1$; then we
recursively define the depth of a node as one more than the depth of its
greatest-depth parent.  If the depth of the node is greater than $k$, then
we fix the structure of its child nodes by disallowing any equivalence
transformation beyond this node. We let $\eqstep_k$ denote this ``depth
limited'' equivalence finding function, where $k$ is the depth limit used.  We
use $\eqstep^{\star:N}_k$ and $\eqstep^\star_k$ to denote the functions to
respectively compute the union of $N$ steps of $\eqstep_k$ and the transitive
closure. This approach is similar to Martel's depth limited equivalent
expression transform~\cite{martel07}, however Martel's method eventually allows
transformation of subexpressions beyond the depth limit, because rules of
equivalence would transform these to have a smaller depth.  This contributes
to a time complexity at least exponential in terms of the expression size. In
contrast, our technique has a time complexity that does not depend on the size
of the input expression, but grows with respect to the depth limit $k$. Note
that the full equivalence closure using the inference system we defined earlier
in~\eqref{so:eq:transitive_closure} is at least $O({2n - 1}!!)$ where $n$ is
the number of terms in an expression, as we discussed earlier.

Finally, we use the iterative algorithm in Figure~\ref{so:alg:eqstep} to
efficiently compute $\eqstep^{\star:N}_k$.  In each iteration, we keep track of
the equivalent expressions that are newly discovered in the current iteration,
so that in the next iteration we apply $\eqgenrel$ only to those expressions,
to avoid redundant computation.  We then continue to prove that this algorithm
indeed computes $\eqstep^{\star:N}_k$.
\begin{figure}[ht]
    \centering
    \begin{algorithmic}
        \Function{Equivalent}{$\epsilon$, $k$, $N$}
            \State{$s_0 \gets \epsilon$}
            \State{$s^\prime_0 \gets \epsilon$}
            \For{$i \gets 1, \ldots, N$}
                \State{$s^\prime_i \gets
                    \eqstep_k \left(s^\prime_{i-1}\right) - s_{i-1}$}
                \State{$s_i \gets s_{i-1} \cup s^\prime_i$}
                \If{$s^\prime_i \neq \emptyset$}
                    \State{\Return{$s_i$}}
                \EndIf{}
            \EndFor{}
            \State{\Return{$s_i$}}
        \EndFunction{}
    \end{algorithmic}
    \caption{%
        Our algorithm to discover a set of equivalent expressions from an
        initial set $\epsilon$.
    }\label{so:alg:eqstep}
\end{figure}
\begin{theorem}
    In the algorithm in Figure~\ref{so:alg:eqstep}, at iteration
    $n$, the set of equivalent expressions $s_n$ computes exactly
    $\eqstep^{\star:n}_k(\epsilon)$.
\end{theorem}
\begin{proof}
    We start by assuming that at iteration $m > 0$, $s_m =
    \eqstep^{\star:m}_k(\epsilon)$, and we prove this equality still holds if
    substitute $m$ with $m + 1$.  From the algorithm, we can deduce:
    \begin{equation*}
    \begin{aligned}
        s_{m+1}
         &= s_m \cup s^\prime_{m+1} \\
         &= s_m \cup \left( \eqstep_k \left( s^\prime_m \right) - s_m \right) \\
         &= s_m \cup \eqstep_k \left( s^\prime_m \right) \\
         &= s_m \cup \eqstep_k \left(
                \eqstep_k \left( s^\prime_{m-1} \right) - s_{m-1}
            \right)
    \end{aligned}
    \end{equation*}
    We substitute $s_m$ using Lemma~\ref{so:lem:transitive} to get:
    \begin{equation*}
        s_{m+1}
          = \epsilon \cup \eqstep_k \left( s_{m-1} \right) \cup
            \eqstep_k \left(
                \eqstep_k \left( s^\prime_{m-1} \right) - s_{m-1}
            \right)
    \end{equation*}
    Using distributivity of $\eqstep_k$ over $\cup$ and the iteration $m$ of
    the algorithm, we can derive:
    \begin{equation*}
    \begin{aligned}
        s_{m+1}
         &= \epsilon \cup \eqstep_k \left(
                s_{m-1} \cup \left(
                    \eqstep_k \left( s^\prime_{m-1} \right) - s_{m-1}
                \right)
            \right) \\
         &= \epsilon \cup \eqstep_k \left( s_m \right)
    \end{aligned}
    \end{equation*}
    Finally, we make use of the assumption $s_m =
    \eqstep^{\star:m}_k(\epsilon)$, followed by Lemma~\ref{so:lem:transitive}
    to show:
    \begin{equation*}
        s_{m+1}
        = \epsilon \cup \eqstep_k \left(
            \eqstep^{\star:m}_k(\epsilon)
        \right)
        = \eqstep^{\star:m+1}_k(\epsilon)
    \end{equation*}
    It is trivial that $s_0 = \epsilon = \eqstep^{\star:0}_k(\epsilon)$, by
    induction, $s_n = \eqstep^{\star:n}_k(\epsilon)$ thus holds for all $n \in
    \naturalset$.
\end{proof}

\subsubsection{Pareto Frontier}

Because we optimize expressions in two quality metrics, \ie~the accuracy of
computation and the estimate of FPGA resource utilization, there is a trade-off
between them. We desire the largest subset of all equivalent expressions
$E$ discovered such that in this subset, no expression dominates any other
expression, in terms of having both better area and better accuracy. This
subset is known as the Pareto frontier.  Figure~\ref{so:alg:pareto} shows
a simplified algorithm for calculating the Pareto frontier for a set of
equivalent expressions $\epsilon$.
\begin{figure}[ht]
    \centering
    \begin{algorithmic}
        \Function{Frontier}{$\epsilon$}
            \State{$\mathit{frontier} \gets \epsilon$}
            \For{$e \in \epsilon$}
                \For{$e^\prime \in \epsilon$}
                    \If{$\mathit{Area}(e^\prime) < \mathit{Area}(e)$ and
                        $\abserr(e^\prime) < \abserr(e)$}
                        \State{%
                            $\mathit{frontier} \gets
                                \mathit{frontier} / \{ e \}$}
                    \EndIf{}
                \EndFor{}
            \EndFor{}
            \State{\Return{$\mathit{frontier}$}}
        \EndFunction%
    \end{algorithmic}
    \caption{The Pareto frontier from a set of equivalent expressions.
    }\label{so:alg:pareto}
\end{figure} \\
Here, $\mathit{frontier} / \{ e \}$ is a set identical to $\mathit{frontier}$,
except that the element $e$ is removed.  We use the function $\abserr$ to
analyze the magnitudes of error bounds, which is defined as follows:
\begin{equation}
    \begin{aligned}
        \abserr(e) &= \max\left(
            \left| \mu^\sharp_{\min} \right|,
            \left| \mu^\sharp_{\max} \right|
        \right) \\
        & \quad \text{where~}
        \left(
            x^\sharp, \left[ \mu^\sharp_{\min}, \mu^\sharp_{\max} \right]
        \right) = \error(e)
    \end{aligned}
\end{equation}

\subsubsection{Equivalent Expressions Semantics}

Similar to the analysis of accuracy and resource usage, a set of equivalent
expressions can be computed with semantics. That is, we define structures,
\ie~sets of equivalent expressions, that can be manipulated with arithmetic
operators. In our equivalent expressions semantics, an element of
$\powersetof\exprset$ is used to assign a set of expressions to each node
in an expression parse tree. To begin with, at each leaf of the tree, the
variable or constant is assigned a set containing itself, as for $x$, the set
$\epsilon_x$ of equivalent expressions is $\epsilon_x = \{x\}$. After this, we
propagate the equivalence expressions in the parse tree's direction of flow,
using~\eqref{so:eq:equivalence_semantics} defined below:
\begin{equation}
    \begin{aligned}
        \epsilon_x \circ \epsilon_y &= \frontier\left(
            \eqstep^\star_k \left(
                E_\circ \left( \epsilon_x, \epsilon_y \right)
            \right) \right) \\
        & \text{where~}
        E_\circ(\epsilon_x, \epsilon_y) = \{
            e_x \circ e_y \mid e_x \in \epsilon_x \wedge e_y \in \epsilon_y
        \}, \\
        & \text{and~} \circ\in\{+, -, \times\}
    \end{aligned}
    \label{so:eq:equivalence_semantics}
\end{equation}
The equation implies that in the propagation procedure, it recursively
constructs a set of equivalent subexpressions for the parent node from
two child expressions, and uses the depth limited equivalence function
$\eqstep^\star_k$ to work out a larger set of equivalent expressions. To reduce
computation effort, we select only those expressions on the Pareto frontier
for the propagation in the DFG\@. Although in worst case the complexity of
this process is exponential, the selection by Pareto optimality accelerates
the algorithm significantly. For example, for the subexpression $a + b$ of our
sample expression:
\begin{equation}
    \begin{aligned}
        \epsilon_a + \epsilon_b
            &= \frontier\left(
                    \eqstep^\star_k \left(
                        E_\circ \left( \epsilon_a, \epsilon_b \right)
                    \right)
                \right) \\
            &= \frontier\left(
                    \eqstep^\star_k \left(
                        E_\circ \left( \{a\}, \{b\} \right)
                    \right)
                \right) \\
            &= \frontier\left(
                    \{ a + b, b + a \}
                \right)
    \end{aligned}
\end{equation}
Alternatively, we could view the semantics in terms of DFGs representing
the algorithm for finding equivalent expressions. The parsing of an
expression directly determines the structure of its DFG\@. For instance,
the expression $(a + b) \times (a + b)$ generates the DFG illustrated in
Figure~\ref{so:fig:tree_expr_flow}. The circles labeled $3$ and $7$ in this
diagram are shorthands for the operation $E_+$ and $E_\times$ respectively,
where $E_+$ and $E_\times$ is defined in~\eqref{so:eq:equivalence_semantics}.
\begin{figure}[ht]
    \centering
    \includegraphics[scale=0.6]{tree_expr_flow}
    \caption{The DFG for finding equivalent expressions of
    $(a + b) \times (a + b)$.}\label{so:fig:tree_expr_flow}
\end{figure}

For our example in Figure~\ref{so:fig:tree_expr_flow},
similar to the construction of data flow equations in
Section~\ref{bg:sec:abstract_interpretation} of Chapter~\ref{chp:background},
we can produce a set of equations from the data flow of the DFG, which now
produces equivalent expressions:
\begin{equation}
    \begin{aligned}
        \enter{1} &= \enter{1} \cup \{a\} &
        \enter{2} &= \enter{2} \cup \{b\} \\
        \enter{3} &= E_+(\enter{1}, \enter{2}) &
        \enter{4} &= \enter{3} \cup \enter{5} \\
        \enter{5} &= \eqstep_k(\enter{4}) &
        \enter{6} &= \frontier(\enter{5}) \\
        \enter{7} &= E_\times(\enter{6}, \enter{6}) &
        \enter{8} &= \enter{7} \cup \enter{9} \\
        \enter{9} &= \eqstep_k(\enter{8}) &
        \enter{10} &= \frontier(\enter{9})
    \end{aligned}
    \label{so:eq:tree_expr_flow}
\end{equation}
Because of loops in the DFG, it is no longer trivial to find the solution.
In general, the analysis equations are solved iteratively. We can
regard the set of equations as a single transfer function $F$ as in
\eqref{so:eq:transfer_function}, where the function $F$ takes as input
the variables $A(1), \ldots, A(10)$ appearing in the right-hand sides of
\eqref{so:eq:tree_expr_flow} and outputs the values $A(1), \ldots, A(10)$
appearing in the left-hand sides. Our aim is then to find an input $\vec{x}$ to
$F$ such that $F(\vec{x}) = \vec{x}$, \ie~a fixpoint of $F$.
\begin{equation}
      F((\enter{1}, \ldots, \enter{10}))
    = (\enter{1} \cup \{a\}, \ldots, \frontier(\enter{9}))
    \label{so:eq:transfer_function}
\end{equation}
Initially we assign $\enter{i} = \varnothing$ for $i\in\{1,2,\ldots,10\}$,
and we denote $\vec\varnothing = (\varnothing, \ldots, \varnothing)$.
Then we compute iteratively $F(\vec\varnothing)$, $F^2(\vec\varnothing) =
F(F(\vec\varnothing))$, and so forth, until the fixpoint solution $\fix F$ is
reached for some iteration $n$, that is:
\begin{equation}
    \fix F = F^n(\vec\varnothing) =
    F(F^n(\vec\varnothing)) = F^{n + 1}(\vec\varnothing)
\end{equation}
The fixpoint solution $\fix F$ gives a set of equivalent expressions derived
using our method, which is found at $\enter{10}$. In essence, the depth limit
acts as a sliding window.  The semantics allow hierarchical transformation of
subexpressions using a depth-limited search and the propagation of a set of
subexpressions that are locally Pareto optimal to the parent expressions in a
bottom-up hierarchy.

The problem with the semantics above is that the time complexity of
$\eqstep^\star_k$ scales poorly, since the worst case number of subexpressions
needed to explore increases exponentially with $k$. Therefore an alternative
method is to optimize it by changing the structure of the DFG slightly, as
shown in Figure~\ref{so:fig:tree_expr_flow_greedy}. The difference is that at
each iteration, the Pareto frontier filters the results to decrease the number
of expressions to process for the next iteration.
\begin{figure}[ht]
    \centering
    \includegraphics[scale=0.6]{tree_expr_flow_greedy}
    \caption{The alternative DFG for $(a + b) \times (a + b)$.
    }\label{so:fig:tree_expr_flow_greedy}
\end{figure}

In the rest of this chapter, we use \frontiertrace{} to indicate our equivalent
expression finding semantics, and \greedytrace{} to represent the alternative
method.

\section{Implementation}
\label{so:sec:implementation}

The majority of \soap, is implemented in Python.  For computing errors in
real arithmetic, we use exact arithmetic based on rational numbers within the
\gls{gmp} library~\cite{gmp}.  In case when exact arithmetic is not possible
because of high computational costs, floating-point arithmetic can be used
to efficiently and safely bound round-off error values.  We also use the
\gls{mpfr} library~\cite{mpfr} for access to floating-point rounding modes and
arbitrary precision floating-point computation.

Because of the workload of equivalent expression finding, the underlying
algorithm is optimized in many ways. First, for each iteration, the relation
finding function $\eqstep_k$ is only applied to newly discovered expressions
in the previous iteration, using the algorithm in Figure~\ref{so:alg:closure}.
The second optimization is to cache results of function calls such as
$\eqstep_k$, $\area$ and $\error$, since there is a large chance that these
results from subexpressions are reused several times, subexpressions are also
maximally shared to eliminate duplication in memory.  Thirdly, the computation
of $\eqstep_k$ is fully multi-threaded.

The resource statistics of operators are provided using FloPoCo~\cite{flopoco}
and \gls{xst}~\cite{xst}.  Initially, For each combination of an operator,
an exponent width between 5 and 16, and a mantissa width ranging from 10 to
113, a total of 2496 distinct implementations are generated using FloPoCo.
All of them are optimized to use \gls{dsp} blocks.  They are then synthesized
using \gls{xst}, targeting a Virtex-6 \gls{fpga} device (XC6VLX760).  Because
\glspl{lut} are generally more constrained resources than \gls{dsp} blocks in
floating-point computations, we provide synthesis statistics in \glspl{lut}
only.  Finally, an \gls{rtl} code generation backend can produce synthesizable
code from an optimized candidate expression.

\section{Results}
\label{so:sec:results}

Because Martel's approach defers selecting optimal options until the end of
equivalent expression discovery, we developed a method that could produce
exactly the same set of equivalent expressions from the traces computed by
Martel, and has the same time complexity. The difference is that we adopted it
to generate a Pareto frontier from the discovered expressions, instead of only
error bounds.  This allows us to compare \marteltrace{}, \ie~our implementation
of Martel's method, against our methods \frontiertrace{} and \greedytrace{}
discussed in Section~\ref{so:sec:equivalent}.  Figure~\ref{so:fig:martel}
optimizes the expression ${(\vara + \varb)}^2$ using the three methods above,
all using depth limit $3$, and the input ranges are $\vara \in [5, 10]$
and $\varb \in [0, 0.001]$~\cite{martel07}. The IEEE 754 single-precision
floating-point format with rounding to nearest was used for the evaluation
of accuracy and area estimation. The scatter points represent different
implementations of the original expression that have been explored and
analyzed, and the (overlapping) lines denote the Pareto frontiers. In this
example, our methods produce the same Pareto frontier that Martel's method
could discover, while having up to 50\% shorter run time. Because we consider
an accuracy/area trade-off, we find that we can not only have the most accurate
implementation discovered by Martel, but also an option that is only 0.0005\%
less accurate, but uses 7\% fewer \glspl{lut}.

We go beyond the optimization of a small expression, by generating results in
Figure~\ref{so:fig:multi_expr_32} to show that the same technique is applicable
to simultaneous optimization of multiple large expressions. The expressions
$e_1$ and $e_2$, with input ranges $\vara \in [1, 2], \varb \in [10, 20], \varc
\in [10, 200]$ are used as our example:
\begin{equation}
    \begin{aligned}
    e_1 =&
        (\vara + \vara + \varb) \times
        (\vara + \varb + \varb) \times
        (\varb + \varb + \varc) \times {} \\
        &
        (\varb + \varc + \varc) \times
        (\varc + \varc + \vara) \times
        (\varc + \vara + \vara), \\
    e_2 =&
        (1 + \varb + \varc) \times
        (\vara + 1 + \varc) \times
        (\vara + \varb + 1).
    \end{aligned}
\end{equation}

We generated and optimized \gls{rtl} implementations of $e_1$ and
$e_2$ simultaneously using \frontiertrace{} and \greedytrace{}
with the depth limits indicated by the numbers in the legend of
Figure~\ref{so:fig:multi_expr_32}. Note that because the expressions evaluate
to large values, the errors are also relatively large. We set the depth limit
to $2$ and found that \greedytrace{} executes up to $10\times$ faster than
\frontiertrace{}, while discovering a sizable subset of the Pareto frontier of
\frontiertrace{}. Also our methods are significantly faster and more scalable
than \marteltrace{}, because of its poor scalability discussed earlier, our
computer ran out of 8 GB of memory before we could produce any results. If we
normalize the time allowed for each method and compare the performance, we
found that \greedytrace{} with a depth limit $3$ takes takes slightly less time
than \frontiertrace{} with a depth limit $2$, but produces a generally better
Pareto frontier. The alternative implementations of the original expression
provided by the Pareto frontier of \greedytrace{} can either reduce the
\glspl{lut} used by approximately 10\% when accuracy is not crucial, or can
be about 10\% more accurate if resource is not our concern.  It also enables
the ability to choose different trade-off options, such as an implementation
that is 7\% more accurate and uses 7\% fewer \glspl{lut} than the original
expression.

Furthermore, Figure~\ref{so:fig:multi_expr_vary_width} varies the mantissa
width of the floating-point format, and presents the Pareto frontier
of both $e_1$ and $e_2$ together under optimization. Floating-point
formats with mantissa widths ranging from 10 to 112 bits were used to
optimize and evaluate the expressions for both accuracy and area usage. It
turns out that some implementations originally on the Pareto frontier of
Figure~\ref{so:fig:multi_expr_32} are no longer desirable, as by varying the
mantissa width, new implementations are both more accurate and less resource
demanding.

Besides the large example expressions above, Figure~\ref{so:fig:taylor_sin}
and Figure~\ref{so:fig:motzkin} are produced by optimizing expressions with
real applications under single precision. Figure~\ref{so:fig:taylor_sin} shows
the optimization of the Taylor expansion of $\sin(x + y)$, where $x\in[-0.1,
0.1]$ and $y\in[0, 1]$, using \greedytrace{} with a depth limit $3$. The
function $\mathrm{taylor}(f, d)$ indicates the Taylor expansion of function
$f(x, y)$ at $x = y = 0$ with a maximum degree of $d$. For order 5 we reduced
error by more than 60\%. Figure~\ref{so:fig:motzkin} illustrates the results
obtained using the depth limit $3$ with the Motzkin polynomial~\cite{demmel}
$x^6 + y^4 z^2 + y^2 z^4 - 3 x^2 y^2 z^2$, which is known to be difficult to
evaluate accurately, especially using inputs $x\in[-0.99, 1]$, $y\in[1, 1.01]$,
$z\in[-0.01, 0.01]$.

All these above results are generated with the same type of floating-point
operators in each expression.  Although in this chapter we do not analyze
the number of \glspl{dsp} used in synthesized circuits, the \gls{dsp} count
increases linearly with the estimated \gls{lut} count.  In the next chapter
we further introduce the estimation of \gls{dsp} elements used as another
objective to optimize.

Because of the scalability problem of the depth limit $k$ mentioned in
Section~\ref{so:sec:equivalent}, $k \leq 3$ for all of our experiments.  By
setting $k = 4$, the tool does not terminate in reasonable amount of time and
saturates the memory (16 GB) of our system.  In the following chapters, we
propose methods to limit the number of iterations and the number of equivalent
expressions discovered to mitigate the lack of scalability of $k$.

Finally, Figure~\ref{so:fig:area} demonstrates the accuracy of the area
estimation used in our analysis. It compares the actual \glspl{lut} necessary
with the estimated number of \glspl{lut} using our semantics, by synthesizing
more than 6000 equivalent expressions derived from $\vara + \varb + \varc$,
$(\vara + 1) \times (\varb + 1) \times (\varc + 1)$, $e_1$, and $e_2$ using
varying mantissa widths. The dotted line indicates exact area estimation, a
scatter points that is close to the line means the area estimation for that
particular implementation is accurate. The solid black line represents the
linear regression line of all scatter points. On average, our area estimation
is a 6.1\% over-approximation of the actual number of \glspl{lut}, and the
worst case over-approximation is 7.7\%.
\newcommand{\figsize}{0.5}
\begin{figure}[ht]
    \centering
    \includegraphics[scale=\figsize]{martel}
    \caption{Optimization of ${(\vara + \varb)}^2$.}\label{so:fig:martel}
\end{figure}
\begin{figure}[ht]
    \centering
    \includegraphics[scale=\figsize]{multi_expr_32}
    \caption{%
        Simultaneous optimization of both $e_1$ and $e_2$.
    }\label{so:fig:multi_expr_32}
\end{figure}
\begin{figure}[ht]
    \centering
    \includegraphics[scale=\figsize]{multi_expr_vary_width}
    \caption{%
        Varying the mantissa width of Figure~\ref{so:fig:multi_expr_32}.
    }\label{so:fig:multi_expr_vary_width}
\end{figure}
\begin{figure}[ht]
    \centering
    \includegraphics[scale=\figsize]{taylor_sin}
    \caption{The Taylor expansion of $\sin(x + y)$.}\label{so:fig:taylor_sin}
\end{figure}
\begin{figure}[ht]
    \centering
    \includegraphics[scale=\figsize]{motzkin}
    \caption{The Motzkin polynomial $e_m$.}\label{so:fig:motzkin}
\end{figure}
\begin{figure}[ht]
    \centering
    \includegraphics[scale=\figsize]{area}
    \caption{Accuracy of Area Estimation.}\label{so:fig:area}
\end{figure}

High-level synthesis tools are typically designed to adhere to a rigid
specification which outlines their behaviour.  It is a traditional practice
to design this specification and the subsequent tool to ensure that the
synthesized circuits perform functionally identical to the original source
program written in high-level language.  It is also viewed as a good practice
because it has predicable outcomes.  Guided by the rules of the language,
programmers translate mathematical objects such as algorithms and physical
information respectively into source code and numerical data, in a way similar
to tools adhering to their specifications.  This manual process of translation
is unfortunately an approximate one.  Computations as simple as $\sqrt{3}$ must
be approximated, \eg~they are carried out in floating-point arithmetic, because
of the finite nature of computing machines.  Therefore, \gls{hls} tools cannot
be relied upon for an exact interpretation of the mathematical objects we wish
to implement, even if they guarantee the functional equivalence between the
source code and the synthesized result.

Despite the awareness of the approximate characteristic of numerical
software/hardware implementations using floating-point operations, engineers
often take the risks of neglecting this fact, and anticipate their designs to
behave identically to the mathematical algorithms visioned in real arithmetic
within a reasonable but not well-defined error margin.  As it was shown in
Section~\ref{bg:sub:expression_accuracy} in Chapter~\ref{chp:background},
round-off errors when accumulated, could have detrimental effects on our daily
life.  The aforementioned functional equivalence between source and circuit
guaranteed by \gls{hls} tools is therefore unable to regain any lost accuracies
due to approximation.

Traditional \gls{ir}-level \gls{hls} program optimization consist of a series
of transformation passes.  Most of these passes do not predict whether
they have negative impact on the resulting circuit, and they limit their
capabilities by preserving functional equivalence.  Varying the order of
these passes could have significant impact on the quality, as these passes
interact with one another in a complicated manner, it is difficult to predict
the overall impact on performance~\cite{huang15}.  For $n$ passes, there are
$n{\,!}$ distinct ways to order, it is thus a considerable challenge to decide
the optimal pass ordering, which is exacerbated by the fact that it could be
highly dependent on the input program~\cite{cong13}.

These above shortcomings of traditional \gls{hls} tools and optimizing
compilers provide a strong motivation for the work proposed in this thesis.

Firstly, we can apply the philosophy of relaxing the functional equivalence
required by \gls{hls} tools.  In the mean time, we preserve the equivalence of
the underlying mathematical objects in real arithmetic which hardware designs
are approximating.  One can often improve the numerical accuracy by choosing a
better alternative among these equivalences.

Secondly, by the same paradigm shift, a wide range of optimization
opportunities can be explored to minimize throughput and resource utilization.
These opportunities were previously lost out to the necessity of ensuring
consistent behaviour.

Finally, optimization can be carried out by applying steps of equivalence
rewrites driven by a prediction model.  Traditional optimization passes can
be broken up into much smaller common parts made of equivalence rules can be
easily proved mathematically correct.  By using models to predict run time,
resources and accuracy to guide the optimization process, it is possible to
explore multiple designs that trade-off the three performance metrics while
removing concerns about the ordering problem.  Many optimization passes, such
as constant propagation, dead code removal, common subexpression elimination
and~\etc, are naturally subsumed by the new approach.  As the computational
power of machines increases exponentially, we can foresee an increase in the
scale of the vast search space to be explored in the future.

This thesis therefore broadens the horizon of \gls{hls} tools, and equips
them with the new program optimization paradigm by leveraging these above
observations.  Specifically, the trade-off relationship among numerical
accuracy, resource utilization and throughput are optimized in floating-point
numerical programs for \gls{hls}\@.  Here we summarize the contributions of
this thesis.

To the best of our knowledge, this thesis is the first to introduce
multiple-objective performance optimization in a unified framework for
discovering equivalence in programs.  Chapter~\ref{chp:stropt} implements
this framework and optimizes a suite of expressions that are difficult to
optimize by hand, and improve numerical accuracy and area automatically.
In the experimental results, it turns out that the two central goals,
\ie~improving accuracy and minimizing area, are often not in conflict, as
optimized expressions can enjoy almost all enhancements that can be achieved
in both metrics.  Guided by the concept of abstract interpretation, it further
introduces the semantics-based program analyses to jointly reason about
safe ranges of round-off errors and resource utilization, and subsequently,
discovery of equivalent expressions.  This technique lays the necessary
foundation for program equivalence beyond simple arithmetic expressions.

The infinite size of the equivalent program space, coupled with undecidability
of program properties, makes the program optimization an even more
challenging task than the one of arithmetic expressions.  For this,
Chapter~\ref{chp:progopt} introduces a new graph-based intermediate
representation, \gls{mir}, for capturing the semantics of numerical programs.
This approach reduces the size of the search space, and the \gls{ir} itself
is derived from the formal semantics of programs to ensure the correctness
of equivalent \glspl{mir} and the back-and-forth translation between C and
\gls{mir}\@.  It further eliminates the problem of optimization pass ordering,
because by using the equivalence discovery framework, the Pareto frontier can
be extended incrementally with small steps of rewrites to multiple candidates.
Traditional compiler optimizations are naturally subsumed and further enhanced
by the \glspl{mir}, as many optimization techniques such as loop splitting and
loop fusion that previously must be profiled to justify enabling them, can
emerge automatically from the optimization process.  By optimizing a suite of
resource-efficient benchmark examples, the tool improves the numerical accuracy
by up to 65\%.

Formerly, \gls{hls} tools' ability to pipeline loops is fundamentally constrained
by intra-iteration dependencies.  Traditional optimization techniques such as
partial loop unrolling may have minimal effects on the initiation interval of
pipelined loops, as these do not impact the data-path structure, which ensures
that the functional equivalence is preserved.  Encouraged by the promising
effects of Nicolau~\etal's tree height reduction technique~\cite{nicolau91}
and LegUp's recurrence minimization~\cite{canis14}, Chapter~\ref{chp:latopt}
further incorporates latency analysis into the unified program optimization
framework.  It was found that traditional optimization techniques when used
in tandem with the arithmetic equivalence rules and memory access reduction
rules can significantly improve the latency and accuracy of a numerical
program.  In Chapter~\ref{chp:progopt}, the experimental results identifies
that the static analysis of round-off errors for each candidate explored is
the key factor to the speed of optimization.  This problem is addressed in
this chapter by graph partitioning and candidate pruning algorithms.  It
further enables deeper partial loop unrolling factors that was not explored in
Chapter~\ref{chp:progopt}.  Often as we optimize numerical programs by spending
more resources, latency and round-off error can be simultaneously minimized, as
more resources would allow greater flexibility to discover equivalent programs
that often perform well in terms of run time and accuracy.  By optimizing
a suite of benchmark examples from PolyBench and Livermore loops, the tool
improves the latency and accuracy of each by up to 12$\times$ and 7$\times$
respectively, at a cost of 4$\times$ more resource utilization.


\section{Future Prospects}
\label{cc:sec:future_prospects}

In its current form, the new approach to program optimization explained in this
thesis forms the underlying basis for a much larger set of future work.  Even
though it is precursory on its own, the promising experimental results showcase
the powerful optimization it can bring to optimizing compilers and \gls{hls}
tools.  Here, a list of potential directions of future research is discussed
that could further widen the scope of our technique for a broader range of
applications.

\textbf{\gls{llvmir}-Level Program Optimization.}  We could envision
a back-and-forth translator from \gls{llvmir}~\cite{llvm, llvm_ir} to
\gls{mir} graphs.  This could enable a much wider applicability of the
technique presented in the thesis to both \gls{llvm}-based \gls{hls} tools
and software compilers.  Additionally, it could benefit from existing
\gls{llvm} optimizations passes by using the optimized \gls{llvmir} code
as inputs.  There are however obstacles in migrating to \gls{llvmir} as
the source language.  Firstly, \gls{llvmir} is \gls{ssa}-based.  Since it
uses temporary variables for intermediate results in computation, a full
liveness analysis~\cite{hathhorn12, nielson99, boissinot08} may be necessary
to eliminate temporary variables from the resulting \gls{mir}\@.  Secondly,
control-flows in \gls{llvmir} are more freely structured.  Unlike C, which
defines \iflit~statements and \whilelit~loops and discourages the use of
\verb|goto| statements, control-flow in \gls{llvmir} are composed by basic
blocks and branches between pairs of them.  This requires the \gls{mir} to be
further extended to cope with complex control-flow patterns.  Conventionally,
programs written with branches are often analyzed using \emph{continuation
style semantics}~\cite{felleisen88}.  It is not evident how this semantics can
be embedded within \glspl{mir}.

\textbf{Tighter bounds on round-off errors.} As an alternative for interval
analysis, the accuracy analysis could enjoy more sophisticated abstract domains
that capture the correlations between variables, and produce tighter bounds
for results.  Currently, the analysis cannot produce meaningful, \ie~finite,
bounds on the round-off errors of certain numerical programs.  If the analysis
fails to bound errors, then currently the optimization cannot be directed
to a more accurate implementation.  By using abstract relational domains,
it is possible to produce a much tighter bound on the values of program
variables, and the associated errors.  There are a few relational domains-based
static analysis techniques of floating-point errors~\cite{mine07_2, putot04,
goubault11, astree}, however making use of them still poses challenges.  Each
floating-point operation introduces an independent error term as a new variable
in the formulation of these relational domains, and it may be difficult to
determine how to collapse these error terms into a smaller set of variables,
as the optimization in this thesis can introduce a large number of error
variables.

\textbf{Special and fused operators.} There could be a lot of interest in the
\gls{hls} community on how \soap~can be incorporated with existing work on
fused floating-point data-path synthesis.  Langhammer~\etal~\cite{langhammer}
propose that normalization and denormalization stages could be regarded as
redundant between operators in a floating-point data-path.  By removing
these stages, subsets of the data-path become fixed-point data-paths, in the
meanwhile saving resources and improving throughput at a cost of accuracy.  It
will be compelling to isolate the normalization/denormalization stages into
operators in the \soap~framework, so that a mixed floating-point/fixed-point
program can more efficiently trade-off resources, accuracy and latency.

\textbf{Multiple word-lengths.}  In this thesis, experiments have been carried
out on floating-point operations with a fixed mantissa only.  It would be
beneficial to further integrate fixed-point support.  Additionally, by further
supporting multiple precisions in the data-path, \ie~allowing each operators to
compute with different precisions, the trade-off relationship among our three
primary performance measures can be even more effective.  Techniques, known
as multiple word-length optimization~\cite{constantinides, lee06, cantin02},
exist to apply a heuristic approach to perturb the precisions in a data-path,
so that a performance metric can be optimized while round-off errors of outputs
satisfies an error budget.  Instituting such techniques in the \soap~framework
is rewarding as it can further reduce the area and latency requirement of a
synthesized circuit for a given accuracy.  All of these approaches optimize a
fixed data-path, whereas in \soap~the structure of the data- and control-paths
are varying as we optimize them.  Analyzing each of the candidates for the
optimal precision assignment to each operator is very inefficient because of
the number of candidates explored.  Moreover, current techniques work with a
predetermined error budget, and yet in fact a Pareto frontier exists for each
data-path to trade-off accuracy, resources and latency.

\textbf{Numerical analysis and linear algebra.}  There are two distinct
approaches to the analysis of round-off errors.  One focuses on the round-off
errors by statically analyzing numerical programs, and apply this in a way
which is as general as possible, similar to the method presented in this
thesis.  On the other hand, there are techniques employed by numerical analysts
to evaluates and improve the numerical accuracy and stability of particular
algorithms analytically.  Many creative solutions to challenges are invented
in this process.  For instance, \emph{Kahan's compensated summation} algorithm
is an accurate way to compute a sum of $n$ values, $\sum_{i = 0}^{n-1}
x_i$~\cite{kahan65} is shown in Figure~\ref{co:lst:sum}.  This algorithm
cannot be discovered easily using the method outlined in this thesis, and a
way to extend the framework to optimize programs as creatively as humans still
eludes us at the moment.  Higham~\etal~\cite{higham02} discuss in great depth
many existing numerical accuracy problems encountered in finite-precision
computation of polynomials and linear algebra subprograms and how to analyze
and overcome inaccuracies, often in terms of relative errors.  Bridging the
gap between computational and mathematical approaches for numerical analysis
will allow us to automate many accuracy optimizations that were previously
unexplored by the tool.
\begin{figure}[ht]
    \centering
\begin{lstlisting}[]
    float compensated_summation(float X[N])
    {
        float sum = 0.0f;
        float e = 0.0f;
        for (i = 0; i < n; i++)
        {
            float tmp = sum;
            float y = X[i] + e;
            sum = tmp + y;
            e = (temp - sum) + y;
        }
        return sum;
    }
\end{lstlisting}
    \caption{%
        Kahan's compensated summation algorithm to accurately compute the sum
        of $n$ elements $\sum_{i = 0}^{n-1} x_i$.
    }\label{co:lst:sum}
\end{figure}

\textbf{Continuity analysis and optimization.} The robustness of programs
are very important to us.  In many cases, we wish our algorithms to be
free from discontinuity, \ie~a small change in the initial condition
would not result in an undesirably large jump in the outputs.  For this,
Chaudhuri~\etal~\cite{chaudhuri11} and Goubault~\etal~\cite{goubault13}
respectively propose methods to analyze the robustness of programs.  The
former approach formally proves whether an algorithm is ill-conditioned in
terms of the existence of discontinuity, whereas the latter statically analyze
programs to determine if round-off errors introduce significant discontinuous
behaviour.  To illustrate, consider an \iflit~branch,
``\lstinline[basicstyle=\tt]{if ($e$ > 0) $c_1$ else $c_2$}'', where $e$
is a floating-point expression.  When $e$ is positive and very close to
$0$ when evaluated in real arithmetic, the floating-point result of $e$
could be non-positive, due to the effects of the round-off errors.  In
these extraordinary cases, the $c_2$ branch may be executed instead of the
intended $c_1$.  These above new techniques could inspire us to implement the
optimization of discontinuous behaviour, such as the one shown in the example,
as another objective.

\textbf{Memory partitioning.} The experimental results in this work see a
diminishing performance return when loops are deeply unrolled, because of a
memory bottleneck.  As memory accesses saturate in loop execution, \ie~all
memory ports are working in 100\% utilization, it is unable to gain further
performance improvements.  Currently the tool stops exploring further loop
unrolling when this happens.  By automatically partition arrays upon hitting
such a memory bottleneck, further throughput improvements can be achieved.

\textbf{Other practical considerations.}  Finally, we may consider design
perspectives that could make the resulting tool much more usable.  For
instance, programs may still be optimized by not having any knowledge about the
input variables.  Herbie~\cite{panchekha15} makes no assumption about the input
space, and can nevertheless optimize arithmetic expressions, by splitting the
input space into regimes.


% \section{Tool Usage}
% \label{cc:sec:usage}

% \soap~is a source-to-source optimizer that specifically targets numerical
% program statements written in a subset of standard C99.  The tool supports
% arithmetic and Boolean expressions, assignment statements, \iflit{} statements,
% \whilelit{} loops and \forlit{} loops.  The numerical data types we allow are
% $\inttype$ and $\floattype$, as well as single- and multi-dimensional array
% types.

% The program below is an example usage of \soap~in a C program.  Note that it
% specifies the input values are respectively a two-dimensional array \verb|A|,
% where its elements are single-precision floating point values between 0 and 1,
% and an integer \verb|T| equals to $20$.  It also indicates the only output that
% we care from this code is the resultant \verb|A|.
% \begin{lstlisting}
  % #define N 1024
  % #pragma soap begin
  % #pragma soap in float A[N][N]=[0,1], int T=20
  % #pragma soap out A
  % for (int t = 0; t < T; t++)
    % for (int i = 1; i < N-1; i++)
      % for (int j = 1; j < N-1; j++)
        % A[i][j] = 0.2f * (A[i-1][j] +
          % A[i][j-1] + A[i][j] +
          % A[i][j+1] + A[i+1][j]);
  % #pragma soap end
% \end{lstlisting}

% \soap~is an open-source command-line utility, which only requires the user
% to provide a program written in C extended with the above \verb|#pragma|
% statements.  The Pareto optimal programs are all automatically generated by
% \soap, each is accompanied with our estimations of its latency and resource
% usage, and an analyzed bound on round-off errors.  These programs can then be
% given to \gls{vhls} to be synthesized into circuits.


\section{Final Remarks}
\label{cc:sec:final_remarks}

This thesis adapts existing techniques such as accuracy, latency and resource
usage analysis, and further introduces novel approaches, \eg~\gls{mir} and
efficient equivalence discovery, and delivers them in a unified framework.
The functional equivalence relaxation paradigm is relatively under-explored,
because these optimizations are often highlighted as \emph{unsafe} by the
\gls{hls} tools, as they cannot analyze the numerical implications of these
optimizations.  \Gls{hls} tools therefore have very limited optimization
options base on this particular concept.  With the constructive results
produced by this thesis, optimizations based on our concept can not only raise
performance measures, but also result in even \emph{safer} implementations
as we improve numerical accuracies.  The equivalence discovery algorithm in
tandem with \glspl{mir} could have great potential in compiler optimization
based on our concept.  Furthermore, since machine learning algorithms are
error-resilient~\cite{lesser11, kim09, holt91, zhu03}, the methods demonstrated
in this thesis have promising capabilities to improve the resource usage,
latency and accuracy of them.


\chapter{Numerical Program Optimization}
\label{chp:progopt}

\chapter{Latency}


\chapter{Conclusion}

\chapter{Future Work}

We believe that it is possible to extend our tool for the multi-objective
optimization of arithmetic expressions in the following ways. First, Secondly,
it would be useful to further allow transformations of expressions while
allowing different mantissa widths in the subexpressions, this further
increases the number options in the Pareto frontier, as well as leads to more
optimized expressions. Thirdly, as an alternative for interval analysis, we
could employ more sophisticated abstract domains that capture the correlations
between variables, and produce tighter bounds for results. Finally, there
could be a lot of interest in the HLS community on how our tool can be
incorporated with Martin Langhammer's work on fused floating-point datapath
synthesis~\cite{langhammer}.


\cleardoublepage


% Back matter
{%
\setstretch{1.1}
\renewcommand{\bibfont}{\normalfont\small}
\setlength{\biblabelsep}{0pt}
\setlength{\bibitemsep}{0.5\baselineskip plus 0.5\baselineskip}
\printbibliography[nottype=online]
\printbibliography[heading=subbibliography,title={Webseiten},type=online,prefixnumbers={@}]
}
\cleardoublepage

\listoffigures
\cleardoublepage

\listoftables
\cleardoublepage

% \input{colophon}
% \cleardoublepage

% \input{declaration}
% \clearpage
% \newpage
% \mbox{}

\end{document}
