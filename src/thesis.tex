\documentclass[thesis.tex]{subfiles}

\graphicspath{{stropt/fig/}}
\DeclareGraphicsExtensions{.pdf}

\newcommand{\eg}{\textit{e.g.}}
\newcommand{\etc}{\textit{etc.}}
\newcommand{\etal}{\textit{et~al.}}
\newcommand{\ie}{\textit{i.e.}}

\newcommand{\naturalset}{\ensuremath\mathbb{N}}
\newcommand{\realset}{\ensuremath\mathbb{R}}
\newcommand{\floatset}{\ensuremath\mathbb{F}}
\newcommand{\powersetof}[1]{\ensuremath\mathcal{P}\left({#1}\right)}
\newcommand{\intervalset}{\ensuremath\mathbf{Interval}}
\newcommand{\floatintervalset}{\ensuremath\intervalset_\floatset}
\newcommand{\errorset}{\ensuremath\mathbb{E}^\sharp}
\newcommand{\labelset}{\ensuremath\mathbf{Label}}
\newcommand{\exprset}{\ensuremath\mathbf{Expr}}
\newcommand{\varset}{\ensuremath\mathbf{Var}}
\newcommand{\env}[1]{\ensuremath\mathbf{Env}_{#1}}
\newcommand{\eqrel}{\ensuremath\mathbin{\rhd}}
\DeclareMathOperator{\roundup}{\uparrow^\sharp_\circ}
\DeclareMathOperator{\rounddown}{\downarrow^\sharp_\circ}
\DeclareMathOperator{\fresh}{\mathit{fresh}}
\DeclareMathOperator{\eqstep}{\blacktriangleright}
\DeclareMathOperator{\dom}{\mathrm{Dom}}
\DeclareMathOperator{\area}{\mathrm{Area}}
\DeclareMathOperator{\error}{\mathrm{Error}}
\DeclareMathOperator{\abserr}{\mathrm{AbsError}}
\DeclareMathOperator{\abs}{\mathrm{abs}}
\DeclareMathOperator{\frontier}{\textsc{Frontier}}

\newcommand{\varx}{\ensuremath\texttt{x}}
\newcommand{\stmtsep}{\ensuremath;~}
\newcommand{\ifstmt}[3]{
    \ensuremath\texttt{if~} {#1} \texttt{~then~} ({#2}) \texttt{~else~} ({#3})
}
\newcommand{\whilestmt}[2]{
    \ensuremath\texttt{while~} {#1} \texttt{~do~} ({#2})
}

\newcommand{\enter}[1]{\ensuremath{A({#1})}}

\newcommand{\interval}[2]{\ensuremath\left[{#1}, {#2}\right]}

\newcommand{\lattice}[2]{\ensuremath\left<{#1}, {#2}\right>}
\newcommand{\join}{\ensuremath\sqcup}
\newcommand{\meet}{\ensuremath\sqcap}

\newcommand{\marteltrace}{\texttt{martel\_trace}}
\newcommand{\frontiertrace}{\texttt{frontier\_trace}}
\newcommand{\greedytrace}{\texttt{greedy\_trace}}

\begin{document}

% Front matter
\pagenumbering{roman}           % roman page numbing (invisible for empty page style)
\pagestyle{empty}               % no header or footers
% !TEX root = ../thesis-example.tex
%
% ------------------------------------  --> cover title page
\begin{titlepage}
	\pdfbookmark[0]{Cover}{Cover}
	\flushright
	\hfill
	\vfill
	{\LARGE\thesisTitle \par}
	\rule[5pt]{\textwidth}{.4pt} \par
	{\Large\thesisName}
	\vfill
	\textit{\large\thesisDate} \\
	Version: \thesisVersion
\end{titlepage}


% ------------------------------------  --> main title page
\begin{titlepage}
	\pdfbookmark[0]{Titlepage}{Titlepage}
	\tgherosfont
	\centering

	\includegraphics[width=6cm]{imperial_logo} \\[4mm]
    % {\Large \thesisUniversity} \\[4mm]
	\textsf{\thesisUniversityDepartment} \\
	% \textsf{\thesisUniversityInstitute} \\
	\textsf{\thesisUniversityGroup} \\

	\vfill
	{\LARGE \color{ctcolortitle}\textbf{\thesisTitle} \\[10mm]}
	{\Large \thesisName} \\

	\vfill
	% \begin{minipage}[t]{.27\textwidth}
		% \raggedleft
		% \textit{1. Reviewer}
	% \end{minipage}
	% \hspace*{15pt}
	% \begin{minipage}[t]{.65\textwidth}
		% {\Large \thesisFirstReviewer} \\
		  % {\small \thesisFirstReviewerDepartment} \\[-1mm]
		% {\small \thesisFirstReviewerUniversity}
	% \end{minipage} \\[5mm]
	% \begin{minipage}[t]{.27\textwidth}
		% \raggedleft
		% \textit{2. Reviewer}
	% \end{minipage}
	% \hspace*{15pt}
	% \begin{minipage}[t]{.65\textwidth}
		% {\Large \thesisSecondReviewer} \\
		  % {\small \thesisSecondReviewerDepartment} \\[-1mm]
		% {\small \thesisSecondReviewerUniversity}
	% \end{minipage} \\[10mm]
	\begin{minipage}[t]{.27\textwidth}
		\raggedleft
		\textit{Supervisor}
	\end{minipage}
	\hspace*{15pt}
	\begin{minipage}[t]{.65\textwidth}
		\thesisFirstSupervisor%\ and \thesisSecondSupervisor
	\end{minipage} \\[10mm]
    \begin{minipage}[t]{.8\textwidth}
        {\small
            Submitted in part fulfilment of the requirements for the degree of
            Doctor of Philosophy of Imperial College London and the Diploma of
            Imperial College London
        }
    \end{minipage} \\[10mm]

	\thesisDate \\

\end{titlepage}


% ------------------------------------  --> lower title back for single page layout
\hfill
\vfill
{
	\small
	\textbf{\thesisName} \\
	\textit{\thesisTitle} \\
	\thesisDate \\
	% Reviewers: \thesisFirstReviewer\ and \thesisSecondReviewer \\
	Supervisor: \thesisFirstSupervisor%\ and \thesisSecondSupervisor
	\\[1.5em]
    \textbf{\thesisUniversity} \\
	\textit{\thesisUniversityGroup} \\
	% \thesisUniversityInstitute \\
	\thesisUniversityDepartment \\
	\thesisUniversityStreetAddress \\
	\thesisUniversityPostalCode\ and \thesisUniversityCity
}
                   % INCLUDE: all titlepages
\cleardoublepage

\pagestyle{plain}               % display just page numbers
\pdfbookmark[0]{Abstract}{Abstract}
\chapter*{Abstract}
\label{sec:abstract}
\vspace*{-10mm}

This thesis introduces a new technique, and its associated tool \soap, to
automatically perform source-to-source optimization of numerical programs,
specifically targeting the trade-off among numerical accuracy, latency, and
resource usage as a \acrlong{hls} flow for \acrshort{fpga} implementations.  A
new intermediate representation, \acrshort{mir}, is introduced to carry out
the abstraction and optimization of numerical programs.  Equivalent structures
in \acrshortpl{mir} are efficiently discovered using methods based on formal
semantics by taking into account axiomatic rules from real arithmetic, such as
associativity, distributivity and others, in tandem with program equivalence
rules that enable control-flow restructuring and eliminate redundant array
accesses.  For the first time, we bring rigorous approaches from software
static analysis, specifically formal semantics and \acrlong{ai}, to bear
on program transformation for \acrlong{hls}.  New abstract semantics are
developed to generate a computable subset of equivalent \acrshortpl{mir}
from an original \acrshort{mir}.  Using formal semantics, three objectives
are calculated for each \acrshort{mir} representing a pipelined numerical
program: the accuracy of computation and an estimate of resource utilization
in \acrshort{fpga} and the latency of program execution.  The optimization of
these objectives produces a Pareto frontier consisting of a set of equivalent
\acrshortpl{mir}.  We thus go beyond existing literature by not only optimizing
the precision requirements of an implementation, but changing the structure
of the implementation itself.  Using \soap{} to optimize the structure of a
variety of real world and artificially generated arithmetic expressions in
single precision, we improve either their accuracy or the resource utilization
by up to 60\%.  When applied to a suite of computational intensive numerical
programs from PolyBench and Livermore Loops benchmarks, \soap{} has generated
circuits that enjoy up to a 12$\times$ speedup, with a simultaneous 7$\times$
increase in accuracy, at a cost of up to 4$\times$ more \acrshortpl{lut}.
                % INCLUDE: the abstracts (english and german)
\cleardoublepage

% !TEX root = ../thesis-example.tex
%
\pdfbookmark[0]{Acknowledgement}{Acknowledgement}
\chapter*{Acknowledgement}
\label{sec:acknowledgement}
\vspace*{-10mm}

\todo{Acknowledgement goes here\textellipsis}
         % INCLUDE: acknowledgement
\cleardoublepage

\setcounter{tocdepth}{2}        % define depth of toc
\tableofcontents                % display table of contents
\cleardoublepage

% Body matter
\pagenumbering{arabic}          % arabic page numbering
\setcounter{page}{1}            % set page counter
\pagestyle{maincontentstyle}    % fancy header and footer

% \doublespacing

% Main text
\chapter{Introduction}

\chapter{Background}

\chapter{Structural Optimization of Arithmetic Expressions}
\label{chp:stropt}

% \begin{abstract}
    % This paper introduces SOAP, a new tool to automatically optimize the
% structure of arithmetic expressions for FPGA implementation as part of a
% high level synthesis flow, taking into account axiomatic rules derived
% from real arithmetic, such as distributivity, associativity and others. We
% explicitly target an optimized area/accuracy trade-off, allowing arithmetic
% expressions to be automatically re-written for this purpose. For the first
% time, we bring rigorous approaches from software static analysis, specifically
% formal semantics and abstract interpretation, to bear on source-to-source
% transformation for high-level synthesis. New abstract semantics are developed
% to generate a computable subset of equivalent expressions from an original
% expression. Using formal semantics, we calculate two objectives, the accuracy
% of computation and an estimate of resource utilization in FPGA\@. The
% optimization of these objectives produces a Pareto frontier consisting of a
% set of expressions. This gives the synthesis tool the flexibility to choose
% an implementation satisfying constraints on both accuracy and resource
% usage. We thus go beyond existing literature by not only optimizing the
% precision requirements of an implementation, but changing the structure of the
% implementation itself. Using our tool to optimize the structure of a variety of
% real world and artificially generated examples in single precision, we improve
% either their accuracy or the resource utilization by up to 60\%.
% \end{abstract}

\section{Introduction}
\label{sec:introduction}

The IEEE 754 standard~\cite{ieee754} for floating-point computation is
ubiquitous in computing machines. In practice, it is often neglected that
floating-point computations almost always have roundoff errors. In fact,
associativity and distributivity properties which we consider to be fundamental
laws of real numbers no longer hold under floating-point arithmetic. This opens
the possibility of using these rules to generate an expression equivalent to
the original expression in real arithmetic, which could have better quality
than the original when evaluated in floating point computation.

By exploiting rules of equivalence in arithmetic, such as associativity $(a
+ b) + c \equiv a + (b + c)$ and distributivity $(a + b) \times c \equiv a
\times c + b \times c$, it is possible to automatically generate different
implementations of the same arithmetic expression. We optimize the structures
of arithmetic expressions in terms of the following two quality metrics
relevant to FPGA implementation: the resource usage when synthesized into
circuits, and a bound on roundoff errors when evaluated. Our goal is the joint
minimization of these two quality metrics. This optimization process provides
a Pareto optimal set of implementations. For example, our tool discovered that
with single precision floating-point representation, if $a \in [0.1, 0.2]$,
then the expression ${(a + 1)}^2$ uses fewest resources when implemented in the
form $(a + 1) \times (a + 1)$ but most accurate when expanded into $(((a \times
a) + a) + a) + 1$. However it turns out that a third alternative, $((1 + a)
+ a) + (a \times a)$, is never desirable because it is neither more accurate
nor uses fewer resources than the other two possible structures. Our aim is to
automatically detect and utilize such information to optimize the structure of
expressions.

A na{\"\i}ve implementation of equivalent expression finding would be to
explore all possible equivalent expressions to find optimal choices, However
this would result in combinatorial explosion~\cite{ioualalen}. For instance,
in worst case, the parsing of a simple summation of $n$ variables could
result in $(2n - 1)!! = 1\times3\times5\times\cdots\times(2n - 1)$ distinct
expressions~\cite{ioualalen, mouilleron}. This is further complicated by
distributivity as ${(a + b)}^k$ could expand into an expression with a
summation of $2^k$ terms each with $k - 1$ multiplications. Therefore, usually
it would be infeasible to generate a complete set of equivalent expressions
using the rules of equivalence, since an expression with a moderate number of
terms will have a very large number of equivalent expressions. The methodology
explained in this paper makes use of formal semantics as well as abstract
interpretation~\cite{cousot77} to significantly reduce the space and time
requirements and produce a subset of the Pareto frontier.

In order to further increase the options available in the Pareto frontier,
we introduce freedom in choosing mantissa widths for the evaluation of the
expressions. Generally as the precision of the evaluation increases, the
utilization of resources increases for the same expression. This gives
flexibility in the trade-off between resource usage and precision. Our
approach and its associated tool, SOAP, allow high-level synthesis flows to
automatically determine whether it is a better choice to rewrite an expression,
or change its precision in order to meet optimization goals.

The three contributions of this paper are:
\begin{enumerate}
    \item Efficient methods for discovering equivalent structures of
    arithmetic expressions.
    \item A semantics-based program analysis that allows joint reasoning about
    the resource usage and safe ranges of values and errors in floating-point
    computation of arithmetic expressions.
    \item A tool which produces RTL implementations on the area-accuracy
    trade-off curve derived from structural optimization.
\end{enumerate}

This paper is structured as follows. Section~\ref{sec:related_work}
discusses related existing work in high-level synthesis and the
optimization of arithmetic expressions. We explain the basic
concepts of semantics with abstract interpretation used in this
paper in Section~\ref{sec:abstract_interpretation}. Using this,
Section~\ref{sec:semantics} explains the concrete and abstract semantics
for finding equivalent structure in arithmetic expressions, as well as
the analysis of their resource usage estimates and bounds of errors.
Section~\ref{sec:implementation} gives an overview of the implementation
details in our tool. Then we discuss the results of optimized example
expressions in Section~\ref{sec:results} and end with concluding remarks in
Section~\ref{sec:conclusion}.

\section{Related Work}
\label{sec:related_work}

High-level synthesis (HLS) is the process of compiling a high-level
representation of an application (usually in C, C++ or MATLAB) into
register-transfer-level (RTL) implementation for FPGA~\cite{coussy, gajski}.
HLS tools enable us to work in a high-level language, as opposed to facing
labor-intensive tasks such as optimizing timing, designing control logic in
the RTL implementation. This allows application designers to instead focus on
the algorithmic and functional aspects of their implementation~\cite{coussy}.
Another advantage of using HLS over traditional RTL tools is that a C
description is smaller than a traditional RTL description by a factor
of 10~\cite{coussy, bdti}, which means HLS tools are in general more
productive and less error-prone to work with. HLS tools benefit us in their
ability to automatically search the design space with a reasonable design
cost~\cite{bdti}, explore a large number of trade-offs between performance,
cost and power~\cite{mcfarland}, which is generally much more difficult to
achieve in RTL tools. HLS has received a resurgence of interest recently,
particularly in the FPGA community. Xilinx now incorporates a sophisticated
HLS flow into its Vivado design suite~\cite{vivado_hls} and the open-source
HLS tool, LegUp~\cite{legup}, is gaining significant traction in the research
community.

However, in both commercial and academic HLS tools, there is very little
support for static analysis of numerical algorithms. LLVM-based HLS
tools such as Vivado HLS and LegUp usually have some traditional static
analysis-based optimization passes such as constant propagation, alias
analysis, bitwidth reduction or even expression tree balancing to reduce
latency for numerical algorithms. There are also academic tools that
perform precision-performance trade-off by optimizing word-lengths of data
paths~\cite{constantinides}. However there are currently no HLS tools that
perform the trade-off optimization between accuracy and resource usage by
varying the \emph{structure} of arithmetic expressions.

Even in the software community, there are only a few existing techniques
for optimizing expressions by transformation, none of which consider
accuracy/run-time trade-offs. Darulova~\etal~\cite{darulova} employ a
metaheuristic technique. They use genetic programming to evolve the structure
of arithmetic expressions into more accurate forms. However there are several
disadvantages with metaheuristics, such as convergence can only be proved
empirically and scalability is difficult to control because there is no
definitive method to decide how long the algorithm must run until it reaches a
satisfactory goal. Hosangadi~\etal~\cite{hosangadi} propose an algorithm for
the factorization of polynomials to reduce addition and multiplication counts,
but this method is only suitable for factorization and it is not possible to
choose different optimization levels. Peymandoust~\etal~\cite{peymandoust}
present an approach that only deals with the factorization of polynomials
in HLS using Gr\"obner bases. The shortcomings of this are its dependence
on a set of library expressions~\cite{hosangadi} and the high computational
complexity of Gr\"obner bases. The method proposed by Martel~\cite{martel07}
is based on operational semantics with abstract interpretation, but even
their depth limited strategy is, in practice, at least exponentially complex.
Finally Ioualalen~\etal~\cite{ioualalen} introduce the abstract interpretation
of equivalent expressions, and creates a polynomially sized structure to
represent an exponential number of equivalent expressions related by rules of
equivalence. However it restricts itself to only a handful of these rules to
avoid combinatorial explosion of the structure and there are no options for
tuning its optimization level.

Since none of these above captures the optimization of both accuracy and
performance by restructuring arithmetic expressions, we base ourselves on the
software work of Martel~\cite{martel07}, but extend this work in the following
ways. Firstly, we develop new hardware-appropriate semantics to analyze not
only accuracy but also resource usage, seamlessly taking into account common
subexpression elimination. Secondly, because we consider both resource usage
and accuracy, we develop a novel multi-objective optimization approach to
scalably construct the Pareto frontier in a hierarchical manner, allowing fast
design exploration. Thirdly, equivalence finding is guided by prior knowledge
on the bounds of the expression variables, as well as local Pareto frontiers of
subexpressions while it is optimizing expression trees in a bottom-up approach,
which allows us to reduce the complexity of finding equivalent expressions
without sacrificing our ability to optimize expressions.

We begin with an introduction to formal semantics in the following section,
later in Section~\ref{sec:semantics}, we explain our approach by extending the
semantics to reason about errors, resource usage and equivalent expressions.

\section{Abstract Interpretation}
\label{bg:sec:abstract_interpretation}

As our way of living is becoming increasingly dependent on programs, errors
in safety-critical system can incur huge expenses, and even costs of lives.
For example, the maiden flight of Ariane 5 resulted in a failure, because of
a software instruction failed to convert a 64-bit floating-point number into
16-bit signed integer, as the result is too large to be represented by a 16-bit
signed integer~\cite{dowson97}.  The Patriot defense system failed to intercept
an incoming missile because of an accumulated round-off error in the system's
internal clock, resulted in the deaths of 28 people in 1991~\cite{patriot}.
\emph{Static analysis}, a process of analyzing a piece of program written in
an HLL without executing it, is therefore a research topic of great importance
to prevent similar catastrophic errors and mitigate costs of failure in the
future.

It is unfortunate that because of the halting problem~\cite{turing37} and
a direct consequence of it, Rice's theorem~\cite{rice53}, any nontrivial
property on the outcome of a program is in general undecidable.  It means
that an interesting property---a yes or no question---which is never
always true or always false for all programs, is \emph{undecidable}, or
in other words, cannot be answered.  Even a question as simple as ``does
this program return zero'' is difficult to answer.  A static analyzer can
therefore produce an answer which is either a definite ``yes'' or ``I don't
know''~\cite{mine04}, and designing one which answers a ``yes''.  Producing a
meaningful ``yes'' in an efficient manner poses a challenging task to static
analyzers.  Static analyzers rely heavily on formal techniques to perform
well.  Typical techniques employed include symbolic execution, model checking,
satisfiability modulo theories~\cite{demoura08}, data-flow analysis based on
lattices~\cite{nielson99}, and abstract interpretation~\cite{cousot77}.

This section starts by introducing the data-flow analysis framework to analyze
a simple program, abstract interpretation is then applied to this example, and
the properties of the resulting analysis are further discussed.

% This is then extended to define a scalable analysis capturing accuracy.
% Later in Chapter~\ref{chp:progopt} we accommodate sequential statements,
% \iflit~branches and \whilelit~loops in the accuracy analysis, and in
% Chapter~\ref{chp:latopt}, we further improve our analysis by supporting
% multi-dimensional arrays.

\subsection{Data-Flow Analysis Framework}
\label{bg:sub:data_flow}

\begin{figure}[ht]
    \centering
    \begin{minipage}{0.5\textwidth}
    \begin{lstlisting}
    float simple(float x)
    {
        while (x > 1.0)
            x *= 0.9;
        return x;
    }
    \end{lstlisting}
    \end{minipage}
    \caption{%
        A simple program example to be statically analyzed.
    }\label{bg:lst:simple}
\end{figure}
In this section, we use the \emph{data-flow analysis} (DFA)
framework~\cite{nielson99} to statically analyze a program named \verb|simple|
in Figure~\ref{bg:lst:simple}, which consists of only one variable \verb|x|.
We further assume an initial set $\iota \subseteq \realset$ of values of
\verb|x|, and the property that concerns us is answering whether a particular
value $x_\mathrm{invalid}$ is not in the set of all reachable final values
of \verb|x|.  A sensible definition for the set of values can be reached
by \verb|x| is a subset of all real numbers $\realset$, \ie~an element of
$\powersetof\realset$, where $\powersetof\realset$ denotes the \emph{power set}
of $\realset$, also known as the set of all subsets of $\realset$.

The first step of DFA is to translate the body of \verb|simple| into a
control/data-flow graph (CDFG), as shown in Figure~\ref{bg:fig:cdfg} where
each block consists of a single statement or conditional, and the edges in
the graph model the data- and control-flows.  The \textbf{tt} and \textbf{ff}
respectively highlight the control-flow branch taken when the conditional
\mbox{``\texttt{x < 1}''} evaluates to either true or false.
\begin{figure}[ht]
    \centering
    \tikzstyle{block} = [
        draw,
        fill=white,
        rectangle,
        minimum height=2em,
        minimum width=4em
    ]
    \begin{tikzpicture}[node distance=4em]
        \node(entr) {\textbf{entry}};
        \node(cond) [block, below of=entr] {\texttt{x > 1}};
        \node(stmt) [block, below of=cond] {\texttt{x = 0.9 * x;}};
        \node(midl) [coordinate, left of=stmt, xshift=-1.5em] {};
        \node(midr) [coordinate, right of=stmt, xshift=1.5em] {};
        \node(rtrn) [coordinate, below of=stmt, yshift=1em]
            {\texttt{return x;}};
        \node(exit) [below of=rtrn, yshift=1em] {\textbf{exit}};
        \draw[->] (entr) -- node[auto]{0} (cond);
        \draw[->] (cond) -- node[right]{\textbf{tt}} node[left]{1} (stmt);
        \draw[- ] (stmt) -- (midl);
        \draw[->] (midl) |- node[auto]{2} (cond);
        % \draw[->] (stmt) to[out=180, in=180] (cond);
        \draw[- ] (cond) -| node[auto, near start]{\textbf{ff}} (midr);
        \draw[- ] (midr) |- node[auto]{3} (rtrn);
        % \draw[->] (cond) to[out=0, in=0] (rtrn);
        \draw[->] (rtrn) -- (exit);
    \end{tikzpicture}
    \caption{%
        The CDFG of \texttt{simple} in Figure~\ref{bg:lst:simple}.
    }\label{bg:fig:cdfg}
\end{figure}

The individual blocks in the CDFG can therefore be defined as functions $f:
\powersetof\realset \to \powersetof\realset$ that admits an $S$ and produces
$T$, where both $S$ and $T$ are elements of $\powersetof\realset$.  For
instance, for the statement ``\texttt{x *= 0.9;}'', a function $f_1$ can be
defined as follows:
\begin{equation}
    f_1(S) = \{ 0.9 v \mid v \in S \}.
\end{equation}
Here, the definition of $f_1$ indicates that for all possible input values
$v$ of \verb|x| in the set $S$, we multiply it by $0.9$ and collect the
multiplied results into a new set as the output of $f_1$.  Similarly, because
\mbox{``\texttt{x > 1}''} has two conditional branches, two functions,
$f_{2,\truelit}$ and $f_{2, \falselit}$, respectively for both true- and
false-branches of it can be defined:
\begin{equation}
    \begin{aligned}
        f_{2, \truelit}(S) &= S \cap \{ v \in \realset \mid v > 1 \}, \\
        f_{2, \falselit}(S) &= S \cap \{ v \in \realset \mid v \leq 1 \}.
    \end{aligned}
\end{equation}
where $X \cap Y$ computes the intersection of the two sets $X$ and $Y$.

In the next step, the edges of the CDFG are labelled with numbers 0, 1, 2
and 3 to signify different locations of the program.  For each edge labelled
$i$, it is now possible to compute an $A(i)$, a set of values that could be
reached by \verb|x| in a program execution at each location $i$, by wiring up
the functions $f_1$, $f_{2, \truelit}$ and $f_{2, \falselit}$ that correspond
to program statements.  This gives rise to the following system of data-flow
equations:
\begin{align}
    A(0) &= \iota,
        \label{bg:eq:dfa0} \\
    A(1) &= f_{2, \truelit}(A(0) \cup A(2)),
        \label{bg:eq:dfa1} \\
    A(2) &= f_1(A(1)),
        \label{bg:eq:dfa2} \\
    A(3) &= f_{2, \falselit}(A(0) \cup A(2)),
        \label{bg:eq:dfa3}
\end{align}
where $A(0) \cup A(1)$ is the union of $A(0)$ and $A(1)$.

Unfortunately, computationally solving this system of equations is not an
easy task.  In Sections~\ref{bg:sub:most_precise} and~\ref{bg:sub:intervals},
the two significant impediments are explained, and subsequently, theories are
introduced to unravel them.


\subsection{The Most Precise Solution to a Data-Flow Equation}
\label{bg:sub:most_precise}

There are multiple solutions to this system.  For example, we can solve
it manually by substituting $A(0)$ and $A(2)$ in~\eqref{bg:eq:dfa1}
with~\eqref{bg:eq:dfa0} and~\eqref{bg:eq:dfa2}.  We arrive at:
\begin{equation}
    A(1) = \left(
        \iota \cup \left\{ 0.9 v \mid v \in A(1) \right\}
    \right) \cap \{ v \in \realset \mid v > 1 \}.
    \label{bg:eq:dfa_a1}
\end{equation}
It turns out that the set of all real numbers greater than $1$, or:
\begin{equation}
    A(1) = \{ v \in \realset \mid v > 1 \}
    \label{bg:eq:a11}
\end{equation}
is a solution to~\eqref{bg:eq:dfa_a1}.  Substituting $A(1)$ in the right-hand
side of~\eqref{bg:eq:dfa_a1} with this value proves that it is indeed the
solution for this equation, assuming all sets below are subsets of $\realset$
to simplify the derivation:
\begin{equation}
    \begin{aligned}
        A(1)
        &= \bigg( \iota \cup \Big\{ 0.9 v \mid v \in
                \left\{ v^\prime \mid v^\prime > 1 \right\}
           \Big\} \bigg) \cap \{ v \mid v > 1 \} \\
        &= \bigg( \iota \cup \left\{ 0.9 v \mid v > 1 \right\} \bigg) \cap
           \{ v \mid v > 1 \} \\
        &= \bigg( \iota \cup \left\{ v \mid v > 0.9 \right\} \bigg) \cap
           \{ v \mid v > 1 \} \\
        &= \bigg( \iota \cap \{ v \mid v > 1 \} \bigg) \cup
           \bigg(
               \left\{ v \mid v > 0.9 \right\} \cap \{ v  \mid v > 1 \}
           \bigg) \\
        &= \bigg( \iota \cap \{ v \mid v > 1 \} \bigg) \cup
           \{ v \mid v > 1 \} \\
        &= \{ v \mid v > 1 \}.
    \end{aligned}
\end{equation}

Intuitively, a manual inspection of \verb|simple| finds that \verb|x| can reach
values $v$, $0.9 v$, $0.9^2 v$, and \textellipsis, such that all values in this
sequence are greater than $1$, for each $v \in \iota$; or more succinctly, an
alternative solution to $A(1)$ should be:
\begin{equation}
    A(1) = \{ v \in \iota \mid 0.9^k v > 1 \wedge k \in \naturalset \}.
    \label{bg:eq:a12}
\end{equation}
Here $k \in \naturalset$ denotes $k$ is one of $0, 1, 2, \mathellipsis$, \ie~a
natural number.

It is evident to us the latter solution~\eqref{bg:eq:a12} is more precise,
hence more desirable, than the former~\eqref{bg:eq:a11}.  Not only does it
contain information the former has, \ie~all values reachable by $A(1)$ is
greater than 1, it also expresses the fact that it only consists of values of
the form $0.9^k v$, where $v \in \realset$ and $k \in \naturalset$.  A useful
definition of preciseness is therefore the subset relation ``$\subseteq$''.
If it is known that $X \subseteq X^\prime$, and $X$ and $X^\prime$ are both
solution to a system of data-flow equations, then $X$ is clearly more appealing
than $X^\prime$.

The set $\powersetof\realset$, with a preciseness ordering ``$\subseteq$'', is
a \emph{partially ordered set}.  It has three following properties for any $X,
Y, Z \in \powersetof\realset$: it is \emph{reflexive}, $X \subseteq X$; it has
the \emph{antisymmetry} property, \ie~if $X \subseteq Y$ and $Y \subseteq X$,
then $X = Y$; and finally it is transitive, if $X \subseteq Y$ and $Y \subseteq
Z$, then $X \subseteq Y$.  In contrast to a total ordered set such as the set
of reals $\realset$, not every two elements in $\powersetof\realset$ can be
compared, \eg~neither of the sets $\{1, 2, 3\}$ and $\{2, 3, 4\}$ is a subset
of each other.

For the purpose of computing the solution to $A(1)$'s
equation~\eqref{bg:eq:dfa_a1}, a function $f: \powersetof\realset \to
\powersetof\realset$ can be defined:
\begin{equation}
    f(X) = \left(
        \iota \cup \left\{ 0.9 v \mid v \in X \right\}
    \right) \cap \{ v \in \realset \mid v > 1 \},
    \label{bg:eq:transfer}
\end{equation}
so that all solutions of the original equation~\eqref{bg:eq:dfa_a1} are now in
this following set, which are known as the \emph{fixpoints} of $f$:
\begin{equation}
    \mathrm{Fix}(f) = \left\{
        X \in \powersetof\realset \mid
        f(X) = X
    \right\}.
\end{equation}

By using this particular definition of preciseness, two important questions
however arise:
\begin{enumerate}

    \item Is the most precise solution unique?  A unique most precise solution
    is defined as the only one which is the most precise among all possible
    solutions to the systems of data-flow equations.  In other words, if
    it exists, then it is defined as the \emph{least fixpoint} (LFP) of
    $f$ which is a subset of all other fixpoints, \ie~$\lfp (f) \subseteq
    \mathrm{Fix}(f)$.  As we have discussed earlier, multiple fixpoints exist,
    and it is possible that these fixpoint solutions are not comparable.

    \item If a unique solution exists and it is unique, how do we find it?
    This is equivalent to finding a way to compute the LFP $\lfp(f)$ using $f$.

\end{enumerate}

As it turns out, the first question can be answered by
Theorem~\ref{bg:thr:tarski}~\cite{tarski55, nielson99}, which proves that $\lfp
(f)$ is indeed unique.
\begin{theorem}
    \textup{[Tarski's fixpoint theorem]}
    If $\mathsf{L}$ is a complete lattice, and a function $g:
    \mathsf{L} \to \mathsf{L}$ is a monotone function, then $\lfp(g)$,
    the LFP of $g$ is the greatest lower bound of all fixpoints
    $\mathrm{Fix}(g)$.\label{bg:thr:tarski}
\end{theorem}

In our case, $f$ is a \emph{monotone} function, because by definition a
\emph{monotone} function satisfies the condition that if $X \subseteq Y$,
then $g(X) \subseteq g(Y)$.  In the DFA of \verb|simple|, $\mathsf{L} =
\powersetof\realset$, which is a \emph{complete lattice}\footnote{%
    Exact definitions of complete lattice and complete partial order are not
    required in this section.  Both of them can be found in~\cite{nielson99}.
}, since all power sets are complete lattices~\cite{nielson99}.  The LFP of
$f$, that is the intersection of all elements in $\mathrm{Fix}(f)$, or the
greatest lower bound of all fixpoints, can therefore be written concisely as:
\begin{equation}
    \lfp (f) = \bigcap \mathrm{Fix}(f).
\end{equation}

Secondly, another theorem~\cite{kleene52}, which is closely
related to Theorem~\ref{bg:thr:tarski}, states:
\begin{theorem}
    \textup{[Kleene's fixpoint theorem]}
    If $\mathsf{L}$ is a complete partial order (CPO), and $g: \mathsf{L} \to
    \mathsf{L}$ is a Scott-continuous function, then the $\lfp (g)$ can be
    computed as the least upper bound of all values in the sequence $\bot$,
    $g(\bot)$, $g^2(\bot)$, $g^3(\bot)$, \textellipsis{}\label{bg:thr:kleene}
\end{theorem}
Here, $\bot$ denotes the least element in $\mathsf{L}$.  A function of the form
$h^n(x)$, where $h: \mathsf{M} \to \mathsf{M}$ for any domain $\mathsf{M}$ and
$n \in \naturalset$, is recursively defined as:
\begin{equation}
    h^n(x) = \left\{
        \begin{aligned}
            & h(h^{n-1}(x)) \quad && \text{if~} n > 0, \\
            & x && \text{if~} n = 0.
        \end{aligned}
    \right.
\end{equation}

Our function $f$ is \emph{Scott-continuous}: it is monotone; and for any chain
of $X_0 \subseteq X_1 \subseteq X_2 \subseteq X_3 \subseteq \mathellipsis$,
where $X_i \in \powersetof\realset$:
\begin{equation}
    \bigcup_{i \in \naturalset} f(X_i) = f \left(
        \bigcup_{i \in \naturalset} X_i
    \right).
\end{equation}
As a CPO is more general that a complete lattice, and the least
element in $\powersetof\realset$ is the empty set $\varnothing$, using
Theorem~\ref{bg:thr:kleene} in our example analysis, the most precise solution
of $A(1)$ can therefore be computed using:
\begin{equation}
    \lfp (f) = \bigcup_{k \in \naturalset} f^k (\varnothing).
\end{equation}
The functions $f^k(\varnothing)$ for the first $k+1$ iterations can be
evaluated as follows:
\begin{equation}
    \begin{aligned}
        f^0(\varnothing) &= \varnothing, \quad\quad
        f^1(\varnothing) = \iota \cap \{ v \mid v > 1 \}, \\
        f^2(\varnothing) &= f(f^1(\varnothing))
               = \left(
                     \iota \cup
                     \{ 0.9v \mid v \in \iota \}
                 \right) \cap \{ v \mid v > 1 \}, \\
        f^3(\varnothing) &= \left(
                     \iota \cup
                     \{ 0.9v \mid v \in \iota \} \cup
                     \{ 0.9^2 v \mid v \in \iota \}
                 \right) \cap \{ v \mid v > 1 \}, \mathellipsis, \\
        f^k(\varnothing) &= \left(
                     \iota \cup
                     \{ 0.9v \mid v \in \iota \} \cup
                     \mathellipsis \cup
                     \{ 0.9^{k-1} v \mid v \in \iota \}
                 \right) \cap \{ v \mid v > 1 \}.
    \end{aligned}
\end{equation}
Finally, the most precise solution to~\eqref{bg:eq:dfa_a1} can be computed
using the LFP formula for $f$, which is exactly the same as the alternative
solution that was manually computed in~\eqref{bg:eq:a12}:
\begin{equation}
    \begin{aligned}
        \lfp (f)
            &= \bigcup_{k \in \naturalset} f^k (\varnothing) \\
            &= \{ v \mid v > 1 \} \cap
               \bigcup_{k \in \naturalset} \{ 0.9^k v \mid v \in \iota \} \\
            &= \{ v \in \iota \mid 0.9^k v > 1 \wedge k \in \naturalset \}.
    \end{aligned}
\end{equation}

Even though we have derive a method to statically analyze a program,
significant obstacles still prevent us from using it efficiently.  Firstly
in the \verb|simple| case study, because the LFP is evaluated as the union
of $f^k(\varnothing)$ in a sequence, this sequence is likely to be infinite,
and thus cannot be computed fully.  Secondly, the set of input values,
$\iota$, not only determines the number of iterations necessary in order to
calculate the LFP, but also impacts the amount of computation required in
each iteration.  For instance if $\iota = {4}$ then it is only necessary to
track the computation for a single input value $4$, whereas when $\iota =
\{ v \mid 0 \leq v \leq 1000 \}$, there are infinitely many values in the
set.  As a result, in general, the LFP of an arbitrary self-map function $f:
\mathsf{L} \to \mathsf{L}$, where $\mathsf{L}$ is a complete lattice, is thus
not computable in finite amount of time.  In Section~\ref{bg:sub:intervals},
a method known as abstract interpretation is introduced to overcome the
computability problem.


\subsection{Abstract Interpretation with Intervals}
\label{bg:sub:intervals}

A framework of methods, known as \emph{abstract interpretation} (AI), is
proposed by Cousot~\etal~\cite{cousot77} to formally mitigate the problem of
computability in program analysis.  Instead of finding the LFP, which may not
be computable, it is much more efficient to work out an \emph{approximation} of
the LFP\@.  Despite the outcome of an AI-based static analysis not as precise
as the LFP, the significant benefits of AI is two-fold.  Firstly, the program
analysis framework can now produce a ``yes'' or ``I don't know'' answer to a
query of program property in a finite amount of time.  Secondly, it provides
the means to prove the correctness of an answer produced by the static analyzer
using AI in formal mathematics.

We illustrate these concepts by putting the familiar idea of \emph{interval
arithmetic}~\cite{moore} in the framework of abstract interpretation. As
an illustration, consider the following expression and its DFG in
Figure~\ref{bg:fig:sample_tree}\@:
\begin{equation}
    (a + b) \times (a + b)
    \label{bg:eq:absint_sample}
\end{equation}
\begin{figure}[ht]
    \centering
    \includegraphics[scale=0.6]{sample_tree}
    \caption{The DFG for the sample expression.}\label{bg:fig:sample_tree}
\end{figure}

We may wish to ask: if initially $a$ and $b$ are real numbers in the range of
$[0.2, 0.3]$ and $[2, 3]$ respectively, what would be the outcome of evaluating
this expression with real arithmetic? A straightforward approach is simulation.
Evaluating the expression for a large quantity of inputs will produce a set
of possible outputs of the expression. However the simulation approach is
unsafe, since there are infinite number of real-valued inputs possible and it
is infeasible to simulate for all.

A better method might be to represent the possible values of $a$ and $b$ using
ranges. To compute the ranges of its output values, we could operate on ranges
rather than values (note that the superscript $\sharp$ denotes ranges). Assume
that $a^\sharp_{init} = [0.2, 0.3]$, $b^\sharp_{init} = [2, 3]$, which are the
input ranges of $a$ and $b$, and $\enter{l}$ where $l \in \{1, 2, 3, 4\}$ are
the intervals of the outputs of the boxes labelled with $l$ in the DFG\@. We
extract the data flow from the DFG to produce the following set of equations:
\begin{equation}
    \begin{aligned}
        \enter{1} &= a^\sharp_{init} \\
        \enter{2} &= b^\sharp_{init} \\
        \enter{3} &= \enter{1} + \enter{2} \\
        \enter{4} &= \enter{3} \times \enter{3}
    \end{aligned}
    \label{bg:eq:absint_sample_analysis}
\end{equation}
For the equations above to make sense, addition and multiplication need to be
defined on intervals. We may define the following interval operations:
\begin{equation}
    \begin{aligned}
        \interval{a}{b} + \interval{c}{d} &= \interval{a + c}{b + d} \\
        \interval{a}{b} - \interval{c}{d} &=  \interval{a - d}{b - c} \\
        \interval{a}{b} \times \interval{c}{d} &=
            \interval{\min(s)}{\max(s)} \\
        \text{where~} s &= \{ a \times c, a \times d, b \times c, b \times d \}
    \end{aligned}
    \label{bg:eq:interval_operations}
\end{equation}
The solution to the set of~\eqref{bg:eq:absint_sample_analysis} for $\enter{4}$
is $[4.84, 10.89]$, which represents a safe bound on the output at the end
of program execution. Note that in actual execution of the program, the
semantics represent the values of intermediate variables, which are real
values. In our case, a set of real values forms the set of all possible
values produced by our code. However computing this set precisely is not,
in general, a possible task. Instead, we use abstract interpretation based
on intervals, which gives the abstract semantics of this program. Here, we
have achieved a classical interval analysis by \emph{defining} the meaning of
addition and multiplication on abstract mathematical structures (in this case
intervals) which capture a safe approximation of the original semantics of the
program.

Later in Sections~\ref{so:sec:resource}~and~\ref{so:sec:equivalent} of
Chapter~\ref{chp:stropt}, we further generalize the idea by defining the
meaning of these operations on more complex abstract structures which allow
us to scalably reason about the area of FPGA implementations and equivalent
program structures respectively.


\subsection{Accuracy Analysis}
\label{bg:sub:accuracy}

Because we optimize numerical programs in a way that may have significant
impact on accuracy, and one of our objectives is to minimize round-off error
in the process, it is necessary to perform accuracy analysis on optimized
candidates.

Since our numerical programs make use of floating-point arithmetic, we first
introduce the concepts of the floating-point representation~\cite{ieee754}. Any
values $v$ representable in floating-point with standard exponent offset can be
expressed with the format given by the following equation:
\begin{equation}
    v = s \times 2^{e + 2^{k - 1} - 1} \times 1.{m_1 m_2 m_3 \ldots m_p}
    \label{bg:eq:floating_point}
\end{equation}
In~\eqref{bg:eq:floating_point}, the bit $s$ is the sign bit, the $k$-bit
unsigned integer $e$ is known as the exponent bits, and the $p$-bits $m_1 m_2
m_3 \ldots m_p$ are the mantissa bits, here we use $1.{m_1 m_2 m_3 \ldots m_p}$
to indicate a fixed-point number represented in unsigned binary format.

Because of the finite characteristic of IEEE 754 floating-point format, it
is not always possible to represent exact values with it. Computations in
floating-point arithmetic often induces roundoff errors. Therefore, following
Martel~\cite{martel07}, we bound with ranges the values of floating-point
calculations, as well as their roundoff errors. Our accuracy analysis
determines the bounds of all possible outputs and their associated range of
roundoff errors for expressions. For example, assume that real variables $a
\in [0.2, 0.3]$, $b \in [2.3, 2.4]$, it is possible to derive that in single
precision floating-point computation with rounding to the nearest, ${(a + b)}^2
\in [6.24999857, 7.29000187]$ and the error caused by this computation is
bounded by $[-1.60634534\times10^{-6}, 1.60634534\times10^{-6}]$.

We employ abstract error semantics for the calculation of errors described
in~\cite{ioualalen, martel07}. First we define the domain $\errorset
= \floatintervalset\times\intervalset$, where $\intervalset$ and
$\floatintervalset$ respectively represent the set of real intervals, and
the set of floating-point intervals (intervals exactly representable in
floating-point arithmetic). The value $(x^\sharp, \mu^\sharp) \in \errorset$
represents a safe bound on floating-point values and the accumulated error
represented as a range of real values. Then addition and multiplication can be
defined for the semantics as in~\eqref{bg:eq:error_semantics}:
\begin{equation}
    \begin{aligned}
        \left( x^\sharp_1, \mu^\sharp_1 \right) +
        \left( x^\sharp_2, \mu^\sharp_2 \right)
    &=  \left(
            \roundup{x^\sharp_1 + x^\sharp_2},
            \mu^\sharp_1 + \mu^\sharp_2 +
            \rounddown{x^\sharp_1 + x^\sharp_2}
        \right) \\
        \left( x^\sharp_1, \mu^\sharp_1 \right) -
        \left( x^\sharp_2, \mu^\sharp_2 \right)
    &=  \left(
            \roundup{x^\sharp_1 - x^\sharp_2},
            \mu^\sharp_1 - \mu^\sharp_2 +
            \rounddown{x^\sharp_1 - x^\sharp_2}
        \right) \\
        \left( x^\sharp_1, \mu^\sharp_1 \right) \times
        \left( x^\sharp_2, \mu^\sharp_2 \right)
    &=  \left(
            \roundup{x^\sharp_1 \times x^\sharp_2},
            x^\sharp_1 \times \mu^\sharp_2 + x^\sharp_2 \times \mu^\sharp_1 +
            \mu^\sharp_1 \times \mu^\sharp_2 +
            \rounddown{x^\sharp_1 \times x^\sharp_2}
        \right) \\
    &\qquad\qquad\qquad\qquad\qquad\qquad\text{~for~}
        \left( x^\sharp_1, \mu^\sharp_1 \right) \in \errorset,
        \left( x^\sharp_2, \mu^\sharp_2 \right) \in \errorset
    \end{aligned}
    \label{bg:eq:error_semantics}
\end{equation}

The addition, subtraction and multiplication of intervals follow
the standard rules of interval arithmetic defined earlier
in~\eqref{bg:eq:interval_operations}.  In~\eqref{bg:eq:error_semantics}, the
function $\rounddownop: \intervalset \to \intervalset$ determines the range
of roundoff error due to the floating-point computation under one of the
\emph{rounding modes} $\circ \in \{ -\infty, \infty, 0, \neg0, \sim \}$ which
are round towards negative infinity, towards infinity, towards zero, away from
zero and towards nearest floating-point value respectively. It is defined as:
\begin{equation}
    \begin{aligned}
        & \downarrow^\sharp_\circ([a, b]) = \left\{
            \begin{aligned}
                & \left[ -\frac{z}{2}, \frac{z}{2}\right]
                    & \quad \circ & \text{~is~}\sim \\
                & \left[ -z, z\right]
                    & \quad \circ & \in \{ -\infty, \infty, 0, \neg0 \}
            \end{aligned}
        \right. \\
        & \qquad\qquad\qquad\qquad \text{where~} z = \max(ulp(a), ulp(b))
    \end{aligned}
\end{equation}
Here $z$ denotes the maximum rounding error that can occur for values
within the range $[a, b]$, and the unit of the last place (ulp) function
$ulp(x)$~\cite{muller} characterizes the distance between two adjacent
floating-point values $f_1$ and $f_2$ satisfying $f_1 \leq x \leq
f_2$~\cite{goldberg}. In our analysis, the function $ulp$ is defined as:
\begin{equation}
    ulp(x) = 2^{e(x) + 2^{k - 1} - 1} \times 2^{-p}
\end{equation}
where $e(x)$ is the exponent of $x$, $k$ and $p$ are the parameters of the
floating-point format as defined in~\eqref{bg:eq:floating_point}. The function
$\roundupop: \intervalset \to \floatintervalset$ computes the floating-point
bound from a real bound, by rounding the infimum $a$ and supremum $b$ of the
input interval $[a, b]$:
\begin{equation}
    \roundupop\left(\left[a, b\right]\right)
    = {\left[
        \uparrow_\circ{\left(a\right)},
        \uparrow_\circ{\left(b\right)}
    \right]}_\floatset
\end{equation}
where the subscript $\floatset$ indicates the interval is a floating-point
interval, and we define $\uparrow_\circ: \realset \to \floatset$ to be the
function that rounds a real number to a floating-point value, under the
rounding mode $\circ$.

Expressions can be evaluated for their accuracy by the method as follows.
Initially the expression is parsed into a data flow graph (DFG). By way of
illustration, the sample expression ${(a + b)}^2$ has the tree structure
in Figure~\ref{bg:fig:sample_tree}. Then the exact ranges of values of $a$ and
$b$ are converted into the abstract semantics using a cast operation as in
\eqref{bg:eq:cast}:
\begin{equation}
    \mathrm{cast}\left(x^\sharp\right) = \left(
        \roundup{x^\sharp}, \rounddown{x^\sharp}
    \right)
    \label{bg:eq:cast}
\end{equation}
For example, for the real variable $a \in [0.2, 0.3]$ under single precision
with rounding to nearest,
\begin{equation}
    \mathrm{cast}\left([0.2, 0.3]\right) = \left(
        {\left[0.200000003, 0.300000012\right]}_\floatset,
        \left[-1/67108864, 1/67108864\right]
    \right)
\end{equation}
After this, the propagation of bounds in the data flow graph is carried out as
described in Section~\ref{bg:sub:intervals}, where the difference is the abstract
error semantics defined in~\eqref{bg:eq:error_semantics} is used in lieu of the
interval semantics. At the root of the tree (\ie~the exit of the DFG) we find
the value of the accuracy analysis result for the expression.

\section{Novel Semantics}
\label{so:sec:semantics}

\subsection{Accuracy Analysis}

In Section~\ref{bg:sub:accuracy} of Chapter~\ref{chp:background}, we described
a technique to analyze the round-off error of evaluating an expression tree.
Throughout this chapter, we use the function $\error: \exprset\to\errorset$ to
represent the above-mentioned analysis of evaluation accuracy, where $\exprset$
denotes the set of all expressions.


\subsection{Resource Usage Analysis}

Here we define similar formal semantics which calculate an approximation to the
FPGA resource usage of an expression, taking into account common subexpression
elimination. This is important as, for example, rewriting $a \times b + a
\times c$ as $a \times (b + c)$ in the larger expression $(a \times b + a
\times c) + {(a \times b)}^2$ causes the common subexpression $a \times b$ to
be no longer present in both terms. Our analysis must capture this.

The analysis proceeds by labelling subexpressions. Intuitively, the set of
labels $\labelset$, is used to assign unique labels to unique expressions,
so it is possible to easily identify and reuse them. For convenience, let
the function $\fresh: \exprset\to\labelset$ assign a distinct label to each
expression or variable, where $\exprset$ is the set of all expressions. Before
we introduce the labeling semantics, we define the environment $\lambda:
\labelset\to\exprset\cup\{\bot\}$, which is a function that maps labels to
expressions, and $\env{}$ denotes the set of such environments. A label $l$ in
the domain of $\lambda\in\env{}$ that maps to $\bot$ indicates that $l$ does
not map to an expression. An element $(l, \lambda)\in\labelset\times\env{}$
stands for the labeling scheme of an expression. Initially, we map all labels
to $\bot$, then in the mapping $\lambda$, each leaf of an expression is
assigned a unique label, and the unique label $l$ is used to identify the leaf.
That is for the leaf variable or constant $x$:
\begin{equation}
    (l, \lambda) = (\fresh(x), [\fresh(x)\mapsto{x}])
\end{equation}
This equation uses $[\fresh(x)\mapsto{x}]$ to indicate an environment that
maps the label $\fresh(x)$ to the expression $x$ and all other labels map
to $\bot$, in other words, if $l = \fresh(x)$ and $l^\prime \neq l$, then
$\lambda(l) = x$ and $\lambda(l^\prime) = \bot$.

\begin{figure}[ht]
    \centering
    \includegraphics[scale=0.6]{sample_tree}
    \caption{The DFG for the sample expression.}\label{so:fig:sample_tree}
\end{figure}
For example, for the DFG in Figure~\ref{so:fig:sample_tree}, taken from
Section~\ref{bg:sec:abstract_interpretation} of Chapter~\ref{chp:background},
we have for the variables $a$ and $b$:
\begin{equation}
    \begin{aligned}
        (l_a, \lambda_a) &= (\fresh(a), [\fresh(a)\mapsto{a}])
                   = (l_1, [l_1 \mapsto a]) \\
        (l_b, \lambda_b) &= (l_2, [l_2 \mapsto b])
    \end{aligned}
    \label{so:eq:variable_env}
\end{equation}
Then the environments are propagated in the flow direction of the DFG, using
the following formulation of the labeling semantics:
\begin{equation}
    \begin{aligned}
        (l_x, \lambda_x) \circ (l_y, \lambda_y)
            &= (l, (\lambda_x\odot\lambda_y)
                      [l\mapsto{l_x \circ l_y}]) \\
            \text{where~} l &= \fresh(l_x \circ l_y),
                          \circ\in\{+, -, \times\}
    \end{aligned}
    \label{so:eq:labeling_semantics}
\end{equation}
Specifically, $\lambda=\lambda_x\odot\lambda_y$ signifies that $\lambda_y$
is used to update the mapping in $\lambda_x$, if the mapping does not
exist in $\lambda_x$, and result in a new environment $\lambda$; and
$\lambda[l\mapsto{x}]$ is a shorthand for $\lambda\odot[l\mapsto{x}]$.  As
an example, with the expression in Figure~\ref{so:fig:sample_tree}, using
\eqref{so:eq:variable_env}, recall to mind that $l_1 = l_a$, $l_2 = l_b$, we
derive for the subexpression $a + b$:
\begin{equation}
    \begin{aligned}
        (l_{a + b}, \lambda_{a + b})
            &= (l_a, \lambda_a) + (l_b, \lambda_b) \\
            &= (l_3, (\lambda_a \odot \lambda_b) [l_3\mapsto{l_a + l_b}]) \\
            &  \text{where~} l_3 = \fresh(l_a + l_b) \\
            &= (l_3, [l_1\mapsto{a}]\odot
                     [l_2\mapsto{b}]\odot
                     [l_3\mapsto{l_1 + l_2}]) \\
            &= (l_3, [l_1\mapsto{a}, l_2\mapsto{b}, l_3\mapsto{l_1 + l_2}])
    \end{aligned}
\end{equation}
\todo{George: I am a little confused by addition here.  What is the definition
of ``+'' on labels?  If $l_a + l_b$ should be read purely as a syntactic
construct, why does it need a distinct representation as $l_3$?}
Finally, for the full expression $(a + b) \times (a + b)$:
\begin{equation}
    \begin{aligned}
        (l, \lambda)
            &= (l_{a + b}, \lambda_{a + b}) \times
               (l_{a + b}, \lambda_{a + b}) \\
            &= (l_4, [l_1\mapsto{a}, l_2\mapsto{b},
                      l_3\mapsto{l_1 + l_2}, l_4\mapsto{l_3 \times l_3}])
    \end{aligned}
\end{equation}
From the above derivation, it is clear that the semantics capture the reuse
of subexpressions. The estimation of area is performed by counting, for an
expression, the numbers of additions, subtractions and multiplications in
the final labeling environment, then calculating the number of LUTs used to
synthesize the expression. If the number of operators is $n_\circ$ where
$\circ\in\{+,-,\times\}$, then the number of LUTs in total for the expressions
is estimated as $\sum_{\circ\in\{+,-,\times\}} A_\circ n_\circ$, where the
value $A_\circ$ denotes the number of LUTs per $\circ$ operator, which is
dependent on the type of the operator and the floating-point format used to
generate the operator.

In the following sections, we use the function $\area: \exprset\to\naturalset$
to denote our resource usage analysis.

\subsection{Equivalent Expressions Analysis}
\label{so:sub:equivalent_expressions_analysis}

In earlier sections, we introduce semantics that define additions and
multiplications on intervals, then gradually transition to error semantics that
compute bounds of values and errors, as well as labelling environments that
allow common subexpression elimination, by defining arithmetic operations on
these structures. In this section, we now take the leap from not only analyzing
an expression for its quality, to defining arithmetic operations on sets of
equivalent expressions, and use these rules to discover equivalent expressions.
Before this, it is necessary to formally define equivalent expressions and
functions to discover them.

\subsubsection{Discovering Equivalent Expressions}

From an expression, a set of equivalent expressions can be discovered by our
\emph{equivalence relation} $\eqrel$ on the set of all expressions $\exprset$,
and $\eqrel \subset \exprset\times\exprset$.  It is noteworthy that a relation
is said to be an equivalence relation when it is reflexive, symmetric and
transitive, \ie~for all $e_1, e_2, e_3 \in \exprset$, we have the following
rules in our inference system:
\begin{equation}
    \begin{aligned}
        \text{Reflexivity}
            &: e_1 \eqrel e_1 \\
        \text{Symmetry}
            &: \text{~if~} e_1 \eqrel e_2,
            \text{~then~} e_2 \eqrel e_1 \\
        \text{Transitivity}
            &: \text{~if~} e_1 \eqrel e_2 \text{~and~} e_2 \eqrel e_3,
            \text{~then~} e_1 \eqrel e_3.
    \end{aligned}
    \label{so:eq:equivalence_relation}
\end{equation}

We extend our inference system with additional rules that relate equivalent
expressions.  Let's define $e_1, e_2, e_3 \in \exprset$, $v_1, v_2, v_3 \in
\realset$, and $\circ \in \{+, \times\}$.  First, the arithmetic rules are:
\begin{equation}
    \begin{aligned}
        \text{Associativity}(\circ)
            &: (e_1 \circ e_2) \circ e_3 \eqrel e_1 \circ (e_2 \circ e_3) \\
        \text{Commutativity}(\circ)
            &: e_1 \circ e_2 \eqrel e_2 \circ e_1 \\
        \text{Distributivity}
            &: e_1 \times (e_2 + e_3) \eqrel e_1 \times e_2 + e_1 \times e_3
    \end{aligned}
    \label{so:eq:equivalence_arithmetic}
\end{equation}
Secondly, the reduction rules are:
\begin{equation}
    \begin{aligned}
        \text{Identity}(\times)
            &: e_1 \times 1 \eqrel e_1 \quad &
        \text{Zero Propagation}
            &: e_1 \times 0 \eqrel 0 \\
        \text{Identity}(+)
            &: e_1 + 0 \eqrel e_1 &
        \text{Constant Propagation}(\circ)
            &: \inference{v_3 = v_1 \circ v_2}{v_1 \circ v_2 \eqrel v_3}
    \end{aligned}
    \label{so:eq:equivalence_reduction}
\end{equation}
The Constant Propagation rule states that if an expression is a
summation/multiplication of two values, then it can be simply evaluated to
produce the result. Finally, the following two allow structural induction on
expression trees, \eg~it is possible to derive that $a + (b + c) \eqrel a + (c
+ b)$ from $b + c \eqrel c + b$:
\begin{equation}
    \begin{aligned}
        \text{Tree}(\circ)
            : \inference{e_1 \eqrel e_2}{e_3 \circ e_1 \eqrel e_3 \circ e_2}
        \quad &
        \text{Tree}^\prime(\circ)
            : \inference{e_1 \eqrel e_2}{e_1 \circ e_3 \eqrel e_2 \circ e_3}
    \end{aligned}
    \label{so:eq:equivalence_tree}
\end{equation}

We say that $e_1$ is equivalent to $e_2$ if and only if $e_1 \eqrel e_2$ For
some expressions $e_1$ and $e_2$.  Although for simplicity, we have not defined
rules for subtraction and division, these rules can be easily added and are
present in our framework.


\subsubsection{Scalable Methods}

The above rules of equivalence relates an expression with all of its equivalent
expressions.  In general because of combinatorial explosion, the set of all
equivalent expressions is so large to be derived, which motivates us to develop
scalable methods that execute fast enough even with large expressions.

Instead of deriving the full set of equivalent expressions, we can define
a new relation $\eqgenrel$, a subset of $\eqrel$, which is identical to
our equivalent relation $\eqrel$ except that we do not have transitivity
in~\eqref{so:eq:equivalence_relation} from $\eqrel$, to generate equivalent
expressions in a series of steps.

We define the function $\eqstep: \powersetof\exprset\to\powersetof\exprset$,
where $\powersetof\exprset$ denotes the power set of $\exprset$, which
generates a (possibly larger) set of equivalent expressions from an initial set
of equivalent expressions by one step of $\eqgenrel$, that is:
\begin{equation}
    \eqstep(\epsilon) = \left\{
        e^\prime\in\exprset \mid
        e \eqgenrel e^\prime \wedge e\in\epsilon\right\}
    \label{so:eq:eqstep}
\end{equation}
where $\epsilon$ is a set of equivalent expressions.
\begin{corollary}
    By the definition of $\eqstep$ in~\eqref{so:eq:eqstep}, $\eqstep(\epsilon_a
    \cup \epsilon_b) = \eqstep(\epsilon_a) \cup \eqstep(\epsilon_b)$.
    \label{so:cor:union}
\end{corollary}

From this, we may note that we can define a function
$\eqstep^{\star:N}(\epsilon)$ to generate a set of equivalent expressions,
by taking the union of $N$ steps of $\eqstep$ of $\epsilon$, as given by the
following formula:
\begin{equation}
    \eqstep^{\star:N}(\epsilon) = \bigcup_{i = 0}^N \eqstep^i(\epsilon)
    \label{so:eq:transitive_generator}
\end{equation}
Here we define:
\begin{equation}
    \begin{aligned}
        \eqstep^0(\epsilon) &= \epsilon \quad \text{and~} \\
        \eqstep^i(\epsilon) &= \eqstep\left(
            \eqstep^{i - 1}\left(\epsilon\right)
        \right) \quad \text{for~} i \in \{ 0, 1, 2, \cdots \}
    \end{aligned}
\end{equation}
By allowing $N$ to approach $\infty$, we obtain the full set of equivalent
expressions of $\epsilon$, \ie~the transitive closure:
\begin{equation}
    \eqstep^\star(\epsilon)
    = \eqstep^{\star:\infty}(\epsilon)
    = \bigcup_{i = 0}^\infty \eqstep^i(\epsilon)
    \label{so:eq:transitive_closure}
\end{equation}

\begin{lemma}
    $\eqstep^{\star:N}(\epsilon) = \epsilon \cup
    \eqstep\left(\eqstep^{\star:N-1}(\epsilon)\right)$.
    \label{so:lem:transitive}
\end{lemma}
\begin{proof}
    Following~\eqref{so:eq:transitive_generator}, $\eqstep^{\star:N}(\epsilon)
    = \eqstep^0(\epsilon) \cup \eqstep^1(\epsilon) \cup \cdots \cup
    \eqstep^N(\epsilon)$.  We then apply Corollary~\eqref{so:cor:union} to the
    right-hand side to get $\epsilon \cup \eqstep\left( \eqstep^0(\epsilon)
    \cup \eqstep^1(\epsilon) \cup \cdots \cup \eqstep^{N-1}(\epsilon)\right)$,
    which equals to $\epsilon \cup \eqstep\left( \eqstep^{\star:N-1}(\epsilon)
    \right)$ by definition.
\end{proof}

In practice, it is often infeasible to generate the full transitive closure of
a given expression, we therefore impose further constraints on how we discover
equivalent expressions.

First, instead of exploring the full transitive closure, that is, by allowing
the number of steps $N$ in~\eqref{so:eq:transitive_generator} to be infinite,
we may restrict $N$ to be a small finite value to allow a smaller set of
equivalent expressions to be computed.

Second, the complexity of equivalent expression finding is reduced by fixing
the structure of subexpressions at a certain depth $k$ in the original
expression.  The definition of depth is given as follows: first the root
of the parse tree of an expression is assigned depth $d = 1$; then we
recursively define the depth of a node as one more than the depth of its
greatest-depth parent.  If the depth of the node is greater than $k$, then
we fix the structure of its child nodes by disallowing any equivalence
transformation beyond this node. We let $\eqstep_k$ denote this ``depth
limited'' equivalence finding function, where $k$ is the depth limit used.  We
use $\eqstep^{\star:N}_k$ and $\eqstep^\star_k$ to denote the functions to
respectively compute the union of $N$ steps of $\eqstep_k$ and the transitive
closure. This approach is similar to Martel's depth limited equivalent
expression transform~\cite{martel07}, however Martel's method eventually allows
transformation of subexpressions beyond the depth limit, because rules of
equivalence would transform these to have a smaller depth.  This contributes
to a time complexity at least exponential in terms of the expression size. In
contrast, our technique has a time complexity that does not depend on the size
of the input expression, but grows with respect to the depth limit $k$. Note
that the full equivalence closure using the inference system we defined earlier
in~\eqref{so:eq:transitive_closure} is at least $O({2n - 1}!!)$ where $n$ is
the number of terms in an expression, as we discussed earlier.

Finally, we use the iterative algorithm in Figure~\ref{so:alg:eqstep} to
efficiently compute $\eqstep^{\star:N}_k$.  In each iteration, we keep track of
the equivalent expressions that are newly discovered in the current iteration,
so that in the next iteration we apply $\eqgenrel$ only to those expressions,
to avoid redundant computation.  We then continue to prove that this algorithm
indeed computes $\eqstep^{\star:N}_k$.
\begin{figure}[ht]
    \centering
    \begin{algorithmic}
        \Function{Equivalent}{$\epsilon$, $k$, $N$}
            \State{$s_0 \gets \epsilon$}
            \State{$s^\prime_0 \gets \epsilon$}
            \For{$i \gets 1, \ldots, N$}
                \State{$s^\prime_i \gets
                    \eqstep_k \left(s^\prime_{i-1}\right) - s_{i-1}$}
                \State{$s_i \gets s_{i-1} \cup s^\prime_i$}
                \If{$s^\prime_i \neq \emptyset$}
                    \State{\Return{$s_i$}}
                \EndIf{}
            \EndFor{}
            \State{\Return{$s_i$}}
        \EndFunction{}
    \end{algorithmic}
    \caption{%
        Our algorithm to discover a set of equivalent expressions from an
        initial set $\epsilon$.
    }\label{so:alg:eqstep}
\end{figure}
\begin{theorem}
    In the algorithm in Figure~\ref{so:alg:eqstep}, at iteration
    $n$, the set of equivalent expressions $s_n$ computes exactly
    $\eqstep^{\star:n}_k(\epsilon)$.
\end{theorem}
\begin{proof}
    We start by assuming that at iteration $m > 0$, $s_m =
    \eqstep^{\star:m}_k(\epsilon)$, and we prove this equality still holds if
    substitute $m$ with $m + 1$.  From the algorithm, we can deduce:
    \begin{equation*}
    \begin{aligned}
        s_{m+1}
         &= s_m \cup s^\prime_{m+1} \\
         &= s_m \cup \left( \eqstep_k \left( s^\prime_m \right) - s_m \right) \\
         &= s_m \cup \eqstep_k \left( s^\prime_m \right) \\
         &= s_m \cup \eqstep_k \left(
                \eqstep_k \left( s^\prime_{m-1} \right) - s_{m-1}
            \right)
    \end{aligned}
    \end{equation*}
    We substitute $s_m$ using Lemma~\ref{so:lem:transitive} to get:
    \begin{equation*}
        s_{m+1}
          = \epsilon \cup \eqstep_k \left( s_{m-1} \right) \cup
            \eqstep_k \left(
                \eqstep_k \left( s^\prime_{m-1} \right) - s_{m-1}
            \right)
    \end{equation*}
    Using distributivity of $\eqstep_k$ over $\cup$ and the iteration $m$ of
    the algorithm, we can derive:
    \begin{equation*}
    \begin{aligned}
        s_{m+1}
         &= \epsilon \cup \eqstep_k \left(
                s_{m-1} \cup \left(
                    \eqstep_k \left( s^\prime_{m-1} \right) - s_{m-1}
                \right)
            \right) \\
         &= \epsilon \cup \eqstep_k \left( s_m \right)
    \end{aligned}
    \end{equation*}
    Finally, we make use of the assumption $s_m =
    \eqstep^{\star:m}_k(\epsilon)$, followed by Lemma~\ref{so:lem:transitive}
    to show:
    \begin{equation*}
        s_{m+1}
        = \epsilon \cup \eqstep_k \left(
            \eqstep^{\star:m}_k(\epsilon)
        \right)
        = \eqstep^{\star:m+1}_k(\epsilon)
    \end{equation*}
    It is trivial that $s_0 = \epsilon = \eqstep^{\star:0}_k(\epsilon)$, by
    induction, $s_n = \eqstep^{\star:n}_k(\epsilon)$ thus holds for all $n \in
    \naturalset$.
\end{proof}

\subsubsection{Pareto Frontier}

Because we optimize expressions in two quality metrics, \ie~the accuracy of
computation and the estimate of FPGA resource utilization, there is a trade-off
between them. We desire the largest subset of all equivalent expressions
$E$ discovered such that in this subset, no expression dominates any other
expression, in terms of having both better area and better accuracy. This
subset is known as the Pareto frontier.  Figure~\ref{so:alg:pareto} shows
a simplified algorithm for calculating the Pareto frontier for a set of
equivalent expressions $\epsilon$.
\begin{figure}[ht]
    \centering
    \begin{algorithmic}
        \Function{Frontier}{$\epsilon$}
            \State{$\mathit{frontier} \gets \epsilon$}
            \For{$e \in \epsilon$}
                \For{$e^\prime \in \epsilon$}
                    \If{$\mathit{Area}(e^\prime) < \mathit{Area}(e)$ and
                        $\abserr(e^\prime) < \abserr(e)$}
                        \State{%
                            $\mathit{frontier} \gets
                                \mathit{frontier} / \{ e \}$}
                    \EndIf{}
                \EndFor{}
            \EndFor{}
            \State{\Return{$\mathit{frontier}$}}
        \EndFunction%
    \end{algorithmic}
    \caption{The Pareto frontier from a set of equivalent expressions.
    }\label{so:alg:pareto}
\end{figure} \\
Here, $\mathit{frontier} / \{ e \}$ is a set identical to $\mathit{frontier}$,
except that the element $e$ is removed.  We use the function $\abserr$ to
analyze the magnitudes of error bounds, which is defined as follows:
\begin{equation}
    \begin{aligned}
        \abserr(e) &= \max\left(
            \left| \mu^\sharp_{\min} \right|,
            \left| \mu^\sharp_{\max} \right|
        \right) \\
        & \quad \text{where~}
        \left(
            x^\sharp, \left[ \mu^\sharp_{\min}, \mu^\sharp_{\max} \right]
        \right) = \error(e)
    \end{aligned}
\end{equation}

\subsubsection{Equivalent Expressions Semantics}

Similar to the analysis of accuracy and resource usage, a set of equivalent
expressions can be computed with semantics. That is, we define structures,
\ie~sets of equivalent expressions, that can be manipulated with arithmetic
operators. In our equivalent expressions semantics, an element of
$\powersetof\exprset$ is used to assign a set of expressions to each node
in an expression parse tree. To begin with, at each leaf of the tree, the
variable or constant is assigned a set containing itself, as for $x$, the set
$\epsilon_x$ of equivalent expressions is $\epsilon_x = \{x\}$. After this, we
propagate the equivalence expressions in the parse tree's direction of flow,
using~\eqref{so:eq:equivalence_semantics} defined below:
\begin{equation}
    \begin{aligned}
        \epsilon_x \circ \epsilon_y &= \frontier\left(
            \eqstep^\star_k \left(
                E_\circ \left( \epsilon_x, \epsilon_y \right)
            \right) \right) \\
        & \text{where~}
        E_\circ(\epsilon_x, \epsilon_y) = \{
            e_x \circ e_y \mid e_x \in \epsilon_x \wedge e_y \in \epsilon_y
        \}, \\
        & \text{and~} \circ\in\{+, -, \times\}
    \end{aligned}
    \label{so:eq:equivalence_semantics}
\end{equation}
The equation implies that in the propagation procedure, it recursively
constructs a set of equivalent subexpressions for the parent node from
two child expressions, and uses the depth limited equivalence function
$\eqstep^\star_k$ to work out a larger set of equivalent expressions. To reduce
computation effort, we select only those expressions on the Pareto frontier
for the propagation in the DFG\@. Although in worst case the complexity of
this process is exponential, the selection by Pareto optimality accelerates
the algorithm significantly. For example, for the subexpression $a + b$ of our
sample expression:
\begin{equation}
    \begin{aligned}
        \epsilon_a + \epsilon_b
            &= \frontier\left(
                    \eqstep^\star_k \left(
                        E_\circ \left( \epsilon_a, \epsilon_b \right)
                    \right)
                \right) \\
            &= \frontier\left(
                    \eqstep^\star_k \left(
                        E_\circ \left( \{a\}, \{b\} \right)
                    \right)
                \right) \\
            &= \frontier\left(
                    \{ a + b, b + a \}
                \right)
    \end{aligned}
\end{equation}
Alternatively, we could view the semantics in terms of DFGs representing
the algorithm for finding equivalent expressions. The parsing of an
expression directly determines the structure of its DFG\@. For instance,
the expression $(a + b) \times (a + b)$ generates the DFG illustrated in
Figure~\ref{so:fig:tree_expr_flow}. The circles labeled $3$ and $7$ in this
diagram are shorthands for the operation $E_+$ and $E_\times$ respectively,
where $E_+$ and $E_\times$ is defined in~\eqref{so:eq:equivalence_semantics}.
\begin{figure}[ht]
    \centering
    \includegraphics[scale=0.6]{tree_expr_flow}
    \caption{The DFG for finding equivalent expressions of
    $(a + b) \times (a + b)$.}\label{so:fig:tree_expr_flow}
\end{figure}

For our example in Figure~\ref{so:fig:tree_expr_flow},
similar to the construction of data flow equations in
Section~\ref{bg:sec:abstract_interpretation} of Chapter~\ref{chp:background},
we can produce a set of equations from the data flow of the DFG, which now
produces equivalent expressions:
\begin{equation}
    \begin{aligned}
        \enter{1} &= \enter{1} \cup \{a\} &
        \enter{2} &= \enter{2} \cup \{b\} \\
        \enter{3} &= E_+(\enter{1}, \enter{2}) &
        \enter{4} &= \enter{3} \cup \enter{5} \\
        \enter{5} &= \eqstep_k(\enter{4}) &
        \enter{6} &= \frontier(\enter{5}) \\
        \enter{7} &= E_\times(\enter{6}, \enter{6}) &
        \enter{8} &= \enter{7} \cup \enter{9} \\
        \enter{9} &= \eqstep_k(\enter{8}) &
        \enter{10} &= \frontier(\enter{9})
    \end{aligned}
    \label{so:eq:tree_expr_flow}
\end{equation}
Because of loops in the DFG, it is no longer trivial to find the solution.
In general, the analysis equations are solved iteratively. We can
regard the set of equations as a single transfer function $F$ as in
\eqref{so:eq:transfer_function}, where the function $F$ takes as input
the variables $A(1), \ldots, A(10)$ appearing in the right-hand sides of
\eqref{so:eq:tree_expr_flow} and outputs the values $A(1), \ldots, A(10)$
appearing in the left-hand sides. Our aim is then to find an input $\vec{x}$ to
$F$ such that $F(\vec{x}) = \vec{x}$, \ie~a fixpoint of $F$.
\begin{equation}
      F((\enter{1}, \ldots, \enter{10}))
    = (\enter{1} \cup \{a\}, \ldots, \frontier(\enter{9}))
    \label{so:eq:transfer_function}
\end{equation}
Initially we assign $\enter{i} = \varnothing$ for $i\in\{1,2,\ldots,10\}$,
and we denote $\vec\varnothing = (\varnothing, \ldots, \varnothing)$.
Then we compute iteratively $F(\vec\varnothing)$, $F^2(\vec\varnothing) =
F(F(\vec\varnothing))$, and so forth, until the fixpoint solution $\fix F$ is
reached for some iteration $n$, that is:
\begin{equation}
    \fix F = F^n(\vec\varnothing) =
    F(F^n(\vec\varnothing)) = F^{n + 1}(\vec\varnothing)
\end{equation}
The fixpoint solution $\fix F$ gives a set of equivalent expressions derived
using our method, which is found at $\enter{10}$. In essence, the depth limit
acts as a sliding window.  The semantics allow hierarchical transformation of
subexpressions using a depth-limited search and the propagation of a set of
subexpressions that are locally Pareto optimal to the parent expressions in a
bottom-up hierarchy.

The problem with the semantics above is that the time complexity of
$\eqstep^\star_k$ scales poorly, since the worst case number of subexpressions
needed to explore increases exponentially with $k$. Therefore an alternative
method is to optimize it by changing the structure of the DFG slightly, as
shown in Figure~\ref{so:fig:tree_expr_flow_greedy}. The difference is that at
each iteration, the Pareto frontier filters the results to decrease the number
of expressions to process for the next iteration.
\begin{figure}[ht]
    \centering
    \includegraphics[scale=0.6]{tree_expr_flow_greedy}
    \caption{The alternative DFG for $(a + b) \times (a + b)$.
    }\label{so:fig:tree_expr_flow_greedy}
\end{figure}

In the rest of this chapter, we use \frontiertrace{} to indicate our equivalent
expression finding semantics, and \greedytrace{} to represent the alternative
method.

\section{Implementation}
\label{so:sec:implementation}

The majority of \soap, is implemented in Python.  For computing errors in
real arithmetic, we use exact arithmetic based on rational numbers within the
\gls{gmp} library~\cite{gmp}.  In case when exact arithmetic is not possible
because of high computational costs, floating-point arithmetic can be used
to efficiently and safely bound round-off error values.  We also use the
\gls{mpfr} library~\cite{mpfr} for access to floating-point rounding modes and
arbitrary precision floating-point computation.

Because of the workload of equivalent expression finding, the underlying
algorithm is optimized in many ways. First, for each iteration, the relation
finding function $\eqstep_k$ is only applied to newly discovered expressions
in the previous iteration, using the algorithm in Figure~\ref{so:alg:closure}.
The second optimization is to cache results of function calls such as
$\eqstep_k$, $\area$ and $\error$, since there is a large chance that these
results from subexpressions are reused several times, subexpressions are also
maximally shared to eliminate duplication in memory.  Thirdly, the computation
of $\eqstep_k$ is fully multi-threaded.

The resource statistics of operators are provided using FloPoCo~\cite{flopoco}
and \gls{xst}~\cite{xst}.  Initially, For each combination of an operator,
an exponent width between 5 and 16, and a mantissa width ranging from 10 to
113, a total of 2496 distinct implementations are generated using FloPoCo.
All of them are optimized to use \gls{dsp} blocks.  They are then synthesized
using \gls{xst}, targeting a Virtex-6 \gls{fpga} device (XC6VLX760).  Because
\glspl{lut} are generally more constrained resources than \gls{dsp} blocks in
floating-point computations, we provide synthesis statistics in \glspl{lut}
only.  Finally, an \gls{rtl} code generation backend can produce synthesizable
code from an optimized candidate expression.

\section{Results}
\label{so:sec:results}

Because Martel's approach defers selecting optimal options until the end of
equivalent expression discovery, we developed a method that could produce
exactly the same set of equivalent expressions from the traces computed by
Martel, and has the same time complexity. The difference is that we adopted it
to generate a Pareto frontier from the discovered expressions, instead of only
error bounds.  This allows us to compare \marteltrace{}, \ie~our implementation
of Martel's method, against our methods \frontiertrace{} and \greedytrace{}
discussed in Section~\ref{so:sec:equivalent}.  Figure~\ref{so:fig:martel}
optimizes the expression ${(\vara + \varb)}^2$ using the three methods above,
all using depth limit $3$, and the input ranges are $\vara \in [5, 10]$
and $\varb \in [0, 0.001]$~\cite{martel07}. The IEEE 754 single-precision
floating-point format with rounding to nearest was used for the evaluation
of accuracy and area estimation. The scatter points represent different
implementations of the original expression that have been explored and
analyzed, and the (overlapping) lines denote the Pareto frontiers. In this
example, our methods produce the same Pareto frontier that Martel's method
could discover, while having up to 50\% shorter run time. Because we consider
an accuracy/area trade-off, we find that we can not only have the most accurate
implementation discovered by Martel, but also an option that is only 0.0005\%
less accurate, but uses 7\% fewer \glspl{lut}.

We go beyond the optimization of a small expression, by generating results in
Figure~\ref{so:fig:multi_expr_32} to show that the same technique is applicable
to simultaneous optimization of multiple large expressions. The expressions
$e_1$ and $e_2$, with input ranges $\vara \in [1, 2], \varb \in [10, 20], \varc
\in [10, 200]$ are used as our example:
\begin{equation}
    \begin{aligned}
    e_1 =&
        (\vara + \vara + \varb) \times
        (\vara + \varb + \varb) \times
        (\varb + \varb + \varc) \times {} \\
        &
        (\varb + \varc + \varc) \times
        (\varc + \varc + \vara) \times
        (\varc + \vara + \vara), \\
    e_2 =&
        (1 + \varb + \varc) \times
        (\vara + 1 + \varc) \times
        (\vara + \varb + 1).
    \end{aligned}
\end{equation}

We generated and optimized \gls{rtl} implementations of $e_1$ and
$e_2$ simultaneously using \frontiertrace{} and \greedytrace{}
with the depth limits indicated by the numbers in the legend of
Figure~\ref{so:fig:multi_expr_32}. Note that because the expressions evaluate
to large values, the errors are also relatively large. We set the depth limit
to $2$ and found that \greedytrace{} executes up to $10\times$ faster than
\frontiertrace{}, while discovering a sizable subset of the Pareto frontier of
\frontiertrace{}. Also our methods are significantly faster and more scalable
than \marteltrace{}, because of its poor scalability discussed earlier, our
computer ran out of 8 GB of memory before we could produce any results. If we
normalize the time allowed for each method and compare the performance, we
found that \greedytrace{} with a depth limit $3$ takes takes slightly less time
than \frontiertrace{} with a depth limit $2$, but produces a generally better
Pareto frontier. The alternative implementations of the original expression
provided by the Pareto frontier of \greedytrace{} can either reduce the
\glspl{lut} used by approximately 10\% when accuracy is not crucial, or can
be about 10\% more accurate if resource is not our concern.  It also enables
the ability to choose different trade-off options, such as an implementation
that is 7\% more accurate and uses 7\% fewer \glspl{lut} than the original
expression.

Furthermore, Figure~\ref{so:fig:multi_expr_vary_width} varies the mantissa
width of the floating-point format, and presents the Pareto frontier
of both $e_1$ and $e_2$ together under optimization. Floating-point
formats with mantissa widths ranging from 10 to 112 bits were used to
optimize and evaluate the expressions for both accuracy and area usage. It
turns out that some implementations originally on the Pareto frontier of
Figure~\ref{so:fig:multi_expr_32} are no longer desirable, as by varying the
mantissa width, new implementations are both more accurate and less resource
demanding.

Besides the large example expressions above, Figure~\ref{so:fig:taylor_sin}
and Figure~\ref{so:fig:motzkin} are produced by optimizing expressions with
real applications under single precision. Figure~\ref{so:fig:taylor_sin} shows
the optimization of the Taylor expansion of $\sin(x + y)$, where $x\in[-0.1,
0.1]$ and $y\in[0, 1]$, using \greedytrace{} with a depth limit $3$. The
function $\mathrm{taylor}(f, d)$ indicates the Taylor expansion of function
$f(x, y)$ at $x = y = 0$ with a maximum degree of $d$. For order 5 we reduced
error by more than 60\%. Figure~\ref{so:fig:motzkin} illustrates the results
obtained using the depth limit $3$ with the Motzkin polynomial~\cite{demmel}
$x^6 + y^4 z^2 + y^2 z^4 - 3 x^2 y^2 z^2$, which is known to be difficult to
evaluate accurately, especially using inputs $x\in[-0.99, 1]$, $y\in[1, 1.01]$,
$z\in[-0.01, 0.01]$.

All these above results are generated with the same type of floating-point
operators in each expression.  Although in this chapter we do not analyze
the number of \glspl{dsp} used in synthesized circuits, the \gls{dsp} count
increases linearly with the estimated \gls{lut} count.  In the next chapter
we further introduce the estimation of \gls{dsp} elements used as another
objective to optimize.

Because of the scalability problem of the depth limit $k$ mentioned in
Section~\ref{so:sec:equivalent}, $k \leq 3$ for all of our experiments.  By
setting $k = 4$, the tool does not terminate in reasonable amount of time and
saturates the memory (16 GB) of our system.  In the following chapters, we
propose methods to limit the number of iterations and the number of equivalent
expressions discovered to mitigate the lack of scalability of $k$.

Finally, Figure~\ref{so:fig:area} demonstrates the accuracy of the area
estimation used in our analysis. It compares the actual \glspl{lut} necessary
with the estimated number of \glspl{lut} using our semantics, by synthesizing
more than 6000 equivalent expressions derived from $\vara + \varb + \varc$,
$(\vara + 1) \times (\varb + 1) \times (\varc + 1)$, $e_1$, and $e_2$ using
varying mantissa widths. The dotted line indicates exact area estimation, a
scatter points that is close to the line means the area estimation for that
particular implementation is accurate. The solid black line represents the
linear regression line of all scatter points. On average, our area estimation
is a 6.1\% over-approximation of the actual number of \glspl{lut}, and the
worst case over-approximation is 7.7\%.
\newcommand{\figsize}{0.5}
\begin{figure}[ht]
    \centering
    \includegraphics[scale=\figsize]{martel}
    \caption{Optimization of ${(\vara + \varb)}^2$.}\label{so:fig:martel}
\end{figure}
\begin{figure}[ht]
    \centering
    \includegraphics[scale=\figsize]{multi_expr_32}
    \caption{%
        Simultaneous optimization of both $e_1$ and $e_2$.
    }\label{so:fig:multi_expr_32}
\end{figure}
\begin{figure}[ht]
    \centering
    \includegraphics[scale=\figsize]{multi_expr_vary_width}
    \caption{%
        Varying the mantissa width of Figure~\ref{so:fig:multi_expr_32}.
    }\label{so:fig:multi_expr_vary_width}
\end{figure}
\begin{figure}[ht]
    \centering
    \includegraphics[scale=\figsize]{taylor_sin}
    \caption{The Taylor expansion of $\sin(x + y)$.}\label{so:fig:taylor_sin}
\end{figure}
\begin{figure}[ht]
    \centering
    \includegraphics[scale=\figsize]{motzkin}
    \caption{The Motzkin polynomial $e_m$.}\label{so:fig:motzkin}
\end{figure}
\begin{figure}[ht]
    \centering
    \includegraphics[scale=\figsize]{area}
    \caption{Accuracy of Area Estimation.}\label{so:fig:area}
\end{figure}

High-level synthesis tools are typically designed to adhere to a rigid
specification which outlines their behaviour.  It is a traditional practice
to design this specification and the subsequent tool to ensure that the
synthesized circuits perform functionally identical to the original source
program written in high-level language.  It is also viewed as a good practice
because it has predicable outcomes.  Guided by the rules of the language,
programmers translate mathematical objects such as algorithms and physical
information respectively into source code and numerical data, in a way similar
to tools adhering to their specifications.  This manual process of translation
is unfortunately an approximate one.  Computations as simple as $\sqrt{3}$ must
be approximated, \eg~they are carried out in floating-point arithmetic, because
of the finite nature of computing machines.  Therefore, \gls{hls} tools cannot
be relied upon for an exact interpretation of the mathematical objects we wish
to implement, even if they guarantee the functional equivalence between the
source code and the synthesized result.

Despite the awareness of the approximate characteristic of numerical
software/hardware implementations using floating-point operations, engineers
often take the risks of neglecting this fact, and anticipate their designs to
behave identically to the mathematical algorithms visioned in real arithmetic
within a reasonable but not well-defined error margin.  As it was shown in
Section~\ref{bg:sub:expression_accuracy} in Chapter~\ref{chp:background},
round-off errors when accumulated, could have detrimental effects on our daily
life.  The aforementioned functional equivalence between source and circuit
guaranteed by \gls{hls} tools is therefore unable to regain any lost accuracies
due to approximation.

Traditional \gls{ir}-level \gls{hls} program optimization consist of a series
of transformation passes.  Most of these passes do not predict whether
they have negative impact on the resulting circuit, and they limit their
capabilities by preserving functional equivalence.  Varying the order of
these passes could have significant impact on the quality, as these passes
interact with one another in a complicated manner, it is difficult to predict
the overall impact on performance~\cite{huang15}.  For $n$ passes, there are
$n{\,!}$ distinct ways to order, it is thus a considerable challenge to decide
the optimal pass ordering, which is exacerbated by the fact that it could be
highly dependent on the input program~\cite{cong13}.

These above shortcomings of traditional \gls{hls} tools and optimizing
compilers provide a strong motivation for the work proposed in this thesis.

Firstly, we can apply the philosophy of relaxing the functional equivalence
required by \gls{hls} tools.  In the mean time, we preserve the equivalence of
the underlying mathematical objects in real arithmetic which hardware designs
are approximating.  One can often improve the numerical accuracy by choosing a
better alternative among these equivalences.

Secondly, by the same paradigm shift, a wide range of optimization
opportunities can be explored to minimize throughput and resource utilization.
These opportunities were previously lost out to the necessity of ensuring
consistent behaviour.

Finally, optimization can be carried out by applying steps of equivalence
rewrites driven by a prediction model.  Traditional optimization passes can
be broken up into much smaller common parts made of equivalence rules can be
easily proved mathematically correct.  By using models to predict run time,
resources and accuracy to guide the optimization process, it is possible to
explore multiple designs that trade-off the three performance metrics while
removing concerns about the ordering problem.  Many optimization passes, such
as constant propagation, dead code removal, common subexpression elimination
and~\etc, are naturally subsumed by the new approach.  As the computational
power of machines increases exponentially, we can foresee an increase in the
scale of the vast search space to be explored in the future.

This thesis therefore broadens the horizon of \gls{hls} tools, and equips
them with the new program optimization paradigm by leveraging these above
observations.  Specifically, the trade-off relationship among numerical
accuracy, resource utilization and throughput are optimized in floating-point
numerical programs for \gls{hls}\@.  Here we summarize the contributions of
this thesis.

To the best of our knowledge, this thesis is the first to introduce
multiple-objective performance optimization in a unified framework for
discovering equivalence in programs.  Chapter~\ref{chp:stropt} implements
this framework and optimizes a suite of expressions that are difficult to
optimize by hand, and improve numerical accuracy and area automatically.
In the experimental results, it turns out that the two central goals,
\ie~improving accuracy and minimizing area, are often not in conflict, as
optimized expressions can enjoy almost all enhancements that can be achieved
in both metrics.  Guided by the concept of abstract interpretation, it further
introduces the semantics-based program analyses to jointly reason about
safe ranges of round-off errors and resource utilization, and subsequently,
discovery of equivalent expressions.  This technique lays the necessary
foundation for program equivalence beyond simple arithmetic expressions.

The infinite size of the equivalent program space, coupled with undecidability
of program properties, makes the program optimization an even more
challenging task than the one of arithmetic expressions.  For this,
Chapter~\ref{chp:progopt} introduces a new graph-based intermediate
representation, \gls{mir}, for capturing the semantics of numerical programs.
This approach reduces the size of the search space, and the \gls{ir} itself
is derived from the formal semantics of programs to ensure the correctness
of equivalent \glspl{mir} and the back-and-forth translation between C and
\gls{mir}\@.  It further eliminates the problem of optimization pass ordering,
because by using the equivalence discovery framework, the Pareto frontier can
be extended incrementally with small steps of rewrites to multiple candidates.
Traditional compiler optimizations are naturally subsumed and further enhanced
by the \glspl{mir}, as many optimization techniques such as loop splitting and
loop fusion that previously must be profiled to justify enabling them, can
emerge automatically from the optimization process.  By optimizing a suite of
resource-efficient benchmark examples, the tool improves the numerical accuracy
by up to 65\%.

Formerly, \gls{hls} tools' ability to pipeline loops is fundamentally constrained
by intra-iteration dependencies.  Traditional optimization techniques such as
partial loop unrolling may have minimal effects on the initiation interval of
pipelined loops, as these do not impact the data-path structure, which ensures
that the functional equivalence is preserved.  Encouraged by the promising
effects of Nicolau~\etal's tree height reduction technique~\cite{nicolau91}
and LegUp's recurrence minimization~\cite{canis14}, Chapter~\ref{chp:latopt}
further incorporates latency analysis into the unified program optimization
framework.  It was found that traditional optimization techniques when used
in tandem with the arithmetic equivalence rules and memory access reduction
rules can significantly improve the latency and accuracy of a numerical
program.  In Chapter~\ref{chp:progopt}, the experimental results identifies
that the static analysis of round-off errors for each candidate explored is
the key factor to the speed of optimization.  This problem is addressed in
this chapter by graph partitioning and candidate pruning algorithms.  It
further enables deeper partial loop unrolling factors that was not explored in
Chapter~\ref{chp:progopt}.  Often as we optimize numerical programs by spending
more resources, latency and round-off error can be simultaneously minimized, as
more resources would allow greater flexibility to discover equivalent programs
that often perform well in terms of run time and accuracy.  By optimizing
a suite of benchmark examples from PolyBench and Livermore loops, the tool
improves the latency and accuracy of each by up to 12$\times$ and 7$\times$
respectively, at a cost of 4$\times$ more resource utilization.


\section{Future Prospects}
\label{cc:sec:future_prospects}

In its current form, the new approach to program optimization explained in this
thesis forms the underlying basis for a much larger set of future work.  Even
though it is precursory on its own, the promising experimental results showcase
the powerful optimization it can bring to optimizing compilers and \gls{hls}
tools.  Here, a list of potential directions of future research is discussed
that could further widen the scope of our technique for a broader range of
applications.

\textbf{\gls{llvmir}-Level Program Optimization.}  We could envision
a back-and-forth translator from \gls{llvmir}~\cite{llvm, llvm_ir} to
\gls{mir} graphs.  This could enable a much wider applicability of the
technique presented in the thesis to both \gls{llvm}-based \gls{hls} tools
and software compilers.  Additionally, it could benefit from existing
\gls{llvm} optimizations passes by using the optimized \gls{llvmir} code
as inputs.  There are however obstacles in migrating to \gls{llvmir} as
the source language.  Firstly, \gls{llvmir} is \gls{ssa}-based.  Since it
uses temporary variables for intermediate results in computation, a full
liveness analysis~\cite{hathhorn12, nielson99, boissinot08} may be necessary
to eliminate temporary variables from the resulting \gls{mir}\@.  Secondly,
control-flows in \gls{llvmir} are more freely structured.  Unlike C, which
defines \iflit~statements and \whilelit~loops and discourages the use of
\verb|goto| statements, control-flow in \gls{llvmir} are composed by basic
blocks and branches between pairs of them.  This requires the \gls{mir} to be
further extended to cope with complex control-flow patterns.  Conventionally,
programs written with branches are often analyzed using \emph{continuation
style semantics}~\cite{felleisen88}.  It is not evident how this semantics can
be embedded within \glspl{mir}.

\textbf{Tighter bounds on round-off errors.} As an alternative for interval
analysis, the accuracy analysis could enjoy more sophisticated abstract domains
that capture the correlations between variables, and produce tighter bounds
for results.  Currently, the analysis cannot produce meaningful, \ie~finite,
bounds on the round-off errors of certain numerical programs.  If the analysis
fails to bound errors, then currently the optimization cannot be directed
to a more accurate implementation.  By using abstract relational domains,
it is possible to produce a much tighter bound on the values of program
variables, and the associated errors.  There are a few relational domains-based
static analysis techniques of floating-point errors~\cite{mine07_2, putot04,
goubault11, astree}, however making use of them still poses challenges.  Each
floating-point operation introduces an independent error term as a new variable
in the formulation of these relational domains, and it may be difficult to
determine how to collapse these error terms into a smaller set of variables,
as the optimization in this thesis can introduce a large number of error
variables.

\textbf{Special and fused operators.} There could be a lot of interest in the
\gls{hls} community on how \soap~can be incorporated with existing work on
fused floating-point data-path synthesis.  Langhammer~\etal~\cite{langhammer}
propose that normalization and denormalization stages could be regarded as
redundant between operators in a floating-point data-path.  By removing
these stages, subsets of the data-path become fixed-point data-paths, in the
meanwhile saving resources and improving throughput at a cost of accuracy.  It
will be compelling to isolate the normalization/denormalization stages into
operators in the \soap~framework, so that a mixed floating-point/fixed-point
program can more efficiently trade-off resources, accuracy and latency.

\textbf{Multiple word-lengths.}  In this thesis, experiments have been carried
out on floating-point operations with a fixed mantissa only.  It would be
beneficial to further integrate fixed-point support.  Additionally, by further
supporting multiple precisions in the data-path, \ie~allowing each operators to
compute with different precisions, the trade-off relationship among our three
primary performance measures can be even more effective.  Techniques, known
as multiple word-length optimization~\cite{constantinides, lee06, cantin02},
exist to apply a heuristic approach to perturb the precisions in a data-path,
so that a performance metric can be optimized while round-off errors of outputs
satisfies an error budget.  Instituting such techniques in the \soap~framework
is rewarding as it can further reduce the area and latency requirement of a
synthesized circuit for a given accuracy.  All of these approaches optimize a
fixed data-path, whereas in \soap~the structure of the data- and control-paths
are varying as we optimize them.  Analyzing each of the candidates for the
optimal precision assignment to each operator is very inefficient because of
the number of candidates explored.  Moreover, current techniques work with a
predetermined error budget, and yet in fact a Pareto frontier exists for each
data-path to trade-off accuracy, resources and latency.

\textbf{Numerical analysis and linear algebra.}  There are two distinct
approaches to the analysis of round-off errors.  One focuses on the round-off
errors by statically analyzing numerical programs, and apply this in a way
which is as general as possible, similar to the method presented in this
thesis.  On the other hand, there are techniques employed by numerical analysts
to evaluates and improve the numerical accuracy and stability of particular
algorithms analytically.  Many creative solutions to challenges are invented
in this process.  For instance, \emph{Kahan's compensated summation} algorithm
is an accurate way to compute a sum of $n$ values, $\sum_{i = 0}^{n-1}
x_i$~\cite{kahan65} is shown in Figure~\ref{co:lst:sum}.  This algorithm
cannot be discovered easily using the method outlined in this thesis, and a
way to extend the framework to optimize programs as creatively as humans still
eludes us at the moment.  Higham~\etal~\cite{higham02} discuss in great depth
many existing numerical accuracy problems encountered in finite-precision
computation of polynomials and linear algebra subprograms and how to analyze
and overcome inaccuracies, often in terms of relative errors.  Bridging the
gap between computational and mathematical approaches for numerical analysis
will allow us to automate many accuracy optimizations that were previously
unexplored by the tool.
\begin{figure}[ht]
    \centering
\begin{lstlisting}[]
    float compensated_summation(float X[N])
    {
        float sum = 0.0f;
        float e = 0.0f;
        for (i = 0; i < n; i++)
        {
            float tmp = sum;
            float y = X[i] + e;
            sum = tmp + y;
            e = (temp - sum) + y;
        }
        return sum;
    }
\end{lstlisting}
    \caption{%
        Kahan's compensated summation algorithm to accurately compute the sum
        of $n$ elements $\sum_{i = 0}^{n-1} x_i$.
    }\label{co:lst:sum}
\end{figure}

\textbf{Continuity analysis and optimization.} The robustness of programs
are very important to us.  In many cases, we wish our algorithms to be
free from discontinuity, \ie~a small change in the initial condition
would not result in an undesirably large jump in the outputs.  For this,
Chaudhuri~\etal~\cite{chaudhuri11} and Goubault~\etal~\cite{goubault13}
respectively propose methods to analyze the robustness of programs.  The
former approach formally proves whether an algorithm is ill-conditioned in
terms of the existence of discontinuity, whereas the latter statically analyze
programs to determine if round-off errors introduce significant discontinuous
behaviour.  To illustrate, consider an \iflit~branch,
``\lstinline[basicstyle=\tt]{if ($e$ > 0) $c_1$ else $c_2$}'', where $e$
is a floating-point expression.  When $e$ is positive and very close to
$0$ when evaluated in real arithmetic, the floating-point result of $e$
could be non-positive, due to the effects of the round-off errors.  In
these extraordinary cases, the $c_2$ branch may be executed instead of the
intended $c_1$.  These above new techniques could inspire us to implement the
optimization of discontinuous behaviour, such as the one shown in the example,
as another objective.

\textbf{Memory partitioning.} The experimental results in this work see a
diminishing performance return when loops are deeply unrolled, because of a
memory bottleneck.  As memory accesses saturate in loop execution, \ie~all
memory ports are working in 100\% utilization, it is unable to gain further
performance improvements.  Currently the tool stops exploring further loop
unrolling when this happens.  By automatically partition arrays upon hitting
such a memory bottleneck, further throughput improvements can be achieved.

\textbf{Other practical considerations.}  Finally, we may consider design
perspectives that could make the resulting tool much more usable.  For
instance, programs may still be optimized by not having any knowledge about the
input variables.  Herbie~\cite{panchekha15} makes no assumption about the input
space, and can nevertheless optimize arithmetic expressions, by splitting the
input space into regimes.


% \section{Tool Usage}
% \label{cc:sec:usage}

% \soap~is a source-to-source optimizer that specifically targets numerical
% program statements written in a subset of standard C99.  The tool supports
% arithmetic and Boolean expressions, assignment statements, \iflit{} statements,
% \whilelit{} loops and \forlit{} loops.  The numerical data types we allow are
% $\inttype$ and $\floattype$, as well as single- and multi-dimensional array
% types.

% The program below is an example usage of \soap~in a C program.  Note that it
% specifies the input values are respectively a two-dimensional array \verb|A|,
% where its elements are single-precision floating point values between 0 and 1,
% and an integer \verb|T| equals to $20$.  It also indicates the only output that
% we care from this code is the resultant \verb|A|.
% \begin{lstlisting}
  % #define N 1024
  % #pragma soap begin
  % #pragma soap in float A[N][N]=[0,1], int T=20
  % #pragma soap out A
  % for (int t = 0; t < T; t++)
    % for (int i = 1; i < N-1; i++)
      % for (int j = 1; j < N-1; j++)
        % A[i][j] = 0.2f * (A[i-1][j] +
          % A[i][j-1] + A[i][j] +
          % A[i][j+1] + A[i+1][j]);
  % #pragma soap end
% \end{lstlisting}

% \soap~is an open-source command-line utility, which only requires the user
% to provide a program written in C extended with the above \verb|#pragma|
% statements.  The Pareto optimal programs are all automatically generated by
% \soap, each is accompanied with our estimations of its latency and resource
% usage, and an analyzed bound on round-off errors.  These programs can then be
% given to \gls{vhls} to be synthesized into circuits.


\section{Final Remarks}
\label{cc:sec:final_remarks}

This thesis adapts existing techniques such as accuracy, latency and resource
usage analysis, and further introduces novel approaches, \eg~\gls{mir} and
efficient equivalence discovery, and delivers them in a unified framework.
The functional equivalence relaxation paradigm is relatively under-explored,
because these optimizations are often highlighted as \emph{unsafe} by the
\gls{hls} tools, as they cannot analyze the numerical implications of these
optimizations.  \Gls{hls} tools therefore have very limited optimization
options base on this particular concept.  With the constructive results
produced by this thesis, optimizations based on our concept can not only raise
performance measures, but also result in even \emph{safer} implementations
as we improve numerical accuracies.  The equivalence discovery algorithm in
tandem with \glspl{mir} could have great potential in compiler optimization
based on our concept.  Furthermore, since machine learning algorithms are
error-resilient~\cite{lesser11, kim09, holt91, zhu03}, the methods demonstrated
in this thesis have promising capabilities to improve the resource usage,
latency and accuracy of them.



\chapter{Numerical Program Optimization}
\label{chp:progopt}

% This paper introduces a new technique, and its associated open source tool to
% automatically perform source-to-source optimization of numerical programs,
% specifically targeting the trade-off between numerical accuracy and resource
% usage as a high-level synthesis flow for FPGA implementations.  We introduce
% a new intermediate representation, which we call metasemantic intermediate
% representation (MIR), to empower the abstraction and optimization of
% numerical programs.  We efficiently discover equivalent structures in
% MIRs by exploiting the rules of real arithmetic, such as associativity
% and distributivity, and rules that enable control flow restructuring, and
% produce Pareto frontiers of equivalent programs that trades off LUTs, DSPs
% and accuracy.  Additionally, we further broaden the Pareto frontier in our
% optimization flow to automatically explore the numerical implications of
% partial loop unrolling and loop splitting.  In real applications, our tool
% discovers a wide range of Pareto optimal options, and the most accurate one
% improves the accuracy of numerical programs by up to 65\%.

\section{Introduction}
\label{sec:introduction}

The IEEE 754 standard~\cite{ieee754} for floating-point computation is
ubiquitous in computing machines. In practice, it is often neglected that
floating-point computations almost always have roundoff errors. In fact,
associativity and distributivity properties which we consider to be fundamental
laws of real numbers no longer hold under floating-point arithmetic. This opens
the possibility of using these rules to generate an expression equivalent to
the original expression in real arithmetic, which could have better quality
than the original when evaluated in floating point computation.

By exploiting rules of equivalence in arithmetic, such as associativity $(a
+ b) + c \equiv a + (b + c)$ and distributivity $(a + b) \times c \equiv a
\times c + b \times c$, it is possible to automatically generate different
implementations of the same arithmetic expression. We optimize the structures
of arithmetic expressions in terms of the following two quality metrics
relevant to FPGA implementation: the resource usage when synthesized into
circuits, and a bound on roundoff errors when evaluated. Our goal is the joint
minimization of these two quality metrics. This optimization process provides
a Pareto optimal set of implementations. For example, our tool discovered that
with single precision floating-point representation, if $a \in [0.1, 0.2]$,
then the expression ${(a + 1)}^2$ uses fewest resources when implemented in the
form $(a + 1) \times (a + 1)$ but most accurate when expanded into $(((a \times
a) + a) + a) + 1$. However it turns out that a third alternative, $((1 + a)
+ a) + (a \times a)$, is never desirable because it is neither more accurate
nor uses fewer resources than the other two possible structures. Our aim is to
automatically detect and utilize such information to optimize the structure of
expressions.

A na{\"\i}ve implementation of equivalent expression finding would be to
explore all possible equivalent expressions to find optimal choices, However
this would result in combinatorial explosion~\cite{ioualalen}. For instance,
in worst case, the parsing of a simple summation of $n$ variables could
result in $(2n - 1)!! = 1\times3\times5\times\cdots\times(2n - 1)$ distinct
expressions~\cite{ioualalen, mouilleron}. This is further complicated by
distributivity as ${(a + b)}^k$ could expand into an expression with a
summation of $2^k$ terms each with $k - 1$ multiplications. Therefore, usually
it would be infeasible to generate a complete set of equivalent expressions
using the rules of equivalence, since an expression with a moderate number of
terms will have a very large number of equivalent expressions. The methodology
explained in this paper makes use of formal semantics as well as abstract
interpretation~\cite{cousot77} to significantly reduce the space and time
requirements and produce a subset of the Pareto frontier.

In order to further increase the options available in the Pareto frontier,
we introduce freedom in choosing mantissa widths for the evaluation of the
expressions. Generally as the precision of the evaluation increases, the
utilization of resources increases for the same expression. This gives
flexibility in the trade-off between resource usage and precision. Our
approach and its associated tool, SOAP, allow high-level synthesis flows to
automatically determine whether it is a better choice to rewrite an expression,
or change its precision in order to meet optimization goals.

The three contributions of this paper are:
\begin{enumerate}
    \item Efficient methods for discovering equivalent structures of
    arithmetic expressions.
    \item A semantics-based program analysis that allows joint reasoning about
    the resource usage and safe ranges of values and errors in floating-point
    computation of arithmetic expressions.
    \item A tool which produces RTL implementations on the area-accuracy
    trade-off curve derived from structural optimization.
\end{enumerate}

This paper is structured as follows. Section~\ref{sec:related_work}
discusses related existing work in high-level synthesis and the
optimization of arithmetic expressions. We explain the basic
concepts of semantics with abstract interpretation used in this
paper in Section~\ref{sec:abstract_interpretation}. Using this,
Section~\ref{sec:semantics} explains the concrete and abstract semantics
for finding equivalent structure in arithmetic expressions, as well as
the analysis of their resource usage estimates and bounds of errors.
Section~\ref{sec:implementation} gives an overview of the implementation
details in our tool. Then we discuss the results of optimized example
expressions in Section~\ref{sec:results} and end with concluding remarks in
Section~\ref{sec:conclusion}.

\section{Related Work}
\label{sec:related_work}

High-level synthesis (HLS) is the process of compiling a high-level
representation of an application (usually in C, C++ or MATLAB) into
register-transfer-level (RTL) implementation for FPGA~\cite{coussy, gajski}.
HLS tools enable us to work in a high-level language, as opposed to facing
labor-intensive tasks such as optimizing timing, designing control logic in
the RTL implementation. This allows application designers to instead focus on
the algorithmic and functional aspects of their implementation~\cite{coussy}.
Another advantage of using HLS over traditional RTL tools is that a C
description is smaller than a traditional RTL description by a factor
of 10~\cite{coussy, bdti}, which means HLS tools are in general more
productive and less error-prone to work with. HLS tools benefit us in their
ability to automatically search the design space with a reasonable design
cost~\cite{bdti}, explore a large number of trade-offs between performance,
cost and power~\cite{mcfarland}, which is generally much more difficult to
achieve in RTL tools. HLS has received a resurgence of interest recently,
particularly in the FPGA community. Xilinx now incorporates a sophisticated
HLS flow into its Vivado design suite~\cite{vivado_hls} and the open-source
HLS tool, LegUp~\cite{legup}, is gaining significant traction in the research
community.

However, in both commercial and academic HLS tools, there is very little
support for static analysis of numerical algorithms. LLVM-based HLS
tools such as Vivado HLS and LegUp usually have some traditional static
analysis-based optimization passes such as constant propagation, alias
analysis, bitwidth reduction or even expression tree balancing to reduce
latency for numerical algorithms. There are also academic tools that
perform precision-performance trade-off by optimizing word-lengths of data
paths~\cite{constantinides}. However there are currently no HLS tools that
perform the trade-off optimization between accuracy and resource usage by
varying the \emph{structure} of arithmetic expressions.

Even in the software community, there are only a few existing techniques
for optimizing expressions by transformation, none of which consider
accuracy/run-time trade-offs. Darulova~\etal~\cite{darulova} employ a
metaheuristic technique. They use genetic programming to evolve the structure
of arithmetic expressions into more accurate forms. However there are several
disadvantages with metaheuristics, such as convergence can only be proved
empirically and scalability is difficult to control because there is no
definitive method to decide how long the algorithm must run until it reaches a
satisfactory goal. Hosangadi~\etal~\cite{hosangadi} propose an algorithm for
the factorization of polynomials to reduce addition and multiplication counts,
but this method is only suitable for factorization and it is not possible to
choose different optimization levels. Peymandoust~\etal~\cite{peymandoust}
present an approach that only deals with the factorization of polynomials
in HLS using Gr\"obner bases. The shortcomings of this are its dependence
on a set of library expressions~\cite{hosangadi} and the high computational
complexity of Gr\"obner bases. The method proposed by Martel~\cite{martel07}
is based on operational semantics with abstract interpretation, but even
their depth limited strategy is, in practice, at least exponentially complex.
Finally Ioualalen~\etal~\cite{ioualalen} introduce the abstract interpretation
of equivalent expressions, and creates a polynomially sized structure to
represent an exponential number of equivalent expressions related by rules of
equivalence. However it restricts itself to only a handful of these rules to
avoid combinatorial explosion of the structure and there are no options for
tuning its optimization level.

Since none of these above captures the optimization of both accuracy and
performance by restructuring arithmetic expressions, we base ourselves on the
software work of Martel~\cite{martel07}, but extend this work in the following
ways. Firstly, we develop new hardware-appropriate semantics to analyze not
only accuracy but also resource usage, seamlessly taking into account common
subexpression elimination. Secondly, because we consider both resource usage
and accuracy, we develop a novel multi-objective optimization approach to
scalably construct the Pareto frontier in a hierarchical manner, allowing fast
design exploration. Thirdly, equivalence finding is guided by prior knowledge
on the bounds of the expression variables, as well as local Pareto frontiers of
subexpressions while it is optimizing expression trees in a bottom-up approach,
which allows us to reduce the complexity of finding equivalent expressions
without sacrificing our ability to optimize expressions.

We begin with an introduction to formal semantics in the following section,
later in Section~\ref{sec:semantics}, we explain our approach by extending the
semantics to reason about errors, resource usage and equivalent expressions.

\section{Syntax Definition}
\label{po:sec:syntax_definition}

Before we discuss program transform, we first look at the syntax definition
used to write numerical programs.  Our program transformation optimizes
\numimp{} programs.  In this section, we formally introduce \numimp, a simple
imperative language which is a subset of C that supports arithmetic and Boolean
expressions, conditional branches, as well as \texttt{while} loops.  Our
language allows numerical data types $\inttype$ and $\floattype$, respectively
stand for integer and floating-point types.

We define $\aexprset, \bexprset$ as the set of arithmetic and Boolean
expressions respectively, and $\stmtset$ denotes the set of program statements.
We then have following syntax definition for expressions and \numimp{}
programs, written in the Backus-Naur Form~\cite{knuth64}:
\newcommand{\syndef}{\ensuremath\mathbin{::=}}%
\newcommand{\synor}{\ensuremath\mathbin{\mid}}%
\begin{equation}
    \begin{aligned}
        a \syndef {} &
            n \synor
            x \synor
            a_1 \odot a_2,
        \quad b \syndef x < a, \\
        s \syndef {} &
            x~\texttt{=}~a \synor
            s_1 \semicolon s_2 \synor
            \mathtt{if}~(b)~\{ s_1 \}~\mathtt{else}~\{ s_2 \} \synor
            \whilelit~(b)~\{s\}
    \end{aligned}
    \label{po:eq:program_syntax}
\end{equation}
We define $\odot \in \left\{ +, -, \times, / \right\}$ to be the arithmetic
operators, $n$ is a numerical constant of type either \inttype{} or \floattype;
$x \in \varset$ is a variable; $a, a_1, a_2 \in \aexprset$ are arithmetic
expressions; $b$ ranges over Boolean expressions, $\bexprset$; and similarly,
$s, s_1, s_2 \in \stmtset$ are program statements.  In our formal definition,
for the purpose of simplicity, we restrict the Boolean expressions to those
of the form $x < a$, where $x$ is a variable and $a$ is an expression;
more complex Boolean expressions are included trivially in our actual
implementation.  Although \forlit~loop is not explicitly defined in the above
syntax definition, it can be trivially derived from a \whilelit~loop.

Furthermore, we introduce the ``\verb|#pragma| \verb|soap begin|'' and
``\verb|#pragma| \verb|soap end|'' directives to delimit the code fragment
to be optimized.  We can also use ``\verb|#pragma| \verb|soap in|'' and
``\verb|#pragma| \verb|soap out|'' to provide input ranges and to declare
output variables, respectively.

As a simple example, the program in Figure~\ref{po:fig:syntax_example} computes
an approximate value of ${\pi^2 a}/6$.  It has two inputs $a$, a floating point
value between 0 and 1, and $n$, an integer value between 10 and 20, which
determines the number of iterations for the loop, and a return variable $y$.

\begin{figure}[ht]
    \begin{lstlisting}
    #pragma soap in \
        float a = [0.0, 1.0], int n = [10, 20]
    #pragma soap out float y
    x = 0;
    y = 0.0;
    while (x < n) {
        x = x + 1;
        y = y + a / (x * x);
    }
    \end{lstlisting}
    \caption{A simple program written with our syntax definition.}
    \label{po:fig:syntax_example}
\end{figure}

Despite the simplicity of our syntax, it includes all the features of
a full programming language rather than an expression language used in
Chapter~\ref{chp:stropt}.  We will add support for arrays and matrices in
Chapter~\ref{chp:latopt}, and show that this can be added with little changes
to our method.

\section{Program to Metasemantic Intermediate Representation}
\label{po:sec:program_to_mir}

The first step of our approach is to analyze the program return value
into a metasemantic intermediate representation (MIR).  This procedure is
called \emph{metasemantic analysis} (MA).  The MA abstracts away irrelevant
information, and preserves the essence of program execution.  Details such
as temporary variables and the ordering of program statements are discarded,
whereas the abstraction still retains dataflow dependencies and keeps only
computations that contribute to the final results.

We work with the MIR as an abstraction of the program because the discovery of
equivalent structures can be much simplified.  For instance, the program ``$x
\assign 1 \semicolon y \assign 2$'' is the same as ``$y \assign 2 \semicolon x
\assign 1$'' because interleaving of non-dependent statements does not change
program semantics.  If we were to base our transformations on the program
syntax, we will need to enable this kind of equivalent relation even though
it has zero impact on our optimization with respect to resource usage and
accuracy.  A much simpler intermediate representation means that we can explore
a much smaller search space.

Our method analyzes a program by recursively dividing the program into
smaller parts, where each part can be separately analyzed into an MIR and
composed together to form a single MIR\@.  An MIR is a mathematical object
that associates each program variable with a semantic expression.  A semantic
expression is an arithmetic expression, but with additional syntactic features
to support \iflit{} statements and \whilelit{} loops.  We represent semantic
expressions with \emph{directed acyclic graphs} (DAGs) that share common
structures and define $\sexprset$ as the set of semantic expressions, and
$\mirset$ as the set of MIRs.  Because an MIR pairs a variable with an
expression, we can view it as a function $\varset \to \sexprset$ that maps
a variable into a semantic expression.  For instance, $\mu(x)$ returns the
associated expression of the variable $x \in \varset$ for the MIR $\mu \in
\mirset$.  For each variable, its semantic expression in itself provides a
complete picture of how computations can lead to the resulting value of the
variable.  In the rest of this section, we progressively explain how each type
of program statement defined in~\eqref{po:eq:program_syntax} is analyzed into
an MIR\@.

\subsection{Skip}

Because $\skipstmt$ has no effect on program states, the MIR of $\skipstmt$
also does not alter the program state.  Its MIR is defined as $[x \mapsto
x]_{x \in \varset}$, the subscript $x \in \varset$ denotes that the MIR is
constructed by collecting for each program variable $x$, the mapping shown in
the bracket, \ie~the paring of the variable $x$ with the expression $x$.  The
subscript $x \in \varset$ means that for each $x$, the MIR pairs it with $x \in
\sexprset$, which is a semantic expression containing just the variable $x$
itself.  We will repeatedly make use of the notation of constructing an MIR,
$[x \mapsto e(x)]_{p(x)}$ in the following sections, where $x$ is a variable
and $p(x)$ is a predicate on $x$, the MIR is constructed by collecting for all
$x$ the mapping from $x$ to the expression $e(x)$, when $p(x)$ is true.

\subsection{Assignment}

An assignment statement is in the form of ``$x \assign e$'', where $x \in Var$
is a program variable and $e \in \aexprset$ is an arithmetic expression.  The
metasemantic analysis of it produces an MIR as follows, where the operator
$\join$ merges the two MIRs together:
\begin{equation}
    \left[
        y \mapsto \left\{
            \begin{aligned}
                & e && \text{if~} y = x \\
                & y && \text{if~} y \neq x
            \end{aligned}
        \right.
    \right]_{y \in \varset}
    \label{po:eq:mir_assign}
\end{equation}
The MIR in~\eqref{po:eq:mir_assign} signifies for a variable $y \in \varset$,
if $y$ is $x$, then we assign the expression $e$ to the variable $x$, where
$e$ is a semantic expression represented with a DAG\@.  The DAG shares all
common subexpressions in $e$.  For instance, an expression written as $(x + 1)
\times (x + 1)$ shares the subexpression node $x + 1$ by reusing the node in
the DAG\@.  For each other program variable $y \in \varset$, where $y \neq x$,
$y$ is associated with a semantic expression $y \in \sexprset$, representing
that the MIR does not the value of all program variables except $x$, because
only $x$ is updated in the statement.

For example, for a program with two variables $x$ and $y$, analyzing the
statement ``$y \assign x \times 2$'' produces the following MIR\@, note the
variable $x$ is shared between two semantic expressions:
\mirfig{mir_assign_2}

\subsection{Sequential statements}
\label{po:sub:sequential_statements}

A sequential statement, ``$s_1 \semicolon s_2$'' is formed by joining together
$s_1$ and $s_2$, where $s_1, s_2 \in \stmtset$ are statements.  It signifies
that $s_1$ and $s_2$ are executed in sequence.  Therefore, it is necessary to
\emph{append} the effect of executing $s_1$ to that of $s_2$, to arrive at
the full MIR of ``$s_1 \semicolon s_2$''.  This concept can be realized by
defining a new operator $\expand$, the substitution operator, such that the MIR
of ``$s_1 \semicolon s_2$'' is equal to $\mu_2 \expand \mu_1$, where $\mu_1$
and $\mu_2$ are the MIRs of $s_1$ and $s_2$ respectively.  The resulting MIR
of $\mu_2 \expand \mu_1$ is constructed by substituting, for every expression
$e \in \sexprset$ in $\mu_2$, each variable $x$ in $e$ with $\mu_1(x)$, which
is the associated expression of $x$ in $\mu_1$.  Furthermore, the operator
allows the format $e \expand \mu$, where $e \in \sexprset$ is called the target
expression and $\mu \in \mirset$ is the source MIR, to mean the variables in
$e$ is substituted with $\mu$ using the substitution strategy above.

We illustrate this by using a simple example program $p = {}$``$x
\assign x + 1 \semicolon y \assign x \times 2$''.  Using the MIR of
assignments, the MIR of $p$ can be derived, as shown in the right-hand
side of equation~\eqref{po:eq:mir_seq_1}.  By substituting the variables
with corresponding expressions, we arrive at the left-hand side of
\eqref{po:eq:mir_seq_1}:
\mirfig{mir_seq_1}

\subsection{Conditional Branches}

Conditional branches, or \iflit~statements, are represented with ``$\iflit~
b~ \thenlit~ s_1~ \elselit~ s_2$''.  Here $b \in \bexprset$ is a Boolean
expression, and $s_1, s_2 \in \stmtset$ are respectively the true- and
false-branches.  Our analysis of \iflit~statements is slightly more complex,
as we start to consider control flows.  The analysis is carried out in two
steps.  The first step is to compute recursively, the MIRs $\mu_1, \mu_2
\in \mirset$ of the respective true- and false-branches, namely, $s_1$ and
$s_2$.  We introduce the conditional node ``$?$'', which is derived from
C syntax, to signify conditional branches in expressions.  The left-most,
middle and right-most children of this node are respectively the Boolean
expression, the true- and false-expressions.  Then the second step is to
compute a new MIR, where each program variable $x \in \varset$ is associated
with a conditional node with three children, the Boolean expression $b$,
$\mu_1(x)$ and $\mu_2(x)$, that is, $[x \mapsto \select{b}{\mu_1}{\mu_2}]_{x
\in \varset}$.  As an example we consider the program ``$\iflit~(x < 0)$
$\thenlit~(y \assign x \times 2)$ $\elselit~\skipstmt$'', where the set of
program variables is $\{x, y\}$.  Its MIR is shown in the left-hand side
of~\eqref{po:eq:mir_if_2}.  Because both true- and false-expressions of $x$
are the same, regardless of the truth value of $x < 0$, the two expressions
will evaluated to the same value.  In our analysis we further simplify
the expression of $x$, and the resulting MIR is in the right-hand side
of~\eqref{po:eq:mir_if_2}.
\mirfig{mir_if_2}

The traditional approach of program abstraction uses control and data flow
graphs (CDFGs)~\cite{namballa04}, which preserves the ordering of sequential
statements, uses a one-to-one mapping from assignment statements to assignment
nodes, storage nodes are used to store the result of assignments, \ie~it allows
nodes to act as a memory to store values, and finally, uses cycles in graphs
to represent program loops.  In contrast, our MIRs, from our analysis point of
view, use no local storage, discard unnecessary intermediate statements, and
most importantly, we treat control structures as operators in expressions, in
the same way as arithmetic computations.  In comparison with CDFGs, these above
facts make MIRs a more suitable candidate for program transformations.

\subsection{``While'' Loops}

For \whilelit~loops, ``$\whilelit\, b\, \dolit\, s$'', where $b \in \bexprset$
and $s \in \stmtset$, we begin by proposing a new operator for semantic
expressions, ``$\fix$'', which we call the \emph{fixpoint} operator.  It is
used to stand for \whilelit~loop structures in semantic expressions.  It has
three child nodes, the Boolean expression $b$, the loop body represented by an
MIR, and the loop exit variable.  The loop body MIR can be obtained with our MA
of the loop body, and the loop exit variable denotes which variable we use on
loop exit as the evaluated value of the fixpoint expression.  We let $\mu_s$ to
be the MIR of the loop body $s$, and derive the MIR of ``$\whilelit\, b\, s$'',
by computing the fixpoint expression for each variable:
\begin{equation}
    \left[
        x \mapsto \left\{
            \begin{aligned}
                & \begin{aligned}
                    \includegraphics[scale=\mirfigscale]{fix_expr}
                \end{aligned} && \text{if~} x \in \varfunc{\mu_s} \\
                & \quad ~ ~ x && \text{otherwise}
            \end{aligned}
        \right.
    \right]_{x \in \varset}
    \label{po:eq:mir_while}
\end{equation}
Here, $\varfunc{\mu_s}$ computes the set of variables that is assigned in the
loop body $\mu_s$.  If a program variable $x$ is in the set $\varfunc{\mu_s}$,
it is paired with its fixpoint expression $\fixpoint{b, \mu_s, x}$; otherwise
the variable $x$ is not updated in the loop, and the loop has no effect on its
value, therefore it is paired with an expression $x$.

\section{Transformations}
\label{po:sec:transformations}

The next step is to use the analyses of accuracy and resource usage of
equivalent structures in MIRs to efficiently discover optimized equivalent
MIRs.  We start by providing an overview of how the accuracy of simple
arithmetic expressions are analyzed with the \soap{} framework, and since it
only allows arithmetic expressions with simple operators $\{+, -, \times\}$, we
explain how it can be extend fully to analyze MIRs and semantic expressions.
Then this section is followed by a detailed explanation of how resources
in MIRs can be shared and how to analyze the resource utilization of MIRs.
Finally, equivalent relations are defined for the discovery of equivalent
structures, and we guide this process efficiently with our analyses of accuracy
and resource usage.

\section{Accuracy Analysis}
\label{po:sec:accuracy_analysis}

Because we make use of accuracy analysis to navigate the Pareto optimization
of program candidates, we start by providing an overview of how the accuracy
of simple arithmetic expressions are analyzed with the \soap{} framework, and
since it only allows arithmetic expressions with simple operators $\{+, -,
\times\}$, we explain how it can be extend fully to analyze \glspl{mir} and
semantic expressions.

In a typical program execution, values of variables, typically integers
$\integerset$ and floating-point values $\floatset$, are modified according
to the effect of the program statements, and they are propagated through
arithmetic operators from the beginning to the end of the program.  In
Section~\ref{so:sec:accuracy} of Chapter~\ref{chp:stropt}, an alternative
semantics, are proposed to instead propagate ranges of values together with
the associated round-off error bounds, \ie~the value-error bound $(v^\sharp,
e^\sharp) \in \errorset$, in order to analyze the accuracy of floating-point
numerical programs.  In this section, this technique is further generalized to
numerical programs.

Initially, we formalize the analyzed program values of a program as an abstract
program state using the domain $\errordom = \varset \to \errorset$, and a
$\sigma^\sharp \in \errordom$ maps each variable $\varx$ to their associated
value-error bound $\sigma^\sharp(\varx) \in \errorset$.

For instance, we assume an abstract state $\sigma^\sharp_0 \in \errordom$ which
provides the input values of $\vara$ and $\varb$ to a program, where:
\begin{equation}
    \sigma^\sharp_0 = \left[
        \vara \mapsto \left(\interval{0}{1}, e^\sharp_0\right),
        \varb \mapsto \left(\interval{1}{2}, e^\sharp_0\right) \right].
    \label{so:eq:example_state}
\end{equation}
This means that initially $\vara$ and $\varb$ are floating-point values bounded
by $[0, 1]$ and $[1, 2]$ respectively, and the error interval $e^\sharp_0 =
[0, 0]$ denotes the absence of round-off errors.

In Section~\ref{so:sec:accuracy} of Chapter~\ref{chp:stropt} we introduced the
function $\error: \aexprset \to \errorset$ to evaluate the bounds on the result
and its round-off error of computing an expression.  Here we introduce a new
function $\exprerrorop: \sexprset \to \errordom \to \errorset$ which further
accepts initial bounds on the values and errors of variables.  The formula
$\exprerrorfunc{e}{\sigma^\sharp}$, where $e \in \sexprset$ and $\sigma^\sharp
\in \errordom$, is used to denote the accuracy analysis of the expression $e$
with the input state $\sigma^\sharp$.

Then in single-precision, the error analysis of $\vara + \varb$, given the
initial bounds $\sigma^\sharp_0$ in~\eqref{so:eq:example_state}, produces the
following result:
\begin{equation}
    \exprerrorfunc{\vara + \varb}{\sigma^\sharp_0} = \left(
        \interval{1}{3},
        \interval{-1.19209304 \times 10^{-7}}{1.19209304 \times 10^{-7}}
    \right).
\end{equation}
This means that the result of this computation is in the range of $v^\sharp =
[1, 3]$, and the round-off error induced by this computation is bounded by the
interval $e^\sharp$.

The method outlined in Chapter~\ref{chp:stropt} to analyze the accuracy of
arithmetic expressions supports only addition, subtraction, multiplication and
division.  In this section, we explain in detail how it is extended to support
\glspl{mir}, and our additional operators in semantic expressions, \ie~the
composition, ternary conditional and fixpoint operators.

\subsection{MIR}

In the same way that an expression can be analyzed for its accuracy, an
\gls{mir}, which is a mapping of variables to semantic expressions, can
be analyzed by performing the $\exprerrorop$ analysis for each of its
expressions.  For instance, the accuracy of a \gls{mir}\@:
\begin{equation}
    \mu_0 = \left[
        \vara \mapsto \vara + \varb,
        \varb \mapsto \vara \times 0.5
    \right],
\end{equation}
with an input state:
\begin{equation}
    \sigma^\sharp_0 = \left[
        \vara \mapsto \left(
            \interval{0}{1}, \interval{0}{0}
        \right),
        \varb \mapsto \left(
            \interval{1}{2}, \interval{0}{0}
        \right)
    \right],
\end{equation}
can be analyzed as follows.

First we analyze the individual expressions $\mu_0(\vara) = \vara + \varb$
and $\mu_0(\varb) = \vara \times 0.5$, which produce respectively the results
below:
\begin{align}
    \left(
        v^\sharp_\vara, e^\sharp_\vara
    \right) &= \left(
        \interval{1}{3},
        \interval{-1.19209304 \times 10^{-7}}{1.19209304 \times 10^{-7}}
    \right), \\
    \left(
        v^\sharp_\varb, e^\sharp_\varb
    \right) &= \left(
        \interval{0}{0.5},
        \interval{-2.98023259 \times 10^{-8}}{2.98023259 \times 10^{-8}}
    \right).
\end{align}

Then the analyzed results are collected into an abstract state assigning the
value-error bounds to their corresponding variables, that is:
\begin{equation}
    \left[
        \vara \mapsto (v^\sharp_\vara, e^\sharp_\vara),
        \varb \mapsto (v^\sharp_\varb, e^\sharp_\varb)
    \right].
\end{equation}

To generalize, we can formally define a function,
$\mirerrorfunc{\mu}{\sigma^\sharp}$, to perform the above analysis,
which takes as inputs the \gls{mir} $\mu \in \mirset$ and an abstract
input state $\sigma^\sharp \in \errordom$.  It computes a new state
${\sigma^\sharp}^\prime$, where for each variable $\varx \in \varset$,
${\sigma^\sharp}^\prime(\varx)$ is the analyzed value-error range of the
expression $\mu(\varx)$.  The error analysis of a \gls{mir} is therefore:
\begin{equation}
    \mirerrorfunc{\mu}{\sigma^\sharp} = {\left[
        \varx \mapsto \exprerrorfunc{\mu(\varx)}{\sigma^\sharp}
    \right]}_{\varx \in \varfunc{\mu}}.
    \label{po:eq:mir_accuracy}
\end{equation}
The notation ${[\varx \mapsto
\exprerrorfunc{\mu(\varx)}{\sigma^\sharp}]}_{\varx \in \varfunc{\mu}}$ means
that the mapping is constructed by collecting for each variable $\varx \in
\varfunc\mu$, the pairing of $\varx$ with the analyzed value-error bound of the
semantic expression $\mu(\varx)$.

\subsection{Composition Operator}

The analysis of an expression $e \expand \mu$, where $e \in \sexprset$ and $\mu
\in \mirset$ is carried out in two steps.  Initially, given an input state
$\sigma^\sharp$, $\mu$ is analyzed using~\eqref{po:eq:mir_accuracy}, and we
write ${\sigma^\sharp}^\prime$ as the analyzed state.  Then the expression $e$
is analyzed for its accuracy as usual, using ${\sigma^\sharp}^\prime$ as the
input state.  Equivalently, this procedure can be defined as:
\begin{equation}
    \exprerrorfunc{e \expand \mu}{\sigma^\sharp}
    = \exprerrorfunc{e}{\left( \mirerrorfunc{\mu}{\sigma^\sharp} \right)}.
\end{equation}

\subsection{Ternary Conditional Operator}

A conditional expression is written as $\select{b}{e_1}{e_2}$, where $b
\in \bexprset$ and $e_1, e_2 \in \sexprset$.  The truth value of Boolean
expression $b$ determines whether $e_1$ or $e_2$ is evaluated to be the
resulting value of the expression.  Correspondingly, in our accuracy analysis,
we impose a constraint defined by the Boolean expression $b$ on the value
ranges of input variables, such that $e_1$ is evaluated with the ranges of
values satisfying the constraint, while $e_2$ is computed with ranges that
violate the constraint.

For example, we analyze an expression $\select{(\vara < 0)}{(\vara -
0.1)}{\vara}$ in single-precision.  Initially, we assume the program state
consists of a variable $\vara$, which is a floating-point value that has no
associated round-off error and is bounded by $\interval{-1}{10}$, that is:
\begin{equation}
    \sigma^\sharp_0 = \left[
        \vara \mapsto \left(
            \interval{-1}{10},
            \interval{0}{0}
        \right)
    \right].
\end{equation}
We consider two cases, when the condition $\vara < 0$ is respectively true and
false.  For $\vara < 0$ to be true, $\vara$ must be in the range of $v^\sharp_b
= [-1, 0^{-}]$, where $0^{-}$ is the greatest single-precision floating-point
value less than 0, because $\vara$ must be strictly smaller than $0$.  Then we
restrict the range of $\vara$ to $v^\sharp_b$, so $\vara - 0.1$ is analyzed to
be $(v^\sharp_1, e^\sharp_1)$, where:
\begin{equation}
    \left( v^\sharp_1, e^\sharp_1 \right) = \left(
        \interval{-1.10000002}{-0.100000001},
        \interval{-5.81145372\times10^{-8}}{6.10947666\times10^{-8}}
    \right).
\end{equation}
Similarly, when $\vara < 0$ is false, we restrict the bound on $\vara$
with $v^\sharp_{\neg b} = \interval{0}{10}$ and analysis of the expression
$\vara$ simply gives $v^\sharp_2 = \interval{0}{10}$ and no round-off error,
$e^\sharp_2 = \interval{0}{0}$.  Finally, the analyzed value and error ranges
for the expression can be obtained by joining these two cases together,
by respectively evaluating $v^\sharp_1 \join v^\sharp_2$, which produces
$v^\sharp = \interval{-1.10000002}{10}$, and the error bounds $e^\sharp_1 \join
e^\sharp_2$.  The final result is therefore:
\begin{equation}
    \begin{aligned}
        & \exprerrorfunc{
            \select{\left(\vara < 0\right)}{\left(\vara + 0.1\right)}{\vara}
        }{\sigma^\sharp_0} \\
        {}={} & \left(
            \interval{-1.10000002}{10},
            \interval{-5.81145372e-08}{6.10947666e-08}
        \right).
    \end{aligned}
\end{equation}

Our analysis is based on interval arithmetic which is very efficient but it
sacrifices accuracy by computing an over-approximation of the exact results.
For instance, the above analysis cannot capture the fact that all evaluated
result $v$ satisfies $-1.1 \leq v < 0$.  The majority of such information
losses occur because of a loose bound on the analyzed floating-point results.
In contrast, joining two error bounds generally produce a precise bound,
because error bounds are often much less correlated, and they often overlap
as there is a high chance that there exists an arithmetic computation which
produces an exact floating-point outcome.  In Section~\ref{po:sec:results},
empirical results show that despite the accuracy analysis potentially produces
loose bounds, it can still be used effectively as an indicator of the round-off
errors in actual executions.

We now provide a formal definition of the above example analysis.  We
use the notation $\sigma^\sharp|_b$ and $\sigma^\sharp|_{\neg b}$, where
$\sigma^\sharp \in \errordom$ is the program state, and $b \in \bexprset$ is
the Boolean expression, to respectively mean the program state $\sigma^\sharp$
is constrained by either $b$ being true or false.  Therefore, the following
formula is used to perform the accuracy analysis on a conditional expression:
\begin{equation}
    \exprerrorfunc{\ternarymir{$\qop$}{$b$}{$e_1$}{$e_2$}}{\sigma^\sharp}
    =
    \left(\exprerrorfunc{e_1}{\sigma^\sharp|_b}\right) \join
    \left(\exprerrorfunc{e_2}{\sigma^\sharp|_{\neg b}}\right).
\end{equation}

\subsection{Fixpoint Operator}
\label{po:sub:fixpoint}

An expression with a fixpoint operator, \fixexprmir, has three child nodes,
the Boolean expression $b \in \bexprset$, the loop body represented with
a \gls{mir} $\mu_s \in \mirset$, and the return variable $x \in \varset$.
Similar to executing a \whilelit~loop, evaluating the expression is to
iteratively evaluate $b$ for its truth value, if $b$ is true, then the loop
\gls{mir} $\mu_s$ is used to update the program state for the next iteration
and we repeat the process and iterate until $b$ is evaluated to false.

Before we explain how a fixpoint expression can be analyzed for its accuracy,
we introduce the concept of \gls{li}.  In our context, a \gls{li} of a
\whilelit~loop is a set of bounds on loop variables that holds invariantly on
entry to each loop iteration.  In Section~\ref{bg:sec:abstract_interpretation}
of Chapter~\ref{chp:background}, we explain how the \gls{li} of a simple
program loop can be computed, here we further extend this concept to general
programs expressed in \glspl{mir}.

For instance, we consider the \verb|basel| example in
Figure~\ref{po:lst:syntax_example}.  If our input is $n = 10$, then the
\gls{li} on the variable $x$ is that its value is an integer, and is bounded
by $[0, 9]$ on loop entry, whereas on loop exit, $x$ is always equal to $10$.
The reason for inferring the \gls{li} is as follows.  Since we optimize the
fixpoint expression's child nodes in a bottom-up hierarchy, the optimization of
$\mu_s$ precedes \fixexprmir~itself.  Hence we use the \gls{li} as the input
state to optimize $\mu_s$, as the \gls{li} encompasses all possible program
states the loop body $\mu_s$ will encounter when executed.

Our accuracy analysis of \fixexprmir~follows the above pattern.
Initially, we start with an input state $\sigma^\sharp_0$.  In the first
iteration $k = 0$, $\sigma^\sharp_k = \sigma^\sharp_0$ is split into two
disjoint parts, namely, $\sigma^\sharp_0|_b$ and $\sigma^\sharp_0|_{\neg b}$,
they respectively satisfies and violates the Boolean constraint $b$.  The state
$\sigma^\sharp_0|_b$ represents all possible program states that enters the
loop $\mu_s$, so $\sigma^\sharp_1 = \mirerrorfunc{\mu_s}{\sigma^\sharp_0|_b}$
captures all possible program states after the loop body.  This procedure
is repeated for iterations $k = 1, 2, 3, \ldots$, until a certain iteration
$n$, where $\sigma^\sharp_n = \sigma^\sharp_{n-1}$.  Hence, we can obtain the
\gls{li} by computing $\sigma^\sharp_0|_b \join \sigma^\sharp_1|_b \join \cdots
\join \sigma^\sharp_n|_b$, and the loop exit states with its counterpart,
\ie~$\sigma^\sharp_0|_{\neg b} \join \sigma^\sharp_1|_{\neg b} \join \cdots
\join \sigma^\sharp_n|_{\neg b}$.  The meaning of $\join$ operator on states is
similar to joining intervals and value-error bounds, which is defined to join
the two value-error bounds in respective states for each variable, which is
defined as follows, where $\sigma^\sharp_a, \sigma^\sharp_b \in \errordom$:
\begin{equation}
    \sigma^\sharp_a \join \sigma^\sharp_b =
        {%
            [ x \mapsto \sigma^\sharp_a(x) \join \sigma^\sharp_b(x) ]
        }_{x \in \varset}.
\end{equation}

Alternatively, we can compute the \gls{li} as the \gls{lfp} of $g: \errordom
\to \errordom$, where:
\begin{equation}
    g(y) = \mirerrorfunc{\mu_s}{
        \left.\left( y \join \sigma^\sharp \right)\right|_b
    }.
\end{equation}

This \gls{lfp} above can be computed with the algorithm in
Figure~\ref{po:alg:fix}.  The value $\bot$ indicates an empty or
unreachable state, and for any state $\sigma^\sharp \in \errordom$, we
have $\bot \join \sigma^\sharp = \sigma^\sharp$.  The return values
$\loopinvar$ and $\loopexit$ are respectively the \gls{li} and the result of
$\exprerrorfunc{\fixexprmir}{\sigma^\sharp}$.  In the rest of this chapter, we
use $\mathsf{E}_\mathsf{s}^\mathrm{LI} \left[\fixexprmir\right]\sigma^\sharp$
to signify the former return value $\loopinvar$.
\begin{figure}[ht]
    \centering
    \newcommand{\statett}{\ensuremath\sigma^\sharp_\truelit}
    \newcommand{\stateff}{\ensuremath\sigma^\sharp_\falselit}
    \begin{algorithmic}
        \singlespacing%
        \Function{FixpointAccuracyAnalysis}{
                \protect\fixexprmir, $\sigma^\sharp$}
            \State{%
                $\sigma^\sharp_0 \gets \sigma^\sharp$;\,
                $\loopinvar \gets \bot$;\,
                $\loopexit \gets \bot$;\,
                $k \gets 0$
            }
            \Loop%
                \State{%
                    $\statett \gets \sigma^\sharp_k|_b$
                }
                \State{%
                    $\stateff \gets \sigma^\sharp_k|_{\neg b}$
                }
                \State{%
                    $\loopinvar \gets \loopinvar \join \statett$
                }
                \State{%
                    $\loopexit \gets \loopexit \join \stateff$
                }
                \State{%
                    $\sigma^\sharp_{k + 1} \gets
                    \mirerrorfunc{\mu_s}{\sigma^\sharp_{tt}}$}
                \If{%
                        $\sigma^\sharp_{k + 1} = \sigma^\sharp_{k}$
                        $\vee$
                        $k \geq \mathrm{max\_iter}$
                }
                    \State{\Return{$\loopinvar$, $\loopexit$}}
                \EndIf%
                \State{$k \gets k + 1$}
            \EndLoop%
        \EndFunction%
    \end{algorithmic}
    \caption{%
        The accuracy analysis of a fixpoint expression.
    }\label{po:alg:fix}
\end{figure}

Our method extends the iterative method we have previously explained in
Section~\ref{bg:sec:abstract_interpretation} of Chapter~\ref{chp:background},
by not only evaluating the \gls{li}, but also the loop exit state.
Because this iterative process may not terminate, we introduce a constant
$\mathrm{max\_iter}$ to limit the number of iterations.  Widening
operators~\cite{cousot04} are additionally used to accelerate the fixpoint
computation.

\section{Resource Usage Analysis}
\label{so:sec:resource}

Here we define similar formal semantics which calculate an approximation to
the \gls{fpga} resource usage of an expression, taking into account common
subexpression elimination. This is important as, for example, rewriting $\vara
\times \varb + \vara \times \varc$ as $\vara \times (\varb + \varc)$ in the
larger expression $(\vara \times \varb + \vara \times \varc) + {(\vara \times
\varb)}^2$ causes the common subexpression $\vara \times \varb$ to be no longer
present in both terms. Our analysis must capture this.

The analysis proceeds by labelling subexpressions. Intuitively, the
set of labels $\labelset$, is used to assign unique labels to unique
expressions, so it is possible to easily identify and reuse them. For
convenience, let the function $\fresh: \aexprset\to\labelset$ assign a
distinct label to each expression or variable, where $\aexprset$ is the set
of all expressions.  It is noteworthy that $\fresh$ is a bijection. Before
we introduce the labeling semantics, we define the environment $\lambda:
\labelset\to\aexprset\cup\{\bot\}$, which is a function that maps labels to
expressions, and $\env{}$ denotes the set of such environments. A label $l$ in
the domain of $\lambda\in\env{}$ that maps to $\bot$ indicates that $l$ does
not map to an expression. An element $(l, \lambda)\in\labelset\times\env{}$
stands for the labeling scheme of an expression. Initially, we map all labels
to $\bot$, then in the mapping $\lambda$, each leaf of an expression is
assigned a unique label, and the unique label $l$ is used to identify the leaf.
That is for the leaf variable or constant $x$:
\begin{equation}
    (l, \lambda) = (\fresh(x), [\fresh(x)\mapsto{x}]).
\end{equation}
This equation uses $[\fresh(x)\mapsto{x}]$ to indicate an environment that
maps the label $\fresh(x)$ to the expression $x$ and all other labels map
to $\bot$, in other words, if $l = \fresh(x)$ and $l^\prime \neq l$, then
$\lambda(l) = x$ and $\lambda(l^\prime) = \bot$.

\begin{figure}[ht]
    \centering
    \tikzstyle{block} = [
        draw,
        fill=white,
        rectangle,
        minimum height=1.5em,
        minimum width=1.5em
    ]
    \begin{tikzpicture}
        \node (+) [draw, fill=white, circle] at (0, 0) {$+$};
        \node (a) [block, left=of +, yshift=4mm] {\vara};
        \node (b) [block, left=of +, yshift=-4mm] {\varb};
        \node (*) [draw, fill=white, circle, right=of +] {$\times$};
        \draw[->] (a) -- node[auto, pos=0]{1} (+);
        \draw[->] (b) -- node[auto, swap, pos=0]{2} (+);
        \draw[->] (+) -- (*);
        \draw[->] (+)
            to[out=45, in=135]
            node[auto, pos=0.05]{3}
            node[auto, pos=0.95]{4} (*);
    \end{tikzpicture}
    \caption{%
        The \gls{dfg} for the sample expression.
    }\label{so:fig:sample_tree}
\end{figure}
For example, for the \gls{dfg} in Figure~\ref{so:fig:sample_tree} we have for
the variables $\vara$ and $\varb$:
\begin{equation}
    \begin{aligned}
        (l_\vara, \lambda_\vara)
            &= (\fresh(\vara), [\fresh(\vara)\mapsto{\vara}])
             = (l_1, [l_1 \mapsto \vara]), \\
        (l_\varb, \lambda_\varb) &= (l_2, [l_2 \mapsto \varb]).
    \end{aligned}
    \label{so:eq:variable_env}
\end{equation}
Then the environments are propagated in the flow direction of the \gls{dfg},
using the following formulation of the labeling semantics:
\begin{equation}
    \begin{aligned}
        (l_x, \lambda_x) \opsymbol (l_y, \lambda_y)
            &= (l, (\lambda_x\odot\lambda_y)
                      [l\mapsto{l_x \opsymbol l_y}]), \\
            \text{where~} l &= \fresh(l_x \opsymbol l_y),
                          \opsymbol\in\{+, -, \times\}.
    \end{aligned}
    \label{so:eq:labeling_semantics}
\end{equation}
Specifically, $\lambda=\lambda_x\odot\lambda_y$ signifies that $\lambda_y$
is used to update the mapping in $\lambda_x$, if the mapping does not
exist in $\lambda_x$, and result in a new environment $\lambda$; and
$\lambda[l\mapsto{x}]$ is a shorthand for $\lambda\odot[l\mapsto{x}]$.  As
an example, with the expression in Figure~\ref{so:fig:sample_tree}, using
\eqref{so:eq:variable_env}, recall to mind that $l_1 = l_\vara$, $l_2 =
l_\varb$, we derive for the subexpression $\vara + \varb$:
\begin{equation}
    \begin{aligned}
        (l_{\vara + \varb}, \lambda_{\vara + \varb})
            &= (l_\vara, \lambda_\vara) + (l_\varb, \lambda_\varb) \\
            &= \left(l_3,
                \left( \lambda_\vara \odot \lambda_\varb \right)
                    \left[ l_3\mapsto{l_\vara + l_\varb} \right]
               \right) \\
            &  \text{where~} l_3 = \fresh(l_a + l_\varb) \\
            &= (l_3, [l_1\mapsto{\vara}]\odot
                     [l_2\mapsto{\varb}]\odot
                     [l_3\mapsto{l_1 + l_2}]) \\
            &= (l_3, [
                 l_1\mapsto{\vara},
                 l_2\mapsto{\varb},
                 l_3\mapsto{l_1 + l_2}
               ]),
    \end{aligned}
\end{equation}
where $l_\vara + l_\varb$ is a syntactic construct to signify that the
subexpressions with labels $l_\vara$ and $l_\varb$ are added to form an
expression.
% \todo{George: I am a little confused by addition here.  What is the
% definition of ``+'' on labels?  If $l_\vara + l_\varb$ should be read purely
% as a syntactic construct, why does it need a distinct representation as
% $l_3$?}
Finally, for the full expression $(\vara + \varb) \times (\vara + \varb)$:
\begin{equation}
    \begin{aligned}
        (l, \lambda)
            &= (l_{a + \varb}, \lambda_{a + \varb}) \times
               (l_{a + \varb}, \lambda_{a + \varb}) \\
            &= (l_4, [l_1\mapsto{\vara}, l_2\mapsto{\varb},
                      l_3\mapsto{l_1 + l_2}, l_4\mapsto{l_3 \times l_3}]).
    \end{aligned}
\end{equation}
From the above derivation, it is clear that the semantics capture the reuse
of subexpressions. The estimation of area is performed by counting, for an
expression, the numbers of additions, subtractions and multiplications in the
final labeling environment, then calculating the number of \glspl{lut} used
to synthesize the expression. If the number of operators is $n_\opsymbol$
where $\opsymbol \in \opset$, where $\aopset$ denotes the set of arithmetic
operators, then the number of \glspl{lut} in total for the expressions is
estimated as:
\begin{equation}
    r_\mathrm{LUT} = \sum_{\mathclap{\opsymbol\in\aopset}}
        R^\mathrm{LUT}_\opsymbol n_\opsymbol,
\end{equation}
where the value $R^\mathrm{LUT}_\opsymbol$ denotes the number of \glspl{lut}
per $\opsymbol$ operator, which is dependent on the type of the operator and
the floating-point format used to generate the operator.

In the following sections, we use the function $\area: \aexprset\to\naturalset$
to denote our resource usage analysis.

\section{Discovering Equivalent Programs}
\label{bg:sec:discovering_equivalent_programs}

In this section, we explain existing optimization methods to restructure
numerical programs with arithmetic equivalences.  Because general numerical
programs---consisting of program statements, conditional branches and
loops---supersede arithmetic expressions, we start by introducing optimization
methods of expressions, followed by those of general numerical programs.


\subsection{Arithmetic Expressions}
\label{bg:sub:arithmetic_expressions}

LLVM-based HLS tools such as Vivado HLS and LegUp usually have some
traditional static analysis-based optimization passes such as constant
propagation and alias analysis.  These optimization passes, however, limit
themselves by producing circuits that do not impact numerical accuracy,
\ie~they compute the same output given identical inputs.  For this reason,
new optimization passes are designed to allow dependence graphs to be
restructured to improve loop parallelism, which allows more computation
across loop iterations to overlap, and in turn faster programs.  Tree height
reduction~\cite{nicolau91} aims to balance an arithmetic expression tree using
associativity and distributivity.  Xilinx's Vivado HLS has a similar feature
called \emph{expression balancing}~\cite{vivado_hls}.  Both of these methods do
not produce optimal loop pipelining, as they do not examine the implications
of loop-carried dependences.  Canis \etal~\cite{canis14} propose a similar
approach called \emph{recurrence minimization}. They specifically tackle
loop pipelining by incrementally restructuring dependence graphs to minimize
longest paths of recurrences.  Their method is subsequently incorporated in
LegUp~\cite{legup}, an open-source academic HLS tool.  However, both LegUp and
Vivado HLS only apply associativity in their restructuring.

Most importantly, none of the above mentioned techniques and tools aim to
minimize, or even analyze, the impact of their transformations on resource
usage and accuracy. In many numerically sensitive programs, small round-off
errors would result in catastrophic inaccurate results.  HLS tools therefore
generally disable this feature by default for floating-point computations.
There are also academic tools that perform precision-performance trade-off by
optimizing word-lengths of data paths~\cite{constantinides}; and this safely
bounds round-off errors.  However, no existing HLS tools can perform the
trade-off optimization among accuracy, resource usage and latency by varying
the \emph{structure} of numerical programs.

A further handful of techniques were proposed to minimizes resources, and none
of these considers accuracy implications.  Hosangadi~\etal~\cite{hosangadi}
propose an algorithm for the factorization of polynomials to reduce
addition and multiplication counts, but this method is only suitable for
factorization and it is not possible to choose different optimization
levels. Peymandoust~\etal~\cite{peymandoust} present an approach that
only deals with the factorization of polynomials in HLS using Gr\"obner
bases. The shortcomings of this are its dependence on a set of library
expressions~\cite{hosangadi} and the high computational complexity of Gr\"obner
bases.

Even in the software community, there are only a few existing techniques
for optimizing expressions by transformation, none of which consider
accuracy/run-time trade-offs.  With regard to the structural optimization
of only arithmetic expressions without control structures, currently
there are only a handful of tools that could optimize by \emph{truly
restructuring}, \ie~they exploit any of the three equivalence relations in
real arithmetic, namely associativity, commutativity and distributivity.
Darulova~\etal~\cite{darulova} employ a metaheuristic technique. They use
genetic programming to evolve the structure of arithmetic expressions into more
accurate forms. However there are several disadvantages with metaheuristics,
such as convergence can only be proved empirically and scalability is difficult
to control because there is no definitive method to decide how long the
algorithm must run until it reaches a satisfactory goal. The method proposed
by Martel~\cite{martel07} is based on operational semantics with abstract
interpretation, but even their depth limited strategy is, in practice, at least
exponentially complex.  Finally Ioualalen~\etal~\cite{ioualalen} introduce the
abstract interpretation of equivalent expressions, and creates a polynomially
sized structure to represent an exponential number of equivalent expressions
related by rules of equivalence. However it restricts itself to only a handful
of these rules to avoid combinatorial explosion of the structure and there are
no options for tuning its optimization level.

The tool, \soap{}, proposed in this thesis was therefore designed to be the
only tool that could trade off area and accuracy in this category.


\subsection{Numerical Programs}
\label{bg:sub:numerical_programs}

The technique we have explored so far has only been limited to individual
arithmetic expressions; for a complete numerical program transformation, not
only is it necessary to support sequential execution of straight-line code,
but also control-flow structures such as conditional branches and loops.
Methods has been developed by Tate~\etal~\cite{tate09}, Martel~\cite{martel09}
and Damouche~\etal~\cite{damouche15} to utilize program semantics and
abstract interpretation~\cite{cousot77}.  However, all of these techniques
can neither unroll loops partially, nor optimize across loop boundaries.
Tate~\etal~\cite{tate09} specifically target discovering equivalent structures
only, it is however difficult to optimize for numerical accuracy using
their approach, as we have discussed in Section~\ref{bg:sec:intermediate}.
Martel~\cite{martel09} and Damouche~\etal~\cite{damouche15} optimize
numerical accuracy only, and both do not optimize across basic blocks.
Furthermore, Martel found that frequently their technique produces much slower
implementations, while we also consider performance, by improving accuracy,
latency and the resource usage of programs.


\section{Code Generation}
\label{po:sec:code_generation}

The final stage is to translate the optimized \gls{mir} back to a program
in its original syntax.  As discussed earlier, the \gls{ma} produces an
abstraction of the program, which means there are generally many ways of
generating different programs from the same \gls{mir}\@.  For this reason,
certain heuristic optimizations are performed before or during code generation,
such as branch- and loop-fusion transformations explained in our resource
usage analysis to produce a unique and deterministic translation from the
\gls{mir}\@.

Our code generation is carried out in three stages.  The first stage applies
the transformations outlined in Section~\ref{po:sec:resource} to perform
loop- and branch-fusion, and to allow sharing of expressions across nested
\glspl{mir}.

After the first stage, we create a topological sort of all nodes, which
produces a linear ordering of all nodes such that the control- and
data-dependences are preserved.  After this, we perform a simple one-to-one
mapping from the list of nodes to program code.  An arithmetic node is
translated into an assignment statement which assigns a temporary variable with
the result of the arithmetic computation.  A ternary conditional and a fixpoint
node respectively generate an \iflit~statement and a \whilelit~loop.  Finally,
a composition node \binarymir{$\expand$}{$e$}{$\mu$} ensures that $\mu$ is
generated before $e$.

The final and optional step of our code generation, is to perform code sinking,
which moves parts of the code so that when their results are not needed, they
are not executed~\cite{llvm}.  For instance, the result of the statement
``\lstinline[basicstyle=\tt]{y = x + 1;}''
is only used in the true-branch of the program:
\begin{lstlisting}
    y = x + 1;
    if (x < 1)
        y = y + 1;
    else
        y = x;
\end{lstlisting}
This statement can therefore be moved into the true-branch of the
\iflit~statement, so it may be evaluated only when needed.

\section{Results}
\label{so:sec:results}

Because Martel's approach defers selecting optimal options until the end of
equivalent expression discovery, we developed a method that could produce
exactly the same set of equivalent expressions from the traces computed by
Martel, and has the same time complexity. The difference is that we adopted it
to generate a Pareto frontier from the discovered expressions, instead of only
error bounds.  This allows us to compare \marteltrace{}, \ie~our implementation
of Martel's method, against our methods \frontiertrace{} and \greedytrace{}
discussed in Section~\ref{so:sec:equivalent}.  Figure~\ref{so:fig:martel}
optimizes the expression ${(\vara + \varb)}^2$ using the three methods above,
all using depth limit $3$, and the input ranges are $\vara \in [5, 10]$
and $\varb \in [0, 0.001]$~\cite{martel07}. The IEEE 754 single-precision
floating-point format with rounding to nearest was used for the evaluation
of accuracy and area estimation. The scatter points represent different
implementations of the original expression that have been explored and
analyzed, and the (overlapping) lines denote the Pareto frontiers. In this
example, our methods produce the same Pareto frontier that Martel's method
could discover, while having up to 50\% shorter run time. Because we consider
an accuracy/area trade-off, we find that we can not only have the most accurate
implementation discovered by Martel, but also an option that is only 0.0005\%
less accurate, but uses 7\% fewer \glspl{lut}.

We go beyond the optimization of a small expression, by generating results in
Figure~\ref{so:fig:multi_expr_32} to show that the same technique is applicable
to simultaneous optimization of multiple large expressions. The expressions
$e_1$ and $e_2$, with input ranges $\vara \in [1, 2], \varb \in [10, 20], \varc
\in [10, 200]$ are used as our example:
\begin{equation}
    \begin{aligned}
    e_1 =&
        (\vara + \vara + \varb) \times
        (\vara + \varb + \varb) \times
        (\varb + \varb + \varc) \times {} \\
        &
        (\varb + \varc + \varc) \times
        (\varc + \varc + \vara) \times
        (\varc + \vara + \vara), \\
    e_2 =&
        (1 + \varb + \varc) \times
        (\vara + 1 + \varc) \times
        (\vara + \varb + 1).
    \end{aligned}
\end{equation}

We generated and optimized \gls{rtl} implementations of $e_1$ and
$e_2$ simultaneously using \frontiertrace{} and \greedytrace{}
with the depth limits indicated by the numbers in the legend of
Figure~\ref{so:fig:multi_expr_32}. Note that because the expressions evaluate
to large values, the errors are also relatively large. We set the depth limit
to $2$ and found that \greedytrace{} executes up to $10\times$ faster than
\frontiertrace{}, while discovering a sizable subset of the Pareto frontier of
\frontiertrace{}. Also our methods are significantly faster and more scalable
than \marteltrace{}, because of its poor scalability discussed earlier, our
computer ran out of 8 GB of memory before we could produce any results. If we
normalize the time allowed for each method and compare the performance, we
found that \greedytrace{} with a depth limit $3$ takes takes slightly less time
than \frontiertrace{} with a depth limit $2$, but produces a generally better
Pareto frontier. The alternative implementations of the original expression
provided by the Pareto frontier of \greedytrace{} can either reduce the
\glspl{lut} used by approximately 10\% when accuracy is not crucial, or can
be about 10\% more accurate if resource is not our concern.  It also enables
the ability to choose different trade-off options, such as an implementation
that is 7\% more accurate and uses 7\% fewer \glspl{lut} than the original
expression.

Furthermore, Figure~\ref{so:fig:multi_expr_vary_width} varies the mantissa
width of the floating-point format, and presents the Pareto frontier
of both $e_1$ and $e_2$ together under optimization. Floating-point
formats with mantissa widths ranging from 10 to 112 bits were used to
optimize and evaluate the expressions for both accuracy and area usage. It
turns out that some implementations originally on the Pareto frontier of
Figure~\ref{so:fig:multi_expr_32} are no longer desirable, as by varying the
mantissa width, new implementations are both more accurate and less resource
demanding.

Besides the large example expressions above, Figure~\ref{so:fig:taylor_sin}
and Figure~\ref{so:fig:motzkin} are produced by optimizing expressions with
real applications under single precision. Figure~\ref{so:fig:taylor_sin} shows
the optimization of the Taylor expansion of $\sin(x + y)$, where $x\in[-0.1,
0.1]$ and $y\in[0, 1]$, using \greedytrace{} with a depth limit $3$. The
function $\mathrm{taylor}(f, d)$ indicates the Taylor expansion of function
$f(x, y)$ at $x = y = 0$ with a maximum degree of $d$. For order 5 we reduced
error by more than 60\%. Figure~\ref{so:fig:motzkin} illustrates the results
obtained using the depth limit $3$ with the Motzkin polynomial~\cite{demmel}
$x^6 + y^4 z^2 + y^2 z^4 - 3 x^2 y^2 z^2$, which is known to be difficult to
evaluate accurately, especially using inputs $x\in[-0.99, 1]$, $y\in[1, 1.01]$,
$z\in[-0.01, 0.01]$.

All these above results are generated with the same type of floating-point
operators in each expression.  Although in this chapter we do not analyze
the number of \glspl{dsp} used in synthesized circuits, the \gls{dsp} count
increases linearly with the estimated \gls{lut} count.  In the next chapter
we further introduce the estimation of \gls{dsp} elements used as another
objective to optimize.

Because of the scalability problem of the depth limit $k$ mentioned in
Section~\ref{so:sec:equivalent}, $k \leq 3$ for all of our experiments.  By
setting $k = 4$, the tool does not terminate in reasonable amount of time and
saturates the memory (16 GB) of our system.  In the following chapters, we
propose methods to limit the number of iterations and the number of equivalent
expressions discovered to mitigate the lack of scalability of $k$.

Finally, Figure~\ref{so:fig:area} demonstrates the accuracy of the area
estimation used in our analysis. It compares the actual \glspl{lut} necessary
with the estimated number of \glspl{lut} using our semantics, by synthesizing
more than 6000 equivalent expressions derived from $\vara + \varb + \varc$,
$(\vara + 1) \times (\varb + 1) \times (\varc + 1)$, $e_1$, and $e_2$ using
varying mantissa widths. The dotted line indicates exact area estimation, a
scatter points that is close to the line means the area estimation for that
particular implementation is accurate. The solid black line represents the
linear regression line of all scatter points. On average, our area estimation
is a 6.1\% over-approximation of the actual number of \glspl{lut}, and the
worst case over-approximation is 7.7\%.
\newcommand{\figsize}{0.5}
\begin{figure}[ht]
    \centering
    \includegraphics[scale=\figsize]{martel}
    \caption{Optimization of ${(\vara + \varb)}^2$.}\label{so:fig:martel}
\end{figure}
\begin{figure}[ht]
    \centering
    \includegraphics[scale=\figsize]{multi_expr_32}
    \caption{%
        Simultaneous optimization of both $e_1$ and $e_2$.
    }\label{so:fig:multi_expr_32}
\end{figure}
\begin{figure}[ht]
    \centering
    \includegraphics[scale=\figsize]{multi_expr_vary_width}
    \caption{%
        Varying the mantissa width of Figure~\ref{so:fig:multi_expr_32}.
    }\label{so:fig:multi_expr_vary_width}
\end{figure}
\begin{figure}[ht]
    \centering
    \includegraphics[scale=\figsize]{taylor_sin}
    \caption{The Taylor expansion of $\sin(x + y)$.}\label{so:fig:taylor_sin}
\end{figure}
\begin{figure}[ht]
    \centering
    \includegraphics[scale=\figsize]{motzkin}
    \caption{The Motzkin polynomial $e_m$.}\label{so:fig:motzkin}
\end{figure}
\begin{figure}[ht]
    \centering
    \includegraphics[scale=\figsize]{area}
    \caption{Accuracy of Area Estimation.}\label{so:fig:area}
\end{figure}

High-level synthesis tools are typically designed to adhere to a rigid
specification which outlines their behaviour.  It is a traditional practice
to design this specification and the subsequent tool to ensure that the
synthesized circuits perform functionally identical to the original source
program written in high-level language.  It is also viewed as a good practice
because it has predicable outcomes.  Guided by the rules of the language,
programmers translate mathematical objects such as algorithms and physical
information respectively into source code and numerical data, in a way similar
to tools adhering to their specifications.  This manual process of translation
is unfortunately an approximate one.  Computations as simple as $\sqrt{3}$ must
be approximated, \eg~they are carried out in floating-point arithmetic, because
of the finite nature of computing machines.  Therefore, \gls{hls} tools cannot
be relied upon for an exact interpretation of the mathematical objects we wish
to implement, even if they guarantee the functional equivalence between the
source code and the synthesized result.

Despite the awareness of the approximate characteristic of numerical
software/hardware implementations using floating-point operations, engineers
often take the risks of neglecting this fact, and anticipate their designs to
behave identically to the mathematical algorithms visioned in real arithmetic
within a reasonable but not well-defined error margin.  As it was shown in
Section~\ref{bg:sub:expression_accuracy} in Chapter~\ref{chp:background},
round-off errors when accumulated, could have detrimental effects on our daily
life.  The aforementioned functional equivalence between source and circuit
guaranteed by \gls{hls} tools is therefore unable to regain any lost accuracies
due to approximation.

Traditional \gls{ir}-level \gls{hls} program optimization consist of a series
of transformation passes.  Most of these passes do not predict whether
they have negative impact on the resulting circuit, and they limit their
capabilities by preserving functional equivalence.  Varying the order of
these passes could have significant impact on the quality, as these passes
interact with one another in a complicated manner, it is difficult to predict
the overall impact on performance~\cite{huang15}.  For $n$ passes, there are
$n{\,!}$ distinct ways to order, it is thus a considerable challenge to decide
the optimal pass ordering, which is exacerbated by the fact that it could be
highly dependent on the input program~\cite{cong13}.

These above shortcomings of traditional \gls{hls} tools and optimizing
compilers provide a strong motivation for the work proposed in this thesis.

Firstly, we can apply the philosophy of relaxing the functional equivalence
required by \gls{hls} tools.  In the mean time, we preserve the equivalence of
the underlying mathematical objects in real arithmetic which hardware designs
are approximating.  One can often improve the numerical accuracy by choosing a
better alternative among these equivalences.

Secondly, by the same paradigm shift, a wide range of optimization
opportunities can be explored to minimize throughput and resource utilization.
These opportunities were previously lost out to the necessity of ensuring
consistent behaviour.

Finally, optimization can be carried out by applying steps of equivalence
rewrites driven by a prediction model.  Traditional optimization passes can
be broken up into much smaller common parts made of equivalence rules can be
easily proved mathematically correct.  By using models to predict run time,
resources and accuracy to guide the optimization process, it is possible to
explore multiple designs that trade-off the three performance metrics while
removing concerns about the ordering problem.  Many optimization passes, such
as constant propagation, dead code removal, common subexpression elimination
and~\etc, are naturally subsumed by the new approach.  As the computational
power of machines increases exponentially, we can foresee an increase in the
scale of the vast search space to be explored in the future.

This thesis therefore broadens the horizon of \gls{hls} tools, and equips
them with the new program optimization paradigm by leveraging these above
observations.  Specifically, the trade-off relationship among numerical
accuracy, resource utilization and throughput are optimized in floating-point
numerical programs for \gls{hls}\@.  Here we summarize the contributions of
this thesis.

To the best of our knowledge, this thesis is the first to introduce
multiple-objective performance optimization in a unified framework for
discovering equivalence in programs.  Chapter~\ref{chp:stropt} implements
this framework and optimizes a suite of expressions that are difficult to
optimize by hand, and improve numerical accuracy and area automatically.
In the experimental results, it turns out that the two central goals,
\ie~improving accuracy and minimizing area, are often not in conflict, as
optimized expressions can enjoy almost all enhancements that can be achieved
in both metrics.  Guided by the concept of abstract interpretation, it further
introduces the semantics-based program analyses to jointly reason about
safe ranges of round-off errors and resource utilization, and subsequently,
discovery of equivalent expressions.  This technique lays the necessary
foundation for program equivalence beyond simple arithmetic expressions.

The infinite size of the equivalent program space, coupled with undecidability
of program properties, makes the program optimization an even more
challenging task than the one of arithmetic expressions.  For this,
Chapter~\ref{chp:progopt} introduces a new graph-based intermediate
representation, \gls{mir}, for capturing the semantics of numerical programs.
This approach reduces the size of the search space, and the \gls{ir} itself
is derived from the formal semantics of programs to ensure the correctness
of equivalent \glspl{mir} and the back-and-forth translation between C and
\gls{mir}\@.  It further eliminates the problem of optimization pass ordering,
because by using the equivalence discovery framework, the Pareto frontier can
be extended incrementally with small steps of rewrites to multiple candidates.
Traditional compiler optimizations are naturally subsumed and further enhanced
by the \glspl{mir}, as many optimization techniques such as loop splitting and
loop fusion that previously must be profiled to justify enabling them, can
emerge automatically from the optimization process.  By optimizing a suite of
resource-efficient benchmark examples, the tool improves the numerical accuracy
by up to 65\%.

Formerly, \gls{hls} tools' ability to pipeline loops is fundamentally constrained
by intra-iteration dependencies.  Traditional optimization techniques such as
partial loop unrolling may have minimal effects on the initiation interval of
pipelined loops, as these do not impact the data-path structure, which ensures
that the functional equivalence is preserved.  Encouraged by the promising
effects of Nicolau~\etal's tree height reduction technique~\cite{nicolau91}
and LegUp's recurrence minimization~\cite{canis14}, Chapter~\ref{chp:latopt}
further incorporates latency analysis into the unified program optimization
framework.  It was found that traditional optimization techniques when used
in tandem with the arithmetic equivalence rules and memory access reduction
rules can significantly improve the latency and accuracy of a numerical
program.  In Chapter~\ref{chp:progopt}, the experimental results identifies
that the static analysis of round-off errors for each candidate explored is
the key factor to the speed of optimization.  This problem is addressed in
this chapter by graph partitioning and candidate pruning algorithms.  It
further enables deeper partial loop unrolling factors that was not explored in
Chapter~\ref{chp:progopt}.  Often as we optimize numerical programs by spending
more resources, latency and round-off error can be simultaneously minimized, as
more resources would allow greater flexibility to discover equivalent programs
that often perform well in terms of run time and accuracy.  By optimizing
a suite of benchmark examples from PolyBench and Livermore loops, the tool
improves the latency and accuracy of each by up to 12$\times$ and 7$\times$
respectively, at a cost of 4$\times$ more resource utilization.


\section{Future Prospects}
\label{cc:sec:future_prospects}

In its current form, the new approach to program optimization explained in this
thesis forms the underlying basis for a much larger set of future work.  Even
though it is precursory on its own, the promising experimental results showcase
the powerful optimization it can bring to optimizing compilers and \gls{hls}
tools.  Here, a list of potential directions of future research is discussed
that could further widen the scope of our technique for a broader range of
applications.

\textbf{\gls{llvmir}-Level Program Optimization.}  We could envision
a back-and-forth translator from \gls{llvmir}~\cite{llvm, llvm_ir} to
\gls{mir} graphs.  This could enable a much wider applicability of the
technique presented in the thesis to both \gls{llvm}-based \gls{hls} tools
and software compilers.  Additionally, it could benefit from existing
\gls{llvm} optimizations passes by using the optimized \gls{llvmir} code
as inputs.  There are however obstacles in migrating to \gls{llvmir} as
the source language.  Firstly, \gls{llvmir} is \gls{ssa}-based.  Since it
uses temporary variables for intermediate results in computation, a full
liveness analysis~\cite{hathhorn12, nielson99, boissinot08} may be necessary
to eliminate temporary variables from the resulting \gls{mir}\@.  Secondly,
control-flows in \gls{llvmir} are more freely structured.  Unlike C, which
defines \iflit~statements and \whilelit~loops and discourages the use of
\verb|goto| statements, control-flow in \gls{llvmir} are composed by basic
blocks and branches between pairs of them.  This requires the \gls{mir} to be
further extended to cope with complex control-flow patterns.  Conventionally,
programs written with branches are often analyzed using \emph{continuation
style semantics}~\cite{felleisen88}.  It is not evident how this semantics can
be embedded within \glspl{mir}.

\textbf{Tighter bounds on round-off errors.} As an alternative for interval
analysis, the accuracy analysis could enjoy more sophisticated abstract domains
that capture the correlations between variables, and produce tighter bounds
for results.  Currently, the analysis cannot produce meaningful, \ie~finite,
bounds on the round-off errors of certain numerical programs.  If the analysis
fails to bound errors, then currently the optimization cannot be directed
to a more accurate implementation.  By using abstract relational domains,
it is possible to produce a much tighter bound on the values of program
variables, and the associated errors.  There are a few relational domains-based
static analysis techniques of floating-point errors~\cite{mine07_2, putot04,
goubault11, astree}, however making use of them still poses challenges.  Each
floating-point operation introduces an independent error term as a new variable
in the formulation of these relational domains, and it may be difficult to
determine how to collapse these error terms into a smaller set of variables,
as the optimization in this thesis can introduce a large number of error
variables.

\textbf{Special and fused operators.} There could be a lot of interest in the
\gls{hls} community on how \soap~can be incorporated with existing work on
fused floating-point data-path synthesis.  Langhammer~\etal~\cite{langhammer}
propose that normalization and denormalization stages could be regarded as
redundant between operators in a floating-point data-path.  By removing
these stages, subsets of the data-path become fixed-point data-paths, in the
meanwhile saving resources and improving throughput at a cost of accuracy.  It
will be compelling to isolate the normalization/denormalization stages into
operators in the \soap~framework, so that a mixed floating-point/fixed-point
program can more efficiently trade-off resources, accuracy and latency.

\textbf{Multiple word-lengths.}  In this thesis, experiments have been carried
out on floating-point operations with a fixed mantissa only.  It would be
beneficial to further integrate fixed-point support.  Additionally, by further
supporting multiple precisions in the data-path, \ie~allowing each operators to
compute with different precisions, the trade-off relationship among our three
primary performance measures can be even more effective.  Techniques, known
as multiple word-length optimization~\cite{constantinides, lee06, cantin02},
exist to apply a heuristic approach to perturb the precisions in a data-path,
so that a performance metric can be optimized while round-off errors of outputs
satisfies an error budget.  Instituting such techniques in the \soap~framework
is rewarding as it can further reduce the area and latency requirement of a
synthesized circuit for a given accuracy.  All of these approaches optimize a
fixed data-path, whereas in \soap~the structure of the data- and control-paths
are varying as we optimize them.  Analyzing each of the candidates for the
optimal precision assignment to each operator is very inefficient because of
the number of candidates explored.  Moreover, current techniques work with a
predetermined error budget, and yet in fact a Pareto frontier exists for each
data-path to trade-off accuracy, resources and latency.

\textbf{Numerical analysis and linear algebra.}  There are two distinct
approaches to the analysis of round-off errors.  One focuses on the round-off
errors by statically analyzing numerical programs, and apply this in a way
which is as general as possible, similar to the method presented in this
thesis.  On the other hand, there are techniques employed by numerical analysts
to evaluates and improve the numerical accuracy and stability of particular
algorithms analytically.  Many creative solutions to challenges are invented
in this process.  For instance, \emph{Kahan's compensated summation} algorithm
is an accurate way to compute a sum of $n$ values, $\sum_{i = 0}^{n-1}
x_i$~\cite{kahan65} is shown in Figure~\ref{co:lst:sum}.  This algorithm
cannot be discovered easily using the method outlined in this thesis, and a
way to extend the framework to optimize programs as creatively as humans still
eludes us at the moment.  Higham~\etal~\cite{higham02} discuss in great depth
many existing numerical accuracy problems encountered in finite-precision
computation of polynomials and linear algebra subprograms and how to analyze
and overcome inaccuracies, often in terms of relative errors.  Bridging the
gap between computational and mathematical approaches for numerical analysis
will allow us to automate many accuracy optimizations that were previously
unexplored by the tool.
\begin{figure}[ht]
    \centering
\begin{lstlisting}[]
    float compensated_summation(float X[N])
    {
        float sum = 0.0f;
        float e = 0.0f;
        for (i = 0; i < n; i++)
        {
            float tmp = sum;
            float y = X[i] + e;
            sum = tmp + y;
            e = (temp - sum) + y;
        }
        return sum;
    }
\end{lstlisting}
    \caption{%
        Kahan's compensated summation algorithm to accurately compute the sum
        of $n$ elements $\sum_{i = 0}^{n-1} x_i$.
    }\label{co:lst:sum}
\end{figure}

\textbf{Continuity analysis and optimization.} The robustness of programs
are very important to us.  In many cases, we wish our algorithms to be
free from discontinuity, \ie~a small change in the initial condition
would not result in an undesirably large jump in the outputs.  For this,
Chaudhuri~\etal~\cite{chaudhuri11} and Goubault~\etal~\cite{goubault13}
respectively propose methods to analyze the robustness of programs.  The
former approach formally proves whether an algorithm is ill-conditioned in
terms of the existence of discontinuity, whereas the latter statically analyze
programs to determine if round-off errors introduce significant discontinuous
behaviour.  To illustrate, consider an \iflit~branch,
``\lstinline[basicstyle=\tt]{if ($e$ > 0) $c_1$ else $c_2$}'', where $e$
is a floating-point expression.  When $e$ is positive and very close to
$0$ when evaluated in real arithmetic, the floating-point result of $e$
could be non-positive, due to the effects of the round-off errors.  In
these extraordinary cases, the $c_2$ branch may be executed instead of the
intended $c_1$.  These above new techniques could inspire us to implement the
optimization of discontinuous behaviour, such as the one shown in the example,
as another objective.

\textbf{Memory partitioning.} The experimental results in this work see a
diminishing performance return when loops are deeply unrolled, because of a
memory bottleneck.  As memory accesses saturate in loop execution, \ie~all
memory ports are working in 100\% utilization, it is unable to gain further
performance improvements.  Currently the tool stops exploring further loop
unrolling when this happens.  By automatically partition arrays upon hitting
such a memory bottleneck, further throughput improvements can be achieved.

\textbf{Other practical considerations.}  Finally, we may consider design
perspectives that could make the resulting tool much more usable.  For
instance, programs may still be optimized by not having any knowledge about the
input variables.  Herbie~\cite{panchekha15} makes no assumption about the input
space, and can nevertheless optimize arithmetic expressions, by splitting the
input space into regimes.


% \section{Tool Usage}
% \label{cc:sec:usage}

% \soap~is a source-to-source optimizer that specifically targets numerical
% program statements written in a subset of standard C99.  The tool supports
% arithmetic and Boolean expressions, assignment statements, \iflit{} statements,
% \whilelit{} loops and \forlit{} loops.  The numerical data types we allow are
% $\inttype$ and $\floattype$, as well as single- and multi-dimensional array
% types.

% The program below is an example usage of \soap~in a C program.  Note that it
% specifies the input values are respectively a two-dimensional array \verb|A|,
% where its elements are single-precision floating point values between 0 and 1,
% and an integer \verb|T| equals to $20$.  It also indicates the only output that
% we care from this code is the resultant \verb|A|.
% \begin{lstlisting}
  % #define N 1024
  % #pragma soap begin
  % #pragma soap in float A[N][N]=[0,1], int T=20
  % #pragma soap out A
  % for (int t = 0; t < T; t++)
    % for (int i = 1; i < N-1; i++)
      % for (int j = 1; j < N-1; j++)
        % A[i][j] = 0.2f * (A[i-1][j] +
          % A[i][j-1] + A[i][j] +
          % A[i][j+1] + A[i+1][j]);
  % #pragma soap end
% \end{lstlisting}

% \soap~is an open-source command-line utility, which only requires the user
% to provide a program written in C extended with the above \verb|#pragma|
% statements.  The Pareto optimal programs are all automatically generated by
% \soap, each is accompanied with our estimations of its latency and resource
% usage, and an analyzed bound on round-off errors.  These programs can then be
% given to \gls{vhls} to be synthesized into circuits.


\section{Final Remarks}
\label{cc:sec:final_remarks}

This thesis adapts existing techniques such as accuracy, latency and resource
usage analysis, and further introduces novel approaches, \eg~\gls{mir} and
efficient equivalence discovery, and delivers them in a unified framework.
The functional equivalence relaxation paradigm is relatively under-explored,
because these optimizations are often highlighted as \emph{unsafe} by the
\gls{hls} tools, as they cannot analyze the numerical implications of these
optimizations.  \Gls{hls} tools therefore have very limited optimization
options base on this particular concept.  With the constructive results
produced by this thesis, optimizations based on our concept can not only raise
performance measures, but also result in even \emph{safer} implementations
as we improve numerical accuracies.  The equivalence discovery algorithm in
tandem with \glspl{mir} could have great potential in compiler optimization
based on our concept.  Furthermore, since machine learning algorithms are
error-resilient~\cite{lesser11, kim09, holt91, zhu03}, the methods demonstrated
in this thesis have promising capabilities to improve the resource usage,
latency and accuracy of them.



\chapter{Accurate and Resource Efficient Pipelining of Numerical Programs}

% Loops are pervasive in numerical programs, so state-of-the-art high-level
% synthesis (HLS) tools use pipelining to schedule them efficiently. Still, the
% run time performance of the resultant FPGA implementation is limited by data
% dependences between loop iterations. Some of these dependence constraints can
% be alleviated by rewriting the program according to arithmetic identities
% (\eg~associativity and distributivity), memory access reductions, and
% control flow optimizations (\eg~partial loop unrolling). HLS tools cannot
% safely enable such rewrites by default because they may spoil the accuracy
% of floating-point computations and increase area usage. In this paper,
% we introduce the first open-source program optimizer for automatically
% rewriting a given program to optimize latency while controlling for accuracy
% and area. Our tool reports a three-dimensional Pareto frontier that the
% programmer can use to resolve the trade-off according to their needs. When
% applied to a suite of PolyBench and Livermore Loops benchmarks, our tool has
% generated programs that enjoy up to a 12$\times$ speedup, with a simultaneous
% 7$\times$ increase in accuracy, at a cost of up to 4$\times$ more LUTs.

\section{Introduction}
\label{sec:introduction}

The IEEE 754 standard~\cite{ieee754} for floating-point computation is
ubiquitous in computing machines. In practice, it is often neglected that
floating-point computations almost always have roundoff errors. In fact,
associativity and distributivity properties which we consider to be fundamental
laws of real numbers no longer hold under floating-point arithmetic. This opens
the possibility of using these rules to generate an expression equivalent to
the original expression in real arithmetic, which could have better quality
than the original when evaluated in floating point computation.

By exploiting rules of equivalence in arithmetic, such as associativity $(a
+ b) + c \equiv a + (b + c)$ and distributivity $(a + b) \times c \equiv a
\times c + b \times c$, it is possible to automatically generate different
implementations of the same arithmetic expression. We optimize the structures
of arithmetic expressions in terms of the following two quality metrics
relevant to FPGA implementation: the resource usage when synthesized into
circuits, and a bound on roundoff errors when evaluated. Our goal is the joint
minimization of these two quality metrics. This optimization process provides
a Pareto optimal set of implementations. For example, our tool discovered that
with single precision floating-point representation, if $a \in [0.1, 0.2]$,
then the expression ${(a + 1)}^2$ uses fewest resources when implemented in the
form $(a + 1) \times (a + 1)$ but most accurate when expanded into $(((a \times
a) + a) + a) + 1$. However it turns out that a third alternative, $((1 + a)
+ a) + (a \times a)$, is never desirable because it is neither more accurate
nor uses fewer resources than the other two possible structures. Our aim is to
automatically detect and utilize such information to optimize the structure of
expressions.

A na{\"\i}ve implementation of equivalent expression finding would be to
explore all possible equivalent expressions to find optimal choices, However
this would result in combinatorial explosion~\cite{ioualalen}. For instance,
in worst case, the parsing of a simple summation of $n$ variables could
result in $(2n - 1)!! = 1\times3\times5\times\cdots\times(2n - 1)$ distinct
expressions~\cite{ioualalen, mouilleron}. This is further complicated by
distributivity as ${(a + b)}^k$ could expand into an expression with a
summation of $2^k$ terms each with $k - 1$ multiplications. Therefore, usually
it would be infeasible to generate a complete set of equivalent expressions
using the rules of equivalence, since an expression with a moderate number of
terms will have a very large number of equivalent expressions. The methodology
explained in this paper makes use of formal semantics as well as abstract
interpretation~\cite{cousot77} to significantly reduce the space and time
requirements and produce a subset of the Pareto frontier.

In order to further increase the options available in the Pareto frontier,
we introduce freedom in choosing mantissa widths for the evaluation of the
expressions. Generally as the precision of the evaluation increases, the
utilization of resources increases for the same expression. This gives
flexibility in the trade-off between resource usage and precision. Our
approach and its associated tool, SOAP, allow high-level synthesis flows to
automatically determine whether it is a better choice to rewrite an expression,
or change its precision in order to meet optimization goals.

The three contributions of this paper are:
\begin{enumerate}
    \item Efficient methods for discovering equivalent structures of
    arithmetic expressions.
    \item A semantics-based program analysis that allows joint reasoning about
    the resource usage and safe ranges of values and errors in floating-point
    computation of arithmetic expressions.
    \item A tool which produces RTL implementations on the area-accuracy
    trade-off curve derived from structural optimization.
\end{enumerate}

This paper is structured as follows. Section~\ref{sec:related_work}
discusses related existing work in high-level synthesis and the
optimization of arithmetic expressions. We explain the basic
concepts of semantics with abstract interpretation used in this
paper in Section~\ref{sec:abstract_interpretation}. Using this,
Section~\ref{sec:semantics} explains the concrete and abstract semantics
for finding equivalent structure in arithmetic expressions, as well as
the analysis of their resource usage estimates and bounds of errors.
Section~\ref{sec:implementation} gives an overview of the implementation
details in our tool. Then we discuss the results of optimized example
expressions in Section~\ref{sec:results} and end with concluding remarks in
Section~\ref{sec:conclusion}.

\section{Motivation}
\label{lo:sec:motivation}

\newcommand\barr[3]{\texttt{\underline{#1[#2][#3]}}}

Figure~\ref{lo:fig:seidel_prog} gives an implementation of the Seidel stencil
computation, extracted from PolyBench~\cite{polybench}, where initially all
values in the array \verb|A| are single-precision floating-point values between
$0$ and $1$.  It resembles the typical code frequently used in fluid dynamic
simulations for solving partial differential equations and systems of linear
equations.

\begin{figure}[ht]
    \begin{lstlisting}
        #define N 1024
        for (int t = 0; t < 20; t++)
          for (int i = 1; i < N-1; i++)
            for (int j = 1; j < N-1; j++)
              $\barr{A}{i}{j}$ = 0.2 * (A[i-1][j] + $\barr{A}{i}{j-1}$ +
                  A[i][j] + A[i][j+1] + A[i+1][j]);
    \end{lstlisting}
    \caption{%
        An excerpt from the Seidel stencil~\cite{polybench}.  The
        inter-iteration data-dependence of the innermost loop is underlined
        ($\barr{A}{i}{j}$ and $\barr{A}{i}{j-1}$).
    }\label{lo:fig:seidel_prog}
\end{figure}

We start by synthesizing this program in \gls{vhls}\@.  We enable \emph{loop
pipelining} in \gls{vhls}, which asks it to optimize the loop by overlapping
its iterations.  However, we can observe that this program has very limited
opportunity for pipelining, because each iteration \verb|j| of the innermost
loop ends by writing to \verb|A[i][j]|, and the next iteration \verb|j+1|
begins by reading from \verb|A[i][j]|; this inter-iteration dependence is
highlighted in Figure~\ref{lo:fig:seidel_prog}.  Hence, it serves as our
example to motivate a better \soap~to efficiently pipeline loops.

\Gls{vhls} generates a schedule where the depth of the loop $D$ is $49$, and
\gls{ii} as enforced by the data-dependences above is $46$.  The trip count of
the innermost loop is $N = 1022$.  The overall latency of the innermost loop is
therefore $((N-1) \times \II) + D = 47,015$ cycles.

We then enable \gls{vhls}'s \gls{eb} optimization.  When synthesized, this
optimization pass tries to reorder the sequence of additions in the loop body
into a tree structure, thus reducing the $\II$ to 28 cycles, and the depth $D$
to 42 cycles, while the trip count $N = 1022$ remains the same.  The overall
latency is now $((N-1) \times \II) + D = 28,630$ cycles.  The overall resource
usage remains roughly the same.

However, as mentioned in Section~\ref{bg:sec:discovering_equivalent_programs}
of Chapter~\ref{chp:background}, \gls{vhls}'s \gls{eb} has two shortcomings.
Firstly, it is not aware of the inter-iteration data-dependence and misses the
opportunity to further pipeline this loop.  Secondly, and most importantly,
\gls{vhls} does not guarantee that this optimization will not result in
catastrophic numerical inaccuracies.

We further discover that if the loop is partially unrolled, \gls{vhls}'s
\gls{eb} did not improve the total run time, despite using a lot more
resources.  Additionally, \gls{eb} only makes use of associativity, but
not other equivalence rules.  These limitations pose great restrictions on
\gls{vhls}'s ability to produce a significantly faster implementation.

We then use the enhanced \soap~of this chapter to automatically discover
equivalent programs from the program in Figure~\ref{lo:fig:seidel_prog}.
Because \soap~explores a large number of paths that lead to a Pareto frontier
of implementations, here we illustrate one of the many paths that could be
taken by minimizing latency, while trying to optimize accuracy and resource
usage.  By using just arithmetic equivalences, \soap~specifically applies
transformations to alleviate the constraints on the inter-iteration dependence,
and discovers that the innermost loop can be rewritten to minimize latency as
shown in Figure~\ref{lo:fig:seidel_prog_2}.

\begin{figure}[ht]
    \begin{lstlisting}
        for (int j = 1; j < 1023; j++)
            A[i][j] = 0.2 * (A[i][j-1] +
                ((A[i][j] + A[i][j+1]) +
                 (A[i+1][j] + A[i-1][j])));
    \end{lstlisting}
    \caption{The optimized program using only arithmetic equivalences.}
    \label{lo:fig:seidel_prog_2}
\end{figure}

Although this loop still has a data-dependence between consecutive iterations,
this transformation greatly reduces latency because most of the loop iterations
can now be overlapped.  We find that this simple transformation can reduce
$\II$ to 19, which speeds up the original program by 2.3$\times$, using almost
the same number of \glspl{lut} and \gls{dsp} elements as the original program.
At the same time, the sequence of additions are now reordered to minimize
round-off errors, improving the accuracy by 18\%.

\soap~also supports more complex control-flow restructuring transformations,
such as partial loop unrolling, in tandem with rules that optimize memory
accesses and arithmetic calculations.  This can further reduce the loop's
latency.  In this example, unrolling the loop by a factor of two (\ie~updating
two matrix elements on every iteration and halving the trip count) and applying
other rules, results in a program with $\II = 19, D = 152, N = 511$.  When
implemented on a device it is 4.8$\times$ faster than the original, and
almost twice as accurate, at a cost of 17\% more \glspl{lut}, as shown in
Figure~\ref{lo:fig:seidel_prog_3}.

\begin{figure}[ht]
\begin{lstlisting}
  for (int j = 1; j < 1023; j += 2) {
    float t0 = A[i][j-1], t1 = A[i][j+1];
    float t2 = (A[i][j] + t1) + (A[i+1][j] + A[i-1][j]);
    float t3 = 0.04f * t2 + 0.2f *
        ((t1 + A[i][j+2]) + (A[i+1][j+1] + A[i-1][j+1]));
    A[i][j] = 0.2f * (t0 + t2);
    A[i][j+1] = 0.04f * t0 + t3;
  }
\end{lstlisting}
    \caption{The optimized program using arithmetic equivalences in tandem with
    control-flow restructuring and memory access optimization.}
    \label{lo:fig:seidel_prog_3}
\end{figure}

Further increasing the optimization effort, which enables the loop to be
deeper partially unrolled, leads to a program that is 7$\times$ faster
than the original, but uses 2.8$\times$ \glspl{lut}.  To summarize, in
Table~\ref{lo:tab:seidel_results}, we compare \gls{vhls} with \gls{eb}, against
one of the many implementations that we have explored using \soap~with the
increased optimization effort.  The three columns respectively shows the
original program with loop pipelining enabled, what \gls{vhls} can achieve
alone, and the capability of \soap.  It is important to note that the round-off
error is unknown for \gls{vhls} with \gls{eb}, because it cannot predict the
impact of its unsafe optimizations on accuracy.  We performed place-and-route
for exact statistics.

\begin{table}[ht]
    \centering\singlespacing%
    \renewcommand\arraycolsep{4.0mm}
    \caption{%
        Comparison among the optimized implementations generated by
        \gls{vhls}'s expression balancing and our optimizer.  The row ``Total
        run time (s)'' indicates the wall-clock time in seconds of running the
        synthesized circuits.
    }\label{lo:tab:seidel_results}
    $\begin{array}{rrrr}
        \toprule
        \multicolumn{1}{c}{} &
        \multicolumn{1}{c}{\text{\gls{vhls}}} &
        \multicolumn{1}{c}{\text{\gls{vhls} with \gls{eb}}} &
        \multicolumn{1}{c}{\text{\gls{vhls} with \soap}}
        \\ \midrule\midrule

        \text{Clock period (ns)} &
        2.60 & 2.65 & 2.66 \\ \midrule

        \Shade \text{Inner latency (cycles)} &
        \Shade 47.0\unitk & \Shade 28.6\unitk & \Shade 6.59\unitk \\ \midrule

        \text{Total run time (s)} &
        2.50\unitm & 1.56\unitm & 0.358\unitm \\ \midrule

        \Shade \text{\glspl{lut}} &
        \Shade 620 & \Shade 623 & \Shade 1778 \\ \midrule

        \text{\gls{dsp} elements} &
        5 & 5 & 8 \\ \midrule

        \Shade \text{Round-off error} &
        \Shade 10.68\unitmu &
        \Shade \text{unknown} &
        \Shade 4.31\unitmu \\ \bottomrule
    \end{array}$
\end{table}

\section{High-Level Overview}
\label{lo:sec:overview}

\begin{figure}[ht]
    \centering
    \includegraphics[scale=0.8]{overview}
    \caption{%
        An overview of our automatic program optimization process. The shaded
        region shows our internal tool flow.
    }\label{lo:fig:overview}
\end{figure}

We start by introducing a high-level overview of our program optimization
process (Figure~\ref{lo:fig:overview}).  Our automatic optimization process
starts by taking as an input, the original numerical program written in C, and
translates it into a \gls{mir}\@.  A \gls{mir} is a \gls{dag}, and it serves as
an abstract representation of the original program.  It discards information
about \emph{how} a program is executed, which is dependent on how the program
is structured in C, but retains the \emph{effect} of program execution, keeping
only the structure that leads to the final results.  This procedure, explained
in detail in Section~\ref{lo:sec:intermediate}, greatly reduces the number
of program transformations we need to explore.  We then discover equivalent
\glspl{mir} using our efficient optimization procedure discussed in detail
in Section~\ref{lo:sec:structural_optimization}.  The optimized C programs
can then be generated from the \glspl{mir}, using the \soap{} framework's
code generation routines, to be synthesized in \gls{vhls} to obtain \gls{rtl}
implementations.

As we apply transformation rules to discover equivalent \glspl{mir}, we
estimate latency, resource usage and analyze round-off errors for each
\gls{mir} we have discovered.  Non-Pareto-optimal \glspl{mir}---the ones with
all three performance metrics (latency, resource usage and accuracy) worse
than any other \glspl{mir}---are pruned immediately to keep the size of total
\glspl{mir} discovered tractable.  Section~\ref{lo:sec:performance_analysis}
explains in depth how we analyze latency, resource usage and accuracy.

\section{Intermediate Representations}
\label{bg:sec:intermediate}

\Glspl{ir} are data structures designed to be independent of the machine
architecture and source language.  They are often invented with the intention
to ease program analysis and optimization in mind, by abstracting information
from the original program that are irrelevant to our objectives.  In this
section, we introduce several categories of existing \glspl{ir}, and delve
deeper into the advantages and disadvantages of each.

\subsection{Static Single Assignment Form and Control-Flow Graph}
\label{bg:sub:ssa_cfg}

Traditionally, \gls{ssa} form~\cite{alpern88, rau92} together with
the \gls{cfg} are used to represent data- and control-flow of a
program~\cite{cytron91}, because they are more favorable program
representations on which optimization passes can be implemented, when
compared to the original \gls{hll} or the output language.  \Gls{ssa} can be
advantageous in implementing conventional optimization techniques, \eg~code
motion~\cite{cytron86}, removing redundant computations~\cite{rosen88}, and
constant propagation~\cite{cytron91}.  Because the \gls{llvmir}~\cite{llvm_ir}
is based on \gls{ssa} and \glspl{cfg}, and is commonly used in many \gls{hls}
tools such as LegUp~\cite{legup}, we introduce \gls{ssa} and \glspl{cfg}
by compiling the dot-product example in Figure~\ref{bg:lst:dotprod} into
\gls{llvmir} as shown in Figure~\ref{bg:lst:dotprod_ll}.
\begin{figure}[ht]
    \centering
    \begin{lstlisting}[language=LLVM]
define float @dotprod(
    float* nocapture readonly %A,
    float* nocapture readonly %B) #0
{
; <label>:0
  br label %2

; <label>:1         ; preds = %2
  ret float %8

; <label>:2         ; preds = %2, %0
  %i.02 = phi i32 [ 0, %0 ], [ %9, %2 ]
  %d.01 = phi float [ 0.000000e+00, %0 ], [ %8, %2 ]
  %3 = getelementptr inbounds float, float* %A, i32 %i.02
  %4 = load float, float* %3, align 4, !tbaa !2
  %5 = getelementptr inbounds float, float* %B, i32 %i.02
  %6 = load float, float* %5, align 4, !tbaa !2
  %7 = fmul float %4, %6
  %8 = fadd float %d.01, %7
  %9 = add nuw nsw i32 %i.02, 1
  %exitcond = icmp eq i32 %9, 1024
  br i1 %exitcond, label %1, label %2
}
    \end{lstlisting}
    \caption{%
        The compiled and optimized \gls{llvmir} output from the dot-product
        example in Figure~\ref{bg:lst:dotprod}.
    }\label{bg:lst:dotprod_ll}
\end{figure}

The \gls{llvmir} of our example function consists of parts that are known
as \glspl{bb}.  Each \gls{bb} in turn often contains a label that uniquely
identifies the \gls{bb}, a list of \gls{llvmir} statements in \gls{ssa} form
without any branches, \ie~the statements are executed sequentially, and a
terminator instruction, which is typically a branch instruction that leads the
control-flow to a different \gls{bb}, by referencing a \gls{bb} label or a
function return.

The \gls{llvm} framework implicitly constructs a \gls{cfg} from the
\gls{ir} code, which is a directed graph representing the control-flow of
a program.  The vertices in the \gls{cfg} constitute \glspl{bb}, while
the edges indicate the control-flow directions (\ie~branches to other
\glspl{bb}), often with predicate attributes to determine whether the branch
is taken.  For instance, we consider the first line of the third \gls{bb} in
Figure~\ref{bg:lst:dotprod_ll}:
\begin{lstlisting}[language=LLVM]
    ; <label>:2     ; preds = %2 %0
\end{lstlisting}\vspace{-15pt}
which indicates it has a label value $2$ and the control-flow coming to this
\gls{bb} is from either \gls{bb}2 or \gls{bb}0, here we use \gls{bb}$n$ as a
shorthand denoting a \gls{bb} labelled $n$.  Additionally, this \gls{bb} ends
with the branch terminator instruction:
\begin{lstlisting}[language=LLVM]
    br i1 %exitcond, label %1, label %2
\end{lstlisting}\vspace{-15pt}
This instruction directs the control-flow to \gls{bb}1 or \gls{bb}2,
and the variable \verb|%exitcond| in the terminator instruction decides
which branch is taken.  Finally, the complete \gls{cfg} is shown in
Figure~\ref{bg:fig:dotprod_cfg}.  It is noteworthy that \gls{bb}2 has two edges
that leads to either \verb|BB1| or \verb|BB2| itself.  If \verb|%exitcond|
evaluates to false (\textbf{ff}), then another iteration of \gls{bb}2 will
commence, otherwise (\textbf{tt}) the exit condition is satisfied and will lead
the control-flow to \verb|BB1|.
\begin{figure}[ht]
    \centering
    \begin{tikzpicture}
        \node [] (entry) {\textbf{entry}};
        \node [rect, below of=entry, node distance=4em] (bb0) {\texttt{BB0}};
        \node [rect, below of=bb0, node distance=4em] (bb1) {\texttt{BB1}};
        \node [rect, below of=bb1, node distance=4em] (bb2) {\texttt{BB2}};
        \node [coordinate, right of=bb1, node distance=5em] (bb1tr) {};
        \node [coordinate, left of=bb2, node distance=8em] (bb2tl) {};
        \node [coordinate, right of=bb2, node distance=8em] (bb2tr) {};
        \node [below of=bb2, node distance=4em] (exit) {\textbf{exit}};
        \draw [->] (entry) -- (bb0);
        \draw [- ] (bb0)    to[out=0, in=90]    (bb1tr);
        \draw [->] (bb1tr)  to[out=-90, in=0]   (bb2);
        \draw [- ] (bb1)    to[out=180, in=90]  (bb2tl);
        \draw [->] (bb2tl)  to[out=-90, in=180] (exit);
        \draw [->] (bb2) -- node[auto, swap]{\textbf{tt}} (bb1);
        \draw [->] (bb2) to[out=-150, in=150, loop]
            node[auto, swap]{\textbf{ff}} (bb2);
    \end{tikzpicture}
    \caption{%
        The \gls{cfg} of the \gls{llvmir} code in
        Figure~\ref{bg:lst:dotprod_ll}.
    }\label{bg:fig:dotprod_cfg}
\end{figure}

Each \gls{bb} contains sequential computations that are represented by
\gls{ssa} instructions.  The \gls{ssa} form describes the operations in the
original program, such that each variable in it is assigned exactly one value.

The sequence of instructions that assigns to \verb|%3|-\verb|%9| in
Figure~\ref{bg:lst:dotprod_ll} carries out most of the computations in the
program.  It starts by reading \verb|A[i]| and \verb|B[i]|, multiplying them
together, then adding the result with \verb|d| to form a new variable, and
finally, the iteration value is incremented by $1$.  It may seem unusual that
the accumulated sum of products and the iteration value are not assigned
to \verb|d| and \verb|i| respectively.  We can imagine two \glspl{bb}, one
initializes \verb|d| and \verb|i| to zeros, the other accumulates these two
variables in a loop.  As all variables must be assigned once only, one of
the \glspl{bb} should use different names for these two variables.  When the
control-flows of these two \glspl{bb} join, we must introduce a way to read
from the variables that are assigned in the two \glspl{bb} in the succeeding
\gls{bb}\@.  A new instruction, called the $\phi$-function is therefore defined
for our purpose.  The $\phi$-function accepts two variable names as its inputs,
and produces the value of either variable as its output, determined by which
preceding \gls{bb} the control-flow came from.  For example, in \gls{llvmir},
the instruction:
\begin{lstlisting}[language=LLVM]
    %d.01 = phi float [ 0.000000e+00, %0 ], [ %8, %2 ]
\end{lstlisting}\vspace{-15pt}
shows that if the control-flow originated from \gls{bb}0, then a constant zero
is returned, otherwise the control-flow had to come from \gls{bb}2 and the
value of \verb|%8| is used instead.

The rationale of \gls{ssa} is that we can abstract away anti- and output
dependences by never assigning to the same variable twice, while only true
data-flow dependences remain.  An anti-dependence is a dependence relation
when a read operation must precede a write to the same variable, and an output
dependence is when two writes refer to the same location.  By removing these
dependences and deferring the analysis of them, certain program optimization
analyses can run much more efficiently.  Analyses that may benefit from this
include scheduling~\cite{rau94}, liveness analysis (estimating the lifetime
of variables to reduce register requirements)~\cite{cytron91}, detecting
opportunities for parallelism~\cite{cytron87}, and finding equivalent parts in
the program~\cite{alpern88}.

In a cyclic \gls{cfg}, the control-flow could potentially revisit a \gls{bb},
and instructions in this \gls{bb} will inevitably assign a different value to
the same variable, which forms anti- and output dependences, which could have
a detrimental effect on efficient loop pipelining in some computing machines.
An alternative \gls{ir}, the \gls{dsa} form~\cite{rau92} can therefore be used
in place of the \gls{ssa} to address this issue.  The \gls{dsa} defines a
linear sequence of virtual registers for each variable, such that every time
the variable is assigned in a dynamic execution path, a new virtual register is
used.

\subsubsection{Alternatives to \gls{ssa} and \gls{cfg}}

There are a number of alternative \glspl{ir} that are similar in construction
to the \gls{ssa} and \gls{cfg} approach in \gls{llvmir}\@.  For instance, the
data-dependence graph~\cite{rau94} introduced in Section~\ref{bg:sub:sdc} are
designed for the purpose of capturing data-flow dependences in polyhedral
methods. \gls{dfg} is a popular alternative to \gls{ssa}, which is often
a \gls{dag}.  In general, a \gls{dfg}'s vertices are input, output and
operation nodes, and the edges capture the dependences between these nodes.
A \glspl{dfg} however generally does not preserve enough information for us
to reconstruct a program from the graph itself.  A group of data structures,
known as \gls{cdfg}~\cite{orailoglu86}, is commonly used to represent programs
in \gls{hls} tools, \eg~SPARK~\cite{gupta04}.  A \gls{cdfg} resembles a
\gls{cfg} such as the one used in \gls{llvmir}, but in lieu of using sequential
instructions in \gls{ssa} form in graph vertices, each vertex contains a
\gls{dfg}, where no \gls{ssa} temporary variables are used and data-flow
dependences can by explicitly identified by edges.


\subsection{Equality Saturation}
\label{bg:sub:equality_saturation}

The \glspl{ir} we discussed above are all used to analyze and transform the
underlying program structure, so as to produce a new representation of the
optimized program.  In a conventional optimizing compiler, program optimization
is often carried out in a sequence of transformation passes, where each pass
accepts a program, often written in a certain \gls{ir}, and produces an
optimized program in the \gls{ir}\@.  The traditional practice is to always
apply these optimization phases in a fixed order, but a good ordering of these
phases is crucial to achieve a good optimized result, and the optimal ordering
varies across applications being compiled~\cite{almagor04}.  The process of
finding the optimal ordering is known as the \emph{phase-ordering problem},
which is in general undecidable~\cite{touati06}.  Moreover, programs running
on \glspl{cpu} or \glspl{gpu} are usually quantified by their throughput or
latency, in contrast, designs on \glspl{fpga} concern us with additional
objectives besides run time, such as power consumption and resource utilization
that impact the quality of synthesized circuit.  Multiple designs which
trade-off these objectives could exist, and which design to choose relies
on the specifics of the use case.  It is therefore sensible to explore the
design space by optimizing multiple objectives simultaneously.  For the above
reasons, it is desirable to have an \gls{ir} and the associated optimization
procedures to efficiently discover equivalent structures that lead to different
implementations of the original program.

In software, a novel approach called \emph{equality saturation} is proposed
in~\cite{tate09} to find multiple possible optimized variants of the original
program, and subsequently deal with the phase-ordering problem.  It creates a
new graph-based \gls{ir}, \gls{peg}, to encode the effects of executing the
program.

To begin, we review the structure of the \gls{peg}, by considering a simple
loop example in Figure~\ref{bg:fig:factorial}.  By understanding how the
\gls{peg} can be evaluated for the output values, we can interpret how the
\gls{peg} captures the control- and data-flow information of the program.
\glspl{peg}, similar to arithmetic expressions expressed in a tree structure,
can be evaluated in a bottom-up fashion, by recursively propagating computed
values from the leaf nodes to the root of the tree.  However, unlike arithmetic
expressions which are acyclic, edges in \glspl{peg} may form cycles to express
loops in the original program.
\begin{figure}[ht]
    \newsavebox{\factlstbox}
    \begin{lrbox}{\factlstbox}
    \begin{lstlisting}
int x = 1;
int y = 1;
while (y <= 10) {
    x = y * x;
    y = y + 1;
}
    \end{lstlisting}
    \end{lrbox}
    \centering
    \subfloat[The original program.]{%
        \begin{minipage}{0.4\textwidth}
            \usebox{\factlstbox}
        \end{minipage}\label{bg:lst:factorial_c}
    }
    \subfloat[The resulting \gls{peg}.]{%
        \begin{minipage}{0.5\textwidth}
        \begin{tikzpicture}[mir]
            \foreach \i/\v/\op in { 1/x/*, 2/y/+ } {
                \node (\v) [mirnode] at (\i * 30mm, 0) {$\v$};
                \node (e\i) [mirnode, right=of \v] {$\mathit{eval}_1$};
                \node (t\i) [mirnodealt, below left=of e\i, yshift=-10mm]
                    {$\theta_1$};
                \node (1\i) [mirnode, below left=of t\i] {$1$};
                \node (o\i) [mirnode, below right=of t\i] {\texttt{\op}};
                \draw[|->] (\v) -- (e\i);
                \draw[-] (e\i) -- (t\i.100);
                \draw[-] (t\i) -- (1\i);
                \node (mid\i) [coordinate, right=of o\i, yshift=-2mm] {};
                \draw[-] (t\i) -- (o\i)
                    to[out=-45, in=-135] (mid\i)
                    to[out=45, in=45] (t\i.60);
            }
            \node (12) [mirnode, below left=of o2] {$1$};
            \draw[-] (o2) -- (12);

            \node (p) [mirnode, below right=10mm of e2] {$\mathit{pass}_1$};
            \node (neg) [mirnode, below=of p] {$\neg$};
            \node (leq) [mirnode, below=of neg] {$\leq$};
            \node (10) [mirnode, below right=of leq] {$10$};
            \draw[-] (e1) to[out=-45, in=160] (p);
            \draw[-] (e2) -- (p.90);
            \draw[-] (p) -- (neg) -- (leq) -- (10);
            \node (mid3) [coordinate, left=12mm of neg] {};
            \draw[-] (leq) to[out=-135, in=-45] (mid3)
                           to[out=135, in=50] (t2.80);

            \node (mid4) [coordinate, below=5mm of o1, xshift=2mm] {};
            \draw[-] (o1) to[out=-135, in=180] (mid4) to[out=0, in=135] (t2);
        \end{tikzpicture}
        \end{minipage}\label{bg:fig:factorial_peg}
    }
    \caption{%
        A simple loop which computes the factorial of 10, and the resulting
        \gls{peg}\@.  This example and its \gls{peg}, showing computations
        that lead to the final \texttt{x} and \texttt{y}, is taken
        from~\cite{tate09}.
    }\label{bg:fig:factorial}
\end{figure}

\subsubsection{Data-Flow Nodes}

All loops in the \gls{peg} are formed by $\theta$ nodes, which is used in the
following form:
\begin{center}
    \vspace{-16.5pt}
    \begin{tikzpicture}
        \node (theta) at (0, 0) {$\theta$};
        \node (init) at (-5ex, -5ex) {$\iota$};
        \node (func) at (5ex, -5ex) {$\mathrm{func}(~~)$};
        \node[coordinate] (input) at (7ex, -5ex) {};
        \node[coordinate] (mid1) at (8.5ex, -7ex) {};
        \node[coordinate] (mid2) at (10ex, -5ex) {};
        \draw[-] (theta) -- (init);
        \draw[-] (theta) -- (func);
        \draw[-] (input) to[out=-90, in=180] (mid1);
        \draw[-] (mid1) to[out=0, in=-90] (mid2);
        \draw[-] (mid2) to[out=90, in=45] (theta);
    \end{tikzpicture}
    \vspace{-16.5pt}
\end{center}
where it accepts two child graphs, $\iota$ and $\mathrm{func}$, and
$\mathrm{func}$ further takes the $\theta$ node as one of its inputs to
form a complete cycle.  Evaluating a $\theta$ node produces a list of
values computed iteratively by the node's subgraph.  The first value in the
list, is the computed result of $\iota$, which we name $i$, and values in
the rest of the sequence are iteratively computed by $\mathrm{func}$.  In
functional programming, this is similar to iteratively computing the fixpoint
$\mathbf{fix}\,F$ of an initial list $[i]$, which is defined as:
\begin{equation}
    \mathbf{fix}\,F ([i]) = \lim_{n \to \infty} F^n ([i]),
    \quad\text{where~}
    F(v) = \mathrm{prepend}\left(
        i, \mathrm{map}\left( \mathrm{func}, v \right)
    \right).
\end{equation}
Here, $\mathrm{map}(\mathrm{func}, v)$ applies the subgraph computation
$\mathrm{func}$ to all elements in the list $v$, and $\mathrm{prepend}(i,
v^\prime)$ prepends the element $i$ to the list $v^\prime$.

For example, the following subgraph extracted from
Figure~\ref{bg:fig:factorial_peg} evaluates to the sequence $[1, 2, 3, 4,
\mathellipsis]$:
\begin{center}
    \vspace{-16.5pt}
    \begin{tikzpicture}
        \node (theta) at (0, 0) {$\theta_1$};
        \node (init) at (-5ex, -5ex) {$1$};
        \node (add) at (5ex, -5ex) {$+$};
        \node (one) at (0, -10ex) {$1$};
        \node[coordinate] (mid) at (8ex, -5ex) {};
        \draw[-] (theta) -- (init);
        \draw[-] (theta) -- (add);
        \draw[-] (add) -- (one);
        \draw[-] (add) to[out=-45, in=-90] (mid);
        \draw[-] (mid) to[out=90, in=45] (theta);
    \end{tikzpicture}
    \vspace{-16.5pt}
\end{center}

It is noteworthy that $\theta$ nodes may have subscripts.  For instance,
in Figure~\ref{bg:fig:factorial_peg}, both nodes $\theta_1$ share the same
subscript $1$.  This is used to indicate that the two sequences produced
by both $\theta$ nodes iterate simultaneously, \ie~they share the same
iteration count so that a new value for \verb|x| can be computed as we update
\verb|y|.  Therefore, the $\theta_1$ node in the left of this figure produces
a sequence of the factorials of $[1, 2, 3, \mathellipsis]$, \ie~$[1, 2, 6, 24,
\mathellipsis]$.

Computation nodes, such as arithmetic $+$ and Boolean operators $\leq$ and
$\neg$ in Figure~\ref{bg:fig:factorial_peg}, operates on a list of values, by
performing the computation on each value in the list.

For instance, the $\leq$ node accepts two inputs, the sequence $[1, 2, 3,
\mathellipsis]$ derived earlier, and a scalar $10$, computes the result of $x
\leq 10$ for each element $x$ in the sequence, and finally produces the list,
where $\truelit$ and $\falselit$ respectively denote true and false Boolean
values:
\begin{equation}
    [
        \truelit, \truelit, \truelit, \truelit, \truelit,
        \truelit, \truelit, \truelit, \truelit, \truelit,
        \falselit, \falselit, \falselit, \mathellipsis
    ].
\end{equation}
The subsequent $\neg$ node then negates all elements in the list:
\begin{equation}
    [
        \falselit, \falselit, \falselit, \falselit, \falselit,
        \falselit, \falselit, \falselit, \falselit, \falselit,
        \truelit, \truelit, \truelit, \mathellipsis
    ].
    \label{bg:eq:bool_seq}
\end{equation}

\subsubsection{Control-Flow Nodes}

The $\theta$ node encodes an infinite sequence of computed values, whereas
the output value of the program is a scalar.  By further representing
control-flow information in \gls{peg}, it becomes possible to refer to a
single value in this sequence, as the output of the program.  To do so,
Tate~\etal~\cite{tate09} further introduce $\mathrm{pass}$ and $\mathrm{eval}$
nodes.  The $\mathrm{pass}$ node finds the first true ($\truelit$) value
in a sequence of Boolean values, and returns the index of this value, and
$\mathrm{eval}$ takes two child nodes, where the first node evaluates to a list
$v$ of values, and the second is an scalar $n$ used to select a scalar value
from $l$, as the output of this node.

To illustrate, the $\mathrm{pass}_1$ node finds the first $\truelit$ in the
list~\eqref{bg:eq:bool_seq}, $11$.  The $\mathrm{eval}$ node of the output
variable \verb|y| subsequently fetches the $11$-th item from the list $[1, 2,
3, 4, \mathellipsis]$ we produced earlier, which is $11$.  Similarly we can
apply the same process to find that the output \verb|x| is $10!\,$, \ie~the
factorial of 10.

By using $\mathrm{pass}$ and $\mathrm{eval}$ nodes to represent the
control-flow in an algebraic fashion, and mixing data- and control-flows
together in the graphical representation, \glspl{peg} provide us with
greater opportunities to optimize data-flow across control-flow boundaries
and \emph{vice versa}.  Simple equivalence rules can be defined for these
nodes algebraically.  For instance, arithmetic operators can be distributed
over $\theta$ and $\mathrm{eval}$, \eg~$\mathrm{eval}(j, k) + i \equiv
\mathrm{eval}(j + i, k)$.  Complex transformations can therefore be deductively
constructed from these simple equivalence rules.

\subsubsection{Equivalence Finding}

By applying transformation passes to the \gls{peg}, their approach detects
incremental modifications, and appends these changes to the original
\gls{peg}\@.  The new changes, represented as extra structures in the
\gls{peg}, are linked to their corresponding equivalent nodes by equivalence
edges.  These edges indicate a pair of subgraphs are equivalent, forming a
\gls{epeg}, that could capture multiple \glspl{peg} in the same structure.  The
resulting \gls{epeg} is similar to the one in Figure~\ref{bg:fig:epeg}, where
dashed edges indicate equivalences.  It is notable that each edge allows a
binary choice, therefore the number of \glspl{peg} can be represented in an
\gls{epeg} could be exponential in the number of equivalence edges.
\begin{figure}[ht]
    \centering
    \begin{tikzpicture}[mir]
        \node (*1) [mirnode] at (0, 0) {$*$};
        \node (t1) [mirnodealt, below left=10mm of *1] {$\theta$};
        \node (51) [mirnode, below right=of *1] {$5$};
        \node (01) [mirnode, below left=of t1] {$0$};
        \node (f1) [mirnode, below right=of t1] {$\phi$};
        \node (d1) [mirnode, below left=of f1] {$\delta$};
        \node (+1) [mirnodealt, below=5mm of f1] {$+$};
        \node (31) [mirnode, below left=of +1] {$3$};
        \node (+2) [mirnodealt, below right=of +1] {$+$};
        \node (11) [mirnode, below left=of +2] {$1$};
        \draw[-] (*1) -- (t1.60);
        \draw[-] (*1) -- (51);
        \draw[-] (t1) -- (f1) -- (+1) -- (+2) -- (11);
        \draw[-] (t1) -- (01);
        \draw[-] (f1) -- (d1);
        \draw[-] (f1) to[bend left] (+2);
        \draw[-] (+1) -- (31);
        \draw[-] (+2) -- (11);
        \draw[-] (+2) to[out=-45, in=45, looseness=2] (t1);

        \node (t2) [mirnode, right=20mm of *1] {$\theta$};
        \draw[dashed] (*1) -- (t2);
        \node (*2) [mirnode, below left=of t2] {$*$};
        \node (02) [mirnode, right=of *2] {$0$};
        \node (03) [mirnode, below left=of *2] {$0$};
        \node (52) [mirnode, below right=of *2] {$5$};
        \draw[-] (t2) -- (*2) -- (03);
        \draw[-] (*2) -- (52);
        \draw[dashed] (*2) -- (02);

        \node (*3) [mirnode, below right=of t2] {$*$};
        \node (53) [mirnode, below right=of *3] {$5$};
        \draw[-] (t2) -- (*3) -- (53);
        \draw[-] (*3)
            .. controls ($(*3) +(-0.5,-1.5)$) and ($(f1) +(1,1)$) .. (f1);

        \node (f2) [mirnode, right=15mm of *3] {$\phi$};
        \node (d2) [mirnode, below left=of f2] {$\delta$};
        \node (*4) [mirnode, below=of f2] {$*$};
        \node (54) [mirnode, below right=of *4] {$5$};
        \draw[dashed] (*3) -- (f2);
        \draw[-] (f2) -- (*4) -- (54);
        \draw[-] (f2) -- (d2);
        \draw[-] (*4)
            .. controls ($(*4) +(-1,-1)$) and ($(+1) +(1,1)$) .. (+1);

        \node (+3) [mirnode, right=15mm of *4] {$+$};
        \node (*5) [mirnode, below left=of +3] {$*$};
        \node (*6) [mirnode, below right=of +3] {$*$};
        \node (151) [mirnode, right=of *5] {$15$};
        \node (32) [mirnode, below left=of *5] {$3$};
        \node (55) [mirnode, below right=of *5] {$5$};
        \node (56) [mirnode, below right=of *6] {$5$};
        \draw[dashed] (*4) -- (+3);
        \draw[dashed] (*5) -- (151);
        \draw[-] (+3) -- (*5) -- (32);
        \draw[-] (*5) -- (55);
        \draw[-] (f2)
            .. controls ($(f2) +(0.5,-0.5)$) and ($(*6) +(-.5,1.5)$) .. (*6)
            .. controls ($(*6) +(-1,-2)$) and ($(+2) +(1,1)$) .. (+2);
        \draw[-] (+3) -- (*6) -- (56);

        \node (+4) [mirnode, right=15mm of *6] {$+$};
        \node (*7) [mirnode, below left=of +4] {$*$};
        \node (57) [mirnode, right=of *7] {$5$};
        \node (12) [mirnode, below left=of *7] {$1$};
        \node (58) [mirnode, below right=of *7] {$5$};
        \draw[dashed] (*6) -- (+4);
        \node (m1) [coordinate, right=of +4] {};
        \draw[-] (+4) to[out=-45, in=-90, looseness=1.5] (m1)
            .. controls ($(m1) +(0,1)$) and ($ (t2) +(1,1)$) .. (t2);
        \draw[-] (+4) -- (*7) -- (12);
        \draw[-] (*7) -- (58);
        \draw[dashed] (*7) -- (57);
    \end{tikzpicture}
    \caption{%
        An simple \gls{epeg} example, taken from~\cite{tate09}.
    }\label{bg:fig:epeg}
\end{figure}

By repeatedly applying all possible passes to the \gls{epeg}, this graph will
eventually saturate, \ie~no more equivalent structure can be added to the graph
because all possible equivalences are now discovered.  This process and the
resulting \gls{epeg} is more space-time efficient than enumerating all possible
\glspl{peg} along the path, because \gls{epeg} encourages sharing common
subgraphs, even across equivalent edges.  This saturated graph can always be
produced regardless of in what order we apply the passes, hence preventing
the phase-ordering problem.  Furthermore, \gls{epeg} defers the decision
of whether an optimization should be committed until we have reached full
saturation, allowing the global optima to be discovered.  In contrast, because
each optimization pass in a conventional compiler is performed once, the
compilers must make the decision to commit changes immediately after applying
the optimization, which consequently often results in local optima.

\section{Structural Optimization}
\label{lo:sec:structural_optimization}

From a numerical program, we can generate a \gls{mir} using the translation
process in Section~\ref{lo:sec:intermediate}.  The next step is to transform
the \gls{mir}, and discover \glspl{mir} that are equivalent to the original
\gls{mir} in real arithmetic, but may execute differently in finite-precision
arithmetic because of round-off errors.

\subsection{Improved Algorithm}
\label{lo:sub:algorithm}

% Our optimization starts by partitioning the \gls{mir} into sub-\glspl{mir}.
% This further reduces the size of the search space, \ie~equivalent \glspl{mir}
% reachable using our transformation rules.  This speeds up the rest of the
% optimization process, because these rules are not applied on partition
% boundaries.

As discussed in previous chapters, even a small expression could have a huge
number of equivalent ones.  Exhaustively discovering all equivalent \glspl{mir}
would result in combinatorial explosion of the number of equivalent \glspl{mir}
in the search space.  For this reason, we base ourselves on an algorithm from
Section~\ref{po:sec:equivalence_analysis} of Chapter~\ref{chp:progopt} that
searches efficiently by discovering equivalences in a bottom-up hierarchy.  In
this section, we discuss two major improvements to the algorithm which further
increase its performance.

\subsubsection{Partitioning}

Instead of optimizing the \gls{mir} immediately, we start by partitioning the
\gls{mir} into multiple smaller sub-\glspl{mir}.  The partition boundaries are
determined by $\updateop$ operators.  For instance, we consider the partially
unrolled Fibonacci example in Figure~\ref{lo:fig:fib2}.  The \gls{mir} of the
loop body is shown in Figure~\ref{lo:fig:fib_mir2}.  The partition boundaries
are indicated by the region surrounded by the red dotted curve \tikz{\draw[red,
dashed, rounded corners=1ex] (0,0) rectangle (20pt,2ex);}.  A multiply shared
subexpression, such as $\texttt{i} - 1$, also determines the partition boundary
by merging its partition with one of its parents with the smallest partition by
node count.  If all parents contain the same number of nodes then a choice is
made randomly.
\begin{figure}[ht]
    \centering
    \newsavebox{\fiblstb}
    \begin{lrbox}{\fiblstb}
        \begin{lstlisting}
for (int i=3; i<1023; i+=2)
{
    A[i-1] = A[i-2]+A[i-3];
    A[i] = A[i-1]+A[i-2];
}
        \end{lstlisting}
    \end{lrbox}
    \subfloat[The partially unrolled loop.]{%
        \begin{minipage}{0.45\textwidth}
            \usebox{\fiblstb}
        \end{minipage}\label{lo:fig:fib2}
    }
    \subfloat[The \gls{mir} of the partially unrolled loop body.]{%
        \begin{tikzpicture}[mir]
            \node[mirnode] (mA) at (0, 0) {\texttt{A}};
            \node[mirnode] (upd) [right=of mA] {$\updateop$};
            \node[mirnode] (updi) [below=of upd] {\texttt{i}};
            \node[mirnode] (+) [below right=of upd] {$+$};
            \node[mirnode] (acc1) [below left=of +] {$\accessop$};
            \node[mirnode] (acc2) [below right=of +] {$\accessop$};
            \node[mirnode] (-1) [below right=6mm of acc1] {$-$};
            \node[mirnode] (1) [below right=of -1] {$1$};
            \node[mirnode] (-1i) [below left=of -1] {\texttt{i}};
            \draw[<-] (upd) -- (+);
            \draw[<-] (+) -- (acc1);
            \draw[<-] (+) -- (acc2);
            \draw[<-] (acc1) -- (-1);
            \draw[<-] (-1) -- (1);
            \draw[<-] (upd) -- (updi);
            \draw[<-] (-1) -- (-1i);

            \node[mirnode] (2upd) [below left=8mm of acc1] {$\updateop$};
            \node[mirnode] (2+) [below right=8mm of 2upd] {$+$};
            \node[mirnode] (2acc1) [below left=of 2+] {$\accessop$};
            \node[mirnode] (2acc2) [below right=of 2+] {$\accessop$};
            \node[mirnode] (-3) [below right=6mm of 2acc1] {$-$};
            \node[mirnode] (3) [below right=of -3] {$3$};
            \node[mirnode] (-3i) [below left=of -3] {\texttt{i}};
            \node[mirnode] (Ain) [below left=8mm of 2acc1] {\texttt{A}};
            \draw[<-] (2upd) -- (2+);
            \draw[<-] (2+) -- (2acc1);
            \draw[<-] (2+) -- (2acc2);
            \draw[<-] (2acc1) -- (-3);
            \draw[<-] (2acc1) -- (Ain);
            \draw[<-] (-3) -- (3);
            \draw[<-] (2upd) -- (Ain);
            \draw[<-, channel] (2upd) to[out=-70, in=180] (-1);
            \draw[<-, channel] (2acc2) -- (Ain);
            \draw[<-] (-3) -- (-3i);

            \node[mirnode] (-2) [below right=6mm of acc2, yshift=-5mm] {$-$};
            \node[mirnode] (2) [below right=of -2] {$2$};
            \node[mirnode] (-2i) [below left=of -2] {\texttt{i}};
            \draw[<-] (2acc2) to[out=0, in=180] (-2);
            \draw[<-] (-2) -- (-2i);
            \draw[<-] (-2) -- (2);
            \draw[<-] (acc2) -- (-2);

            \draw[<-] (upd) -- (2upd);
            \draw[<-] (acc1) -- (2upd);
            \draw[<-, channel] (acc2) -- (2upd);

            \node[mirnode] (mi) [right=15mm of upd] {\texttt{i}};
            \node[mirnode] (i+2) [right=of mi] {$+$};
            \node[mirnode] (i+2 i) [below left=of i+2] {\texttt{i}};
            \node[mirnode] (i+2 2) [below right=of i+2] {$2$};
            \draw[|->] (mA) -- (upd);
            \draw[|->] (mi) -- (i+2);
            \draw[<-] (i+2) -- (i+2 i);
            \draw[<-] (i+2) -- (i+2 2);

            \draw[red, dashed, rounded corners]
                (3.south east) -- (-3i.south) --
                (Ain.south west) -- (Ain.north west) --
                (2upd.north west) -- (2upd.north east) --
                (2acc2.north east) -- cycle;
            \draw[red, dashed, rounded corners]
                (2.north east) -- (2.south east) -- (-2i.south west) --
                (-1i.south west) -- (acc1.south west) --
                (upd.north west) -- (upd.north east) --
                (acc2.north east) -- cycle;
            \draw[red, dashed, rounded corners]
                (i+2.north) -- (i+2 i.north west) -- (i+2 i.south west) --
                (i+2 2.south east) -- (i+2 2.north east) -- cycle;
            \brackets{(current bounding box)};
        \end{tikzpicture}
        {}\label{lo:fig:fib_mir2}
    }
    \caption{An example to illustrate how \acrshortpl{mir} are partitioned.}
\end{figure}

Because transformation rules can only be applied to each partition but not
across them, the size of the search space can be reduced further.  In turn,
each are optimized separately and generate a set of partitions equivalent to
the original.  We then select combinations from these partitions to be merged,
this generates a set of \glspl{mir} that are equivalent to the original.
Finally, we preserve those \glspl{mir} merged on the Pareto frontier.

\subsubsection{Optimization}

% Figure~\ref{lo:alg:optimize} shows the pseudocode of the optimization
% algorithm.  It takes as an input a \gls{mir} graph, and produces a set of
% equivalent graphs that are estimated to be Pareto-optimal when converted into
% C programs and synthesized into circuits.  Although this algorithm deals with
% a special case, \ie~a root node $op$ with two child subtrees $e_1, e_2$, it
% can easily be generalized to an arbitrary number of child subtrees.  Here,
% $e \stackrel{r}{\eqgenrel} e'$ means $e^\prime$ can be obtained by
% transforming part of the graph $e$ in accordance with the transformation rule
% $r$.  The next section discusses the transformation rules we used.

% The algorithm starts by discovering equivalences in the leaves of a
% \gls{mir}, and progresses upwards for equivalent structures of the individual
% components that make up the graph, until the roots of the graph, where we
% have a set of \glspl{mir} equivalent to the original \gls{mir}\@.  As it
% traverses through the \gls{mir}, the algorithm calculates the performance
% metrics at each node, using the analyses presented in the next section.
% Transformations that are not Pareto-optimal are immediately pruned from the
% search space, thus reducing the average complexity of the algorithm.

Previously in \soap, as we optimize parts of \glspl{mir}, the Pareto
frontier is used to filter discovered equivalent candidates (\glspl{mir} or
semantic expressions), which keeps the size of the set relatively small and
manageable.  As we optimize larger programs, the run time of the tool increases
significantly.  Currently, the \soap{} framework prunes the \glspl{mir} that
are Pareto-suboptimal, leaving only those that are on the Pareto frontier.
However, because our Pareto frontier is three-dimensional, there is a large
increase in the number of Pareto-optimal \glspl{mir}.  This Pareto pruning
approach is no longer feasible for our benchmark examples.  Therefore in this
chapter, not only do we use the Pareto frontier to filter candidates, we also
introduce a \textsc{Prune} function to further reduce the size of Pareto
frontier.

We rely on the \textsc{Prune} function to efficiently steer the direction
of our Pareto frontier as we discover new candidates.  It takes as an
input the set of Pareto-optimal equivalent candidates that we have
discovered, and prunes elements in this set to reduce its size by sampling,
keeping the number of discovered \glspl{mir} tractable.  The pruning
algorithm is inspired by Poisson-disk sampling algorithm~\cite{bridson07}.
Our algorithm in Figure~\ref{lo:alg:sample} starts by first randomly
selecting one point from the Pareto frontier $\epsilon_0$ (denoted by
$\textsc{RandomSample}(\epsilon_0)$).  It then grows the set of points by
adding the neighbours from the point that are separated by at least a certain
distance $\delta$, where the distance $\delta$ is decreased iteratively by
a factor $\zeta = 0.8$, until $\epsilon$ contains at least $\eta = 20\%$
of all points in the Pareto frontier, or a maximum number of attempts
($\mathrm{attempt\_count}$) is reached.  The distance between two options $e$
and $e^\prime$ is computed as follows:
\begin{equation}
    \mathrm{dist} \left(e, e^\prime\right)
    = \sqrt{
        \sum_{f \in F} {\left(
            \frac{f(e) - f(e^\prime)}{\max(f(e), f(e^\prime))}
        \right)}^2
    },
\end{equation}
where $f \in F$ enumerates each function that evaluates the performance of a
candidate $e$, \ie~the function $f \in F$ computes either the accuracy, area
or latency of $e$.

\begin{figure}[ht]
    \centering
    \begin{algorithmic}
        \singlespacing%
        \Function{Sample}{$\epsilon_0$}
            \State{$\delta = 1.0$}
            \State{%
                $\epsilon = \left\{
                    \textsc{RandomSample}\left(\epsilon_0\right)
                \right\}$}
            \For{%
                $i = 1, 2, \mathellipsis, \mathrm{attempt\_count}$
            }
                \For{$e \in \epsilon_0$}
                    \State{$\mathrm{in\_range} = \falselit$}
                    \For{$e^\prime \in \epsilon$}
                        \If{$\mathrm{dist}\left(e, e^\prime\right) < \delta$}
                            \State{$\mathrm{in\_range} = \truelit$}
                            \State\textbf{break}
                        \EndIf%
                        \If{$\neg \mathrm{in\_range}$}
                            \State{%
                                $\epsilon = \epsilon \cup \left\{ e \right\}$}
                        \EndIf%
                    \EndFor%
                \EndFor%
                \If{$\left|\epsilon\right| \geq \eta \left|\epsilon_0\right|$}
                    \State\textbf{break}
                \EndIf%
                \State{$\delta = \zeta\delta$}
            \EndFor%
            \State{\Return{$\epsilon$}}
        \EndFunction%
    \end{algorithmic}
    \caption{%
        The algorithm used to sample the Pareto frontier.
    }\label{lo:alg:sample}
\end{figure}

This method is superior to random sampling, because random sampling often
samples points that are close together, which usually are very similar
implementations.

We found that with all improvements above and a faster accuracy analysis in
Section~\ref{lo:sub:accuracy}, the algorithm is significantly faster than
the original optimization algorithm in Section~\ref{po:sub:discovering} of
Chapter~\ref{chp:progopt}.  Even though this algorithm may discover potentially
fewer candidates on the Pareto frontier, we can now explore greater partial
loop unrolling depths to widen the swing of the Pareto frontier in the same
amount of time.

% \todo{Move this algorithm to Chapter~stropt.}
% \begin{figure}[ht]
%     \centering
%     \begin{algorithmic}
% \Function{Optimize}{$op(e_1, e_2)$}
%     \State{$s_1 \gets$ \Call{Optimize}{$e_1$}}
%     \State{$s_2 \gets$ \Call{Optimize}{$e_2$}}
%     \State{$s^\prime \gets \varnothing$}
%     \State{%
%         $s \gets \left\{
%             op(e^\prime_1, e^\prime_2) \mid
%             e^\prime_1 \in s_1 \wedge e^\prime_2 \in s_2
%         \right\}$}
%     \While{$s \neq s^\prime$}
%         \State{$s^\prime \gets s$}
%         \State{$s^{\prime\prime} \gets \varnothing$}
%         \For{$r \in \mathrm{transformation\_rules}, e \in s$}
%             \For{%
%                 $e^\prime \text{~where~}
%                     e \stackrel{r}{\eqgenrel} e^\prime$
%             }
%                 \State{%
%                     $s^{\prime\prime} \gets
%                         s^{\prime\prime} \cup \left\{ e^\prime \right\}$}
%             \EndFor
%         \EndFor
%         \State{$s \gets$ \Call{Prune}{$s^{\prime\prime}$}}
%     \EndWhile
%     \State{\Return{$s$}}
% \EndFunction
%     \end{algorithmic}
%     \caption{%
%         The algorithm we used for the efficient discovery of equivalent
%         structures in \glspl{mir}.
%     }
% \end{figure}


\subsection{Transformation Rules}
\label{lo:sub:transformation_rules}

This section details the new transformation rules in the equivalence relation
$\equiv$ and consequently in the structural optimization relation $\eqgenrel$.
Each transformation rule on its own is not revolutionary, but for the
first time, they are used in tandem with arithmetic rules and control-flow
restructuring rules introduced respectively in Chapters~\ref{chp:stropt}
and~\ref{chp:progopt}.  This enables a much better automatic structural
optimization on the latency, resource usage and accuracy of numerical programs,
than is possible using only a subset of them.

In Chapters~\ref{chp:stropt} and~\ref{chp:progopt}, \soap{} provides a range
of equivalence rules that are used in the optimization, such as associativity,
distributivity, commutativity, constant propagation, and partial loop
unrolling.  In Table~\ref{lo:tab:rules}, we list those rules that proved
effective when minimizing loop latencies.  Although these rules are used to
transform \glspl{mir}, we present before-and-after examples written in C to
allow the effect of each rule to be readily understood.
\begin{table}[t]
    \caption{%
        Before-and-after examples to demonstrate the access reduction rules.
    }\label{lo:tab:rules}
    \centering
    \begin{tabular}{ll}
        \toprule
        \multicolumn{2}{c}{\textbf{Access Reduction Rules}}
        \\\midrule\midrule
        \Shade\emph{Multiple reads} & \Shade%
            \texttt{x=A[i-{}-]; y=A[i+1];} $~\eqgenrel~$
            \texttt{x=A[i-{}-]; y=x;}
        \\\midrule
        \emph{Multiple writes} &
            \texttt{A[i++]=x; A[i-1]=y;} $~\eqgenrel~$
            \texttt{A[i++]=y;}
        \\\midrule
        \Shade\emph{Read after write} & \Shade%
            \texttt{A[i++]=x; y=A[i-1];} $~\eqgenrel~$
            \texttt{A[i++]=x; y=x;}
        \\\midrule
        \emph{Indep.\ accesses} (where $\texttt{i}\not\equiv\texttt{j}$) &
            \texttt{A[i]=x; y=A[j];} $~\eqgenrel~$
            \texttt{y=A[j]; A[i]=x;}
        \\\bottomrule%
    \end{tabular}%
\end{table}%

Our new rules, the access reduction rules, with formal definitions below and
examples in Table~\ref{lo:tab:rules}, remove extraneous data-dependences that
arise after partial unrolling.  These rules, along with partial loop unrolling,
mostly do not really impact latency, because they are very well studied in
polyhedral loop dependence analysis, and tools such as \gls{vhls} can make use
of them automatically.  However, they give the necessary freedom to arithmetic
rules to affect latency.  The rules are as follows, where $A$ is an array,
$\bar{\imath}, \bar{\jmath}$ are subscripts, and $e, e^\prime$ are expressions:
\begin{itemize}

    \item \emph{Multiple reads}, eliminates the second of two reads of the
    same location.  This arises naturally from the \gls{mir}, as common
    subexpressions are shared.

    \item \emph{Multiple writes}, eliminates a write that is overwritten:
    \vspace{-11pt}
    \begin{equation}
        \update{\update{A}{\bar{\imath}}{e}}{\bar{\imath}}{e^\prime}
            \eqgenrel \update{A}{\bar{\imath}}{e^\prime}.
    \end{equation}

    \item \emph{Read after write}, eliminates a read from a location
    that has just been written:
    \vspace{-11pt}
    \begin{equation}
        \access{\update{A}{\bar{\imath}}{e}}{\bar{\imath}} \eqgenrel e.
    \end{equation}

    \item \emph{Independent accesses}, allows two array operations to be
    reordered if it can be proved that they never access the same location:
    \vspace{-11pt}
    \begin{equation}
        \access{\update{A}{\bar{\imath}}{e}}{\bar{\jmath}}
            \eqgenrel \access{A}{\bar{\jmath}},
        \text{if~} \bar{\imath} \not\equiv \bar{\jmath}.
    \end{equation}
    We also visualize this rule in Figure~\ref{lo:fig:indep_example}, which
    shows a sample \gls{mir} transformation.
\end{itemize}

\begin{figure}[ht]
    \begin{equation*}
        \begin{tikzpicture}[mir]
            \node[mirnode] (var_y) at (0,0) {\texttt{y}};
            \node[mirnode] (access)[right=of var_y] {$\accessop$};
            \node[mirnode] (j1)    [below right=of access] {\texttt{j}};
            \node[mirnode] (update)[below left=of access] {$\updateop$};
            \node[mirnode] (var_A) [left=of update] {\texttt{A}};
            \node[mirnode] (a2)    [below left=of update] {\texttt{A}};
            \node[mirnodealt] (i2) [below=5mm of update] {\texttt{i}};
            \node[mirnode] (x2)    [below right=of update] {\texttt{x}};

            \draw[|->] (var_y) -- (access);
            \draw[|->] (var_A) -- (update);
            \draw[<-] (access) -- (j1);
            \draw[<-] (access) -- (update);
            \draw[<-] (update) -- (x2);
            \draw[<-] (update) -- (i2);
            \draw[<-] (update) -- (a2);
            \brackets{(current bounding box)}
        \end{tikzpicture}
        \eqgenrel
        \begin{tikzpicture}[mir]
            \node[mirnode] (var_y) at (0,0) {\texttt{y}};
            \node[mirnode] (access)[right=of var_y] {$\accessop$};
            \node[mirnode] (j1)    [below right=of access] {\texttt{j}};
            \node[mirnode] (update)[below left=of access] {$\updateop$};
            \node[mirnode] (var_A) [left=of update] {\texttt{A}};
            \node[mirnode] (a2)    [below left=of update] {\texttt{A}};
            \node[mirnodealt] (i2) [below=5mm of update] {\texttt{i}};
            \node[mirnode] (x2)    [below right=of update] {\texttt{x}};

            \draw[|->] (var_y) -- (access);
            \draw[|->] (var_A) -- (update);
            \draw[<-] (access) -- (j1);
            \draw[<-] (update) -- (x2);
            \draw[<-] (update) -- (i2);
            \draw[<-] (update) -- (a2);
            \draw[<-, channel] (access) to[bend left=30] (a2);
            \brackets{(current bounding box)}
        \end{tikzpicture}
    \end{equation*}
    \caption{%
        A sample \acrshort{mir} transformation using the \emph{independent
        accesses} rule.
    }\label{lo:fig:indep_example}
\end{figure}

These rules may not seem powerful on their own, but when combined with other
structural rules, they enable \soap{} to detect dependences that can be removed
in the \gls{mir}\@.  This in turn allows more opportunities for the rules
to further reduce loop latency.  By way of illustration, we optimize the
Fibonacci series example program in Figure~\ref{lo:fig:fib} for latency.
By partially unrolling the loop with a factor 2, we obtain the program in
Figure~\ref{lo:fig:fib2}.  We can see that because of the rigid array access
pattern, associativity cannot be applied easily to the loop kernel.  However,
by applying the above access reduction rules first, we give associativity the
freedom to reduce latency by half and improve accuracy by 50\%, as shown in
Figure~\ref{lo:lst:fib_opt}.
\begin{figure}[ht]
    \centering
    \begin{lstlisting}
    for (int i = 2; i < 1023; i += 2) {
        float t2 = A[i - 2], t3 = A[i - 3];
        A[i - 1] = t2 + t3;
        A[i] = 2 * t2 + t3;
    }
    \end{lstlisting}
    \caption{%
        The optimized program that computes the Fibonacci sequence.  It reduces
        latency of the original in Figure~\ref{lo:fig:fib} by half and improves
        accuracy by 50\%.
    }\label{lo:lst:fib_opt}
\end{figure}

Without the above access reduction rules, it is therefore not possible to
reach this optimized implementation.  Conversely, it is not possible to
relax scheduling constraints due to inter-iteration dependences without
arithmetic equivalence rules, as these reduction rules are there to assist
transformation rules that make a difference in latency.  Therefore the rules
in Table~\ref{lo:tab:rules} must be used in conjunction with arithmetic and
control-flow rules to optimize latency in numerical programs.

\section{Performance Analysis}
\label{lo:sec:performance_analysis}

This section explains how we analyze \glspl{mir} for our three performance
metrics: latency, resource usage, and accuracy.

\subsection{Latency Analysis}
\label{lo:sub:latency}

The purpose of our latency analysis is not to create a complete scheduling
of numerical programs, as this would be computationally expensive, and would
need to be repeated for tens of thousands of equivalent programs.  Instead, it
computes a lower bound of the loop's \gls{ii}, the \acrfull{mii}.  (Recall that
the initiation interval is the number of clock cycles that must elapse between
the starts of two consecutive loop iterations, and is determined by data
dependences and resource constraints.)  We then compute the overall latency of
the loop, and subsequently, the total latency of the program.

Following LegUp~\cite{legup}, we compute \gls{mii} values using the first
few steps of modulo \gls{sdc} scheduling~\cite{canis14} introduced in depth
in Section~\ref{bg:sub:sdc} of Chapter~\ref{chp:background}, by viewing
\glspl{mir} as dependence graphs, as the structure of \glspl{mir} already
captures intra-iteration data-dependences.  In addition to this, we add extra
latency information as attributes on the edges of \glspl{mir}, plus new edges
to form cycles that capture inter-iteration data-dependences.  The analysis is
carried out in three stages.

The analysis starts with the \gls{mir} of the loop under analysis. Each edge
in the \gls{mir}, say $s\rightarrow t$, represents a data-dependence: the
operation at node $s$ must be evaluated fully before the operation at $t$
can begin.  The first step is to add a pair $\pair{l}{d}$ for each edge of
the \gls{mir}\@.  Here, $l$ is the \emph{latency} of the edge (the number of
\emph{clock cycles} that must elapse between the start of $s$ and the start
of $t$) and $d$ is the \emph{dependence distance} (the number of \emph{loop
iterations} that must elapse between the start of $s$ and the start of $t$).
Because all operations in the \gls{mir} are performed in a single iteration,
all edges have $d=0$.  The value of $l$ is given by the latency of the
operation at node $s$; if $s$ corresponds to an input variable or a numerical
constant, then $l=0$.

The second stage is to add edges to form a cyclic dependence graph that
captures \gls{raw} dependences across loop iterations.  This step involves
checking whether each pair of ``$\accessop$'' and ``$\updateop$'' nodes
has a dependence, and if so, adding a new edge between them with latency
and dependence distance attributes.  As an example, consider the \gls{mir}
in~\eqref{lo:eq:array_example} and assume each iteration increments \verb|i|
by 1.  Because in the original program, \verb|A[i]| and \verb|A[i+1]| are
respectively reading from and writing to the same array \verb|A|, we need to
check if these accesses could touch the same memory location in different
iterations.  For this, our analysis formulates an \gls{ilp} problem for
the dependence distance, and solves it using the \gls{isl}~\cite{isl}.
In this example, the dependence distance is 1 because the value written
to \verb|A[i+1]| in the current iteration \verb|i| is immediately used
in the next iteration \verb|i+1|.  Similarly, we also add new edges
for reads and writes to the same variable, which can be treated as a
special array with only one element. Our analysis yields the \gls{mir} in
Figure~\ref{lo:fig:example_latency}.

\begin{figure}[ht]
    \centering
    \begin{tikzpicture}[mir, node distance=8mm, inner sep=0]
        \node[mirnode] (A) at (0,0) {\texttt{A}};
        \node[mirnode, inner sep=1.5mm] (update) [right=of A] {$\updateop$};
        \node[mirnode] (A1)    [below left=of update] {\texttt{A}};
        \node[mirnode] (plus)  [below=of update] {$+$};
        \node[mirnode] (i1)    [below left=of plus] {\texttt{i}};
        \node[mirnode] (n1)    [below right=of plus] {$1$};
        \node[mirnode] (times) [below right=of update, xshift=11mm] {$\times$};
        \node[mirnode] (n2)    [below left=of times] {$2$};
        \node[mirnode] (access)[below right=of times] {$\accessop$};
        \node[mirnode] (A2)    [below left=of access] {\texttt{A}};
        \node[mirnode] (i2)    [below right=of access] {\texttt{i}};

        \draw[|->] (A) -- (update);
        \draw[<-] (update) to[auto, swap, pos=0.8] node{\smallpair00} (A1);
        \draw[<-] (update) to[auto] node{\smallpair{10}0} (plus);
        \draw[<-] (update) to[auto] node{\smallpair70} (times);
        \draw[<-] (plus) to[auto, swap] node{\smallpair00} (i1);
        \draw[<-] (plus) to[auto] node{\smallpair00} (n1);
        \draw[<-] (times) to[auto, swap] node{\smallpair00} (n2);
        \draw[<-] (times) to[auto] node{\smallpair20} (access);
        \draw[<-] (access) to[auto, swap] node{\smallpair00} (A2);
        \draw[<-] (access) to[auto] node{\smallpair00} (i2);
        \brackets{(current bounding box)}
\begin{pgfinterruptboundingbox}
        \draw[<-,dashed] (access) to[auto, swap, pos=0.2, out=45, in=30]
        node{\smallpair{-2}1} (update);
\end{pgfinterruptboundingbox}
    \end{tikzpicture}
    \caption{%
        The \acrshort{mir} with edges labelled with latency attributes.
    }\label{lo:fig:example_latency}
\end{figure}

Note the new dashed edge from the $\updateop$ node to the $\accessop$ node,
which is labeled $\pair{-2}{1}$.  The first value, $-2$, signifies that the
latency of the edge between $\times$ and $\accessop$, which is 2 cycles, is
canceled out because the multiplier can reuse its output from the previous
iteration as the input for the current iteration. The second value, $1$,
indicates that there is a data flow dependence from iteration $i$ to iteration
$i+1$.

We assume no limit on the number of operators we can allocate, so operators
do not constraint \gls{ii}.  However, in \gls{vhls}, each array is usually
translated into a dual-port \gls{ram}, which allows only two accesses per
clock cycle~\cite{vivado_hls}, and thus constraints \gls{mii}.  Following
Section~\ref{bg:sub:sdc}, we evaluate:
\begin{equation}
    \ResMII = \max_{a\,\in\,A} \left\lceil \frac{n_a}{r_a} \right\rceil,
\end{equation}
where $a \in A$ ranges over all arrays in the loop body, $n_a$ is the number of
accesses to the array $a$, \ie~the number of shared $\accessop$ and $\updateop$
nodes accessing the array $a$ in the dependence graph, and $r_a = 2$ is the
maximum number of accesses allowed per cycle per array.

The final step is to calculate \acrfull{recmii} which is defined in
Section~\ref{bg:sub:sdc} as:
\begin{equation}
    \RecMII = \max_{c\,\in\,C} \left\lceil \frac{l_c}{d_c} \right\rceil,
\end{equation}
where $c \in C$ ranges over all cycles in the graph, and $l_c$ and $d_c$ are
respectively the sums of all latencies and dependence distances of the edges in
the cycle.  Because a typical \gls{mir} with array accesses could have a very
large number of cycles, we efficiently search for an \gls{mii} using a modified
Floyd--Warshall algorithm~\cite{floyd62}, following~\cite{rau94}.

Finally, we estimate the total latency $L$ of the loop with:
\begin{equation}
    L = (N - 1) \MII + D,
    \text{~where~}
        \MII = \max\left(\RecMII, \ResMII\right).
    \nonumber
\end{equation}
Recalling from Section~\ref{bg:sub:sdc} in Chapter~\ref{chp:background}, $N$ is
the maximum \emph{trip count}, \ie~the loop's total number of iterations, and
$D$ is the loop's depth, \ie~the total number of cycles per iteration.

Because we optimize \glspl{mir} in a bottom-up hierarchy, when an
expression that does not constitute a inter-iteration dependence is
optimized, its latency is estimated by scheduling its operations by using
an \gls{alap}~\cite{wang_hls} scheduling algorithm, where each operation
is scheduled to the latest opportunity, while respecting the order of data
dependences.

Because the expression is eventually used in a loop, and the \gls{ii} of
the loop is critical to how fast the loop can execute, it is necessary
to start optimizing for \gls{ii} as soon as possible.  Therefore, in our
latency analysis of a \gls{mir} or an expression that is a fragment of a
inter-iteration dependence cycle, our algorithm automatically shortens any
paths between any pairs of dependent accesses in the \gls{mir}\@, as we use the
latency analysis as a component to manoeuvre our optimization on the Pareto
frontier.  Moreover, we place greater weights on dependent accesses with
smaller dependence distances, because these impact the resulting loop \gls{ii}
more significantly than larger distances.  For instance, consider a loop body
that has two dependent accesses across iterations, \ie~the graph contains two
cycles:
\begin{lstlisting}
    A[i] = $f\left( \texttt{A[i-1]},\,\texttt{A[i-2]} \right)$;
\end{lstlisting}
Here as we optimize this program in a bottom-up hierarchy, $f$ is optimized
before the loop body.  For \gls{mii} considerations, the subexpression tree $f$
is thus rewritten such that the following latency cost is minimized:
\begin{equation}
    L = \max \left(
        \frac{l_1}{d_1}, \frac{l_2}{d_2}
    \right).
\end{equation}
Here, $l_1$ and $l_2$ are respectively the lengths
of the longest latency-weighted paths from the nodes
$\access{\texttt{A}}{\overline{\texttt{i}-1}}$ and
$\access{\texttt{A}}{\overline{\texttt{i}-2}}$ to the root of $f$, and $d_1$
and $d_2$ are respectively the dependence distances of the two nodes to the
write of \verb|A|, where $d_1 = 1$ and $d_2 = 2$.


\subsection{Resource Utilization Analysis}
\label{lo:sub:resource}

The hardware resource usage analysis of Chapter~\ref{chp:stropt} captures the
sharing of common subexpressions, but cannot analyze resource binding, which
allows common operations to be shared across clock cycles. For instance, in the
floating-point expression $\vara + (\varb + \varc)$, the two additions can be
computed using one addition operator only.  In this section, we develop a new
resource usage analysis that fully understands how resources are shared in an
\gls{fpga} implementation of numerical programs.

We rely on the foundation of resource usage analysis from
Chapters~\ref{chp:stropt} and~\ref{chp:progopt}, which counts the number
$n_\opsymbol$ of each type of operation $\opsymbol \in \opset$, while maximally
sharing common subexpressions.  In a pipelined loop, we compute a lower bound
$a_\opsymbol$ on the number of instances of $\opsymbol$ that must be allocated,
using the following formula:
\begin{equation}
    a_\opsymbol = \left\{
        \begin{aligned}
            & \left\lceil \frac{n_\opsymbol}{\MII} \right\rceil
            && \quad \text{if $\opsymbol$ is shared}, \\
            & \quad n_\opsymbol
            && \quad \text{otherwise}.
        \end{aligned}
    \right.
\end{equation}
Here, integer operators are typically not shared~\cite{cong15}, so the number
of operations is the number of allocated instances.

For instance, if we know that a pipelined loop has $\MII = 3$, and each
iteration uses 6 multiplications, then we can compute that we need to
synthesize at least 2 multipliers.

For straight-line code, non-pipelined loops, and different loops, we use a
simple \gls{alap} scheduling~\cite{wang_hls} to estimate resource utilization.

Finally, we accumulate the number of \glspl{lut} and \gls{dsp} elements
for all allocated operators.  In addition, we estimate the number of
\glspl{lut} required by multiplexers generated for sharing operators, where
$R^\mathrm{LUTs}_\mathrm{mux}$ approximates $1/n$ of the number of \glspl{lut}
required by an $n$-to-1 multiplexer.  The final result is the estimated
resource utilization for the full program:
\begin{equation}
    \begin{aligned}
        r_\mathrm{LUT} &=
            \sum_{\opsymbol \in \opset} \left(
                a_\opsymbol R^\mathrm{LUTs}_\opsymbol +
                \left( n_\opsymbol - a_\opsymbol \right)
                    R^\mathrm{LUTs}_\mathrm{mux}
            \right), \\
        r_\mathrm{DSP} &=
            \sum_{\opsymbol \in \opset} a_\opsymbol R^\mathrm{DSPs}_\opsymbol,
    \end{aligned}
\end{equation}
where $R^\mathrm{LUTs}_\opsymbol$ and $R^\mathrm{DSPs}_\opsymbol$ denote the
number of \glspl{lut} and \glspl{dsp} required by one operator $\opsymbol$
respectively.


\subsection{Accuracy Analysis}
\label{lo:sub:accuracy}

We extend the accuracy analysis of Chapter~\ref{chp:progopt} to support arrays.
Because our benchmark suite consists of programs with large arrays, we keep
the analysis efficient by treating an \emph{entire} array as a pair of a
floating-point interval and an interval of accumulated round-off errors.  These
intervals accumulate all values that are assigned to the array, and never
shrink the range bounded by these intervals when we assign new values to an
array location.  We therefore define the read and write accesses to an array as
follows in the accuracy analysis:
\begin{equation}
    \begin{aligned}
        \exprerrorfunc{\access{A}{\bar\imath}}\sigma^\sharp
        &= \exprerrorfunc{A}\sigma^\sharp, \\
        \exprerrorfunc{\update{A}{\bar\imath}{e}}\sigma^\sharp
        &= \exprerrorfunc{A}\sigma^\sharp \sqcup
           \exprerrorfunc{e}\sigma^\sharp,
    \end{aligned}
\end{equation}
where $A \in \varset$ is an array variable, $\bar\imath$ is a subscript
to index an element in $A$, $e \in \sexprset$ is an expression, and
$\sigma^\sharp \in \errordom$ is the input abstract program state (recall from
Section~\ref{po:sec:accuracy_analysis} of Chapter~\ref{chp:progopt}).

Alternatively, we can view each element $A[i]$, where $i$ is a tuple with $N$
non-negative integers, in an $N$-dimensional array $A$ as a variable.  Updating
$A$ thus produces an abstract state which collects all elements in $A$:
\begin{equation}
    \begin{aligned}
        \exprerrorfunc{\access{A}{\bar\imath}}\sigma^\sharp
        &= \bigsqcup_{i \in \exprerrorfunc{\bar\imath}\sigma^\sharp}
            \sigma^\sharp \left( A[i] \right), \\
        \exprerrorfunc{\update{A}{\bar\imath}{e}}\sigma^\sharp
        &= {\left[
            A[i] \mapsto \sigma^\sharp \left( A[i] \right) \sqcup
                \exprerrorfunc{e}\sigma^\sharp
        \right]}_{i \in \exprerrorfunc{\bar\imath}\sigma^\sharp},
    \end{aligned}
\end{equation}
where $i \in \exprerrorfunc{\bar\imath}\sigma^\sharp$ ranges over all possible
indices that are tuples with $N$ non-negative integers, where $A$ is an
$N$-dimensional array.

Additionally, because most of the loops in our benchmark programs consist
of nested loops and have large iterations, the fixpoint analysis routine
in Section~\ref{po:sub:fixpoint} is modified to analyze only a small
fraction of the innermost loop execution of a loop nest.  By ensuring the
innermost loop iterator increments by the same amount, we can guarantee the
accuracy analysis to be fair for each equivalent implementation for fixpoint
expressions with different unroll factors.  For the experimental outcomes in
Section~\ref{lo:sec:results}, we analyze $10\%$ of the total executions of an
innermost loop for the purpose of optimization.

Finally, we further use the dependence analysis in the \gls{mir} graph
explained in Section~\ref{lo:sub:latency} to detect whether errors are
accumulated across iterations.  This process is carried out by first analyzing
the \gls{mir} of the loop body for intra-iteration dependences.  The absence
of such dependences indicates the round-off errors do not accumulate across
iterations, and it suffices to analyze the fixpoint expression for one loop
iteration.

\section{Tool Usage}
\label{sec:usage}

\todo{Place this somewhere else.}

Our tool is a source-to-source optimizer that specifically targets numerical
program statements written in a subset of standard C99.  Our tool
supports arithmetic and Boolean expressions, assignment statements, \iflit{}
statements, \whilelit{} loops and \forlit{} loops.  The numerical data types we
allow are $\inttype$ and $\floattype$, as well as single- and multi-dimensional
array types.

The program below is an example usage of our tool in a C program.  Note
that it specifies the input values are respectively a two-dimensional array
\verb|A|, where its elements are single-precision floating point values between
0 and 1, and an integer \verb|T| equals to 20.  It also indicates the only
output that we care from this code is the resultant \verb|A|.
\begin{lstlisting}
  #define N 1024
  #pragma opt begin
  #pragma opt in float A[N][N]=[0,1], int T=20
  #pragma opt out A
  for (int t = 0; t < T; t++)
    for (int i = 1; i < N-1; i++)
      for (int j = 1; j < N-1; j++)
        $\arr{A}{i}{j}$ = 0.2 * ($\arr{A}{i-1}{j}$ +
          $\arr{A}{i}{j-1}$ + $\arr{A}{i}{j}$ +
          $\arr{A}{i}{j+1}$ + $\arr{A}{i+1}{j}$);
  #pragma opt end
\end{lstlisting}

Our tool is an open-source command-line utility, which only requires the user
to provide a program written in C extended with the above \verb|#pragma|
statements.  The Pareto optimal programs are all automatically generated by
our tool, each is accompanied with our estimations of its latency and resource
usage, and an analyzed bound on round-off errors.  These programs can then be
given to Vivado HLS to be synthesized into circuits.

\section{Results}
\label{so:sec:results}

Because Martel's approach defers selecting optimal options until the end of
equivalent expression discovery, we developed a method that could produce
exactly the same set of equivalent expressions from the traces computed by
Martel, and has the same time complexity. The difference is that we adopted it
to generate a Pareto frontier from the discovered expressions, instead of only
error bounds.  This allows us to compare \marteltrace{}, \ie~our implementation
of Martel's method, against our methods \frontiertrace{} and \greedytrace{}
discussed in Section~\ref{so:sec:equivalent}.  Figure~\ref{so:fig:martel}
optimizes the expression ${(\vara + \varb)}^2$ using the three methods above,
all using depth limit $3$, and the input ranges are $\vara \in [5, 10]$
and $\varb \in [0, 0.001]$~\cite{martel07}. The IEEE 754 single-precision
floating-point format with rounding to nearest was used for the evaluation
of accuracy and area estimation. The scatter points represent different
implementations of the original expression that have been explored and
analyzed, and the (overlapping) lines denote the Pareto frontiers. In this
example, our methods produce the same Pareto frontier that Martel's method
could discover, while having up to 50\% shorter run time. Because we consider
an accuracy/area trade-off, we find that we can not only have the most accurate
implementation discovered by Martel, but also an option that is only 0.0005\%
less accurate, but uses 7\% fewer \glspl{lut}.

We go beyond the optimization of a small expression, by generating results in
Figure~\ref{so:fig:multi_expr_32} to show that the same technique is applicable
to simultaneous optimization of multiple large expressions. The expressions
$e_1$ and $e_2$, with input ranges $\vara \in [1, 2], \varb \in [10, 20], \varc
\in [10, 200]$ are used as our example:
\begin{equation}
    \begin{aligned}
    e_1 =&
        (\vara + \vara + \varb) \times
        (\vara + \varb + \varb) \times
        (\varb + \varb + \varc) \times {} \\
        &
        (\varb + \varc + \varc) \times
        (\varc + \varc + \vara) \times
        (\varc + \vara + \vara), \\
    e_2 =&
        (1 + \varb + \varc) \times
        (\vara + 1 + \varc) \times
        (\vara + \varb + 1).
    \end{aligned}
\end{equation}

We generated and optimized \gls{rtl} implementations of $e_1$ and
$e_2$ simultaneously using \frontiertrace{} and \greedytrace{}
with the depth limits indicated by the numbers in the legend of
Figure~\ref{so:fig:multi_expr_32}. Note that because the expressions evaluate
to large values, the errors are also relatively large. We set the depth limit
to $2$ and found that \greedytrace{} executes up to $10\times$ faster than
\frontiertrace{}, while discovering a sizable subset of the Pareto frontier of
\frontiertrace{}. Also our methods are significantly faster and more scalable
than \marteltrace{}, because of its poor scalability discussed earlier, our
computer ran out of 8 GB of memory before we could produce any results. If we
normalize the time allowed for each method and compare the performance, we
found that \greedytrace{} with a depth limit $3$ takes takes slightly less time
than \frontiertrace{} with a depth limit $2$, but produces a generally better
Pareto frontier. The alternative implementations of the original expression
provided by the Pareto frontier of \greedytrace{} can either reduce the
\glspl{lut} used by approximately 10\% when accuracy is not crucial, or can
be about 10\% more accurate if resource is not our concern.  It also enables
the ability to choose different trade-off options, such as an implementation
that is 7\% more accurate and uses 7\% fewer \glspl{lut} than the original
expression.

Furthermore, Figure~\ref{so:fig:multi_expr_vary_width} varies the mantissa
width of the floating-point format, and presents the Pareto frontier
of both $e_1$ and $e_2$ together under optimization. Floating-point
formats with mantissa widths ranging from 10 to 112 bits were used to
optimize and evaluate the expressions for both accuracy and area usage. It
turns out that some implementations originally on the Pareto frontier of
Figure~\ref{so:fig:multi_expr_32} are no longer desirable, as by varying the
mantissa width, new implementations are both more accurate and less resource
demanding.

Besides the large example expressions above, Figure~\ref{so:fig:taylor_sin}
and Figure~\ref{so:fig:motzkin} are produced by optimizing expressions with
real applications under single precision. Figure~\ref{so:fig:taylor_sin} shows
the optimization of the Taylor expansion of $\sin(x + y)$, where $x\in[-0.1,
0.1]$ and $y\in[0, 1]$, using \greedytrace{} with a depth limit $3$. The
function $\mathrm{taylor}(f, d)$ indicates the Taylor expansion of function
$f(x, y)$ at $x = y = 0$ with a maximum degree of $d$. For order 5 we reduced
error by more than 60\%. Figure~\ref{so:fig:motzkin} illustrates the results
obtained using the depth limit $3$ with the Motzkin polynomial~\cite{demmel}
$x^6 + y^4 z^2 + y^2 z^4 - 3 x^2 y^2 z^2$, which is known to be difficult to
evaluate accurately, especially using inputs $x\in[-0.99, 1]$, $y\in[1, 1.01]$,
$z\in[-0.01, 0.01]$.

All these above results are generated with the same type of floating-point
operators in each expression.  Although in this chapter we do not analyze
the number of \glspl{dsp} used in synthesized circuits, the \gls{dsp} count
increases linearly with the estimated \gls{lut} count.  In the next chapter
we further introduce the estimation of \gls{dsp} elements used as another
objective to optimize.

Because of the scalability problem of the depth limit $k$ mentioned in
Section~\ref{so:sec:equivalent}, $k \leq 3$ for all of our experiments.  By
setting $k = 4$, the tool does not terminate in reasonable amount of time and
saturates the memory (16 GB) of our system.  In the following chapters, we
propose methods to limit the number of iterations and the number of equivalent
expressions discovered to mitigate the lack of scalability of $k$.

Finally, Figure~\ref{so:fig:area} demonstrates the accuracy of the area
estimation used in our analysis. It compares the actual \glspl{lut} necessary
with the estimated number of \glspl{lut} using our semantics, by synthesizing
more than 6000 equivalent expressions derived from $\vara + \varb + \varc$,
$(\vara + 1) \times (\varb + 1) \times (\varc + 1)$, $e_1$, and $e_2$ using
varying mantissa widths. The dotted line indicates exact area estimation, a
scatter points that is close to the line means the area estimation for that
particular implementation is accurate. The solid black line represents the
linear regression line of all scatter points. On average, our area estimation
is a 6.1\% over-approximation of the actual number of \glspl{lut}, and the
worst case over-approximation is 7.7\%.
\newcommand{\figsize}{0.5}
\begin{figure}[ht]
    \centering
    \includegraphics[scale=\figsize]{martel}
    \caption{Optimization of ${(\vara + \varb)}^2$.}\label{so:fig:martel}
\end{figure}
\begin{figure}[ht]
    \centering
    \includegraphics[scale=\figsize]{multi_expr_32}
    \caption{%
        Simultaneous optimization of both $e_1$ and $e_2$.
    }\label{so:fig:multi_expr_32}
\end{figure}
\begin{figure}[ht]
    \centering
    \includegraphics[scale=\figsize]{multi_expr_vary_width}
    \caption{%
        Varying the mantissa width of Figure~\ref{so:fig:multi_expr_32}.
    }\label{so:fig:multi_expr_vary_width}
\end{figure}
\begin{figure}[ht]
    \centering
    \includegraphics[scale=\figsize]{taylor_sin}
    \caption{The Taylor expansion of $\sin(x + y)$.}\label{so:fig:taylor_sin}
\end{figure}
\begin{figure}[ht]
    \centering
    \includegraphics[scale=\figsize]{motzkin}
    \caption{The Motzkin polynomial $e_m$.}\label{so:fig:motzkin}
\end{figure}
\begin{figure}[ht]
    \centering
    \includegraphics[scale=\figsize]{area}
    \caption{Accuracy of Area Estimation.}\label{so:fig:area}
\end{figure}

\section{Related Work}
\label{sec:related_work}

High-level synthesis (HLS) is the process of compiling a high-level
representation of an application (usually in C, C++ or MATLAB) into
register-transfer-level (RTL) implementation for FPGA~\cite{coussy, gajski}.
HLS tools enable us to work in a high-level language, as opposed to facing
labor-intensive tasks such as optimizing timing, designing control logic in
the RTL implementation. This allows application designers to instead focus on
the algorithmic and functional aspects of their implementation~\cite{coussy}.
Another advantage of using HLS over traditional RTL tools is that a C
description is smaller than a traditional RTL description by a factor
of 10~\cite{coussy, bdti}, which means HLS tools are in general more
productive and less error-prone to work with. HLS tools benefit us in their
ability to automatically search the design space with a reasonable design
cost~\cite{bdti}, explore a large number of trade-offs between performance,
cost and power~\cite{mcfarland}, which is generally much more difficult to
achieve in RTL tools. HLS has received a resurgence of interest recently,
particularly in the FPGA community. Xilinx now incorporates a sophisticated
HLS flow into its Vivado design suite~\cite{vivado_hls} and the open-source
HLS tool, LegUp~\cite{legup}, is gaining significant traction in the research
community.

However, in both commercial and academic HLS tools, there is very little
support for static analysis of numerical algorithms. LLVM-based HLS
tools such as Vivado HLS and LegUp usually have some traditional static
analysis-based optimization passes such as constant propagation, alias
analysis, bitwidth reduction or even expression tree balancing to reduce
latency for numerical algorithms. There are also academic tools that
perform precision-performance trade-off by optimizing word-lengths of data
paths~\cite{constantinides}. However there are currently no HLS tools that
perform the trade-off optimization between accuracy and resource usage by
varying the \emph{structure} of arithmetic expressions.

Even in the software community, there are only a few existing techniques
for optimizing expressions by transformation, none of which consider
accuracy/run-time trade-offs. Darulova~\etal~\cite{darulova} employ a
metaheuristic technique. They use genetic programming to evolve the structure
of arithmetic expressions into more accurate forms. However there are several
disadvantages with metaheuristics, such as convergence can only be proved
empirically and scalability is difficult to control because there is no
definitive method to decide how long the algorithm must run until it reaches a
satisfactory goal. Hosangadi~\etal~\cite{hosangadi} propose an algorithm for
the factorization of polynomials to reduce addition and multiplication counts,
but this method is only suitable for factorization and it is not possible to
choose different optimization levels. Peymandoust~\etal~\cite{peymandoust}
present an approach that only deals with the factorization of polynomials
in HLS using Gr\"obner bases. The shortcomings of this are its dependence
on a set of library expressions~\cite{hosangadi} and the high computational
complexity of Gr\"obner bases. The method proposed by Martel~\cite{martel07}
is based on operational semantics with abstract interpretation, but even
their depth limited strategy is, in practice, at least exponentially complex.
Finally Ioualalen~\etal~\cite{ioualalen} introduce the abstract interpretation
of equivalent expressions, and creates a polynomially sized structure to
represent an exponential number of equivalent expressions related by rules of
equivalence. However it restricts itself to only a handful of these rules to
avoid combinatorial explosion of the structure and there are no options for
tuning its optimization level.

Since none of these above captures the optimization of both accuracy and
performance by restructuring arithmetic expressions, we base ourselves on the
software work of Martel~\cite{martel07}, but extend this work in the following
ways. Firstly, we develop new hardware-appropriate semantics to analyze not
only accuracy but also resource usage, seamlessly taking into account common
subexpression elimination. Secondly, because we consider both resource usage
and accuracy, we develop a novel multi-objective optimization approach to
scalably construct the Pareto frontier in a hierarchical manner, allowing fast
design exploration. Thirdly, equivalence finding is guided by prior knowledge
on the bounds of the expression variables, as well as local Pareto frontiers of
subexpressions while it is optimizing expression trees in a bottom-up approach,
which allows us to reduce the complexity of finding equivalent expressions
without sacrificing our ability to optimize expressions.

We begin with an introduction to formal semantics in the following section,
later in Section~\ref{sec:semantics}, we explain our approach by extending the
semantics to reason about errors, resource usage and equivalent expressions.

High-level synthesis tools are typically designed to adhere to a rigid
specification which outlines their behaviour.  It is a traditional practice
to design this specification and the subsequent tool to ensure that the
synthesized circuits perform functionally identical to the original source
program written in high-level language.  It is also viewed as a good practice
because it has predicable outcomes.  Guided by the rules of the language,
programmers translate mathematical objects such as algorithms and physical
information respectively into source code and numerical data, in a way similar
to tools adhering to their specifications.  This manual process of translation
is unfortunately an approximate one.  Computations as simple as $\sqrt{3}$ must
be approximated, \eg~they are carried out in floating-point arithmetic, because
of the finite nature of computing machines.  Therefore, \gls{hls} tools cannot
be relied upon for an exact interpretation of the mathematical objects we wish
to implement, even if they guarantee the functional equivalence between the
source code and the synthesized result.

Despite the awareness of the approximate characteristic of numerical
software/hardware implementations using floating-point operations, engineers
often take the risks of neglecting this fact, and anticipate their designs to
behave identically to the mathematical algorithms visioned in real arithmetic
within a reasonable but not well-defined error margin.  As it was shown in
Section~\ref{bg:sub:expression_accuracy} in Chapter~\ref{chp:background},
round-off errors when accumulated, could have detrimental effects on our daily
life.  The aforementioned functional equivalence between source and circuit
guaranteed by \gls{hls} tools is therefore unable to regain any lost accuracies
due to approximation.

Traditional \gls{ir}-level \gls{hls} program optimization consist of a series
of transformation passes.  Most of these passes do not predict whether
they have negative impact on the resulting circuit, and they limit their
capabilities by preserving functional equivalence.  Varying the order of
these passes could have significant impact on the quality, as these passes
interact with one another in a complicated manner, it is difficult to predict
the overall impact on performance~\cite{huang15}.  For $n$ passes, there are
$n{\,!}$ distinct ways to order, it is thus a considerable challenge to decide
the optimal pass ordering, which is exacerbated by the fact that it could be
highly dependent on the input program~\cite{cong13}.

These above shortcomings of traditional \gls{hls} tools and optimizing
compilers provide a strong motivation for the work proposed in this thesis.

Firstly, we can apply the philosophy of relaxing the functional equivalence
required by \gls{hls} tools.  In the mean time, we preserve the equivalence of
the underlying mathematical objects in real arithmetic which hardware designs
are approximating.  One can often improve the numerical accuracy by choosing a
better alternative among these equivalences.

Secondly, by the same paradigm shift, a wide range of optimization
opportunities can be explored to minimize throughput and resource utilization.
These opportunities were previously lost out to the necessity of ensuring
consistent behaviour.

Finally, optimization can be carried out by applying steps of equivalence
rewrites driven by a prediction model.  Traditional optimization passes can
be broken up into much smaller common parts made of equivalence rules can be
easily proved mathematically correct.  By using models to predict run time,
resources and accuracy to guide the optimization process, it is possible to
explore multiple designs that trade-off the three performance metrics while
removing concerns about the ordering problem.  Many optimization passes, such
as constant propagation, dead code removal, common subexpression elimination
and~\etc, are naturally subsumed by the new approach.  As the computational
power of machines increases exponentially, we can foresee an increase in the
scale of the vast search space to be explored in the future.

This thesis therefore broadens the horizon of \gls{hls} tools, and equips
them with the new program optimization paradigm by leveraging these above
observations.  Specifically, the trade-off relationship among numerical
accuracy, resource utilization and throughput are optimized in floating-point
numerical programs for \gls{hls}\@.  Here we summarize the contributions of
this thesis.

To the best of our knowledge, this thesis is the first to introduce
multiple-objective performance optimization in a unified framework for
discovering equivalence in programs.  Chapter~\ref{chp:stropt} implements
this framework and optimizes a suite of expressions that are difficult to
optimize by hand, and improve numerical accuracy and area automatically.
In the experimental results, it turns out that the two central goals,
\ie~improving accuracy and minimizing area, are often not in conflict, as
optimized expressions can enjoy almost all enhancements that can be achieved
in both metrics.  Guided by the concept of abstract interpretation, it further
introduces the semantics-based program analyses to jointly reason about
safe ranges of round-off errors and resource utilization, and subsequently,
discovery of equivalent expressions.  This technique lays the necessary
foundation for program equivalence beyond simple arithmetic expressions.

The infinite size of the equivalent program space, coupled with undecidability
of program properties, makes the program optimization an even more
challenging task than the one of arithmetic expressions.  For this,
Chapter~\ref{chp:progopt} introduces a new graph-based intermediate
representation, \gls{mir}, for capturing the semantics of numerical programs.
This approach reduces the size of the search space, and the \gls{ir} itself
is derived from the formal semantics of programs to ensure the correctness
of equivalent \glspl{mir} and the back-and-forth translation between C and
\gls{mir}\@.  It further eliminates the problem of optimization pass ordering,
because by using the equivalence discovery framework, the Pareto frontier can
be extended incrementally with small steps of rewrites to multiple candidates.
Traditional compiler optimizations are naturally subsumed and further enhanced
by the \glspl{mir}, as many optimization techniques such as loop splitting and
loop fusion that previously must be profiled to justify enabling them, can
emerge automatically from the optimization process.  By optimizing a suite of
resource-efficient benchmark examples, the tool improves the numerical accuracy
by up to 65\%.

Formerly, \gls{hls} tools' ability to pipeline loops is fundamentally constrained
by intra-iteration dependencies.  Traditional optimization techniques such as
partial loop unrolling may have minimal effects on the initiation interval of
pipelined loops, as these do not impact the data-path structure, which ensures
that the functional equivalence is preserved.  Encouraged by the promising
effects of Nicolau~\etal's tree height reduction technique~\cite{nicolau91}
and LegUp's recurrence minimization~\cite{canis14}, Chapter~\ref{chp:latopt}
further incorporates latency analysis into the unified program optimization
framework.  It was found that traditional optimization techniques when used
in tandem with the arithmetic equivalence rules and memory access reduction
rules can significantly improve the latency and accuracy of a numerical
program.  In Chapter~\ref{chp:progopt}, the experimental results identifies
that the static analysis of round-off errors for each candidate explored is
the key factor to the speed of optimization.  This problem is addressed in
this chapter by graph partitioning and candidate pruning algorithms.  It
further enables deeper partial loop unrolling factors that was not explored in
Chapter~\ref{chp:progopt}.  Often as we optimize numerical programs by spending
more resources, latency and round-off error can be simultaneously minimized, as
more resources would allow greater flexibility to discover equivalent programs
that often perform well in terms of run time and accuracy.  By optimizing
a suite of benchmark examples from PolyBench and Livermore loops, the tool
improves the latency and accuracy of each by up to 12$\times$ and 7$\times$
respectively, at a cost of 4$\times$ more resource utilization.


\section{Future Prospects}
\label{cc:sec:future_prospects}

In its current form, the new approach to program optimization explained in this
thesis forms the underlying basis for a much larger set of future work.  Even
though it is precursory on its own, the promising experimental results showcase
the powerful optimization it can bring to optimizing compilers and \gls{hls}
tools.  Here, a list of potential directions of future research is discussed
that could further widen the scope of our technique for a broader range of
applications.

\textbf{\gls{llvmir}-Level Program Optimization.}  We could envision
a back-and-forth translator from \gls{llvmir}~\cite{llvm, llvm_ir} to
\gls{mir} graphs.  This could enable a much wider applicability of the
technique presented in the thesis to both \gls{llvm}-based \gls{hls} tools
and software compilers.  Additionally, it could benefit from existing
\gls{llvm} optimizations passes by using the optimized \gls{llvmir} code
as inputs.  There are however obstacles in migrating to \gls{llvmir} as
the source language.  Firstly, \gls{llvmir} is \gls{ssa}-based.  Since it
uses temporary variables for intermediate results in computation, a full
liveness analysis~\cite{hathhorn12, nielson99, boissinot08} may be necessary
to eliminate temporary variables from the resulting \gls{mir}\@.  Secondly,
control-flows in \gls{llvmir} are more freely structured.  Unlike C, which
defines \iflit~statements and \whilelit~loops and discourages the use of
\verb|goto| statements, control-flow in \gls{llvmir} are composed by basic
blocks and branches between pairs of them.  This requires the \gls{mir} to be
further extended to cope with complex control-flow patterns.  Conventionally,
programs written with branches are often analyzed using \emph{continuation
style semantics}~\cite{felleisen88}.  It is not evident how this semantics can
be embedded within \glspl{mir}.

\textbf{Tighter bounds on round-off errors.} As an alternative for interval
analysis, the accuracy analysis could enjoy more sophisticated abstract domains
that capture the correlations between variables, and produce tighter bounds
for results.  Currently, the analysis cannot produce meaningful, \ie~finite,
bounds on the round-off errors of certain numerical programs.  If the analysis
fails to bound errors, then currently the optimization cannot be directed
to a more accurate implementation.  By using abstract relational domains,
it is possible to produce a much tighter bound on the values of program
variables, and the associated errors.  There are a few relational domains-based
static analysis techniques of floating-point errors~\cite{mine07_2, putot04,
goubault11, astree}, however making use of them still poses challenges.  Each
floating-point operation introduces an independent error term as a new variable
in the formulation of these relational domains, and it may be difficult to
determine how to collapse these error terms into a smaller set of variables,
as the optimization in this thesis can introduce a large number of error
variables.

\textbf{Special and fused operators.} There could be a lot of interest in the
\gls{hls} community on how \soap~can be incorporated with existing work on
fused floating-point data-path synthesis.  Langhammer~\etal~\cite{langhammer}
propose that normalization and denormalization stages could be regarded as
redundant between operators in a floating-point data-path.  By removing
these stages, subsets of the data-path become fixed-point data-paths, in the
meanwhile saving resources and improving throughput at a cost of accuracy.  It
will be compelling to isolate the normalization/denormalization stages into
operators in the \soap~framework, so that a mixed floating-point/fixed-point
program can more efficiently trade-off resources, accuracy and latency.

\textbf{Multiple word-lengths.}  In this thesis, experiments have been carried
out on floating-point operations with a fixed mantissa only.  It would be
beneficial to further integrate fixed-point support.  Additionally, by further
supporting multiple precisions in the data-path, \ie~allowing each operators to
compute with different precisions, the trade-off relationship among our three
primary performance measures can be even more effective.  Techniques, known
as multiple word-length optimization~\cite{constantinides, lee06, cantin02},
exist to apply a heuristic approach to perturb the precisions in a data-path,
so that a performance metric can be optimized while round-off errors of outputs
satisfies an error budget.  Instituting such techniques in the \soap~framework
is rewarding as it can further reduce the area and latency requirement of a
synthesized circuit for a given accuracy.  All of these approaches optimize a
fixed data-path, whereas in \soap~the structure of the data- and control-paths
are varying as we optimize them.  Analyzing each of the candidates for the
optimal precision assignment to each operator is very inefficient because of
the number of candidates explored.  Moreover, current techniques work with a
predetermined error budget, and yet in fact a Pareto frontier exists for each
data-path to trade-off accuracy, resources and latency.

\textbf{Numerical analysis and linear algebra.}  There are two distinct
approaches to the analysis of round-off errors.  One focuses on the round-off
errors by statically analyzing numerical programs, and apply this in a way
which is as general as possible, similar to the method presented in this
thesis.  On the other hand, there are techniques employed by numerical analysts
to evaluates and improve the numerical accuracy and stability of particular
algorithms analytically.  Many creative solutions to challenges are invented
in this process.  For instance, \emph{Kahan's compensated summation} algorithm
is an accurate way to compute a sum of $n$ values, $\sum_{i = 0}^{n-1}
x_i$~\cite{kahan65} is shown in Figure~\ref{co:lst:sum}.  This algorithm
cannot be discovered easily using the method outlined in this thesis, and a
way to extend the framework to optimize programs as creatively as humans still
eludes us at the moment.  Higham~\etal~\cite{higham02} discuss in great depth
many existing numerical accuracy problems encountered in finite-precision
computation of polynomials and linear algebra subprograms and how to analyze
and overcome inaccuracies, often in terms of relative errors.  Bridging the
gap between computational and mathematical approaches for numerical analysis
will allow us to automate many accuracy optimizations that were previously
unexplored by the tool.
\begin{figure}[ht]
    \centering
\begin{lstlisting}[]
    float compensated_summation(float X[N])
    {
        float sum = 0.0f;
        float e = 0.0f;
        for (i = 0; i < n; i++)
        {
            float tmp = sum;
            float y = X[i] + e;
            sum = tmp + y;
            e = (temp - sum) + y;
        }
        return sum;
    }
\end{lstlisting}
    \caption{%
        Kahan's compensated summation algorithm to accurately compute the sum
        of $n$ elements $\sum_{i = 0}^{n-1} x_i$.
    }\label{co:lst:sum}
\end{figure}

\textbf{Continuity analysis and optimization.} The robustness of programs
are very important to us.  In many cases, we wish our algorithms to be
free from discontinuity, \ie~a small change in the initial condition
would not result in an undesirably large jump in the outputs.  For this,
Chaudhuri~\etal~\cite{chaudhuri11} and Goubault~\etal~\cite{goubault13}
respectively propose methods to analyze the robustness of programs.  The
former approach formally proves whether an algorithm is ill-conditioned in
terms of the existence of discontinuity, whereas the latter statically analyze
programs to determine if round-off errors introduce significant discontinuous
behaviour.  To illustrate, consider an \iflit~branch,
``\lstinline[basicstyle=\tt]{if ($e$ > 0) $c_1$ else $c_2$}'', where $e$
is a floating-point expression.  When $e$ is positive and very close to
$0$ when evaluated in real arithmetic, the floating-point result of $e$
could be non-positive, due to the effects of the round-off errors.  In
these extraordinary cases, the $c_2$ branch may be executed instead of the
intended $c_1$.  These above new techniques could inspire us to implement the
optimization of discontinuous behaviour, such as the one shown in the example,
as another objective.

\textbf{Memory partitioning.} The experimental results in this work see a
diminishing performance return when loops are deeply unrolled, because of a
memory bottleneck.  As memory accesses saturate in loop execution, \ie~all
memory ports are working in 100\% utilization, it is unable to gain further
performance improvements.  Currently the tool stops exploring further loop
unrolling when this happens.  By automatically partition arrays upon hitting
such a memory bottleneck, further throughput improvements can be achieved.

\textbf{Other practical considerations.}  Finally, we may consider design
perspectives that could make the resulting tool much more usable.  For
instance, programs may still be optimized by not having any knowledge about the
input variables.  Herbie~\cite{panchekha15} makes no assumption about the input
space, and can nevertheless optimize arithmetic expressions, by splitting the
input space into regimes.


% \section{Tool Usage}
% \label{cc:sec:usage}

% \soap~is a source-to-source optimizer that specifically targets numerical
% program statements written in a subset of standard C99.  The tool supports
% arithmetic and Boolean expressions, assignment statements, \iflit{} statements,
% \whilelit{} loops and \forlit{} loops.  The numerical data types we allow are
% $\inttype$ and $\floattype$, as well as single- and multi-dimensional array
% types.

% The program below is an example usage of \soap~in a C program.  Note that it
% specifies the input values are respectively a two-dimensional array \verb|A|,
% where its elements are single-precision floating point values between 0 and 1,
% and an integer \verb|T| equals to $20$.  It also indicates the only output that
% we care from this code is the resultant \verb|A|.
% \begin{lstlisting}
  % #define N 1024
  % #pragma soap begin
  % #pragma soap in float A[N][N]=[0,1], int T=20
  % #pragma soap out A
  % for (int t = 0; t < T; t++)
    % for (int i = 1; i < N-1; i++)
      % for (int j = 1; j < N-1; j++)
        % A[i][j] = 0.2f * (A[i-1][j] +
          % A[i][j-1] + A[i][j] +
          % A[i][j+1] + A[i+1][j]);
  % #pragma soap end
% \end{lstlisting}

% \soap~is an open-source command-line utility, which only requires the user
% to provide a program written in C extended with the above \verb|#pragma|
% statements.  The Pareto optimal programs are all automatically generated by
% \soap, each is accompanied with our estimations of its latency and resource
% usage, and an analyzed bound on round-off errors.  These programs can then be
% given to \gls{vhls} to be synthesized into circuits.


\section{Final Remarks}
\label{cc:sec:final_remarks}

This thesis adapts existing techniques such as accuracy, latency and resource
usage analysis, and further introduces novel approaches, \eg~\gls{mir} and
efficient equivalence discovery, and delivers them in a unified framework.
The functional equivalence relaxation paradigm is relatively under-explored,
because these optimizations are often highlighted as \emph{unsafe} by the
\gls{hls} tools, as they cannot analyze the numerical implications of these
optimizations.  \Gls{hls} tools therefore have very limited optimization
options base on this particular concept.  With the constructive results
produced by this thesis, optimizations based on our concept can not only raise
performance measures, but also result in even \emph{safer} implementations
as we improve numerical accuracies.  The equivalence discovery algorithm in
tandem with \glspl{mir} could have great potential in compiler optimization
based on our concept.  Furthermore, since machine learning algorithms are
error-resilient~\cite{lesser11, kim09, holt91, zhu03}, the methods demonstrated
in this thesis have promising capabilities to improve the resource usage,
latency and accuracy of them.



\chapter{Conclusion}

\chapter{(Scratch)}

We believe that it is possible to extend our tool for the multi-objective
optimization of arithmetic expressions in the following ways. First, Secondly,
it would be useful to further allow transformations of expressions while
allowing different mantissa widths in the subexpressions, this further
increases the number options in the Pareto frontier, as well as leads to more
optimized expressions. Thirdly, as an alternative for interval analysis, we
could employ more sophisticated abstract domains that capture the correlations
between variables, and produce tighter bounds for results. Finally, there
could be a lot of interest in the HLS community on how our tool can be
incorporated with Martin Langhammer's work on fused floating-point datapath
synthesis~\cite{langhammer}.

We are convinced that with the foundation and framework that we developed,
our tool can be extended in the following ways.  First, it can be trivially
extended to support additional numerical data structure such as arrays and
matrices.  Secondly, the Pareto optimization can be extended to optimize the
latencies of equivalent programs, as restructuring programs and partially
unrolling loops could have a notable impact on the ability to pipeline
program loops, especially when arrays are incorporated.  Finally, fixed point
representations, along with the interaction between our structural optimization
and multiple wordlength optimization~\cite{constantinides} could also generate
a lot of interest from the HLS community.

Currently, our tool sees a diminishing performance return when loops are deeply
unrolled, because of a memory bottleneck. We are exploring an extension to
our tool that enables it to automatically partition arrays upon hitting such
a memory bottleneck. Also, our tool currently supports only single-precision
floating-point data types; we intend to extend this to multiple-precision
types, and explore the impact on our three performance metrics: latency,
resource utilization and numerical accuracy.


\cleardoublepage


% Back matter
{%
\setstretch{1.1}
\renewcommand{\bibfont}{\normalfont\small}
\setlength{\biblabelsep}{0pt}
\setlength{\bibitemsep}{0.5\baselineskip plus 0.5\baselineskip}
\printbibliography[nottype=online]
\printbibliography[heading=subbibliography,title={Webseiten},type=online,prefixnumbers={@}]
}
\cleardoublepage

\listoffigures
\cleardoublepage

\listoftables
\cleardoublepage

% \input{colophon}
% \cleardoublepage

% \input{declaration}
% \clearpage
% \newpage
% \mbox{}

\end{document}
