\documentclass[%
    paper=A4,                   % paper size --> A4 is default in Germany
    twoside=true,               % onesite or twoside printing
    openright,                  % doublepage cleaning ends up right side
    parskip=full,               % spacing value / method for paragraphs
    chapterprefix=true,         % prefix for chapter marks
    11pt,                       % font size
    headings=normal,            % size of headings
    bibliography=totoc,         % include bib in toc
    listof=totoc,               % include listof entries in toc
    titlepage=on,               % own page for each title page
    captions=tableabove,        % display table captions above the float env
    draft=true,                 % value for draft version
]{scrreprt}%

\usepackage[utf8]{inputenc}
\usepackage[english]{babel}
\usepackage[
    figuresep=quad,
    sansserif=false,
    hangfigurecaption=false,
    hangsection=true,
    hangsubsection=true,
    colorize=full,
    colortheme=bluemagenta,
    bibsys=bibtex,
    bibfile=references,
    bibstyle=alphabetic,
    wrapfooter=true,
]{cleanthesis}
% \usepackage[inference]{semantic}
% \usepackage{flushend}   % balance end of page columns
\usepackage[inference]{semantic}
\usepackage{algorithmicx}
\usepackage{algpseudocode}
\usepackage{amsmath}
\usepackage{amssymb}
\usepackage{booktabs}   % for \toprule and \bottomrule
\usepackage{colortbl}   % for \rowcolor, \cellcolor, etc.
\usepackage{color}
\usepackage{courier}    % makes \tt font same size as \rm font
\usepackage{fixltx2e}   % for \textsubscript and \textsuperscript
\usepackage{galois}
\usepackage{graphicx}
\usepackage{listings}   % source code
\usepackage{multirow}   % for \multirow
\usepackage{pgfplotstable}
\usepackage{pgfplots}
\usepackage{stmaryrd}   % double brackets
\usepackage{subfig}
\usepackage{textgreek}  % for \textmu
\usepackage{tikz}       % graphs
\usepackage{todonotes}
\usepackage{verbatim}

\addbibresource{intro/publications.bib}

\usetikzlibrary{positioning,fit,decorations.pathmorphing,shapes}
\graphicspath{{stropt/fig/}{progopt/fig/}{latopt/fig/}}
\DeclareGraphicsExtensions{.pdf}

\pgfplotsset{select coords between index/.style 2 args={%
    x filter/.code={%
        \ifnum\coordindex<#1\def\pgfmathresult{}\fi
        \ifnum\coordindex>#2\def\pgfmathresult{}\fi
    },
    filter discard warning=false
}}

\lstset{%
  language=C,
  basicstyle=\tt,
  columns=fixed,
  mathescape=true,
  keywordstyle=\bfseries,
  otherkeywords={\#pragma,\#define},
}

%\listfiles  % Debug LaTeX Information

\newcommand{\thesisTitle}{Structural Optimization of Numerical Programs}
\newcommand{\thesisName}{Xitong Gao}
\newcommand{\thesisDate}{\today}
\newcommand{\thesisVersion}{0.0.1}

\newcommand{\thesisFirstSupervisor}{George A. Constantinides}
% \newcommand{\thesisSecondSupervisor}{John Smith}

\newcommand{\thesisUniversity}{\protect{Imperial College London}}
\newcommand{\thesisUniversityDepartment}{%
    Department of Electrical and Electronic Engineering}
% \newcommand{\thesisUniversityInstitute}{Institut for Clean Thesis Dev}
\newcommand{\thesisUniversityGroup}{Circuits and Systems Research Group}
\newcommand{\thesisUniversityCity}{London}
\newcommand{\thesisUniversityStreetAddress}{South Kensington Campus}
\newcommand{\thesisUniversityPostalCode}{SW7 2AZ}

\hypersetup{%
    final=true,
    pdftitle={\thesisTitle},    %   - title (PDF meta)
    pdfauthor={\thesisName},    %   - author (PDF meta)
    bookmarks=true,
    plainpages=false,           %   -
    linktoc=all,
    colorlinks=false,           %   - colorize links?
    pdfborder={0 0 0},          %   -
    breaklinks=true,            %   - allow line break inside links
    bookmarksnumbered=true,     %
    bookmarksopen=true          %
}

\newcommand{\eg}{\textit{e.g.}}
\newcommand{\etc}{\textit{etc.}}
\newcommand{\etal}{\textit{et~al.}}
\newcommand{\ie}{\textit{i.e.}}

% names of stuff
\newcommand{\soap}{SOAP}
\newcommand{\SOAP}{SOAP}
\newcommand{\newsoap}{\textsf{SOAP2}}
\newcommand{\numimp}{\textsf{NumImp}}

\DeclareMathOperator{\roundupop}{\uparrow^\sharp_\circ}
\DeclareMathOperator{\rounddownop}{\downarrow^\sharp_\circ}
\DeclareMathOperator{\fresh}{\mathit{fresh}}
\DeclareMathOperator{\eqstep}{\blacktriangleright}
\DeclareMathOperator{\dom}{\mathrm{Dom}}
\DeclareMathOperator{\area}{\mathrm{Area}}
\DeclareMathOperator{\error}{\mathrm{Error}}
\DeclareMathOperator{\abserr}{\mathrm{AbsError}}
\DeclareMathOperator{\abs}{\mathrm{abs}}
\DeclareMathOperator{\frontier}{\textsc{Frontier}}

\newcommand{\integerset}{\ensuremath\mathbb{Z}}
\newcommand{\naturalset}{\ensuremath\mathbb{N}}
\newcommand{\realset}{\ensuremath\mathbb{R}}
\newcommand{\floatset}{\ensuremath\mathbb{F}}
\newcommand{\powersetof}[1]{\ensuremath\mathcal{P}\left({#1}\right)}
\newcommand{\intervalset}{\ensuremath\mathbf{Interval}}
\newcommand{\floatintervalset}{\ensuremath\intervalset_\floatset}
\newcommand{\errorset}{\ensuremath\mathbb{E}^\sharp}
\newcommand{\errordom}{\ensuremath\mathbf{\Sigma}_{\errorset}}
\newcommand{\labelset}{\ensuremath\mathbf{Label}}
\newcommand{\exprset}{\ensuremath\mathbf{Expr}}
\newcommand{\varset}{\ensuremath\mathbf{Var}}
\newcommand{\env}[1]{\ensuremath\mathbf{Env}_{#1}}
\newcommand{\enter}[1]{\ensuremath{A({#1})}}
\newcommand{\interval}[2]{\ensuremath\left[{#1}, {#2}\right]}
\newcommand{\lattice}[2]{\ensuremath\left<{#1}, {#2}\right>}
\newcommand{\join}{\ensuremath\sqcup}
\newcommand{\meet}{\ensuremath\sqcap}

\newcommand{\marteltrace}{\texttt{martel\_trace}}
\newcommand{\frontiertrace}{\texttt{frontier\_trace}}
\newcommand{\greedytrace}{\texttt{greedy\_trace}}

% program objects
\newcommand{\inttype}{\texttt{int}}
\newcommand{\floattype}{\texttt{float}}
\newcommand{\aexprset}{\ensuremath\mathbf{ArithExpr}}
\newcommand{\bexprset}{\ensuremath\mathbf{BoolExpr}}
\newcommand{\stmtset}{\ensuremath\mathbf{Stmt}}

% program syntax
\DeclareMathOperator{\notop}{\mathtt{not}}
\newcommand{\andop}{\ensuremath\mathbin\mathtt{and}}
\newcommand{\orop}{\ensuremath\mathbin\mathtt{or}}
\newcommand{\semicolon}{\ensuremath\mathbin{;}}
\newcommand{\assign}{\ensuremath\mathbin{:=}}
\newcommand{\iflit}{\texttt{if}}
\newcommand{\thenlit}{\texttt{then}}
\newcommand{\elselit}{\texttt{else}}
\newcommand{\whilelit}{\texttt{while}}
\newcommand{\forlit}{\texttt{for}}
\newcommand{\dolit}{\texttt{do}}
\newcommand{\tolit}{\texttt{to}}
\newcommand{\assignstmt}[2]{\ensuremath{#1}\assign{#2}}
\newcommand{\skipstmt}{\ensuremath\mathtt{skip}}
\newcommand{\ifstmt}[2]{%
\ensuremath\iflit \left(#1\right) \thenlit \left(#2\right)}
\newcommand{\ifelsestmt}[3]{%
    \ensuremath\iflit \left(#1\right) \thenlit \left(#2\right)
    \elselit \left(#3\right)}
\newcommand{\whilestmt}[2]{%
    \ensuremath\whilelit \left(#1\right) \dolit \left(#2\right)}
\newcommand{\forstmt}[2]{%
    \ensuremath\forlit \left(#1\right) \dolit \left(#2\right)}

% metasemantic syntax
\newcommand{\qop}{\ensuremath\mathbin{?}}
\newcommand{\colonop}{\ensuremath\mathbin{:}}
\newcommand{\select}[3]{\ensuremath#1\qop#2\colonop#3}
\newcommand{\expand}{\ensuremath\mathbin{\star}}
\DeclareMathOperator{\fix}{\mathrm{fix}}
\newcommand{\fixpoint}[1]{\ensuremath\fix\left({#1}\right)}
\DeclareMathOperator{\accessop}{\mathrm{access}}
\newcommand{\access}[2]{\ensuremath\accessop\left({#1}, {#2}\right)}
\DeclareMathOperator{\updateop}{\mathrm{update}}
\newcommand{\update}[3]{\ensuremath\updateop\left({#1}, {#2}, {#3}\right)}

% metasemantic objects
\newcommand{\sexprset}{\ensuremath\mathbf{SemExpr}}
\newcommand{\mirset}{\ensuremath\mathbf{MIR}}

% metasemantic functions
\DeclareMathOperator{\metasemop}{\mathsf{M}}
\DeclareMathOperator{\mirerrorop}{\mathsf{E}_\mathsf{m}}
\DeclareMathOperator{\exprerrorop}{\mathsf{E}_\mathsf{s}}
\DeclareMathOperator{\ulp}{\mathrm{ulp}}
\newcommand{\varfunc}[1]{%
    \ensuremath\mathrm{var}\left(#1\right)
}
\newcommand{\metasemfunc}[1]{%
    \ensuremath\metasemop\left\llbracket{#1}\right\rrbracket}
\newcommand{\exprerrorfunc}[1]{\ensuremath\exprerrorop\left[{#1}\right]}
\newcommand{\mirerrorfunc}[1]{\ensuremath\mirerrorop\left[{#1}\right]}
\newcommand{\roundup}[1]{\ensuremath\roundupop\left(#1\right)}
\newcommand{\rounddown}[1]{\ensuremath\rounddownop\left(#1\right)}

% equivalent relations
\newcommand{\eqrel}{\ensuremath\equiv}
\newcommand{\eqgenrel}{\ensuremath\mathbin{\rhd}}

% optimization
\DeclareMathOperator{\optimize}{\textsc{O}}
\newcommand{\optfunc}[1]{\ensuremath\optimize\left[#1\right]}
\DeclareMathOperator{\closure}{\textsc{C}}

% misc
\newcommand{\identity}{\mathrm{id}}
\DeclareMathOperator{\labfunc}{\mathrm{Label}}

% latency
\newcommand\II{\mathit{II}}
\newcommand\MII{\mathit{II}_{\rm min}}
\newcommand\RecMII{\mathit{II}_{\rm min}^{\rm rec}}
\newcommand\ResMII{\mathit{II}_{\rm min}^{\rm res}}

\newcommand\opsymbol{\otimes}

% Add brackets around a given set of nodes
\newcommand\brackets[1]{
\newcommand\ticklength{1.5mm}
\node[fit=#1, inner sep=0mm, outer sep=0.5mm] (bb) {};
\draw ([xshift=\ticklength]bb.north west) -- (bb.north west) --
(bb.south west) -- ([xshift=\ticklength]bb.south west);
\draw ([xshift=-\ticklength]bb.north east) -- (bb.north east) --
(bb.south east) -- ([xshift=-\ticklength]bb.south east);
}

\tikzset{%
  mir/.style={%
    node distance=3mm,
    baseline=(current bounding box.center)
  },
  mirnode/.style={%
    inner sep=0.7mm,
    text height=1ex,
    text depth=0ex
  }
}
