As the computational capability of new FPGA devices grows at an exponential
rate, numerical applications running on them become increasingly more complex.
Traditional design methodologies, working on the register-transfer level (RTL),
have become increasingly costly, forcing us to work on a higher abstraction
level~\cite{gajski}, \ie~to implement our designs using high-level languages
such as C\@.

There are many reasons why FPGA implementations of numerical algorithms
are now best obtained via \emph{high-level synthesis} (HLS) from C\@:
less development effort, the abundance of software engineers compared to
hardware designers, the relative ease of testing C code on an ordinary
microprocessor, the opportunities for rapid design space exploration, and
so on~\cite{meeus12}. Great advances have been made in this area recently,
and the output from HLS tools is nowadays competitive with hand-crafted
designs~\cite{bdti_xilinx}.

Numerical C programs are typically written with floating-point arithmetic,
following the \emph{IEEE 754 standard} for floating-point computation.
Floating-point numbers can represent a wide range of real values, and most
programming languages support it seamlessly.  It has become highly ubiquitous,
and is used in most of our software and hardware implementations of numerical
programs.  Recently, Altera has introduced Arria 10 and Stratix 10 devices
to incorporate hardened floating-point DSP blocks in the FPGA fabric.  As a
result, we expect to see floating-point arithmetic continuing to dominate in
the FPGA implementation of numerical applications.

Although we often make use floating-point arithmetic to implement algorithms,
often specified in real arithmetic, in practice, it is often neglected that
floating-point computations almost always have \emph{roundoff errors}, \ie~the
discrepancy between the actual result in real arithmetic and the rounded result
computed with floating-point arithmetic.  Round-off errors, when accumulated,
can have a devastating effect on numerical accuracy~\cite{higham02}.

In fact, properties such as associativity $(a + b) + c \equiv a + (b + c)$
and distributivity $a \times (b + c) \equiv a \times b + a \times c$ which
we consider to be fundamental laws of real numbers no longer hold under
floating-point arithmetic.  For instance, under single-precision floating-point
arithmetic with rounding to the nearest, the result of $(2^{-24} + 2^{-24})
+ 1 = 1.00000012\textellipsis$ is exact, but $(1 + 2^{-24}) + 2^{-24}$ is
rounded to $1$.  Round-off errors in a numerical program are dependent on
every arithmetic operation and every input value, and with the impact on
floating-point accuracy being so esoteric, it is challenging for engineers to
understand the repercussions of switching between ``\verb|(a + b) * c|'' and
``\verb|a * c + b * c|'' in their programs.

Nevertheless, this opens the possibility of using these rules to generate a
program equivalent to the original program in real arithmetic, which could
have better quality than the original when evaluated in floating point
computation.  Experienced engineers often apply such expression rewriting
intuitions in numerical programs.  For instance, when summing a sequence
of floating-point values, one can sometimes reduce round-off error in the
result by summing the inputs in ascending order.  On the other hand, one can
often reduce latency by applying \emph{expression balancing}, \ie~rearranging
operators in an expression to construct a balanced tree, so that more operators
can work in parallel.  These heuristics cover a very limited number of
possible transformations and may not always improve the original code.  A
trivial process therefore does not exist to apply steps of transformations
using equivalence rules to \emph{optimally} trade off latency, resources and
numerical accuracy.

Existing HLS tools consider these rewrites to be unsafe, and thus make very
limited use of them when restructuring floating-point data-paths.  For
instance, Vivado HLS~\cite{vivado_hls} has only a very simple \emph{expression
balancing} feature that uses associativity to improve latency, and only
expressions with either additions or multiplications are optimized.  Moreover,
it does not produce optimal loop pipelining, because it does not take
into account the implications of these transformations on inter-iteration
dependences and does not explore partial loop unrolling.  In addition, Vivado
HLS cannot reason about how this feature affects numerical accuracy; there is
no guarantee that this transformation will not result in a catastrophically
inaccurate implementation.

In response, this thesis proposes new methodologies and the accompanied
tool---a fully automatic source-to-source optimizer that augments Vivado
HLS---to optimize a given numerical C program using these transformations in
tandem with conventional program transformations.  Our optimizer discovers
not only one, but a wide spectrum of program candidates.  When synthesized
in Vivado HLS, these candidates trade off three performance metrics of great
importance to engineers: run time, resource usage and round-off error.  Here,
run time refers to the latency in clock cycles, resource usage refers to
the number of look-up tables (LUTs) and digital signal processing (DSP)
elements.  Some of these performance metrics could be in conflict.  For
example, higher performance tends to require more circuitry, and how to
resolve this trade-off depends on the user's requirements.  As a result, our
tool produces a \emph{set} of optimized programs, known as the \emph{Pareto
frontier}: those programs $P$ for which the tool has found no $P'$ that
improves on $P$ in all three metrics.  We thus go beyond existing literature
by not only optimizing the precision requirements of an implementation, but
changing the structure of the implementation itself.

Generating candidate optimizations na{\"\i}vely would however produce a
combinatorial explosion, even for small input programs.  For instance,
in worst case, the parsing of a simple summation of $n$ variables could
result in $(2n - 1)!! = 1\times3\times5\times\cdots\times(2n - 1)$ distinct
expressions~\cite{ioualalen, mouilleron}.  This is further complicated
by distributivity as the expression ${(a + b)}^k$ could expand into an
expression with a summation of $2^k$ terms each with $k - 1$ multiplications.
Usually, for this reason, it would be infeasible to generate a complete set
of equivalent expressions using the rules of equivalence, since an expression
with a moderate number of terms will have a very large number of equivalent
expressions.  A new approach was therefore developed to specifically tackle
the efficient discovery of equivalent structures in numerical programs, by
intelligently pruning the set of candidates as it progresses up the input
program's abstract syntax tree.
