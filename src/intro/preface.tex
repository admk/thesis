As the computational capability of \gls{fpga} devices grows at an exponential
rate, numerical applications running on them become increasingly more
complex.  Traditional design methodologies, working at the \gls{rtl}, have
become increasingly costly, forcing us to work at a higher abstraction
level~\cite{gajski, bdti_xilinx, meeus12}, \eg~to implement our designs using
\glspl{hll} such as C\@.

There are many reasons why \gls{fpga} implementations of numerical algorithms
are now best obtained via \gls{hls} from C\@: less development effort, the
abundance of software engineers compared to hardware designers, the relative
ease of testing C code on an ordinary microprocessor, the opportunities
for rapid design space exploration, and so on~\cite{gajski94, meeus12,
nane15}. Great advances have been made in this area recently, and the
output from \gls{hls} tools is nowadays competitive with hand-crafted
designs~\cite{bdti_xilinx}.

Numerical C programs are typically written with floating-point
arithmetic, following the \emph{IEEE 754 standard} for floating-point
computation~\cite{ieee754}.  Floating-point numbers can represent a wide
range of real values, and most programming languages support the standard
seamlessly.  The standard has become ubiquitous, and is used in most of our
software and hardware implementations of numerical programs.  Recently,
Altera has introduced Arria 10 and Stratix 10 devices to incorporate hardened
floating-point \gls{dsp} blocks in the \gls{fpga} fabric~\cite{stratix10fp}.
As a result, we expect to see floating-point arithmetic continuing to dominate
in the \gls{fpga} implementation of numerical applications.

Although we make use floating-point arithmetic to implement algorithms,
generally specified in real arithmetic, in practice, it is often neglected that
floating-point computations almost always have \emph{round-off errors}, \ie~the
discrepancy between the actual result in real arithmetic and the rounded result
computed with floating-point arithmetic.  Round-off errors, when accumulated,
can have a devastating effect on numerical accuracy~\cite{higham02}.

In fact, properties such as associativity $(a + b) + c \equiv a + (b +
c)$ and distributivity $a \times (b + c) \equiv a \times b + a \times c$
which we consider to be fundamental laws of real numbers no longer hold
under floating-point arithmetic~\cite{goldberg}.  For instance, under
single-precision floating-point arithmetic with rounding to the nearest, the
result of $(2^{-24} + 2^{-24}) + 1 = 1.00000012\mathellipsis$ is exact, but
$(1 + 2^{-24}) + 2^{-24}$ is rounded to $1$.  Round-off errors in a numerical
program are dependent on every arithmetic operation and every input value, and
with the impact on floating-point accuracy being so esoteric, it is challenging
for engineers to understand the repercussions of switching between
``\verb|(a + b) * c|'' and ``\verb|a * c + b * c|'' in their programs.

These numerical properties nevertheless open the possibility of using these
rules to generate a program equivalent to the original program in real
arithmetic, but which could have better quality than the original when
evaluated in floating-point computation.  Experienced engineers often apply
such expression rewriting intuitions in numerical programs.  For instance,
when summing a sequence of floating-point values, one can sometimes reduce
round-off error in the result by summing the inputs in ascending order.  On
the other hand, one can often reduce latency by applying \emph{expression
balancing}, \ie~rearranging operators in an expression to construct a balanced
tree, so that more operators can work in parallel.  These heuristics cover a
very limited number of possible transformations and may not always improve
the original code.  There does not exist a trivial process to apply steps of
transformations using equivalence rules to \emph{optimally} trade off latency,
resources and numerical accuracy.

Existing \gls{hls} tools consider these rewrites to be
unsafe~\cite{vivado_hls}, and thus make very limited use of them
when restructuring floating-point data-paths.  For instance,
\gls{vhls}~\cite{vivado_hls} has only a very simple \emph{expression balancing}
feature that uses associativity to improve latency, and only expressions with
either additions or multiplications are optimized.  Moreover, it does not
produce optimal loop pipelining, because it does not take into account the
implications of these transformations on inter-iteration dependences and does
not explore partial loop unrolling.  In addition, \gls{vhls} cannot reason
about how this feature affects numerical accuracy; there is no guarantee
that this transformation will not result in a catastrophically inaccurate
implementation.

In response, this thesis proposes new methodologies and an associated
tool---\soap, a fully automatic source-to-source optimizer that augments
\gls{vhls}---to optimize a given numerical C program using these
transformations in tandem with conventional program transformations.  The
optimizer discovers not only one, but a wide spectrum of program candidates.
When synthesized in \gls{vhls}, these candidates trade off three performance
metrics of great importance to engineers: run time, resource usage and
round-off error.  Here, run time refers to the latency in clock cycles,
resource usage refers to the number of \glspl{lut} and \gls{dsp} elements.
Some of these performance metrics could be in conflict.  For example, higher
performance tends to require more circuitry, and how to resolve this trade-off
depends on the user's requirements.  As a result, the tool produces a
\emph{set} of optimized programs, known as the \emph{Pareto frontier}: those
programs $P$ for which the tool has found no $P'$ that improves on $P$ in all
three metrics.  We thus go beyond existing literature by not only optimizing
the precision requirements of an implementation, but changing the structure of
the implementation itself.

The program optimization flow is \emph{safe} and \emph{semantics-directed}.
\emph{Safety} means that because we base our method on formal mathematics to
optimize programs, our approach can be proved correct, in the sense that when
executed using exact real arithmetic, the transformed version produces exactly
the same output values as the original program. \emph{Semantics-directed}
transformation means that not only do we use program syntax, but also the
semantics, such as inferred numerical accuracy, to guide optimization and
guarantee safety properties of the optimized program.  Our technique obtains
when necessary, by analyzing the program, a bound and a round-off error bound
on each variable in every program location.  These information are then used to
guide program optimization, by analyzing and manipulating not only the syntax,
but also the semantics of programs.

Generating candidate optimizations na{\"\i}vely would however produce a
combinatorial explosion, even for small input programs.  For instance, in the
worst case, the parsing of a simple summation of $n$ variables could result
in $(2n - 3)!! = 1 \times 3 \times 5 \times \cdots \times (2n - 3)$ distinct
expressions modulo commutativity, \ie~we make use of associativity but ignore
any distinction caused by commutativity~\cite{ioualalen, mouilleron}.  This is
further complicated by distributivity as the expression ${(a + b)}^k$ could
expand into an expression with a summation of $2^k$ terms each with $k - 1$
multiplications.  Usually, for this reason, it would be infeasible to generate
a complete set of equivalent expressions using the rules of equivalence, since
an expression with a moderate number of terms will have a very large number of
equivalent expressions, and this number grows faster than exponential rate with
the increase of the number of terms.  New approaches were therefore developed
specifically to tackle the efficient discovery of equivalent structures in
numerical programs.  First, we invent a new intermediate representation,
called \gls{mir} to reduce the size of our search space without affecting its
optimality.  Second, we further reduce the effort of exploring the new search
space by intelligently pruning the set of candidates as it progresses up the
input program's abstract syntax tree.

All of the techniques above are designed with \emph{compositionality} in mind.
This means that we recursively break down each component we work on---such
as programs and \glspl{mir}---into smaller components and use our analyses
to calculate the results of each, such as accuracy, area and latency, and
subsequently equivalent candidates; the final results are then constructed from
those of the subcomponents.  When compared with a global approach, there are
two major advantages.  First, because components are considered independent
of each other, once analyzed the results can be reused in the analysis in the
larger enclosing components.  Another advantage of this is that although we
will not formally prove the correctness of the methodologies in this thesis,
they are designed to be easily expressible in a formal language, and by proving
small lemmas, we can deductively prove the correctness of, for instance, a
program-to-\gls{mir} translation.
