We believe that it is possible to extend our tool for the multi-objective
optimization of arithmetic expressions in the following ways. First, Secondly,
it would be useful to further allow transformations of expressions while
allowing different mantissa widths in the subexpressions, this further
increases the number options in the Pareto frontier, as well as leads to more
optimized expressions. Thirdly, as an alternative for interval analysis, we
could employ more sophisticated abstract domains that capture the correlations
between variables, and produce tighter bounds for results. Finally, there
could be a lot of interest in the HLS community on how our tool can be
incorporated with Martin Langhammer's work on fused floating-point data-path
synthesis~\cite{langhammer}.

We are convinced that with the foundation and framework that we developed,
our tool can be extended in the following ways.  First, it can be trivially
extended to support additional numerical data structure such as arrays and
matrices.  Secondly, the Pareto optimization can be extended to optimize the
latencies of equivalent programs, as restructuring programs and partially
unrolling loops could have a notable impact on the ability to pipeline
program loops, especially when arrays are incorporated.  Finally, fixed point
representations, along with the interaction between our structural optimization
and multiple wordlength optimization~\cite{constantinides} could also generate
a lot of interest from the HLS community.

Currently, our tool sees a diminishing performance return when loops are deeply
unrolled, because of a memory bottleneck. We are exploring an extension to
our tool that enables it to automatically partition arrays upon hitting such
a memory bottleneck. Also, our tool currently supports only single-precision
floating-point data types; we intend to extend this to multiple-precision
types, and explore the impact on our three performance metrics: latency,
resource utilization and numerical accuracy.
