\section{Fixpoint Theorems}

\begin{theorem}
    \textup{[Tarski's fixpoint theorem]}
    If $\mathsf{L}$ is a complete lattice\footnotemark[2], and a function
    $g: \mathsf{L} \to \mathsf{L}$ is a monotone function, then $\lfp(g)$,
    the \gls{lfp} of $g$, is the greatest lower bound of all fixpoints
    $\mathrm{Fix}(g)$.\label{bg:thr:tarski}
\end{theorem}
\footnotetext[2]{%
    Exact definitions of \emph{complete lattice} and \emph{complete partial
    order} are not required in this section.  Both of them can be found
    in~\cite{nielson99}.
}

In our case, $f$ is a \emph{monotone} function, because by definition a
\emph{monotone} function satisfies the condition that if $X \subseteq Y$,
then $g(X) \subseteq g(Y)$.  In the \gls{dfa} of \verb|simple|, $\mathsf{L}
= \powersetof\realset$, which is a complete lattice, since all power sets
are complete lattices~\cite{nielson99}.  The \gls{lfp} of $f$, that is the
intersection of all elements in $\mathrm{Fix}(f)$, is given by this equation:
\begin{equation}
    \lfp (f) = \bigcap \mathrm{Fix}(f),
\end{equation}
which implies that $\lfp (f) \subseteq Z$ for all fixpoints $Z \in
\mathrm{Fix}(f)$, thus $\lfp (f)$ is the unique and most precise solution we
are looking for.

Secondly, another theorem~\cite{kleene52}, which is closely
related to Theorem~\ref{bg:thr:tarski}, states:
\begin{theorem}
    \textup{[Kleene's fixpoint theorem]}
    If $\mathsf{L}$ is a \gls{cpo}\footnotemark[2], and $g: \mathsf{L} \to
    \mathsf{L}$ is a Scott-continuous function, then the $\lfp (g)$ can be
    computed as the least upper bound of all values in the sequence $\bot$,
    $g(\bot)$, $g^2(\bot)$, $g^3(\bot)$, \textellipsis{}\label{bg:thr:kleene}
\end{theorem}
Here, $\bot$ denotes the least element in $\mathsf{L}$.  

Our function $f$ is \emph{Scott-continuous}: it is monotone; and for any chain
of $X_0 \subseteq X_1 \subseteq X_2 \subseteq X_3 \subseteq \mathellipsis$,
where $X_i \in \powersetof\realset$:
\begin{equation}
    \bigcup_{i \in \naturalset} f(X_i) = f \left(
        \bigcup_{i \in \naturalset} X_i
    \right).
\end{equation}
As a \gls{cpo} is more general that a complete lattice, and the least
element in $\powersetof\realset$ is the empty set $\varnothing$, using
Theorem~\ref{bg:thr:kleene} in our example analysis, the most precise solution
of $A(1)$ can therefore be computed using:


\section{Round-off Error}

one of the \emph{rounding modes} $\circ \in \{ -\infty, \infty, 0, \neg0, \sim
\}$ which are round towards negative infinity, towards infinity, towards zero,
away from zero and towards nearest floating-point value respectively.  It is
defined as:
\begin{equation}
    \begin{aligned}
        & \downarrow^\sharp_\circ(\interval{a}{b}) \defeq \left\{
            \begin{aligned}
                & \left[ -\frac{z}{2}, \frac{z}{2}\right]
                    & \quad \circ & \text{~is~}\sim, \\
                & \left[ -z, z\right]
                    & \quad \circ & \in \{ -\infty, \infty, 0, \neg0 \},
            \end{aligned}
        \right. \\
        & \qquad\qquad\qquad\qquad \text{where~} z = \max(\ulp(a), \ulp(b)).
    \end{aligned}
\end{equation}
