\section{Resource Usage Analysis}
\label{so:sec:resource}

Here we define similar formal semantics which calculate an approximation to
the \gls{fpga} resource usage of an expression, taking into account common
subexpression elimination. This is important as, for example, rewriting $a
\times b + a \times c$ as $a \times (b + c)$ in the larger expression $(a
\times b + a \times c) + {(a \times b)}^2$ causes the common subexpression $a
\times b$ to be no longer present in both terms. Our analysis must capture
this.

The analysis proceeds by labelling subexpressions. Intuitively, the
set of labels $\labelset$, is used to assign unique labels to unique
expressions, so it is possible to easily identify and reuse them. For
convenience, let the function $\fresh: \aexprset\to\labelset$ assign a
distinct label to each expression or variable, where $\aexprset$ is the set
of all expressions.  It is noteworthy that $\fresh$ is a bijection. Before
we introduce the labeling semantics, we define the environment $\lambda:
\labelset\to\aexprset\cup\{\bot\}$, which is a function that maps labels to
expressions, and $\env{}$ denotes the set of such environments. A label $l$ in
the domain of $\lambda\in\env{}$ that maps to $\bot$ indicates that $l$ does
not map to an expression. An element $(l, \lambda)\in\labelset\times\env{}$
stands for the labeling scheme of an expression. Initially, we map all labels
to $\bot$, then in the mapping $\lambda$, each leaf of an expression is
assigned a unique label, and the unique label $l$ is used to identify the leaf.
That is for the leaf variable or constant $x$:
\begin{equation}
    (l, \lambda) = (\fresh(x), [\fresh(x)\mapsto{x}]).
\end{equation}
This equation uses $[\fresh(x)\mapsto{x}]$ to indicate an environment that
maps the label $\fresh(x)$ to the expression $x$ and all other labels map
to $\bot$, in other words, if $l = \fresh(x)$ and $l^\prime \neq l$, then
$\lambda(l) = x$ and $\lambda(l^\prime) = \bot$.

\begin{figure}[ht]
    \centering
    \includegraphics[scale=0.6]{sample_tree}
    \caption{%
        The \gls{dfg} for the sample expression.
    }\label{so:fig:sample_tree}
\end{figure}
For example, for the \gls{dfg} in Figure~\ref{so:fig:sample_tree} we have for
the variables $a$ and $b$:
\begin{equation}
    \begin{aligned}
        (l_a, \lambda_a) &= (\fresh(a), [\fresh(a)\mapsto{a}])
                   = (l_1, [l_1 \mapsto a]), \\
        (l_b, \lambda_b) &= (l_2, [l_2 \mapsto b]).
    \end{aligned}
    \label{so:eq:variable_env}
\end{equation}
Then the environments are propagated in the flow direction of the \gls{dfg},
using the following formulation of the labeling semantics:
\begin{equation}
    \begin{aligned}
        (l_x, \lambda_x) \circ (l_y, \lambda_y)
            &= (l, (\lambda_x\odot\lambda_y)
                      [l\mapsto{l_x \circ l_y}]), \\
            \text{where~} l &= \fresh(l_x \circ l_y),
                          \circ\in\{+, -, \times\}.
    \end{aligned}
    \label{so:eq:labeling_semantics}
\end{equation}
Specifically, $\lambda=\lambda_x\odot\lambda_y$ signifies that $\lambda_y$
is used to update the mapping in $\lambda_x$, if the mapping does not
exist in $\lambda_x$, and result in a new environment $\lambda$; and
$\lambda[l\mapsto{x}]$ is a shorthand for $\lambda\odot[l\mapsto{x}]$.  As
an example, with the expression in Figure~\ref{so:fig:sample_tree}, using
\eqref{so:eq:variable_env}, recall to mind that $l_1 = l_a$, $l_2 = l_b$, we
derive for the subexpression $a + b$:
\begin{equation}
    \begin{aligned}
        (l_{a + b}, \lambda_{a + b})
            &= (l_a, \lambda_a) + (l_b, \lambda_b) \\
            &= (l_3, (\lambda_a \odot \lambda_b) [l_3\mapsto{l_a + l_b}]) \\
            &  \text{where~} l_3 = \fresh(l_a + l_b) \\
            &= (l_3, [l_1\mapsto{a}]\odot
                     [l_2\mapsto{b}]\odot
                     [l_3\mapsto{l_1 + l_2}]) \\
            &= (l_3, [l_1\mapsto{a}, l_2\mapsto{b}, l_3\mapsto{l_1 + l_2}]),
    \end{aligned}
\end{equation}
where $l_a + l_b$ is a syntactic construct to signify that the subexpressions
with labels $l_a$ and $l_b$ are added to form an expression.
% \todo{George: I am a little confused by addition here.  What is the definition of ``+'' on labels?  If $l_a + l_b$ should be read purely as a syntactic construct, why does it need a distinct representation as $l_3$?}
Finally, for the full expression $(a + b) \times (a + b)$:
\begin{equation}
    \begin{aligned}
        (l, \lambda)
            &= (l_{a + b}, \lambda_{a + b}) \times
               (l_{a + b}, \lambda_{a + b}) \\
            &= (l_4, [l_1\mapsto{a}, l_2\mapsto{b},
                      l_3\mapsto{l_1 + l_2}, l_4\mapsto{l_3 \times l_3}]).
    \end{aligned}
\end{equation}
From the above derivation, it is clear that the semantics capture the reuse
of subexpressions. The estimation of area is performed by counting, for an
expression, the numbers of additions, subtractions and multiplications in the
final labeling environment, then calculating the number of \glspl{lut} used
to synthesize the expression. If the number of operators is $n_\circ$ where
$\circ\in\{+,-,\times\}$, then the number of \glspl{lut} in total for the
expressions is estimated as $\sum_{\circ\in\{+,-,\times\}} A_\circ n_\circ$,
where the value $A_\circ$ denotes the number of \glspl{lut} per $\circ$
operator, which is dependent on the type of the operator and the floating-point
format used to generate the operator.

In the following sections, we use the function $\area: \aexprset\to\naturalset$
to denote our resource usage analysis.
