\section{Implementation}
\label{so:sec:implementation}

The majority of \soap, is implemented in Python.  For computing errors in
real arithmetic, we use exact arithmetic based on rational numbers within the
\gls{gmp} library~\cite{gmp}.  In case when exact arithmetic is not possible
because of high computational costs, floating-point arithmetic can be used
to efficiently and safely bound round-off error values.  We also use the
\gls{mpfr} library~\cite{mpfr} for access to floating-point rounding modes and
arbitrary precision floating-point computation.

Because of the workload of equivalent expression finding, the underlying
algorithm is optimized in many ways.  Firstly, for each iteration, the relation
finding function $\eqstep_k$ is only applied to newly discovered expressions
in the previous iteration, using the algorithm in Figure~\ref{so:alg:closure}.
The second optimization is to cache results of function calls such as
$\eqstep_k$, $\area$ and $\error$, since there is a large chance that these
results from subexpressions are reused several times, subexpressions are also
maximally shared to eliminate duplication in memory.  Thirdly, the computation
of $\eqstep_k$ is fully multi-threaded.

The resource statistics of operators are provided using FloPoCo~\cite{flopoco}
and \gls{xst}~\cite{xst}.  Initially, For each combination of an operator, an
exponent width between 5 and 16, and a mantissa width ranging from 10 to 113,
a total of 2496 distinct implementations are generated using FloPoCo.  All
of them are optimized to use \gls{dsp} blocks.  They are then synthesized
using \gls{xst}, targeting a Virtex-6 \gls{fpga} device (XC6VLX760).  Because
\glspl{lut} are generally more constrained resources than \gls{dsp} blocks
in floating-point computations, only synthesis statistics in \glspl{lut}
is provided.  Finally, an \gls{rtl} code generation backend can produce
synthesizable code from an optimized candidate expression.
