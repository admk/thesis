\section{Results}
\label{so:sec:results}

Because Martel's approach defers selecting optimal options until the end of
equivalent expression discovery, we developed a method that could produce
exactly the same set of equivalent expressions from the traces computed by
Martel, and has the same time complexity, but we adopted it to generate
a Pareto frontier from the discovered expressions, instead of only error
bounds.  This allows us to compare \marteltrace{}, \ie~our implementation
of Martel's method, against our methods \frontiertrace{} and \greedytrace{}
discussed in Section~\ref{so:sub:equivalent_expressions_analysis}.
Figure~\ref{so:fig:martel} optimizes the expression ${(a + b)}^2$ using the
three methods above, all using depth limit $3$, and the input ranges are
$a \in [5, 10]$ and $b \in [0, 0.001]$~\cite{martel07}. The IEEE754 single
precision floating-point format with rounding to nearest was used for the
evaluation of accuracy and area estimation. The scatter points represent
different implementations of the original expression that have been explored
and analyzed, and the (overlapping) lines denote the Pareto frontiers. In this
example, our methods produce the same Pareto frontier that Martel's method
could discover, while having up to 50\% shorter run time. Because we consider
an accuracy/area trade-off, we find that we can not only have the most accurate
implementation discovered by Martel, but also an option that is only 0.0005\%
less accurate, but uses 7\% fewer LUTs.

We go beyond the optimization of a small expression, by generating results in
Figure~\ref{so:fig:multi_expr_32} to show that the same technique is applicable
to simultaneous optimization of multiple large expressions. The expressions
$e_x$ and $e_y$, with input ranges $a \in [1, 2], b \in [10, 20], c \in [10,
200]$ are used as our example:
\begin{equation}
    \begin{aligned}
    e_x =& (a + a + b) \times (a + b + b) \times (b + b + c) \times {} \\
         & (b + c + c) \times (c + c + a) \times (c + a + a) \\
    e_y =& (1 + b + c) \times (a + 1 + c) \times (a + b + 1)
    \end{aligned}
\end{equation}

We generated and optimized RTL implementations of $e_x$ and
$e_y$ simultaneously using \frontiertrace{} and \greedytrace{}
with the depth limits indicated by the numbers in the legend of
Figure~\ref{so:fig:multi_expr_32}. Note that because the expressions
evaluate to large values, the errors are also relatively large. We set the
depth limit to $2$ and found that \greedytrace{} executes much faster than
\frontiertrace{}, while discovering a subset of the Pareto frontier of
\frontiertrace{}. Also our methods are significantly faster and more scalable
than \marteltrace{}, because of its poor scalability discussed earlier, our
computer ran out of memory before we could produce any results. If we normalize
the time allowed for each method and compare the performance, we found that
\greedytrace{} with a depth limit $3$ takes takes slightly less time than
\frontiertrace{} with a depth limit $2$, but produces a generally better
Pareto frontier. The alternative implementations of the original expression
provided by the Pareto frontier of \greedytrace{} can either reduce the LUTs
used by approximately 10\% when accuracy is not crucial, or can be about 10\%
more accurate if resource is not our concern.  It also enables the ability to
choose different trade-off options, such as an implementation that is 7\% more
accurate and uses 7\% fewer LUTs than the original expression.

Furthermore, Figure~\ref{so:fig:multi_expr_vary_width} varies the mantissa
width of the floating-point format, and presents the Pareto frontier
of both $e_x$ and $e_y$ together under optimization. Floating-point
formats with mantissa widths ranging from 10 to 112 bits were used to
optimize and evaluate the expressions for both accuracy and area usage. It
turns out that some implementations originally on the Pareto frontier of
Figure~\ref{so:fig:multi_expr_32} are no longer desirable, as by varying the
mantissa width, new implementations are both more accurate and less resource
demanding.

Besides the large example expressions above, Figure~\ref{so:fig:taylor_sin}
and Figure~\ref{so:fig:motzkin} are produced by optimizing expressions with
real applications under single precision. Figure~\ref{so:fig:taylor_sin} shows
the optimization of the Taylor expansion of $\sin(x + y)$, where $x\in[-0.1,
0.1]$ and $y\in[0, 1]$, using \greedytrace{} with a depth limit $3$. The
function $\mathrm{taylor}(f, d)$ indicates the Taylor expansion of function
$f(x, y)$ at $x = y = 0$ with a maximum degree of $d$. For order 5 we reduced
error by more than 60\%. Figure~\ref{so:fig:motzkin} illustrates the results
obtained using the depth limit $3$ with the Motzkin polynomial~\cite{demmel}
$x^6 + y^4 z^2 + y^2 z^4 - 3 x^2 y^2 z^2$, which is known to be difficult to
evaluate accurately, especially using inputs $x\in[-0.99, 1]$, $y\in[1, 1.01]$,
$z\in[-0.01, 0.01]$.

Finally, Figure~\ref{so:fig:area} demonstrates the accuracy of the area estimation
used in our analysis. It compares the actual LUTs necessary with the estimated
number of LUTs using our semantics, by synthesizing more than 6000 equivalent
expressions derived from $a + b + c$, $(a + 1) \times (b + 1) \times (c + 1)$,
$e_x$, and $e_y$ using varying mantissa widths. The dotted line indicates exact
area estimation, a scatter points that is close to the line means the area
estimation for that particular implementation is accurate. The solid black line
represents the linear regression line of all scatter points. On average, our
area estimation is a 6.1\% over-approximation of the actual number of LUTs, and
the worst case over-approximation is 7.7\%.
\newcommand{\figsize}{0.6}
\begin{figure}[ht]
    \centering
    \includegraphics[scale=\figsize]{martel}
    \caption{Optimization of ${(a + b)}^2$}\label{so:fig:martel}
\end{figure}
\begin{figure}[ht]
    \centering
    \includegraphics[scale=\figsize]{multi_expr_32}
    \caption{Simultaneous optimization of both $e_x$ and $e_y$}
    {}\label{so:fig:multi_expr_32}
\end{figure}
\begin{figure}[ht]
    \centering
    \includegraphics[scale=\figsize]{multi_expr_vary_width}
    \caption{Varying the mantissa width of Figure~\ref{so:fig:multi_expr_32}}
    {}\label{so:fig:multi_expr_vary_width}
\end{figure}
\begin{figure}[ht]
    \centering
    \includegraphics[scale=\figsize]{taylor_sin}
    \caption{The Taylor expansion of $\sin(x + y)$}
    {}\label{so:fig:taylor_sin}
\end{figure}
\begin{figure}[ht]
    \centering
    \includegraphics[scale=\figsize]{motzkin}
    \caption{The Motzkin polynomial $e_m$}\label{so:fig:motzkin}
\end{figure}
\begin{figure}[ht]
    \centering
    \includegraphics[scale=\figsize]{area}
    \caption{Accuracy of Area Estimation}\label{so:fig:area}
\end{figure}
