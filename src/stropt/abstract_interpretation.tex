\section{Abstract Interpretation}
\label{sec:abstract_interpretation}

\todo{Move this to background chapter?}

This section introduces the basic concepts of formal semantics and abstract
interpretation used in this chapter. We illustrate these concepts by putting
the familiar idea of interval arithmetic~\cite{moore} in the framework
of abstract interpretation. This is then further extended later in the
chapter to define a scalable analysis capturing ranges, errors and resource
utilization. As an illustration, consider the following expression and its DFG
in Figure~\ref{fig:sample_tree}\@:
\begin{equation}
    (a + b) \times (a + b)
    \label{eq:absint_sample}
\end{equation}
\begin{figure}[ht]
    \centering
    \includegraphics[scale=0.6]{sample_tree}
    \caption{The DFG for the sample program.}\label{fig:sample_tree}
\end{figure}

We may wish to ask: if initially $a$ and $b$ are real numbers in the range of
$[0.2, 0.3]$ and $[2, 3]$ respectively, what would be the outcome of evaluating
this expression with real arithmetic? A straightforward approach is simulation.
Evaluating the expression for a large quantity of inputs will produce a set
of possible outputs of the expression. However the simulation approach is
unsafe, since there are infinite number of real-valued inputs possible and it
is infeasible to simulate for all.

A better method might be to represent the possible values of $a$ and $b$ using
ranges. To compute the ranges of its output values, we could operate on ranges
rather than values (note that the superscript $\sharp$ denotes ranges). Assume
that $a^\sharp_{init} = [0.2, 0.3]$, $b^\sharp_{init} = [2, 3]$, which are the
input ranges of $a$ and $b$, and $\enter{l}$ where $l \in \{1, 2, 3, 4\}$ are
the intervals of the outputs of the boxes labelled with $l$ in the DFG\@. We
extract the data flow from the DFG to produce the following set of equations:
\begin{equation}
    \begin{aligned}
        \enter{1} &= a^\sharp_{init} \\
        \enter{2} &= b^\sharp_{init} \\
        \enter{3} &= \enter{1} + \enter{2} \\
        \enter{4} &= \enter{3} \times \enter{3}
    \end{aligned}
    \label{eq:absint_sample_analysis}
\end{equation}
For the equations above to make sense, addition and multiplication need to be
defined on intervals. We may define the following interval operations:
\begin{equation}
    \begin{aligned}
        \interval{a}{b} + \interval{c}{d} &= \interval{a + c}{b + d} \\
        \interval{a}{b} - \interval{c}{d} &=  \interval{a - d}{b - c} \\
        \interval{a}{b} \times \interval{c}{d} &=
            \interval{\min(s)}{\max(s)} \\
        \text{where~} s &= \{ a \times c, a \times d, b \times c, b \times d \}
    \end{aligned}
    \label{eq:interval_operations}
\end{equation}
The solution to the set of~\eqref{eq:absint_sample_analysis} for $\enter{4}$
is $[4.84, 10.89]$, which represents a safe bound on the output at the end of
program execution. Note that in actual execution of the program, the semantics
represent the values of intermediate variables, which are real values. In our
case, a set of real values forms the set of all possible values produced by
our code. However computing this set precisely is not, in general, a possible
task. Instead, we use abstract interpretation based on intervals, which gives
the abstract semantics of this program. Here, we have achieved a classical
interval analysis by \emph{defining} the meaning of addition and multiplication
on abstract mathematical structures (in this case intervals) which capture
a safe approximation of the original semantics of the program. Later in
Section~\ref{sec:semantics}, we further generalize the idea by defining the
meaning of these operations on more complex abstract structures which allow us
to scalably reason about the range, error, and area of FPGA implementations.
