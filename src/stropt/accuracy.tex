\section{Accuracy Analysis}
\label{so:sec:accuracy}

The accuracy analysis used by \soap{} follows the method based on abstract
error domain introduced by Martel~\cite{martel07} to analyze the round-off
error of restructured floating-point expressions.  As it was mentioned
in Section~\ref{bg:ssub:accuracy}, they did not have a preference for the
choice of definition of $\ulp$.  Hence, in \soap{}, the distance between two
adjacent floating-point values $f_1$ and $f_2$ satisfying $f_1 \leq x \leq
f_2$~\cite{goldberg} is characterized using the following definition:
\begin{definition}
    In our analysis, the function $\ulp$ is defined as:
    \begin{equation}
        \ulp(x) = \left\{
            \begin{aligned}
                & \infty,  && \text{if $x$ is $-\infty$ or $\infty$}, \\
                & 2^{e(x) + 2^{k - 1} - 1} \times 2^{-p},
                \quad && \text{otherwise}.
            \end{aligned}
        \right.
    \end{equation}
    where $e(x)$ is the exponent of $x$, $k$ and $p$ are the parameters of the
    floating-point format as defined in~\eqref{bg:eq:floating_point}.
    {}\label{so:def:ulp}
\end{definition}

Since Martel~\cite{martel07} does not define the arithmetic operator for
division.  The following equations for division is therefore introduced in this
thesis.  Firstly, divisions on intervals can be implemented as follows:
\begin{equation}
    \frac{\interval{a}{b}}{\interval{c}{d}}
        \defeq \interval{\min(s)}{\max(s)},
\end{equation}
where $\interval{a}{b}, \interval{c}{d} \in \intervalset$ and:
\begin{equation}
    s = \left\{
    \begin{aligned}
        & \{ -\infty, \infty \}
            && \text{if~} c \leq 0 \leq d, \\
        & \left\{
            \frac{a}{c}, \frac{a}{d}, \frac{b}{c}, \frac{b}{d}
        \right\}
            \quad && \text{otherwise}.
    \end{aligned}
    \right.
\end{equation}
By evaluating the sum of error propagated $\frac{ x^\sharp_1 + \mu^\sharp_1 }{
x^\sharp_2 + \mu^\sharp_2 } - \frac{x^\sharp_1}{x^\sharp_2}$ and the round-off
error introduced by division, the division on values in the abstract error
domain can be derived as follows:
\begin{equation}
    \frac{
        \left( x^\sharp_1, \mu^\sharp_1 \right)
    }{
        \left( x^\sharp_2, \mu^\sharp_2 \right)
    }
    \defeq \left(
            \roundup{\frac{x^\sharp_1}{x^\sharp_2}},
            \frac{
                x^\sharp_2 \mu^\sharp_1 - x^\sharp_1 \mu^\sharp_2
            }{
                x^\sharp_1 \left( x^\sharp_2 + \mu^\sharp_2 \right)
            } + \rounddown{\frac{x^\sharp_1}{x^\sharp_2}}
        \right).
\end{equation}

We use the function $\error: \aexprset\to\errorset$ to represent the
analysis of round-off error in an expression tree, as described in
Section~\ref{bg:ssub:accuracy} of Chapter~\ref{chp:background}, using the above
$\ulp$ equation in Definition~\ref{so:def:ulp}, where $\aexprset$ denotes the
set of all arithmetic expressions.

For each expression in a set of equivalent expressions discovered, $e \in
\epsilon$, each expression $e$ evaluates to a distinct value in the abstract
error domain.  Section~\ref{bg:ssub:accuracy} of Chapter~\ref{chp:background}
presents a method to compare against each other with a partial ordering.
However, a total ordering is much more preferable, as all expressions can
be easily compared against one another.  In \soap, the following function
$\abserr$ is used to convert an evaluated outcome $v \in \errorset$ into a
scalar to denote the magnitude of round-off error:
\begin{equation}
    \begin{aligned}
        \abserr(e) &= \max\left(
            \left| \mu^\sharp_{\min} \right|,
            \left| \mu^\sharp_{\max} \right|
        \right) \\
        & \quad \text{where~}
        \left(
            x^\sharp, \left[ \mu^\sharp_{\min}, \mu^\sharp_{\max} \right]
        \right) = \error(e)
    \end{aligned}
    \label{so:eq:abserr}
\end{equation}
