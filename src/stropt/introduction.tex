\section{Introduction}
\label{so:sec:introduction}

% The IEEE 754 standard~\cite{ieee754} for floating-point computation is
% ubiquitous in computing machines. In practice, it is often neglected that
% floating-point computations almost always have roundoff errors. In fact,
% associativity and distributivity properties which we consider to be fundamental
% laws of real numbers no longer hold under floating-point arithmetic. This opens
% the possibility of using these rules to generate an expression equivalent to
% the original expression in real arithmetic, which could have better quality
% than the original when evaluated in floating point computation.

By exploiting rules of equivalence in arithmetic, such as associativity $(a
+ b) + c \equiv a + (b + c)$ and distributivity $(a + b) \times c \equiv a
\times c + b \times c$, it is possible to automatically generate different
implementations of the same arithmetic expression.  We optimize the structures
of arithmetic expressions in terms of the following two quality metrics
relevant to FPGA implementation: the resource usage when synthesized into
circuits, and a bound on roundoff errors when evaluated. Our goal is the joint
minimization of these two quality metrics.  This optimization process provides
a Pareto optimal set of implementations.  For example, our tool discovered that
with single precision floating-point representation, if $a \in [0.1, 0.2]$,
then the expression ${(a + 1)}^2$ uses fewest resources when implemented in the
form $(a + 1) \times (a + 1)$ but most accurate when expanded into $(((a \times
a) + a) + a) + 1$.  However it turns out that a third alternative, $((1 + a)
+ a) + (a \times a)$, is never desirable because it is neither more accurate
nor uses fewer resources than the other two possible structures. Our aim is to
automatically detect and utilize such information to optimize the structure of
expressions.

A na{\"\i}ve implementation of equivalent expression finding
would be to explore all possible equivalent expressions to find
optimal choices.  However this would result in combinatorial
explosion~\cite{ioualalen}.  Since none of the techniques explained
in Section~\ref{bg:sec:discovering_equivalent_programs} of
Chapter~\ref{chp:background} capture the optimization of both accuracy and
performance by restructuring arithmetic expressions, we base ourselves on the
software work of Martel~\cite{martel07}, but extend their work in the following
ways. Firstly, we develop new hardware-appropriate semantics to analyze not
only accuracy but also resource usage, seamlessly taking into account common
subexpression elimination. Secondly, because we consider both resource usage
and accuracy, we develop a novel multi-objective optimization approach to
scalably construct the Pareto frontier in a hierarchical manner, allowing fast
design exploration. Thirdly, equivalence finding is guided by prior knowledge
on the bounds of the expression variables, as well as local Pareto frontiers
of subexpressions while it is optimizing expression trees in a bottom-up
approach, which allows us to reduce the complexity of finding equivalent
expressions without sacrificing our ability to optimize expressions.  Following
Martel~\cite{martel07} and Ioualalen~\etal~\cite{ioualalen}, the methodology
explained in this chapter makes use of formal semantics as well as abstract
interpretation~\cite{cousot77} to significantly reduce the space and time
requirements and produce a subset of the Pareto frontier.

In order to further increase the options available in the Pareto frontier,
we introduce freedom in choosing mantissa widths for the evaluation of the
expressions. Generally as the precision of the evaluation increases, the
utilization of resources increases for the same expression. This gives
flexibility in the trade-off between resource usage and precision. Our
approach and its associated tool, SOAP, allow high-level synthesis flows to
automatically determine whether it is a better choice to rewrite an expression,
or change its precision in order to meet optimization goals.

The three contributions of this chapter are:
\begin{enumerate}
    \item Efficient methods for discovering equivalent structures of
    arithmetic expressions.
    \item A semantics-based program analysis that allows joint reasoning about
    the resource usage and safe ranges of values and errors in floating-point
    computation of arithmetic expressions.
    \item A tool which produces RTL implementations on the area-accuracy
    trade-off curve derived from structural optimization.
\end{enumerate}

This chapter is structured as follows.  Sections~\ref{so:sec:accuracy}
and~\ref{so:sec:resource} extend the basic concepts of semantics with abstract
interpretation explained in Section~\ref{bg:sec:abstract_interpretation} to
respectively analyze accuracy and resources.  Section~\ref{so:sec:equivalent}
discusses various abstract semantics for finding equivalent structure in
arithmetic expressions, as well as the analysis of their resource usage
estimates and bounds of errors.  Section~\ref{so:sec:implementation}
gives an overview of the implementation details in \soap.  Finally,
we discuss the results of optimized example expressions in
Section~\ref{so:sec:results} and end with concluding remarks for this chapter
in Section~\ref{so:sec:conclusion}.
