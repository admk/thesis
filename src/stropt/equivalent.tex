\section{Equivalent Expressions Analysis}
\label{so:sec:equivalent}

In earlier sections, we introduce semantics that define additions and
multiplications on intervals, then gradually transition to error semantics that
compute bounds of values and errors, as well as labelling environments that
allow common subexpression elimination, by defining arithmetic operations on
these structures. In this section, we now take the leap from not only analyzing
an expression for its quality, to defining arithmetic operations on sets of
equivalent expressions, and use these rules to discover equivalent expressions.
Before this, it is necessary to formally define equivalent expressions and
functions to discover them.

\subsection{Discovering Equivalent Expressions}

From an expression, a set of equivalent expressions can be discovered by our
\emph{equivalence relation} $\eqrel$ on the set of all arithmetic expressions
$\aexprset$, and $\eqrel \subset \aexprset\times\aexprset$.  It is noteworthy
that a relation is said to be an equivalence relation when it is reflexive,
symmetric and transitive, \ie~for all $e_1, e_2, e_3 \in \aexprset$, we
have the following rules in our inference system:
\begin{equation}
    \begin{aligned}{}
        & [\mathrm{Reflexivity}] & \quad &
            e_1 \eqrel e_1 \\
        & [\mathrm{Symmetry}] & \quad &
            \inference{e_1 \eqrel e_2}{e_2 \eqrel e_1} \\
        & [\mathrm{Transitivity}] & \quad &
            \inference{e_1 \eqrel e_2 \wedge e_2 \eqrel e_3}{e_1 \eqrel e_3}
    \end{aligned}
    \label{so:eq:equivalence_relation}
\end{equation}
Here, a rule of the form $\inference{a}{b}$ means that if the formula $a$ is
true, then $b$ also holds.  We extend our inference system with additional
arithmetic rules that relate equivalent expressions.  Let's define $e_1, e_2,
e_3 \in \aexprset$, $v_1, v_2, v_3 \in \realset$.  Firstly, we start with a set
of associativity rules:
\begin{equation}
    \begin{aligned}{}
        & [\mathrm{Assoc}_1] & \quad &
            \forall \circ \in \{ +, \times \}:
            (e_1 \circ e_2) \circ e_3 \eqrel e_1 \circ (e_2 \circ e_3) \\
        & [\mathrm{Assoc}_2] & \quad &
            \forall \circ \in \{ +, \times, / \}:
            (e_1 \circ e_2) \circ e_3 \eqrel (e_1 \circ e_3) \circ e_2 \\
        & [\mathrm{Assoc}_3 (/)] & \quad &
            (e_1 / e_2) / e_3 \eqrel e_1 / (e_2 \times e_3) \\
        & [\mathrm{Assoc}_4 (/)] & \quad &
            e_1 / (e_2 / e_3) \eqrel e_1 \times e_3 / e_2
    \end{aligned}
\end{equation}
Secondly, we define commutativity rules for addition and multiplication:
\begin{equation}
    [\mathrm{Commut}] \quad
    \forall \circ \in \{ +, \times \}: e_1 \circ e_2 \eqrel e_2 \circ e_1
\end{equation}
Thirdly, distributivity rules enable further equivalent expressions to be
explored with our relation:
\begin{equation}
    \begin{aligned}{}
        & [\mathrm{Distrib}_1] & \quad &
            \forall \circ \in \{ \times, / \}:
            e_1 \circ e_3 + e_2 \circ e_3 \eqrel (e_1 + e_2) \circ e_3 \\
        & [\mathrm{Distrib}_2 (\times)] & \quad &
            e_1 + e_1 \times e_2 \eqrel (1 + e_2) \times e_1 \\
        & [\mathrm{Distrib}_3 (\times)] & \quad &
            e_1 + e_1 \eqrel 2 \times e_1 \\
        & [\mathrm{Distrib}_4 (/)] & \quad &
            e_1 + e_2 / e_3 \eqrel (e_1 \times e_3 + e_2) / e_3 \\
        & [\mathrm{Distrib}_5 (/)] & \quad &
            e_1 / e_3 + e_2 / e_4 \eqrel
            (e_1 \times e_4 + e_2 \times e_3) / (e_3 \times e_4) \\
        & [\mathrm{Distrib}_6] & \quad &
            \forall \circ \in \{ +, - \}:
            -(e_1 \circ e_2) \eqrel (-e_1) \circ (-e_2) \\
        & [\mathrm{Distrib}_7] & \quad &
            \forall \circ \in \{ \times, / \}:
            -(e_1 \circ e_2) \eqrel (-e_1) \circ e_2 \eqrel e_1 \circ (-e_2)
    \end{aligned}
\end{equation}
The following rule rewrites an expression with subtraction into an addition:
\begin{equation}
    [\mathrm{Subtract}] \quad
    e_1 - e_2 \eqrel e_1 + (-e_2)
\end{equation}
The following reduction rules propagates constant values in expression
trees. The $\mathrm{ConstProp}_1$ rule states that if an expression is an
arithmetic operation of two constant values, then it can be simply evaluated to
produce the result; and $\mathrm{ConstProp}_2$ rewrites an expression that has
a negation on a constant value into a single constant.
\begin{equation}
    \begin{aligned}{}
        & [\mathrm{ConstProp_1}] &~&
        \inference{%
            (v_3 = v_1 \circ v_2) \wedge
            (\circ \in \binopset)
        }{%
            v_1 \circ v_2 \eqrel v_3
        } \\
        & [\mathrm{ConstProp_2} (-)] &~&
        \inference{%
            v_2 = -v_1
        }{%
            -v_1 \eqrel v_2
        }
    \end{aligned}
    \label{so:eq:constant_propagation}
\end{equation}
Here we define $\binopset = \{ +, -, \times, / \}$ to be the set of binary
operators.  We also define another set of reduction rules looking for common
patterns that can be simplified:
\begin{equation}
    \begin{aligned}{}
        & [\mathrm{Identity}_1] &~&
            \forall \circ \in \{ \times, / \}:
            e_1 \circ 1 \eqrel e_1
            & \quad
        & [\mathrm{Elim}_1 (-)] &~&
            e_1 - e_1 \eqrel 0
            \\
        & [\mathrm{Identity}_2] &~&
            \forall \circ \in \{ +, - \}:
            e_1 \circ 0 \eqrel e_1
            & \quad
        & [\mathrm{Elim}_2 (/)] &~&
            e_1 / e_1 \eqrel 1
            \\
        & [\mathrm{ZeroProp} (\times)] &~&
            0 \times e_1 \eqrel 0
            & \quad
        & [\mathrm{DoubleNeg} (-)] &~&
            -(-e_1) \eqrel e_1
            \\
    \end{aligned}
    \label{so:eq:common_reduction}
\end{equation}
Finally, the following two allow structural induction on expression trees,
\eg~it is possible to derive that $a + (b + c) \eqrel a + (c + b)$ from $b + c
\eqrel c + b$:
\begin{equation}
    \begin{aligned}
        & [\mathrm{Tree}_1] &~&
        \inference{%
            \left( e_1 \eqrel e_2 \right) \wedge
            \left( e_3 \eqrel e_4 \right) \wedge
            \left( \circ \in \binopset \right)
        }{%
            e_1 \circ e_3 \eqrel e_2 \circ e_4
        } \\
        & [\mathrm{Tree}_2] &~&
        \inference{%
            e_1 \eqrel e_2
        }{%
            -e_1 \eqrel -e_2
        }
    \end{aligned}
    \label{so:eq:equivalence_tree}
\end{equation}

We say that $e_1$ is equivalent to $e_2$ if and only if $e_1 \eqrel e_2$ For
some expressions $e_1$ and $e_2$.  Many rules are considered redundant and thus
excluded from our equivalence relation, as these can be derived by combining
rules from our inference system.  For instance, $e_1 \times (e_2 + e_3) \eqrel
e_1 \times e_2 + e_1 \times e_3$ can be derived by using $\mathrm{Distrib}_1$,
in tandem with the $\mathrm{Commut}$ rule.


\subsection{Scalable Methods}

The above rules of equivalence relate an expression with all of its equivalent
expressions.  In general because of combinatorial explosion, the set of all
equivalent expressions is so large to be derived, which motivates us to develop
scalable methods that execute fast enough even with large expressions.

Instead of deriving the full set of equivalent expressions, we can
define a new relation $\eqgenrel$, a subset of $\eqrel$, which
is identical to our equivalent relation $\eqrel$ except that we
place a few restrictions on the relation.  Firstly, we do not have
transitivity in~\eqref{so:eq:equivalence_relation} from $\eqrel$, as
we can have the flexibility to apply $\eqgenrel$ in a series of steps
to generate equivalent expressions.  Secondly, we further disallow
symmetry in~\eqref{so:eq:equivalence_relation} for the reduction rules
in~\ref{so:eq:constant_propagation}~and~\ref{so:eq:common_reduction} to reduce
the space-time complexity of the search space, because often performance
metrics, such as accuracy and resource usage, improve when the number of terms
in the expression is reduced.

\begin{definition}
    We call a function an \emph{equivalent expression generator} (EEG) function
if and only if the function takes as an input an initial set of equivalent
expressions, and generates a set of equivalent expressions.
\end{definition}

For instance, an EEG function $\eqstep: \powersetof\aexprset \to
\powersetof\aexprset$, where $\powersetof\aexprset$ denotes the power set of
$\aexprset$, can be defined as follows:
\begin{equation}
    \eqstep(\epsilon) = \left\{
        e^\prime\in\aexprset \mid
        e \eqgenrel e^\prime \wedge e\in\epsilon\right\}
    \label{so:eq:eqstep}
\end{equation}
where $\epsilon$ is a set of equivalent expressions.

\begin{definition}
    We say an EEG function $f$ is \emph{$\cup$-distributive} if and only if
    the function satisfies $f(\epsilon_a \cup \epsilon_b) = f(\epsilon_a) \cup
    f(\epsilon_b)$.
\end{definition}
\begin{corollary}
    By the definition of $\eqstep$ in~\eqref{so:eq:eqstep}, it is clear that
    $\eqstep$ is $\cup$-distributive.
    {}\label{so:cor:union}
\end{corollary}

\renewcommand{\closure}{\ensuremath\mathsf{cl}}

We further define a functional $\closure_N: ( \powersetof\aexprset \to
\powersetof\aexprset ) \to ( \powersetof\aexprset \to$ $\powersetof\aexprset
)$, which takes as an input a EEG function and produces another EEG function:
\begin{equation}
    f^\prime (\epsilon) \stackrel{\text{def}}=
        \bigcup_{i = 0}^N f^i(\epsilon),
    \text{~where~} f^\prime = \closure_N (f)
    \label{so:eq:transitive_generator}
\end{equation}
here, $f$ and $f^\prime$, respectively the input and output of $\closure_N$,
are both EEG functions.  In the rest of the section, we omit the brackets
surrounding the input of $\closure_N$ for simplicity, \eg~$\closure_N (f)$ can
be written as $\closure_N f$.  We further define the meaning of $f^i$ with the
following equations:
\begin{equation}
    \begin{aligned}
        f^0(\epsilon) &= \epsilon \quad \text{and~} \\
        f^i(\epsilon) &= \eqstep\left(
            f^{i - 1}\left(\epsilon\right)
        \right) \quad \text{for~} i \in \{ 0, 1, 2, \cdots \}
    \end{aligned}
\end{equation}

\begin{lemma}
    $\closure_N f(\epsilon) = \epsilon \cup f \left( \closure_{N-1} f(\epsilon)
    \right)$ for any $\cup$-distributive EEG function $f$.
    {}\label{so:lem:transitive}
\end{lemma}
\begin{proof}
    Following~\eqref{so:eq:transitive_generator}, $\closure_N f(\epsilon) =
    f^0(\epsilon) \cup f^1(\epsilon) \cup \cdots \cup f^N(\epsilon)$.  Because
    $f$ is $\cup$-distributive, we apply distributivity to the right-hand side
    to derive:
    \begin{equation}
        \closure_N f(\epsilon) = \epsilon \cup f\left(
            f^0(\epsilon) \cup f^1(\epsilon) \cup \cdots \cup f^{N-1}(\epsilon)
        \right)
    \end{equation}
    which equals to $\epsilon \cup f\left( \closure_{N-1} f(\epsilon) \right)$
    by definition.
\end{proof}

As an example use of the functional $\closure_N$, we may note that we can
substitute $f$ with $\eqstep$ in $\closure_N f (\epsilon)$ to generate a set
of equivalent expressions, by taking the union of $N$ steps of $\eqstep$ of
$\epsilon$.  By further allowing $N$ to approach $\infty$, we obtain the full
set of equivalent expressions of $\epsilon$ that can be discovered using our
inference system, \ie~the transitive closure of equivalent expressions related
by $f$, from an initial set of equivalent expressions $\epsilon$:
\begin{equation}
    \closure_\infty \eqstep (\epsilon) =
        \bigcup_{i = 0}^\infty \eqstep^i(\epsilon)
    \label{so:eq:transitive_closure}
\end{equation}
We may further omit the $\infty$ from $\closure_\infty$ to denote the
transitive closure, \eg~the above example in~\ref{so:eq:transitive_closure} can
be simplified to be $\closure \eqstep (\epsilon)$.

In practice, it is often infeasible to generate the full transitive closure of
a given expression, we therefore impose further constraints on how we discover
equivalent expressions.

First, instead of exploring the full transitive closure, that is, by allowing
the number of steps $N$ in~\eqref{so:eq:transitive_generator} to be infinite,
we may restrict $N$ to be a small finite value to allow a smaller set of
equivalent expressions to be computed.

Second, the complexity of equivalent expression finding is reduced by fixing
the structure of subexpressions at a certain depth $k$ in the original
expression.  The definition of depth is given as follows: first the root of
the parse tree of an expression is assigned depth $d = 1$; then we recursively
define the depth of a node as one more than the depth of its greatest-depth
parent.  If the depth of the node is greater than $k$, then we fix the
structure of its child nodes by disallowing any equivalence transformation
beyond this node. We let $\eqstep_k$ denote this ``depth limited'' equivalence
finding function, where $k$ is the depth limit used.  We can then use
$\closure_N \eqstep_k$ and $\closure \eqstep_k$ to denote the functions to
respectively compute the union of $N$ steps of $\eqstep_k$ and the transitive
closure. This approach is similar to Martel's depth limited equivalent
expression transform~\cite{martel07}, however Martel's method eventually allows
transformation of subexpressions beyond the depth limit, because rules of
equivalence would transform these to have a smaller depth.  This contributes
to a time complexity at least exponential in terms of the expression size. In
contrast, our technique has a time complexity that does not depend on the size
of the input expression, but grows with respect to the depth limit $k$. Note
that the full equivalence closure using the inference system we defined earlier
in~\eqref{so:eq:transitive_closure} is at least $O({2n - 1}!!)$ where $n$ is
the number of terms in an expression, as we discussed earlier.

Finally, we use the iterative algorithm in Figure~\ref{so:alg:closure} to
efficiently compute $\closure_N f$, where $f$ is a $\cup$-distributive EEG\@.
In each iteration, we keep track of the equivalent expressions that are newly
discovered in the current iteration, so that in the next iteration we apply
$\eqgenrel$ only to those expressions, to avoid redundant computation.  We then
continue to prove that this algorithm indeed computes $\closure_N f$.
\begin{figure}[ht]
    \centering
    \begin{algorithmic}
        \Function{Closure}{$f$, $N$, $\epsilon$}
            \State{$s_0 \gets \epsilon$}
            \State{$s^\prime_0 \gets \epsilon$}
            \For{$i \gets 1, \ldots, N$}
                \State{$s^\prime_i \gets
                    f \left(s^\prime_{i-1}\right) - s_{i-1}$}
                \State{$s_i \gets s_{i-1} \cup s^\prime_i$}
                \If{$s^\prime_i = \emptyset$}
                    \State{\Return{$s_i$}}
                \EndIf{}
            \EndFor{}
            \State{\Return{$s_i$}}
        \EndFunction{}
    \end{algorithmic}
    \caption{%
        Our algorithm to discover a set of equivalent expressions with
        a $\cup$-distributive EEG $f$ from an initial set of equivalent
        expressions $\epsilon$.
    }\label{so:alg:closure}
\end{figure}
\begin{theorem}
    In the algorithm in Figure~\ref{so:alg:closure}, at iteration $n$, the set
    of equivalent expressions $s_n$ computes exactly $\closure_n f(\epsilon)$,
    if $f$ is a $\cup$-distributive EEG\@.
\end{theorem}
\begin{proof}
    We start by assuming that at iteration $m > 0$, $s_m = \closure_m
    f(\epsilon)$, and we prove this equality still holds if substitute $m$ with
    $m + 1$.  From the algorithm, we can deduce:
    \begin{equation*}
    \begin{aligned}
        s_{m+1}
         &= s_m \cup s^\prime_{m+1} \\
         &= s_m \cup \left( f \left( s^\prime_m \right) - s_m \right) \\
         &= s_m \cup f \left( s^\prime_m \right) \\
         &= s_m \cup f \left(
                f \left( s^\prime_{m-1} \right) - s_{m-1}
            \right)
    \end{aligned}
    \end{equation*}
    We substitute $s_m$ using Lemma~\ref{so:lem:transitive} to get:
    \begin{equation*}
        s_{m+1}
          = \epsilon \cup f \left( s_{m-1} \right) \cup
            f \left(
                f \left( s^\prime_{m-1} \right) - s_{m-1}
            \right)
    \end{equation*}
    Using distributivity of $f$ over $\cup$ and the iteration $m$ of the
    algorithm, we can derive:
    \begin{equation*}
    \begin{aligned}
        s_{m+1}
         &= \epsilon \cup f \left(
                s_{m-1} \cup \left(
                    f \left( s^\prime_{m-1} \right) - s_{m-1}
                \right)
            \right) \\
         &= \epsilon \cup f \left( s_m \right)
    \end{aligned}
    \end{equation*}
    Finally, we make use of the assumption $s_m = \closure_m f(\epsilon)$,
    followed by Lemma~\ref{so:lem:transitive} to show:
    \begin{equation*}
        s_{m+1}
        = \epsilon \cup f \left(
            \closure_m f(\epsilon)
        \right)
        = \closure_{m+1} f(\epsilon)
    \end{equation*}
    It is trivial that $s_0 = \epsilon = \closure_0 f(\epsilon)$, by induction,
    $s_n = \closure_n f(\epsilon)$ thus holds for all $n \in \naturalset$.
\end{proof}


\subsection{Pareto Frontier}

Because we optimize expressions in two quality metrics, \ie~the accuracy of
computation and the estimate of FPGA resource utilization, there is a trade-off
between them. We desire the largest subset of all equivalent expressions
$E$ discovered such that in this subset, no expression dominates any other
expression, in terms of having both better area and better accuracy. This
subset is known as the Pareto frontier.  Figure~\ref{so:alg:pareto} shows
a simplified algorithm for calculating the Pareto frontier for a set of
equivalent expressions $\epsilon$.
\begin{figure}[ht]
    \centering
    \begin{algorithmic}
        \Function{Frontier}{$\epsilon$}
            \State{$\mathit{frontier} \gets \epsilon$}
            \For{$e \in \epsilon$}
                \For{$e^\prime \in \epsilon$}
                    \If{$\mathit{Area}(e^\prime) < \mathit{Area}(e)$ and
                        $\abserr(e^\prime) < \abserr(e)$}
                        \State{%
                            $\mathit{frontier} \gets
                                \mathit{frontier} / \{ e \}$}
                    \EndIf{}
                \EndFor{}
            \EndFor{}
            \State{\Return{$\mathit{frontier}$}}
        \EndFunction%
    \end{algorithmic}
    \caption{The Pareto frontier from a set of equivalent expressions.
    }\label{so:alg:pareto}
\end{figure} \\
Here, $\mathit{frontier} / \{ e \}$ is a set identical to $\mathit{frontier}$,
except that the element $e$ is removed.  We use the function $\abserr$ to
analyze the magnitudes of error bounds, which is defined as follows:
\begin{equation}
    \begin{aligned}
        \abserr(e) &= \max\left(
            \left| \mu^\sharp_{\min} \right|,
            \left| \mu^\sharp_{\max} \right|
        \right) \\
        & \quad \text{where~}
        \left(
            x^\sharp, \left[ \mu^\sharp_{\min}, \mu^\sharp_{\max} \right]
        \right) = \error(e)
    \end{aligned}
\end{equation}
Here we use the function $\error: \aexprset\to\errorset$ to represent the
analysis of round-off error analysis in an expression tree, described
in Section~\ref{bg:sub:accuracy} of Chapter~\ref{chp:background}, where
$\aexprset$ denotes the set of all arithmetic expressions.


\subsection{Equivalent Expressions Semantics}

Similar to the analysis of accuracy and resource usage, a set of equivalent
expressions can be computed with semantics. That is, we define structures,
\ie~sets of equivalent expressions, that can be manipulated with arithmetic
operators. In our equivalent expressions semantics, an element of
$\powersetof\aexprset$ is used to assign a set of expressions to each node
in an expression parse tree. To begin with, at each leaf of the tree, the
variable or constant is assigned a set containing itself, as for $x$, the set
$\epsilon_x$ of equivalent expressions is $\epsilon_x = \{x\}$. After this, we
propagate the equivalence expressions in the parse tree's direction of flow,
using~\eqref{so:eq:equivalence_semantics} defined below:
\begin{equation}
    \begin{aligned}
        \epsilon_x \circ \epsilon_y &= \frontier\left(
            \closure \eqstep_k \left(
                E_\circ \left( \epsilon_x, \epsilon_y \right)
            \right) \right) \\
        & \text{where~}
        E_\circ(\epsilon_x, \epsilon_y) = \{
            e_x \circ e_y \mid e_x \in \epsilon_x \wedge e_y \in \epsilon_y
        \}, \\
        & \text{and~} \circ\in\{+, -, \times\}
    \end{aligned}
    \label{so:eq:equivalence_semantics}
\end{equation}
The equation implies that in the propagation procedure, it recursively
constructs a set of equivalent subexpressions for the parent node from two
child expressions, and uses the depth limited equivalence function $\closure
\eqstep_k$ to work out a larger set of equivalent expressions. To reduce
computation effort, we select only those expressions on the Pareto frontier
for the propagation in the DFG\@. Although in worst case the complexity of
this process is exponential, the selection by Pareto optimality accelerates
the algorithm significantly. For example, for the subexpression $a + b$ of our
sample expression:
\begin{equation}
    \begin{aligned}
        \epsilon_a + \epsilon_b
            &= \frontier\left(
                    \closure \eqstep_k \left(
                        E_\circ \left( \epsilon_a, \epsilon_b \right)
                    \right)
                \right) \\
            &= \frontier\left(
                    \closure \eqstep_k \left(
                        E_\circ \left( \{a\}, \{b\} \right)
                    \right)
                \right) \\
            &= \frontier\left(
                    \{ a + b, b + a \}
                \right)
    \end{aligned}
\end{equation}
Alternatively, we could view the semantics in terms of DFGs representing
the algorithm for finding equivalent expressions. The parsing of an
expression directly determines the structure of its DFG\@. For instance,
the expression $(a + b) \times (a + b)$ generates the DFG illustrated in
Figure~\ref{so:fig:tree_expr_flow}. The circles labeled $3$ and $7$ in this
diagram are shorthands for the operation $E_+$ and $E_\times$ respectively,
where $E_+$ and $E_\times$ is defined in~\eqref{so:eq:equivalence_semantics}.
\begin{figure}[ht]
    \centering
    \includegraphics[scale=0.6]{tree_expr_flow}
    \caption{The DFG for finding equivalent expressions of
    $(a + b) \times (a + b)$.}\label{so:fig:tree_expr_flow}
\end{figure}

For our example in Figure~\ref{so:fig:tree_expr_flow},
similar to the construction of data flow equations in
Section~\ref{bg:sec:abstract_interpretation} of Chapter~\ref{chp:background},
we can produce a set of equations from the data flow of the DFG, which now
produces equivalent expressions:
\begin{equation}
    \begin{aligned}
        \enter{1} &= \enter{1} \cup \{a\} &
        \enter{2} &= \enter{2} \cup \{b\} \\
        \enter{3} &= E_+(\enter{1}, \enter{2}) &
        \enter{4} &= \enter{3} \cup \enter{5} \\
        \enter{5} &= \eqstep_k(\enter{4}) &
        \enter{6} &= \frontier(\enter{5}) \\
        \enter{7} &= E_\times(\enter{6}, \enter{6}) &
        \enter{8} &= \enter{7} \cup \enter{9} \\
        \enter{9} &= \eqstep_k(\enter{8}) &
        \enter{10} &= \frontier(\enter{9})
    \end{aligned}
    \label{so:eq:tree_expr_flow}
\end{equation}
Because of loops in the DFG, it is no longer trivial to find the solution.
In general, the analysis equations are solved iteratively. We can
regard the set of equations as a single transfer function $F$ as in
\eqref{so:eq:transfer_function}, where the function $F$ takes as input
the variables $A(1), \ldots, A(10)$ appearing in the right-hand sides of
\eqref{so:eq:tree_expr_flow} and outputs the values $A(1), \ldots, A(10)$
appearing in the left-hand sides. Our aim is then to find an input $\vec{x}$ to
$F$ such that $F(\vec{x}) = \vec{x}$, \ie~a fixpoint of $F$.
\begin{equation}
      F((\enter{1}, \ldots, \enter{10}))
    = (\enter{1} \cup \{a\}, \ldots, \frontier(\enter{9}))
    \label{so:eq:transfer_function}
\end{equation}
Initially we assign $\enter{i} = \varnothing$ for $i\in\{1,2,\ldots,10\}$,
and we denote $\vec\varnothing = (\varnothing, \ldots, \varnothing)$.
Then we compute iteratively $F(\vec\varnothing)$, $F^2(\vec\varnothing) =
F(F(\vec\varnothing))$, and so forth, until the fixpoint solution $\fix F$ is
reached for some iteration $n$, that is:
\begin{equation}
    \fix F = F^n(\vec\varnothing) =
    F(F^n(\vec\varnothing)) = F^{n + 1}(\vec\varnothing)
\end{equation}
The fixpoint solution $\fix F$ gives a set of equivalent expressions derived
using our method, which is found at $\enter{10}$. In essence, the depth limit
acts as a sliding window.  The semantics allow hierarchical transformation of
subexpressions using a depth-limited search and the propagation of a set of
subexpressions that are locally Pareto optimal to the parent expressions in a
bottom-up hierarchy.

The problem with the semantics above is that the time complexity of $\closure
\eqstep_k$ scales poorly, since the worst case number of subexpressions needed
to explore increases exponentially with $k$. Therefore an alternative method
is to optimize it by changing the structure of the DFG slightly, as shown in
Figure~\ref{so:fig:tree_expr_flow_greedy}. The difference is that at each
iteration, the Pareto frontier filters the results to decrease the number of
expressions to process for the next iteration.
\begin{figure}[ht]
    \centering
    \includegraphics[scale=0.6]{tree_expr_flow_greedy}
    \caption{The alternative DFG for $(a + b) \times (a + b)$.
    }\label{so:fig:tree_expr_flow_greedy}
\end{figure}

In the rest of this chapter, we use \frontiertrace{} to indicate our equivalent
expression finding semantics, and \greedytrace{} to represent the alternative
method.
