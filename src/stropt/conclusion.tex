\section{Summary}
\label{so:sec:conclusion}

We provide a formal approach to the optimization of arithmetic expressions
for both accuracy and resource usage in high-level synthesis. The method
proposed in this chapter and the associated tool, \soap, encompass three kind
of semantics that describe the accumulated roundoff errors, count operators in
expressions considering common subexpression elimination, and derive equivalent
expressions. For a set of input expressions, the proposed approach works out
the respective sets of equivalent expressions in a hierarchical bottom-up
fashion, with a windowing depth limit and Pareto selection to help reduce the
complexity of equivalent expression discovery. Using our tool, we improve
either the accuracy of our sample expressions or the resource utilization by
up to 60\%, over the originals under single precision. Our tool enables a
high-level synthesis tool to optimize the structure as well as the precision
of arithmetic expressions, then to automatically choose an implementation that
satisfies accuracy and resource usage constraints.

Because we underpin our approach in formal semantics, it provides the necessary
foundation which permits us to extend the method for general numerical program
transformation in high-level synthesis. Therefore in Chapter~\ref{chp:progopt},
we base ourselves on the methodologies developed in this chapter, and propose
a structural approach to program optimization by safely rewriting equivalent
structures in numerical programs.
