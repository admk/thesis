\section{Novel Semantics}
\label{sec:semantics}

\subsection{Accuracy Analysis}

We first introduce the concepts of the floating-point
representation~\cite{ieee754}. Any values $v$ representable in floating-point
with standard exponent offset can be expressed with the format given by the
following equation:
\begin{equation}
    v = s \times 2^{e + 2^{k - 1} - 1} \times 1.{m_1 m_2 m_3 \ldots m_p}
    \label{eq:floating_point}
\end{equation}
In~\eqref{eq:floating_point}, the bit $s$ is the sign bit, the $k$-bit unsigned
integer $e$ is known as the exponent bits, and the $p$-bits $m_1 m_2 m_3
\ldots m_p$ are the mantissa bits, here we use $1.{m_1 m_2 m_3 \ldots m_p}$ to
indicate a fixed-point number represented in unsigned binary format.

Because of the finite characteristic of IEEE 754 floating-point format, it
is not always possible to represent exact values with it. Computations in
floating-point arithmetic often induces roundoff errors. Therefore, following
Martel~\cite{martel07}, we bound with ranges the values of floating-point
calculations, as well as their roundoff errors. Our accuracy analysis
determines the bounds of all possible outputs and their associated range
of roundoff errors for expressions. For example, assume that $a \in [0.2,
0.3]$, $b \in [2.3, 2.4]$, it is possible to derive that in single precision
floating-point computation with rounding to the nearest, ${(a + b)}^2 \in
[6.24999857, 7.29000187]$ and the error caused by this computation is bounded
by $[-1.60634534\times10^{-6}, 1.60634534\times10^{-6}]$.

We employ abstract error semantics for the calculation of errors described
in~\cite{ioualalen, martel07}. First we define the domain $\errorset
= \floatintervalset\times\intervalset$, where $\intervalset$ and
$\floatintervalset$ respectively represent the set of real intervals, and
the set of floating-point intervals (intervals exactly representable in
floating-point arithmetic). The value $(x^\sharp, \mu^\sharp) \in \errorset$
represents a safe bound on floating-point values and the accumulated error
represented as a range of real values. Then addition and multiplication can be
defined for the semantics as in~\eqref{eq:error_semantics}:
\begin{equation}
    \begin{aligned}
        (x^\sharp_1, \mu^\sharp_1) + (x^\sharp_2, \mu^\sharp_2)
    &=  (\roundup(x^\sharp_1 + x^\sharp_2),
         \mu^\sharp_1 + \mu^\sharp_2 +
         \rounddown(x^\sharp_1 + x^\sharp_2)) \\
        (x^\sharp_1, \mu^\sharp_1) - (x^\sharp_2, \mu^\sharp_2)
    &=  (\roundup(x^\sharp_1 - x^\sharp_2),
         \mu^\sharp_1 - \mu^\sharp_2 +
         \rounddown(x^\sharp_1 - x^\sharp_2)) \\
        (x^\sharp_1, \mu^\sharp_1) \times (x^\sharp_2, \mu^\sharp_2)
    &=  (\roundup(x^\sharp_1 \times x^\sharp_2),
            x^\sharp_1 \times \mu^\sharp_2 + x^\sharp_2 \times \mu^\sharp_1 +
            \mu^\sharp_1 \times \mu^\sharp_2 +
            \rounddown(x^\sharp_1 \times x^\sharp_2)) \\
    &\qquad\qquad\qquad\qquad\qquad\qquad\text{~for~}
        (x^\sharp_1, \mu^\sharp_1) \in \errorset,
        (x^\sharp_2, \mu^\sharp_2) \in \errorset
    \end{aligned}
    \label{eq:error_semantics}
\end{equation}
The addition, subtraction and multiplication of intervals follow the standard
rules of interval arithmetic defined earlier in~\eqref{eq:interval_operations}.
In~\eqref{eq:error_semantics}, the function $\rounddown: \intervalset \to
\intervalset$ determines the range of roundoff error due to the floating-point
computation under one of the rounding modes $\circ \in \{ -\infty, \infty, 0,
\neg0, \sim \}$ which are round towards negative infinity, towards infinity,
towards zero, away from zero and towards nearest floating-point value
respectively. It is defined as:
\begin{equation}
    \begin{aligned}
        & \downarrow^\sharp_\circ([a, b]) = \left\{
            \begin{aligned}
                & \left[ -\frac{z}{2}, \frac{z}{2}\right]
                    & \quad \circ & \text{~is~}\sim \\
                & \left[ -z, z\right]
                    & \quad \circ & \in \{ -\infty, \infty, 0, \neg0 \}
            \end{aligned}
        \right. \\
        & \qquad\qquad\qquad\qquad \text{where~} z = \max(ulp(a), ulp(b))
    \end{aligned}
\end{equation}
Here $z$ denotes the maximum rounding error that can occur for values
within the range $[a, b]$, and the unit of the last place (ulp) function
$ulp(x)$~\cite{muller} characterizes the distance between two adjacent
floating-point values $f_1$ and $f_2$ satisfying $f_1 \leq x \leq
f_2$~\cite{goldberg}. In our analysis, the function $ulp$ is defined as:
\begin{equation}
    ulp(x) = 2^{e(x) + 2^{k - 1} - 1} \times 2^{-p}
\end{equation}
where $e(x)$ is the exponent of $x$, $k$ and $p$ are the parameters of the
floating-point format as defined in~\eqref{eq:floating_point}. The function
$\roundup: \intervalset \to \floatintervalset$ computes the floating-point
bound from a real bound, by rounding the infimum $a$ and supremum $b$ of the
input interval $[a, b]$.

Expressions can be evaluated for their accuracy by the method as follows.
Initially the expression is parsed into a data flow graph (DFG). By way of
illustration, the sample expression ${(a + b)}^2$ has the tree structure
in Fig.~\ref{fig:sample_tree}. Then the exact ranges of values of $a$ and
$b$ are converted into the abstract semantics using a cast operation as in
\eqref{eq:cast}:
\begin{equation}
    cast(x^\sharp) = (\roundup(x^\sharp), \rounddown(x^\sharp))
    \label{eq:cast}
\end{equation}
For example, for the variable $a \in [0.2, 0.3]$ under single precision
with rounding to nearest,
\begin{equation}
    cast([0.2, 0.3]) = ([0.200000003, 0.300000012], [-1/67108864, 1/67108864])
\end{equation}
After this, the propagation of bounds in the data flow graph is carried out as
described in Section~\ref{sec:abstract_interpretation}, where the difference is
the abstract error semantics defined in~\eqref{eq:error_semantics} is used in
lieu of the interval semantics. At the root of the tree (\ie~the exit of the
DFG) we find the value of the accuracy analysis result for the expression.

We use the function $\error: \exprset\to\errorset$ throughout this chapter to
represent the analysis of evaluation accuracy, where $\exprset$ denotes the set
of all expressions.

\subsection{Resource Usage Analysis}

Here we define similar formal semantics which calculate an approximation to the
FPGA resource usage of an expression, taking into account common subexpression
elimination. This is important as, for example, rewriting $a \times b + a
\times c$ as $a \times (b + c)$ in the larger expression $(a \times b + a
\times c) + {(a \times b)}^2$ causes the common subexpression $a \times b$ to
be no longer present in both terms. Our analysis must capture this.

The analysis proceeds by labelling subexpressions. Intuitively, the set of
labels $\labelset$, is used to assign unique labels to unique expressions,
so it is possible to easily identify and reuse them. For convenience, let
the function $\fresh: \exprset\to\labelset$ assign a distinct label to each
expression or variable, where $\exprset$ is the set of all expressions. Before
we introduce the labeling semantics, we define the environment $\lambda:
\labelset\to\exprset\cup\{\bot\}$, which is a function that maps labels to
expressions, and $\env{}$ denotes the set of such environments. A label $l$ in
the domain of $\lambda\in\env{}$ that maps to $\bot$ indicates that $l$ does
not map to an expression. An element $(l, \lambda)\in\labelset\times\env{}$
stands for the labeling scheme of an expression. Initially, we map all labels
to $\bot$, then in the mapping $\lambda$, each leaf of an expression is
assigned a unique label, and the unique label $l$ is used to identify the leaf.
That is for the leaf variable or constant $x$:
\begin{equation}
    (l, \lambda) = (\fresh(x), [\fresh(x)\mapsto{x}])
\end{equation}
This equation uses $[\fresh(x)\mapsto{x}]$ to indicate an environment that
maps the label $\fresh(x)$ to the expression $x$ and all other labels map
to $\bot$, in other words, if $l = \fresh(x)$ and $l^\prime \neq l$, then
$\lambda(l) = x$ and $\lambda(l^\prime) = \bot$. For example, for the DFG in
Fig.~\ref{fig:sample_tree}, we have for the variables $a$ and $b$:
\begin{equation}
    \begin{IEEEeqnarraybox}{rCl}
        (l_a, \lambda_a) &=& (\fresh(a), [\fresh(a)\mapsto{a}])
                   = (l_1, [l_1 \mapsto a]) \\
        (l_b, \lambda_b) &=& (l_2, [l_2 \mapsto b])
    \end{IEEEeqnarraybox}
    \label{eq:variable_env}
\end{equation}
Then the environments are propagated in the flow direction of the DFG, using
the following formulation of the labeling semantics:
\begin{equation}
    \begin{IEEEeqnarraybox}{rCl}
        (l_x, \lambda_x) \circ (l_y, \lambda_y)
            &=& (l, (\lambda_x\odot\lambda_y)
                      [l\mapsto{l_x \circ l_y}]) \\
            \text{where~} l &=& \fresh(l_x \circ l_y),
                          \circ\in\{+, -, \times\}
    \end{IEEEeqnarraybox}
    \label{eq:labeling_semantics}
\end{equation}
Specifically, $\lambda=\lambda_x\odot\lambda_y$ signifies that $\lambda_y$
is used to update the mapping in $\lambda_x$, if the mapping does not
exist in $\lambda_x$, and result in a new environment $\lambda$; and
$\lambda[l\mapsto{x}]$ is a shorthand for $\lambda\odot[l\mapsto{x}]$.
As an example, with the expression in Fig.~\ref{fig:sample_tree}, using
\eqref{eq:variable_env}, recall to mind that $l_1 = l_a$, $l_2 = l_b$, we
derive for the subexpression $a + b$:
\begin{equation}
    \begin{IEEEeqnarraybox}{rCl}
        (l_{a + b}, \lambda_{a + b})
            &=& (l_a, \lambda_a) + (l_b, \lambda_b) \\
            &=& (l_3, (\lambda_a \odot \lambda_b) [l_3\mapsto{l_a + l_b}]) \\
            & & \text{where~} l_3 = \fresh(l_a + l_b) \\
            &=& (l_3, [l_1\mapsto{a}]\odot
                      [l_2\mapsto{b}]\odot
                      [l_3\mapsto{l_1 + l_2}]) \\
            &=& (l_3, [l_1\mapsto{a}, l_2\mapsto{b}, l_3\mapsto{l_1 + l_2}])
    \end{IEEEeqnarraybox}
\end{equation}
Finally, for the full expression $(a + b) \times (a + b)$:
\begin{equation}
    \begin{IEEEeqnarraybox}{rCl}
        (l, \lambda)
            &=& (l_{a + b}, \lambda_{a + b}) \times
                (l_{a + b}, \lambda_{a + b}) \\
            &=& (l_4, [l_1\mapsto{a}, l_2\mapsto{b},
                      l_3\mapsto{l_1 + l_2}, l_4\mapsto{l_3 \times l_3}])
    \end{IEEEeqnarraybox}
\end{equation}
From the above derivation, it is clear that the semantics capture the reuse
of subexpressions. The estimation of area is performed by counting, for an
expression, the numbers of additions, subtractions and multiplications in
the final labeling environment, then calculating the number of LUTs used to
synthesize the expression. If the number of operators is $n_\circ$ where
$\circ\in\{+,-,\times\}$, then the number of LUTs in total for the expressions
is estimated as $\sum_{\circ\in\{+,-,\times\}} A_\circ n_\circ$, where the
value $A_\circ$ denotes the number of LUTs per $\circ$ operator, which is
dependent on the type of the operator and the floating-point format used to
generate the operator.

In the following sections, we use the function $\area: \exprset\to\naturalset$
to denote our resource usage analysis.

\subsection{Equivalent Expressions Analysis}
\label{sub:equivalent_expressions_analysis}

In earlier sections, we introduce semantics that define additions and
multiplications on intervals, then gradually transition to error semantics that
compute bounds of values and errors, as well as labelling environments that
allows common subexpression elimination, by defining arithmetic operations on
these structures. In this section, we now take the leap from not only analyzing
an expression for its quality, to defining arithmetic operations on sets of
equivalent expressions, and use these rules to discover equivalent expressions.
Before this, it is necessary to formally define equivalent expressions and
functions to discover them.

\subsubsection{Discovering Equivalent Expressions}

From an expression, a set of equivalent expressions can be discovered by
the following inference system of equivalence relations $\eqrel \subset
\exprset\times\exprset$. Let's define $e_1, e_2, e_3 \in \exprset$, $v_1, v_2,
v_3 \in \realset$, and $\circ \in \{+, \times\}$. First, the arithmetic rules
are:
\begin{equation*}
    \begin{IEEEeqnarraybox}{rCl}
        \text{Associativity}(\circ)
            &:& (e_1 \circ e_2) \circ e_3 \eqrel e_1 \circ (e_2 \circ e_3) \\
        \text{Commutativity}(\circ)
            &:& e_1 \circ e_2 \eqrel e_2 \circ e_1 \\
        \text{Distributivity}
            &:& e_1 \times (e_2 + e_3) \eqrel e_1 \times e_2 + e_1 \times e_3 \\
        \text{Distributivity}^\prime
            &:& e_1 \times e_2 + e_1 \times e_3 \eqrel e_1 \times (e_2 + e_3)
    \end{IEEEeqnarraybox}
\end{equation*}
Secondly, the reduction rules are:
\begin{equation*}
    \begin{IEEEeqnarraybox}{rClCrCl}
        \text{Identity($\times$)}
            &:& e_1 \times 1 \eqrel e_1 & \quad &
        \text{ZeroProp}
            &:& e_1 \times 0 \eqrel 0 \\
        \text{Identity($+$)}
            &:& e_1 + 0 \eqrel e_1 & &
        \text{ConstProp}(\circ)
            &:& \inference{v_3 = v_1 \circ v_2}{v_1 \circ v_2 \eqrel v_3}
    \end{IEEEeqnarraybox}
\end{equation*}
The ConstProp rule states that if an expression is a summation/multiplication
of two values, then it can be simply evaluated to produce the result. Finally,
the following one allows structural induction on expression trees, \ie~it is
possible to derive that $a + (b + c) \eqrel a + (c + b)$ from $b + c \eqrel c +
b$:
\begin{equation}
    \begin{IEEEeqnarraybox}{rCl}
        \text{Tree}(\circ)
            &:& \inference{e_1 \eqrel e_2}{e_3 \circ e_1 \eqrel e_3 \circ e_2}
    \end{IEEEeqnarraybox}
    \label{eq:equivalence_inference}
\end{equation}
It is possible to transform the structure of expressions using these
above-mentioned inference rules. We define the function $\eqstep:
\powersetof\exprset\to\powersetof\exprset$, where $\powersetof\exprset$
denotes the power set of $\exprset$, which generates a (possibly larger)
set of equivalent expressions from an initial set of equivalent expressions
by one step of (\ref{eq:equivalence_inference}), that is:
\begin{equation}
    \eqstep(\epsilon) = \left\{
        e^\prime\in\exprset \mid
        e \eqrel e^\prime \wedge e\in\epsilon\right\}
\end{equation}
where $\epsilon$ is a set of equivalent expressions. From this, we may note
that the set of equivalent expressions generated by taking the union of any
number of steps of~\eqref{eq:equivalence_inference} of $\epsilon$ is the
transitive closure $\eqstep^\star(\epsilon)$, as given by the following
formula, where $\eqstep^i(\epsilon) = \eqstep(\eqstep^{i - 1}(\epsilon))$ and
$\eqstep^0(\epsilon) = \epsilon$:
\begin{equation}
    \eqstep^\star(\epsilon) = \bigcup_{i = 0}^\infty \eqstep^i(\epsilon)
    \label{eq:transitive_closure}
\end{equation}
We say that $e_1$ is equivalent to $e_2$ if and only if $e_1 \in
\eqstep^\star(\{e\})$ and $e_2 \in \eqstep^\star(\{e\})$ for some expression
$e$. In general because of combinatorial explosion, it is not possible to
derive the full set of equivalent expressions by rules of equivalence, which
motivates us to develop scalable methods that executes fast enough even with
large expressions.

\subsubsection{Scalable Methods}

The complexity of equivalent expression finding is reduced by fixing the
structure of subexpressions at a certain depth $k$ in the original expression.
The definition of depth is given as follows: first the root of the parse
tree of an expression is assigned depth $d = 1$; then we recursively define
the depth of a node as one more than the depth of its greatest-depth parent.
If the depth of the node is greater than $k$, then we fix the structure of
its child nodes by disallowing any equivalence transformation beyond this
node. We let $\eqstep_k$ denote this ``depth limited'' equivalence finding
function, where $k$ is the depth limit used, and $\eqstep^\star_k$ denotes
its transitive closure. This approach is similar to Martel's depth limited
equivalent expression transform~\cite{martel07}, however Martel's method
eventually allows transformation of subexpressions beyond the depth limit,
because rules of equivalence would transform these to have a smaller depth.
This contributes to a time complexity at least exponential in terms of the
expression size. In contrast, our technique has a time complexity that does not
depend on the size of the input expression, but grows with respect to the depth
limit $k$. Note that the full equivalence closure using the inference system we
defined earlier in \eqref{eq:transitive_closure} is at least $O({2n - 1}!!)$
where $n$ is the number of terms in an expression, as we discussed earlier.

\subsubsection{Pareto Frontier}

Because we optimize expressions in two quality metrics, \ie~the accuracy of
computation and the estimate of FPGA resource utilization. We desire the
largest subset of all equivalent expressions $E$ discovered such that in this
subset, no expression dominates any other expression, in terms of having both
better area and better accuracy. This subset is known as the Pareto frontier.
Fig.~\ref{alg:pareto} shows a na\"ive algorithm for calculating the Pareto
frontier for a set of equivalent expressions $\epsilon$.
\begin{figure}[ht]
    \centering
    \begin{algorithmic}
        \Function{Frontier}{$\epsilon$}
            \State{$e_1, e_2, \ldots, e_n
                \gets \mathtt{sort\_by\_accuracy}(\epsilon)$}
            \State{$e \gets e_1$; $\mathit{frontier} \gets \{e\}$}
            \For{$i \in \{1, 2, \ldots, n\}$}
                \If{$\area(e_i) < \area(e)$}
                    \State{$\mathit{frontier}
                        \gets \mathit{frontier} \cup \{e_i\}$; $e \gets e_i$}
                \EndIf%
            \EndFor%
            \State{\Return{$\mathit{frontier}$}}
        \EndFunction%
    \end{algorithmic}
    \caption{The Pareto frontier from a set of equivalent expressions.
    }\label{alg:pareto}
\end{figure} \\
The function $\mathtt{sort\_by\_accuracy}$ sorts a set of expressions by their
analyzed bounds of errors in increasing order, where the magnitudes of error
bounds are used to perform the ordering~\cite{martel07}:
\begin{equation}
    \begin{IEEEeqnarraybox}{rCl}
        \abserr(e) &=& \max(\abs(\mu^\sharp_{\min}),
                            \abs(\mu^\sharp_{\max})) \\
                   & & \text{where~}
                       (x^\sharp, [\mu^\sharp_{\min}, \mu^\sharp_{\max}])
                     = \error(e)
    \end{IEEEeqnarraybox}
\end{equation}

\subsubsection{Equivalent Expressions Semantics}

Similar to the analysis of accuracy and resource usage, a set of equivalent
expressions can be computed with semantics. That is, we define structures,
\ie~sets of equivalent expressions, that can be manipulated with arithmetic
operators. In our equivalent expressions semantics, an element of
$\powersetof\exprset$ is used to assign a set of expressions to each node
in an expression parse tree. To begin with, at each leaf of the tree, the
variable or constant is assigned a set containing itself, as for $x$, the set
$\epsilon_x$ of equivalent expressions is $\epsilon_x = \{x\}$. After this, we
propagate the equivalence expressions in the parse tree's direction of flow,
using \eqref{eq:equivalence_semantics}, where $E_\circ(\epsilon_x, \epsilon_y)
= \{ e_x \circ e_y \mid e_x \in \epsilon_x \wedge e_y \in \epsilon_y \}$, and
$\circ\in\{+, -, \times\}$:
\begin{equation}
      \epsilon_x \circ \epsilon_y
    = \frontier\left(
        \eqstep^\star_k \left(
             E_\circ(\epsilon_x, \epsilon_y)
        \right)
      \right)
    \label{eq:equivalence_semantics}
\end{equation}
The equation implies that in the propagation procedure, it recursively
constructs a set of equivalent subexpressions for the parent node from
two child expressions, and uses the depth limited equivalence function
$\eqstep^\star_k$ to work out a larger set of equivalent expressions. To reduce
computation effort, we select only those expressions on the Pareto frontier
for the propagation in the DFG\@. Although in worst case the complexity of
this process is exponential, the selection by Pareto optimality accelerates
the algorithm significantly. For example, for the subexpression $a + b$ of our
sample expression:
\begin{equation}
    \begin{IEEEeqnarraybox}{rCl}
        \epsilon_a + \epsilon_b
            &=& \frontier\left(
                    \eqstep^\star_k \left(
                        E_\circ(\epsilon_a, \epsilon_b)
                    \right)
                \right) \\
            &=& \frontier\left(
                    \eqstep^\star_k \left(
                        E_\circ(\{a\}, \{b\})
                    \right)
                \right) \\
            &=& \frontier\left(
                    \{ a + b, b + a \}
                \right)
    \end{IEEEeqnarraybox}
\end{equation}
Alternatively, we could view the semantics in terms of DFGs representing
the algorithm for finding equivalent expressions. The parsing of an
expression directly determines the structure of its DFG\@. For instance,
the expression $(a + b) \times (a + b)$ generates the DFG illustrated in
Fig.~\ref{fig:tree_expr_flow}. The circles labeled $3$ and $7$ in this diagram
are shorthands for the operation $E_+$ and $E_\times$ respectively, where $E_+$
and $E_\times$ is defined in \eqref{eq:equivalence_semantics}.
\begin{figure}[ht]
    \centering
    \includegraphics[scale=0.4, trim=0 20pt 0 20pt]{tree_expr_flow}
    \caption{The DFG for finding equivalent expressions of
    $(a + b) \times (a + b)$.}\label{fig:tree_expr_flow}
\end{figure}

For our example in Fig.~\ref{fig:tree_expr_flow}, similar to the construction
of data flow equations in Section~\ref{sec:abstract_interpretation}, we can
produce a set of equations from the data flow of the DFG, which now produces
equivalent expressions:
\begin{equation}
    \begin{IEEEeqnarraybox}{rClCrCl}
        \enter{1} &=& \enter{1} \cup \{a\} &~&
        \enter{2} &=& \enter{2} \cup \{b\} \\
        \enter{3} &=& E_+(\enter{1}, \enter{2}) &~&
        \enter{4} &=& \enter{3} \cup \enter{5} \\
        \enter{5} &=& \eqstep_k(\enter{4}) &~&
        \enter{6} &=& \frontier(\enter{5}) \\
        \enter{7} &=& E_\times(\enter{6}, \enter{6}) &~&
        \enter{8} &=& \enter{7} \cup \enter{9} \\
        \enter{9} &=& \eqstep_k(\enter{8}) &~&
        \enter{10} &=& \frontier(\enter{9})
    \end{IEEEeqnarraybox}
    \label{eq:tree_expr_flow}
\end{equation}
Because of loops in the DFG, it is no longer trivial to find the solution.
In general, the analysis equations are solved iteratively. We can
regard the set of equations as a single transfer function $F$ as in
\eqref{eq:transfer_function}, where the function $F$ takes as input the
variables $A(1), \ldots, A(10)$ appearing in the right-hand sides of
\eqref{eq:tree_expr_flow} and outputs the values $A(1), \ldots, A(10)$
appearing in the left-hand sides. Our aim is then to find an input $\vec{x}$ to
$F$ such that $F(\vec{x}) = \vec{x}$, \ie~a fixpoint of $F$.
\begin{equation}
      F((\enter{1}, \ldots, \enter{10}))
    = (\enter{1} \cup \{a\}, \ldots, \frontier(\enter{9}))
    \label{eq:transfer_function}
\end{equation}
Initially we assign $\enter{i} = \varnothing$ for $i\in\{1,2,\ldots,10\}$,
and we denote $\vec\varnothing = (\varnothing, \cdots, \varnothing)$.
Then we compute iteratively $F(\vec\varnothing)$, $F^2(\vec\varnothing)
= F(F(\vec\varnothing))$, and so forth, until the fixpoint
solution $F^n(\vec\varnothing)$ is reached for some iteration
$n$, \ie~$F^n(\vec\varnothing) = F(F^n(\vec\varnothing)) = F^{n +
1}(\vec\varnothing)$. The fixpoint solution $F^n(\vec\varnothing)$ gives
a set of equivalent expressions derived using our method, which is found
at $\enter{10}$. In essence, the depth limit acts as a sliding window.
The semantics allow hierarchical transformation of subexpressions using a
depth-limited search and the propagation of a set of subexpressions that are
locally Pareto optimal to the parent expressions in a bottom-up hierarchy.

The problem with the semantics above is that the time complexity of
$\eqstep^\star_k$ scales poorly, since the worst case number of subexpressions
needed to explore increases exponentially with $k$. Therefore an alternative
method is to optimize it by changing the structure of the DFG slightly, as
shown in Fig.~\ref{fig:tree_expr_flow_greedy}. The difference is that at each
iteration, the Pareto frontier filters the results to decrease the number of
expressions to process for the next iteration.
\begin{figure}[ht]
    \centering
    \includegraphics[scale=0.4, trim=0 20pt 0 20pt]{tree_expr_flow_greedy}
    \caption{The alternative DFG for $(a + b) \times (a + b)$.
    }\label{fig:tree_expr_flow_greedy}
\end{figure}

In the rest of this chapter, we use \frontiertrace{} to indicate our equivalent
expression finding semantics, and \greedytrace{} to represent the alternative
method.
