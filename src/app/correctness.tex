\chapter{Sound Acceleration of Equivalent Expression Discovery}
\label{app:sound_acceleration}

The functions used in Section~\ref{so:sub:scalable} can sometimes be slow
to compute using a na\"ive implementation.  By using \acrlong{ai} and the
properties of certain \acrfull{eeg} functions, we can further accelerate the
computation.  We start by defining a property of the \gls{eeg} functions
used in Section~\ref{so:sub:scalable} known as \emph{$\cup$-distributive},
then propose a new algorithm to accelerate the computation of $\closure_N f
(\epsilon)$, where $f$ is a $\cup$-distributive \gls{eeg}\@.
\begin{definition}
    We say an \gls{eeg} function $f$ is \emph{$\cup$-distributive} if and only
    if the function satisfies $f(\epsilon_a \cup \epsilon_b) = f(\epsilon_a)
    \cup f(\epsilon_b)$.
\end{definition}
\begin{corollary}
    By the definition of $\eqstep$ in~\eqref{so:eq:eqstep}, it is clear that
    $\eqstep$ is $\cup$-distributive.
    {}\label{so:cor:union}
\end{corollary}

We then continue to prove that the algorithm $\textsc{Closure}(f, N, \epsilon)$
in Figure~\ref{so:alg:closure} indeed computes $\closure_N f(\epsilon)$.
Firstly, we prove the following lemma:
\begin{lemma}
    $\closure_N f(\epsilon) = \epsilon \cup f \left( \closure_{N-1} f(\epsilon)
    \right)$ for any $\cup$-distributive \gls{eeg} function $f$.
    {}\label{so:lem:transitive}
\end{lemma}
\begin{proof}
    Following~\eqref{so:eq:transitive_generator}, $\closure_N f(\epsilon) =
    f^0(\epsilon) \cup f^1(\epsilon) \cup \cdots \cup f^N(\epsilon)$.  Because
    $f$ is $\cup$-distributive, we apply distributivity to the right-hand side
    to derive:
    \begin{equation}
        \closure_N f(\epsilon) = \epsilon \cup f\left(
            f^0(\epsilon) \cup f^1(\epsilon) \cup \cdots \cup f^{N-1}(\epsilon)
        \right),
    \end{equation}
    which equals to $\epsilon \cup f\left( \closure_{N-1} f(\epsilon) \right)$
    by definition.
\end{proof}
Then this allows us to deduce that the algorithm indeed computes $\closure_N
f(\epsilon)$:
\begin{theorem}
    In the algorithm in Figure~\ref{so:alg:closure}, at iteration $n$, the set
    of equivalent expressions $s_n$ computes exactly $\closure_n f(\epsilon)$,
    if $f$ is a $\cup$-distributive \gls{eeg}\@.
    \label{so:thm:closure}
\end{theorem}
\begin{proof}
    We start by assuming that at iteration $m > 0$, $s_m = \closure_m
    f(\epsilon)$, and we prove this equality still holds if substitute $m$ with
    $m + 1$.  From the algorithm, we can deduce:
    \begin{equation*}
        s_{m+1}
        = s_m \cup s^\prime_{m+1}
        = s_m \cup \left( f \left( s^\prime_m \right) - s_m \right)
        = s_m \cup f \left( s^\prime_m \right)
        = s_m \cup f \left(
            f \left( s^\prime_{m-1} \right) - s_{m-1}
        \right).
    \end{equation*}
    We substitute $s_m$ using Lemma~\ref{so:lem:transitive} to get:
    \begin{equation*}
        s_{m+1}
        = \epsilon \cup f \left( s_{m-1} \right) \cup
        f \left(
            f \left( s^\prime_{m-1} \right) - s_{m-1}
        \right).
    \end{equation*}
    Using distributivity of $f$ over $\cup$ and the iteration $m$ of the
    algorithm, we can derive:
    \begin{equation*}
        s_{m+1}
        = \epsilon \cup f \left(
            s_{m-1} \cup \left(
                f \left( s^\prime_{m-1} \right) - s_{m-1}
            \right)
        \right)
        = \epsilon \cup f \left( s_m \right).
    \end{equation*}
    Finally, we make use of the assumption $s_m = \closure_m f(\epsilon)$,
    followed by Lemma~\ref{so:lem:transitive} to show:
    \begin{equation*}
        s_{m+1}
        = \epsilon \cup f \left(
            \closure_m f(\epsilon)
        \right)
        = \closure_{m+1} f(\epsilon).
    \end{equation*}
    It is trivial that $s_0 = \epsilon = \closure_0 f(\epsilon)$, by induction,
    $s_n = \closure_n f(\epsilon)$ thus holds for all $n \in \naturalset$.
\end{proof}

Unfortunately, in the alternative method \greedytrace{}, we cannot make use of
the efficient algorithm in Figure~\ref{so:alg:closure} to compute $\closure_N
\left( \frontier \circ \eqstep_k \right)$ directly.  The reason is that in
general, $\frontier \circ \eqstep_k$ is not $\cup$-distributive.  Imagine two
equivalent expressions $e_1$ and $e_2$, and $e_1$ strictly dominates $e_2$,
\ie~$e_1$ is better than $e_2$ in terms of accuracy and resources, then we can
observe that:
\begin{equation}
    \frontier \left( \left\{ e_1 \right\} \cup \left\{ e_2 \right\} \right)
    = \frontier \left( \left\{ e_1, e_2 \right\} \right)
    = \left\{ e_1 \right\},
\end{equation}
which is not equal to:
\begin{equation}
    \frontier \left( \left\{ e_1 \right\} \right) \cup
    \frontier \left( \left\{ e_2 \right\} \right)
    = \left\{ e_1 \right\} \cup \left\{ e_2 \right\}.
\end{equation}

To resolve this, following \acrlong{ai} discussed in
Section~\ref{bg:sec:abstract_interpretation} of Chapter~\ref{chp:background},
we can introduce a new abstract domain to the power set of equivalent
expressions.  This new domain reduces the computation effort of equivalent
expression discovery, because it is a simpler domain with fewer elements than
the power set.

The abstract domain, represented by $\lattice{\absexprset}{\sqsubseteq}$, can
be inductively defined using the following Gal\"ois connection:
\begin{equation}
    \lattice{\eqexprpowerset}{\subseteq}
    \galois{\alpha}{\gamma}
    \lattice{\absexprset}{\sqsubseteq},
\end{equation}
where its abstraction and concretization functions are defined as follows:
\begin{equation}
    \begin{aligned}
        \alpha(\epsilon) &= \frontier(\epsilon), \quad
        \gamma(\epsilon^\sharp) &= \epsilon^\sharp.
    \end{aligned}
\end{equation}
Specifically, $\alpha$ is identical to $\frontier$, and $\gamma$ is simply the
identity function, \ie~it returns the input as its output.

The partial ordering on them ($\sqsubseteq$) and the join operator ($\sqcup$)
can then be defined inductively as follows using the Gal\"ois connection:
\begin{equation}
    \begin{aligned}
        \epsilon^\sharp_1 \sqsubseteq \epsilon^\sharp_2 &\defeq
            \left( \epsilon^\sharp_1 \sqcup \epsilon^\sharp_2 \right)
            = \epsilon^\sharp_2, \\
        \epsilon^\sharp_1 \sqcup \epsilon^\sharp_2 &\defeq
            \alpha\left(
                \gamma\left( \epsilon^\sharp_1 \right) \cup
                \gamma\left( \epsilon^\sharp_2 \right)
            \right).
    \end{aligned}
\end{equation}
Additionally, an abstract variant of an arbitrary \gls{eeg} $f$ can also be
inductively defined:
\begin{equation}
    f^\sharp (\epsilon^\sharp)
    = \alpha \circ f \circ \gamma \left(\epsilon^\sharp\right)
    = \frontier\left(f\left(\epsilon^\sharp\right)\right),
\end{equation}

Finally, the abstract variant of $\closure_N f$ can be proposed with a simple
modification to the algorithm in Figure~\ref{so:alg:closure}, by replacing
$\cup$ with $\sqcup$ in $s_i \gets s_{i-1} \cup s^\prime_i$, we now have
an abstract closure function:
\begin{equation}
    \closure^\sharp_N f^\sharp (\epsilon)
    \sqsubseteq \bigsqcup_{n \in \naturalset} {f^\sharp}^n (\epsilon),
\end{equation}
that operates with an abstract \gls{eeg} $f^\sharp$.  The correctness of this
formulation can be proved by adapting Theorem~\ref{so:thm:closure} to the
modified algorithm with the property:
\begin{equation}
    f^\sharp\left(\epsilon^\sharp_1 \sqcup \epsilon^\sharp_2\right)
    \sqsubseteq f^\sharp\left(\epsilon^\sharp_1\right)
    \sqcup f^\sharp\left(\epsilon^\sharp_2\right).
\end{equation}
The alternative method, \greedytrace{} which computes $\closure^\sharp_N
\eqstep_k^\sharp$, where $\eqstep_k^\sharp = \frontier \circ \eqstep_k$ can
finally be accelerated using the modified algorithm.
