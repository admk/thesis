\chapter{Formal Definitions of Equivalent MIR Discovery}
\label{app:formal}

In Section~\ref{po:sec:equivalence_analysis} of Chapter~\ref{chp:progopt}
informally explained how equivalent semantic expressions and
\glspl{mir} can be discovered.  In this appendix we formally define the
equivalent discovery procedure for all newly introduced operators in
Section~\ref{po:sec:program_to_mir}, \ie~the ternary conditional, composition,
and fixpoint operators, and also extend this definition to \glspl{mir} in a
similar fashion.

Formally, we extend the optimization function $\optfunc{\cdot}: M \to
\errordom \to M$, where $M = \sexprset \cup \mirset$, proposed in
Section~\ref{so:sec:equivalent} of Chapter~\ref{chp:stropt} to these above
structures.  For ternary conditional operators, we have:
\begin{equation}
    \optfunc{\ternarymir{$\qop$}{$b$}{$e_1$}{$e_2$}}\sigma^\sharp
    = f_{\sigma^\sharp} \left( \left\{
        \ternarymir{$\qop$}{$b^\prime$}{$e^\prime_1$}{$e^\prime_2$}
        \left|
        \tikz[baseline=(current bounding box.center), text height=0]{%
            \node at (0, 0) {%
                $b^\prime \in \optfunc{b}\sigma^\sharp,$
            };
            \node at (1.8mm, -5mm) {%
                $e^\prime_1 \in \optfunc{e_1}\sigma^\sharp|_b^\prime,$
            };
            \node at (1.5mm, -10mm) {%
                $e^\prime_2 \in \optfunc{e_2}\sigma^\sharp|_b^\prime$
            };
        }
        \right.
    \right\} \right).
\end{equation}

Similarly, the function can also be defined for the composition operator:
\begin{equation}
    \optfunc{\binarymir{$\expand$}{$e$}{$\mu$}}\sigma^\sharp
    = \frontier\left( \left\{
        \left.
            \binarymir{$\expand$}{$e^\prime$}{$\mu^\prime$}
        \right|
        \tikz[baseline=(current bounding box.center), text height=1mm]{%
            \node at (0, 0) {%
                $e^\prime \in \optfunc{e} \left(
                    \mirerrorfunc{\mu^\prime}\sigma^\sharp
                \right),$

            };
            \node at (0mm, -6mm) {%
                $\mu^\prime \in \optfunc{\mu}\sigma^\sharp$
            };
        }
    \right\}, \sigma^\sharp \right),
\end{equation}
where $\frontier\left(\epsilon, \sigma^\sharp\right)$ computes the
Pareto-optimal set of equivalent semantic expressions from the initial set
$\epsilon$, using the program state $\sigma^\sharp$ to evaluate the quality
metrics (\ie~round-off errors and resource utilization).

In addition, a formal mathematical definition of the process of discovering
equivalent fixpoint expressions is:
\begin{equation}
    \optfunc{
        \fixpointmir{$b$}{$\mu$}{\varx}
    }\sigma^\sharp
    = \frontier \left( \left\{
        \fixpointmir{$b^\prime$}{$\mu^\prime$}{\varx}
        \left|
        \tikz[baseline=(current bounding box.center), text height=5mm]{%
            \node at (0, 0) {%
                $\loopinvar = \mathsf{E}_\mathsf{s}^\mathrm{LI} \left[
                    \fixpointmir{$b$}{$\mu$}{\varx}
                \right]\sigma^\sharp,$
            };
            \node at (0, -9mm) {%
                $b^\prime \in \optfunc{b}\loopinvar,$
            };
            \node at (0, -19mm) {%
                $\displaystyle \mu^\prime \in \bigcup_{k=0}^{K}
                \optfunc{p^k_\mu (\mu)}\loopinvar$
            };
        }
        \right.
    \right\}, \sigma^\sharp \right),
\end{equation}
where $K$ is the partial unroll factor limit.  We set $K = 3$ in the
experimental results discussed in Section~\ref{po:sec:results}.

Finally, \glspl{mir} can also be optimized in a similar fashion, by
recursively discovering equivalent expressions within it:
\begin{equation}
    \optfunc{\mu}\sigma^\sharp = \frontier\left( \left\{
        {\left[ \varx \mapsto e_\varx \right]}_{\varx \in \varfunc{\mu}}
        \left|
        \tikz[baseline=(current bounding box.center), text height=1mm]{%
            \node at (0, 0) {%
                $e_\vary \in \optfunc{\mu(\vary)}\sigma^\sharp,$

            };
            \node at (0mm, -6mm) {%
                $\vary \in \varfunc{\mu}$
            };
        }
        \right.
    \right\}, \sigma^\sharp \right).
\end{equation}
