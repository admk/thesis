\subsection{Accuracy Analysis}
\label{sub:accuracy_analysis}

In a typical program execution, values of variables, typically integers
$\integerset$ and floating-point values $\floatset$, are modified according
to the effect of the program statements, and they are propagated through
arithmetic operators from the beginning to the end of the program.  By
comparison, the static analysis approach~\cite{soap,martel07} analyzes
round-off errors in floating-point numerical programs by following the same
procedure, but instead of propagating actual values of numerical types,
value-error ranges are used.  A value-error range $(v^\sharp, e^\sharp) \in
\errorset$ consists of a floating-point interval that bounds the value,
$v^\sharp \in \intervalset_\floatset$, and its associated round-off errors
caused by arithmetic operations, which is represented as an interval of real
values, $e^\sharp \in \intervalset_\realset$.  Here $\intervalset_\floatset$
and $\intervalset_\realset$ are the sets of floating-point and real value
intervals respectively, and we define $\errorset = \intervalset_\floatset
\times \intervalset_\realset$ to be the set of value-error ranges.

Generally, arithmetic operators work on numerical data types, such as
\inttype{} and \floattype, and they define how these numerical values can be
used to compute another numerical value.  In contrast, by \emph{overriding}
the meaning of arithmetic operators, the accuracy analysis extends each of
these operators to allow value-error ranges as inputs, and compute as an
output a value-error range.  For instance, we assume an abstract input state
$\sigma^\sharp_0 \in \errordom$, where:
\begin{equation}
        \sigma^\sharp_0 = \left[
            a \mapsto ([0, 1], e^\sharp_0),
            b \mapsto ([1, 2], e^\sharp_0) \right]
\end{equation}
which means that initially
$a$ and $b$ are floating-point values bounded by $[0, 1]$ and $[1, 2]$
respectively, and the interval $e^\sharp_0 = [0, 0]$ denotes the absence of
round-off errors, then in single-precision, the floating-point computation $a
+ b$ produces the result $\left( v^\sharp, e^\sharp \right)$:
\begin{equation}
    \left( v^\sharp, e^\sharp \right) = \left(
        [1, 3], [-1.19209304 \times 10^{-7}, 1.19209304 \times 10^{-7}]
    \right)
\end{equation}
This means that the result of this computation is in the range of $v^\sharp =
[1, 3]$, and the round-off error induced by this computation is bounded by the
interval $e^\sharp$.  Here $\errordom$ denotes the set of all abstract states,
where an abstract state $\sigma^\sharp \in \errordom$ is a mapping from a
variable to its associated value-error range; we can use $\sigma^\sharp(x)$ to
query the value-error range of a variable $x \in \varset$ in $\sigma^\sharp$.
The formula $\exprerrorfunc{e}{\sigma^\sharp}$, where $e \in \sexprset$ and
$\sigma^\sharp \in \errordom$, is used to indicate the accuracy analysis of
the expression $e$ with the input state $\sigma^\sharp$.  Therefore the above
example can be expressed concisely by $\exprerrorfunc{a + b}{\sigma^\sharp_0} =
(v^\sharp, e^\sharp)$.

We build on top of the above method outlined in~\cite{soap} to analyze the
accuracy of arithmetic expressions, which supported only addition, subtraction,
multiplication and division.  In this section, we explain in detail how it is
extended to support MIRs, and our additional operators in semantic expressions,
that is, the substitution operator, the ternary conditional operator and the
fixpoint operator.

\subsubsection{MIR}

In the same way that an expression can be analyzed for its accuracy, an MIR,
which is a mapping of variables to semantic expressions, can be analyzed by
computing the round-off error for each of its expressions.  For instance, the
accuracy of an MIR\@:
\begin{align}
    \mu_0 &= [a \mapsto a + b, b \mapsto a \times 0.5] \\
    \intertext{with an input state}
    \sigma^\sharp_0 &= [a \mapsto ([0, 1], [0, 0]), b \mapsto ([1, 2], [0, 0])]
\end{align}
can be analyzed as follows.  First we analyze the individual expressions
$\mu_0(a) = a + b$ and $\mu_0(b) = a \times 0.5$, which produce respectively
the results below:
\begin{align}
    (v^\sharp_a, e^\sharp_a) &= (
        [1, 3], [-1.19209304 \times 10^{-7}, 1.19209304 \times 10^{-7}]) \\
    (v^\sharp_b, e^\sharp_b) &= (
        [0, 0.5], [-2.98023259 \times 10^{-8}, 2.98023259 \times 10^{-8}])
\end{align}
then the analyzed results are collected into an abstract state assigning the
value-error bounds to their corresponding variables, that is:
\begin{equation}
    [a \mapsto (v^\sharp_a, e^\sharp_a), b \mapsto (v^\sharp_b, e^\sharp_b)].
\end{equation}

We define a function, $\mirerrorfunc{\mu}{\sigma^\sharp}$, to perform the
above analysis, which takes as an input the MIR $\mu \in \mirset$, and an
abstract input state $\sigma^\sharp \in \errordom$.  It computes a new
state ${\sigma^\sharp}^\prime$, where for each variable $x \in \varset$,
${\sigma^\sharp}^\prime(x)$ is the analyzed value-error range of the expression
$\mu(x)$.  Formally,
\begin{equation}
    \mirerrorfunc{\mu}{\sigma^\sharp} = {\left[
        x \mapsto \exprerrorfunc{\mu(x)}{\sigma^\sharp}
    \right]}_{x \in \varfunc{\mu}}.
    \label{eq:mir_accuracy}
\end{equation}
The notation ${[x \mapsto \exprerrorfunc{\mu(x)}{\sigma^\sharp}]}_{x \in
\varfunc{\mu}}$ means that the mapping is constructed by collecting for
each variable $x$ in the mapping $\mu$, the pairing of $x$ with the analyzed
value-error bound of the semantic expression $\mu(x)$.

\subsubsection{Substitution Operator}

The analysis of an expression $e \expand \mu$, where $e \in \sexprset$ and $\mu
\in \mirset$ is carried out in two steps.  Initially, given an input state
$\sigma^\sharp$, $\mu$ is analyzed using~\eqref{eq:mir_accuracy}, and we write
${\sigma^\sharp}^\prime$ as the analyzed state.  Then the expression $e$ is
analyzed for its accuracy as usual, using ${\sigma^\sharp}^\prime$ as the input
state.  Equivalently, this procedure can be defined as:
\begin{equation}
    \exprerrorfunc{e \expand \mu}{\sigma^\sharp}
    = \exprerrorfunc{e}{\left( \mirerrorfunc{\mu}{\sigma^\sharp} \right)}
\end{equation}

\subsubsection{Ternary Conditional Operator}

A conditional expression is written as $\select{b}{e_1}{e_2}$, where $b
\in \bexprset$ and $e_1, e_2 \in \sexprset$.  The truth value of Boolean
expression $b$ determines whether $e_1$ or $e_2$ is evaluated to be the
resulting value of the expression.  Correspondingly, in our accuracy analysis,
we impose a constraint defined by the Boolean expression $b$ on the value
ranges of input variables, such that $e_1$ is evaluated with the ranges of
values satisfying the constraint, while $e_2$ is computed with ranges that
violate the constraint.  For example, we analyze an expression $\select{(x
< 0)}{(x + 0.1)}{x}$ in single-precision.  Initially, we assume the program
state consists of a variable $x$, which is a floating-point value that has no
associated round-off error and is bounded by $[-1, 10]$, that is:
\begin{equation}
    \left[ x \mapsto \left( [-1, 10], [0, 0] \right) \right]
\end{equation}
We consider two cases, when the condition $x < 0$ is respectively true and
false.  For $x < 0$ to be true, $x$ must be in the range of $v^\sharp_b = [-1,
0^{-}]$, where $0^{-}$ is the greatest single-precision floating-point value
less than 0, because $x$ must be strictly smaller than $0$.  Then we restrict
the range of $x$ to $v^\sharp_b$, so $x + 0.1$ is analyzed to be $(v^\sharp_1,
e^\sharp_1)$, where:
\begin{equation}
    ( v^\sharp_1, e^\sharp_1 ) = (
        [-0.899999976, 0.100000001],
        [-3.12924406 \times 10^{-8}, 2.83122095 \times 10^{-8}]
    )
\end{equation}
Similarly, when $x < 0$ is false, we restrict the bound on $x$ with
$v^\sharp_{\neg b} = [0, 10]$ and analysis of the expression $x$ simply gives
$v^\sharp_2 = [0, 10]$ and no round-off error, $e^\sharp_2 = [0, 0]$.  Finally,
the analyzed result for the expression can be obtained by joining these two
cases together, by respectively taking the union of the value intervals
$v^\sharp_1$ and $v^\sharp_2$, which produces $v^\sharp = [-0.899999976, 10]$,
and the error bounds $e^\sharp_1$ and $e^\sharp_2$, which is $e^\sharp =
e^\sharp_1$.  The final result is therefore $(v^\sharp, e^\sharp)$:
\begin{equation}
    ( v^\sharp, e^\sharp ) = (
        [-0.899999976, 10],
        [-3.12924406 \times 10^{-8}, 2.83122095 \times 10^{-8}]
    )
\end{equation}

We now provide a formal definition of the above example analysis.  First, we
define the meaning of taking the union of two intervals:
\begin{equation}
    [a, b] \join [c, d] = [\min(a, c), \max(b, d)]
\end{equation}
Then the formula
\begin{equation}
    (v^\sharp_1, e^\sharp_1) \join (v^\sharp_2, e^\sharp_2) =
    (v^\sharp_1 \join v^\sharp_2, e^\sharp_1 \join e^\sharp_2) 
\end{equation}
is used to take the union of both $(v^\sharp_1, e^\sharp_1), (v^\sharp_2,
e^\sharp_2) \in \errorset$.  We then use the notation $\sigma^\sharp|_b$ and
$\sigma^\sharp|_{\neg b}$, where $\sigma^\sharp \in \errordom$ is the program
state, and $b \in \bexprset$ is the Boolean expression, to respectively mean
the program state $\sigma^\sharp$ is constrained by when $b$ is evaluated to be
either true or false.  Therefore, the following formula is used to perform the
accuracy analysis on a conditional expression:
\begin{equation}
    \exprerrorfunc{%
        \begin{aligned}
            \includegraphics[scale=\mirfigscale]{cond_expr}
            \vspace{-5pt}
        \end{aligned}
    }{\sigma^\sharp}
    =
    \left(\exprerrorfunc{e_1}{\sigma^\sharp|_b}\right) \join
    \left(\exprerrorfunc{e_2}{\sigma^\sharp|_{\neg b}}\right).
\end{equation}

\subsubsection{Fixpoint Operator}

An expression with a fixpoint operator, $\fixpoint{b, \mu_s, x}$, has
three child nodes, the Boolean expression $b \in \bexprset$, the loop body
represented with an MIR $\mu_s \in \mirset$, and the return variable $x \in
\varset$.  Similar to executing a \whilelit~loop, evaluating the expression is
to iteratively evaluate $b$ for its truth value, if $b$ is true, then the loop
MIR $\mu_s$ is used to update the program state for the next iteration and we
repeat the process and iterate until $b$ is evaluated to false.

Before we explain how a fixpoint expression can be analyzed for its accuracy,
we introduce the concept of \emph{loop invariants} (LI).  In our context, an LI
of a \whilelit~loop is a set of bounds on loop variables that holds invariantly
on entry to each loop iteration.  For example, in~\eqref{eq:syntax_example},
if our input is $n = 10$, then the LI on the variable $x$ is that its value is
an integer, and is bounded by $[0, 9]$ on loop entry, whereas on loop exit,
$x$ is always equal to $10$.  The reason for inferring the LI is that, because
we optimize the fixpoint expression's child nodes in a bottom-up hierarchy,
the child $\mu_s$ is optimized before the expression $\fixpoint{b, \mu_s, x}$
itself, we are using the LI as the input state to optimize $\mu_s$, because the
LI encompasses all possible program states the loop body $\mu_s$ will encounter
when executed.

Our accuracy analysis follows a similar pattern.  Initially, we start
with an input state $\sigma^\sharp_0$.  In the first iteration $k = 0$,
$\sigma^\sharp_k = \sigma^\sharp_0$ is split into two disjoint parts,
namely, $\sigma^\sharp_0|_b$ and $\sigma^\sharp_0|_{\neg b}$, they
respectively satisfies and violates the Boolean constraint $b$.  The state
$\sigma^\sharp_0|_b$ represents all possible program states that enters the
loop $\mu_s$, so $\sigma^\sharp_1 = \mirerrorfunc{\mu_s}{\sigma^\sharp_0|_b}$
captures all possible program states after the loop body.  This procedure
is repeated for iterations $k = 1, 2, 3, \ldots$, until a certain iteration
$n$, where $\sigma^\sharp_n = \sigma^\sharp_{n-1}$.  Hence, we can obtain the
LI by computing $\sigma^\sharp_0|_b \join \sigma^\sharp_1|_b \join \cdots
\join \sigma^\sharp_n|_b$, and the loop exit states with its counterpart,
\ie~$\sigma^\sharp_0|_{\neg b} \join \sigma^\sharp_1|_{\neg b} \join \cdots
\join \sigma^\sharp_n|_{\neg b}$.  The meaning of $\join$ operator on states
is similar to joining intervals and value-error bounds, which is defined
to join the two value-error bounds in respective states for each variable, which is defined as follows, where $\sigma^\sharp_a, \sigma^\sharp_b \in \errordom$:
\begin{equation}
    \sigma^\sharp_a \join \sigma^\sharp_b =
        {%
            [ x \mapsto \sigma^\sharp_a(x) \join \sigma^\sharp_b(x) ]
        }_{x \in \varset}.
\end{equation}

The process above is achieved with the algorithm in Figure~\ref{alg:fix}.
The value $\bot$ indicates an empty or unreachable state, and for any
state $\sigma^\sharp \in \errordom$, we have $\bot \join \sigma^\sharp =
\sigma^\sharp$.  The return values $li$ and $le$ are respectively the LI and
the loop exit state.
\newcommand{\statett}{\ensuremath\sigma^\sharp_{\mathrm{tt}}}
\newcommand{\stateff}{\ensuremath\sigma^\sharp_{\mathrm{ff}}}

\begin{figure}[ht]
    \centering
    \begin{algorithmic}
        \Function{Fixpoint-Analysis}{$\fixpoint{b, \mu_s, x}, \sigma^\sharp$}
            \State{%
                $\sigma^\sharp_0 \gets \sigma^\sharp$;
                $li \gets \bot$; $le \gets \bot$; $k \gets 0$
            }
            \Loop%
                \State{$\statett \gets \sigma^\sharp_k|_b$;}
                \State{$\stateff \gets \sigma^\sharp_k|_{\neg b}$}
                \State{$li \gets li \join \statett$;}
                \State{$le \gets le \join \stateff$}
                \State{%
                    $\sigma^\sharp_{k + 1} \gets
                    \mirerrorfunc{\mu_s}{\sigma^\sharp_{tt}}$}
                \If{$\sigma^\sharp_{k + 1} = \sigma^\sharp_{k}$}
                    \State{\Return{$li$, $le$}}
                \EndIf%
                \State{$k \gets k + 1$}
            \EndLoop%
        \EndFunction%
    \end{algorithmic}
    \caption{%
        The accuracy analysis of a fixpoint expression.}\label{alg:fix}
\end{figure}
