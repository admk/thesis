\section{Accuracy Analysis}
\label{po:sec:accuracy_analysis}

Because we make use of accuracy analysis to navigate the Pareto optimization
of program candidates, we start by providing an overview of how the accuracy
of simple arithmetic expressions are analyzed with the \soap{} framework, and
since it only allows arithmetic expressions with simple operators $\{+, -,
\times\}$, we explain how it can be extend fully to analyze \glspl{mir} and
semantic expressions.

In a typical program execution, values of variables, typically integers
$\integerset$ and floating-point values $\floatset$, are modified according
to the effect of the program statements, and they are propagated through
arithmetic operators from the beginning to the end of the program.  In
Section~\ref{so:sec:accuracy} of Chapter~\ref{chp:stropt}, an alternative
semantics, are proposed to instead propagate ranges of values together with
the associated round-off error bounds, \ie~the value-error bound $(v^\sharp,
e^\sharp) \in \errorset$, in order to analyze the accuracy of floating-point
numerical programs.  In this section, this technique is further generalized to
numerical programs.

Initially, we formalize the analyzed program values of a program as an abstract
program state using the domain $\errordom = \varset \to \errorset$, and a
$\sigma^\sharp \in \errordom$ maps each variable $\varx$ to their associated
value-error bound $\sigma^\sharp(\varx) \in \errorset$.

For instance, we assume an abstract state $\sigma^\sharp_0 \in \errordom$ which
provides the input values of $\vara$ and $\varb$ to a program, where:
\begin{equation}
    \sigma^\sharp_0 = \left[
        \vara \mapsto \left(\interval{0}{1}, e^\sharp_0\right),
        \varb \mapsto \left(\interval{1}{2}, e^\sharp_0\right) \right].
    \label{so:eq:example_state}
\end{equation}
This means that initially $\vara$ and $\varb$ are floating-point values bounded
by $[0, 1]$ and $[1, 2]$ respectively, and the error interval $e^\sharp_0 =
[0, 0]$ denotes the absence of round-off errors.

In Section~\ref{so:sec:accuracy} of Chapter~\ref{chp:stropt} we introduced the
function $\error: \aexprset \to \errorset$ to evaluate the bounds on the result
and its round-off error of computing an expression.  Here we introduce a new
function $\exprerrorop: \sexprset \to \errordom \to \errorset$ which further
accepts initial bounds on the values and errors of variables.  The formula
$\exprerrorfunc{e}{\sigma^\sharp}$, where $e \in \sexprset$ and $\sigma^\sharp
\in \errordom$, is used to denote the accuracy analysis of the expression $e$
with the input state $\sigma^\sharp$.

Then in single-precision, the error analysis of $\vara + \varb$, given the
initial bounds $\sigma^\sharp_0$ in~\eqref{so:eq:example_state}, produces the
following result:
\begin{equation}
    \exprerrorfunc{\vara + \varb}{\sigma^\sharp_0} = \left(
        \interval{1}{3},
        \interval{-1.19209304 \times 10^{-7}}{1.19209304 \times 10^{-7}}
    \right).
\end{equation}
This means that the result of this computation is in the range of $v^\sharp =
[1, 3]$, and the round-off error induced by this computation is bounded by the
interval $e^\sharp$.

The method outlined in Chapter~\ref{chp:stropt} to analyze the accuracy of
arithmetic expressions supports only addition, subtraction, multiplication and
division.  In this section, we explain in detail how it is extended to support
\glspl{mir}, and our additional operators in semantic expressions, \ie~the
composition, ternary conditional and fixpoint operators.

\subsection{MIR}

In the same way that an expression can be analyzed for its accuracy, an
\gls{mir}, which is a mapping of variables to semantic expressions, can
be analyzed by performing the $\exprerrorop$ analysis for each of its
expressions.  For instance, the accuracy of a \gls{mir}\@:
\begin{equation}
    \mu_0 = \left[
        \vara \mapsto \vara + \varb,
        \varb \mapsto \vara \times 0.5
    \right],
\end{equation}
with an input state:
\begin{equation}
    \sigma^\sharp_0 = \left[
        \vara \mapsto \left(
            \interval{0}{1}, \interval{0}{0}
        \right),
        \varb \mapsto \left(
            \interval{1}{2}, \interval{0}{0}
        \right)
    \right],
\end{equation}
can be analyzed as follows.

First we analyze the individual expressions $\mu_0(\vara) = \vara + \varb$
and $\mu_0(\varb) = \vara \times 0.5$, which produce respectively the results
below:
\begin{align}
    \left(
        v^\sharp_\vara, e^\sharp_\vara
    \right) &= \left(
        \interval{1}{3},
        \interval{-1.19209304 \times 10^{-7}}{1.19209304 \times 10^{-7}}
    \right), \\
    \left(
        v^\sharp_\varb, e^\sharp_\varb
    \right) &= \left(
        \interval{0}{0.5},
        \interval{-2.98023259 \times 10^{-8}}{2.98023259 \times 10^{-8}}
    \right).
\end{align}

Then the analyzed results are collected into an abstract state assigning the
value-error bounds to their corresponding variables, that is:
\begin{equation}
    \left[
        \vara \mapsto (v^\sharp_\vara, e^\sharp_\vara),
        \varb \mapsto (v^\sharp_\varb, e^\sharp_\varb)
    \right].
\end{equation}

To generalize, we can formally define a function,
$\mirerrorfunc{\mu}{\sigma^\sharp}$, to perform the above analysis,
which takes as inputs the \gls{mir} $\mu \in \mirset$ and an abstract
input state $\sigma^\sharp \in \errordom$.  It computes a new state
${\sigma^\sharp}^\prime$, where for each variable $\varx \in \varset$,
${\sigma^\sharp}^\prime(\varx)$ is the analyzed value-error range of the
expression $\mu(\varx)$.  The error analysis of a \gls{mir} is therefore:
\begin{equation}
    \mirerrorfunc{\mu}{\sigma^\sharp} = {\left[
        \varx \mapsto \exprerrorfunc{\mu(\varx)}{\sigma^\sharp}
    \right]}_{\varx \in \varfunc{\mu}}.
    \label{po:eq:mir_accuracy}
\end{equation}
The notation ${[\varx \mapsto
\exprerrorfunc{\mu(\varx)}{\sigma^\sharp}]}_{\varx \in \varfunc{\mu}}$ means
that the mapping is constructed by collecting for each variable $\varx \in
\varfunc\mu$, the pairing of $\varx$ with the analyzed value-error bound of the
semantic expression $\mu(\varx)$.

\subsection{Composition Operator}

The analysis of an expression $e \expand \mu$, where $e \in \sexprset$ and $\mu
\in \mirset$ is carried out in two steps.  Initially, given an input state
$\sigma^\sharp$, $\mu$ is analyzed using~\eqref{po:eq:mir_accuracy}, and we
write ${\sigma^\sharp}^\prime$ as the analyzed state.  Then the expression $e$
is analyzed for its accuracy as usual, using ${\sigma^\sharp}^\prime$ as the
input state.  Equivalently, this procedure can be defined as:
\begin{equation}
    \exprerrorfunc{e \expand \mu}{\sigma^\sharp}
    = \exprerrorfunc{e}{\left( \mirerrorfunc{\mu}{\sigma^\sharp} \right)}.
\end{equation}

\subsection{Ternary Conditional Operator}

A conditional expression is written as $\select{b}{e_1}{e_2}$, where $b
\in \bexprset$ and $e_1, e_2 \in \sexprset$.  The truth value of Boolean
expression $b$ determines whether $e_1$ or $e_2$ is evaluated to be the
resulting value of the expression.  Correspondingly, in our accuracy analysis,
we impose a constraint defined by the Boolean expression $b$ on the value
ranges of input variables, such that $e_1$ is evaluated with the ranges of
values satisfying the constraint, while $e_2$ is computed with ranges that
violate the constraint.

For example, we analyze an expression $\select{(\vara < 0)}{(\vara -
0.1)}{\vara}$ in single-precision.  Initially, we assume the program state
consists of a variable $\vara$, which is a floating-point value that has no
associated round-off error and is bounded by $\interval{-1}{10}$, that is:
\begin{equation}
    \sigma^\sharp_0 = \left[
        \vara \mapsto \left(
            \interval{-1}{10},
            \interval{0}{0}
        \right)
    \right].
\end{equation}
We consider two cases, when the condition $\vara < 0$ is respectively true and
false.  For $\vara < 0$ to be true, $\vara$ must be in the range of $v^\sharp_b
= [-1, 0^{-}]$, where $0^{-}$ is the greatest single-precision floating-point
value less than 0, because $\vara$ must be strictly smaller than $0$.  Then we
restrict the range of $\vara$ to $v^\sharp_b$, so $\vara - 0.1$ is analyzed to
be $(v^\sharp_1, e^\sharp_1)$, where:
\begin{equation}
    \left( v^\sharp_1, e^\sharp_1 \right) = \left(
        \interval{-1.10000002}{-0.100000001},
        \interval{-5.81145372\times10^{-8}}{6.10947666\times10^{-8}}
    \right).
\end{equation}
Similarly, when $\vara < 0$ is false, we restrict the bound on $\vara$
with $v^\sharp_{\neg b} = \interval{0}{10}$ and analysis of the expression
$\vara$ simply gives $v^\sharp_2 = \interval{0}{10}$ and no round-off error,
$e^\sharp_2 = \interval{0}{0}$.  Finally, the analyzed value and error ranges
for the expression can be obtained by joining these two cases together,
by respectively evaluating $v^\sharp_1 \join v^\sharp_2$, which produces
$v^\sharp = \interval{-1.10000002}{10}$, and the error bounds $e^\sharp_1 \join
e^\sharp_2$.  The final result is therefore:
\begin{equation}
    \begin{aligned}
        & \exprerrorfunc{
            \select{\left(\vara < 0\right)}{\left(\vara + 0.1\right)}{\vara}
        }{\sigma^\sharp_0} \\
        {}={} & \left(
            \interval{-1.10000002}{10},
            \interval{-5.81145372e-08}{6.10947666e-08}
        \right).
    \end{aligned}
\end{equation}

Our analysis is based on interval arithmetic which is very efficient but it
sacrifices accuracy by computing an over-approximation of the exact results.
For instance, the above analysis cannot capture the fact that all evaluated
result $v$ satisfies $-1.1 \leq v < 0$.  The majority of such information
losses occur because of a loose bound on the analyzed floating-point results.
In contrast, joining two error bounds generally produce a precise bound,
because error bounds are often much less correlated, and they often overlap
as there is a high chance that there exists an arithmetic computation which
produces an exact floating-point outcome.  In Section~\ref{po:sec:results},
empirical results show that despite the accuracy analysis potentially produces
loose bounds, it can still be used effectively as an indicator of the round-off
errors in actual executions.

We now provide a formal definition of the above example analysis.  We
use the notation $\sigma^\sharp|_b$ and $\sigma^\sharp|_{\neg b}$, where
$\sigma^\sharp \in \errordom$ is the program state, and $b \in \bexprset$ is
the Boolean expression, to respectively mean the program state $\sigma^\sharp$
is constrained by either $b$ being true or false.  Therefore, the following
formula is used to perform the accuracy analysis on a conditional expression:
\begin{equation}
    \exprerrorfunc{\ternarymir{$\qop$}{$b$}{$e_1$}{$e_2$}}{\sigma^\sharp}
    =
    \left(\exprerrorfunc{e_1}{\sigma^\sharp|_b}\right) \join
    \left(\exprerrorfunc{e_2}{\sigma^\sharp|_{\neg b}}\right).
\end{equation}

\subsection{Fixpoint Operator}
\label{po:sub:fixpoint}

An expression with a fixpoint operator, \fixexprmir, has three child nodes,
the Boolean expression $b \in \bexprset$, the loop body represented with
a \gls{mir} $\mu_s \in \mirset$, and the return variable $x \in \varset$.
Similar to executing a \whilelit~loop, evaluating the expression is to
iteratively evaluate $b$ for its truth value, if $b$ is true, then the loop
\gls{mir} $\mu_s$ is used to update the program state for the next iteration
and we repeat the process and iterate until $b$ is evaluated to false.

Before we explain how a fixpoint expression can be analyzed for its accuracy,
we introduce the concept of \gls{li}.  In our context, a \gls{li} of a
\whilelit~loop is a set of bounds on loop variables that holds invariantly on
entry to each loop iteration.  In Section~\ref{bg:sec:abstract_interpretation}
of Chapter~\ref{chp:background}, we explain how the \gls{li} of a simple
program loop can be computed, here we further extend this concept to general
programs expressed in \glspl{mir}.

For instance, we consider the \verb|basel| example in
Figure~\ref{po:lst:syntax_example}.  If our input is $n = 10$, then the
\gls{li} on the variable $x$ is that its value is an integer, and is bounded
by $[0, 9]$ on loop entry, whereas on loop exit, $x$ is always equal to $10$.
The reason for inferring the \gls{li} is as follows.  Since we optimize the
fixpoint expression's child nodes in a bottom-up hierarchy, the optimization of
$\mu_s$ precedes \fixexprmir~itself.  Hence we use the \gls{li} as the input
state to optimize $\mu_s$, as the \gls{li} encompasses all possible program
states the loop body $\mu_s$ will encounter when executed.

Our accuracy analysis of \fixexprmir~follows the above pattern.
Initially, we start with an input state $\sigma^\sharp_0$.  In the first
iteration $k = 0$, $\sigma^\sharp_k = \sigma^\sharp_0$ is split into two
disjoint parts, namely, $\sigma^\sharp_0|_b$ and $\sigma^\sharp_0|_{\neg b}$,
they respectively satisfies and violates the Boolean constraint $b$.  The state
$\sigma^\sharp_0|_b$ represents all possible program states that enters the
loop $\mu_s$, so $\sigma^\sharp_1 = \mirerrorfunc{\mu_s}{\sigma^\sharp_0|_b}$
captures all possible program states after the loop body.  This procedure
is repeated for iterations $k = 1, 2, 3, \ldots$, until a certain iteration
$n$, where $\sigma^\sharp_n = \sigma^\sharp_{n-1}$.  Hence, we can obtain the
\gls{li} by computing $\sigma^\sharp_0|_b \join \sigma^\sharp_1|_b \join \cdots
\join \sigma^\sharp_n|_b$, and the loop exit states with its counterpart,
\ie~$\sigma^\sharp_0|_{\neg b} \join \sigma^\sharp_1|_{\neg b} \join \cdots
\join \sigma^\sharp_n|_{\neg b}$.  The meaning of $\join$ operator on states is
similar to joining intervals and value-error bounds, which is defined to join
the two value-error bounds in respective states for each variable, which is
defined as follows, where $\sigma^\sharp_a, \sigma^\sharp_b \in \errordom$:
\begin{equation}
    \sigma^\sharp_a \join \sigma^\sharp_b =
        {%
            [ x \mapsto \sigma^\sharp_a(x) \join \sigma^\sharp_b(x) ]
        }_{x \in \varset}.
\end{equation}

Alternatively, we can compute the \gls{li} as the \gls{lfp} of $g: \errordom
\to \errordom$, where:
\begin{equation}
    g(y) = \mirerrorfunc{\mu_s}{
        \left.\left( y \join \sigma^\sharp \right)\right|_b
    }.
\end{equation}

This \gls{lfp} above can be computed with the algorithm in
Figure~\ref{po:alg:fix}.  The value $\bot$ indicates an empty or
unreachable state, and for any state $\sigma^\sharp \in \errordom$, we
have $\bot \join \sigma^\sharp = \sigma^\sharp$.  The return values
$\loopinvar$ and $\loopexit$ are respectively the \gls{li} and the result of
$\exprerrorfunc{\fixexprmir}{\sigma^\sharp}$.  In the rest of this chapter, we
use $\mathsf{E}_\mathsf{s}^\mathrm{LI} \left[\fixexprmir\right]\sigma^\sharp$
to signify the former return value $\loopinvar$.
\begin{figure}[ht]
    \centering
    \newcommand{\statett}{\ensuremath\sigma^\sharp_\truelit}
    \newcommand{\stateff}{\ensuremath\sigma^\sharp_\falselit}
    \begin{algorithmic}
        \singlespacing%
        \Function{FixpointAccuracyAnalysis}{
                \protect\fixexprmir, $\sigma^\sharp$}
            \State{%
                $\sigma^\sharp_0 \gets \sigma^\sharp$;\,
                $\loopinvar \gets \bot$;\,
                $\loopexit \gets \bot$;\,
                $k \gets 0$
            }
            \Loop%
                \State{%
                    $\statett \gets \sigma^\sharp_k|_b$
                }
                \State{%
                    $\stateff \gets \sigma^\sharp_k|_{\neg b}$
                }
                \State{%
                    $\loopinvar \gets \loopinvar \join \statett$
                }
                \State{%
                    $\loopexit \gets \loopexit \join \stateff$
                }
                \State{%
                    $\sigma^\sharp_{k + 1} \gets
                    \mirerrorfunc{\mu_s}{\sigma^\sharp_{tt}}$}
                \If{%
                        $\sigma^\sharp_{k + 1} = \sigma^\sharp_{k}$
                        $\vee$
                        $k \geq \mathrm{max\_iter}$
                }
                    \State{\Return{$\loopinvar$, $\loopexit$}}
                \EndIf%
                \State{$k \gets k + 1$}
            \EndLoop%
        \EndFunction%
    \end{algorithmic}
    \caption{%
        The accuracy analysis of a fixpoint expression.
    }\label{po:alg:fix}
\end{figure}

Our method extends the iterative method we have previously explained in
Section~\ref{bg:sec:abstract_interpretation} of Chapter~\ref{chp:background},
by not only evaluating the \gls{li}, but also the loop exit state.
Because this iterative process may not terminate, we introduce a constant
$\mathrm{max\_iter}$ to limit the number of iterations.  Widening
operators~\cite{cousot04} are additionally used to accelerate the fixpoint
computation.
