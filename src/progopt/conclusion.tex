\section{Summary}
\label{po:sec:conclusion}

A new method is proposed and carried out in \soap{} that performs general
numerical program transformation for the trade-off optimization among accuracy,
and two resource related metrics, \gls{lut} and \gls{dsp} utilizations.  To
optimize a numerical program, it starts by abstracting the program into a
\gls{mir}, which we designed to extract the essence of executing the program,
and removes unnecessary informations such as temporary variables, interleaving
of non-dependent statement, \etc{} The \gls{mir} is then optimized efficiently
by discovering a wide range of equivalent trade-off implementations of the
original \gls{mir}\@.  An optimized \gls{mir} can then be chosen to be
translated into a numerical program.  By using \soap, we optimize the accuracy
of our sample applications by up to 65\% in actual program executions.

In Section~\ref{po:sec:results}, our experiments show that accurate
implementations of numerical algorithms often increase their number of
arithmetic operations for reduced round-off errors.  In general, this could
result in circuits that have longer wall-clock time.  However with greater
area budget, we could have a greater freedom in rewriting programs to have
greater throughputs.  In the next chapter, we therefore explore how to optimize
numerical programs consisting of pipelined loops to trade off accuracy,
resources and latency.
