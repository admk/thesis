\section{Program to Metasemantic Intermediate Representation}
\label{po:sec:program_to_mir}

The first step of our approach is to analyze the program return value
into a metasemantic intermediate representation (MIR).  This procedure is
called \emph{metasemantic analysis} (MA).  The MA abstracts away irrelevant
information, and preserves the essence of program execution.  Details such
as temporary variables and the ordering of program statements are discarded,
whereas the abstraction still retains data-flow dependencies and keeps only
computations that contribute to the final results.

We work with the MIR as an abstraction of the program because the discovery of
equivalent structures can be much simplified.  For instance, the program ``$x
\assign 1 \semicolon y \assign 2$'' is the same as ``$y \assign 2 \semicolon x
\assign 1$'' because interleaving of non-dependent statements does not change
program semantics.  If we were to base our transformations on the program
syntax, we will need to enable this kind of equivalent relation even though
it has zero impact on our optimization with respect to resource usage and
accuracy.  A much simpler intermediate representation means that we can explore
a much smaller search space.

Our method analyzes a program by recursively dividing the program into
smaller parts, where each part can be separately analyzed into an MIR and
composed together to form a single MIR\@.  An MIR is a mathematical object
that associates each program variable with a semantic expression.  A semantic
expression is an arithmetic expression, but with additional syntactic features
to support \iflit{} statements and \whilelit{} loops.  We represent semantic
expressions with \emph{directed acyclic graphs} (DAGs) that share common
structures and define $\sexprset$ as the set of semantic expressions, and
$\mirset$ as the set of MIRs.  Because an MIR pairs a variable with an
expression, we can view it as a function $\varset \to \sexprset$ that maps
a variable into a semantic expression.  For instance, $\mu(x)$ returns the
associated expression of the variable $x \in \varset$ for the MIR $\mu \in
\mirset$.  For each variable, its semantic expression in itself provides a
complete picture of how computations can lead to the resulting value of the
variable.  In the rest of this section, we progressively explain how each type
of program statement defined in~\eqref{po:eq:program_syntax} is analyzed into
an MIR\@.

\subsection{Skip}

Because $\skipstmt$ has no effect on program states, the MIR of $\skipstmt$
also does not alter the program state.  Its MIR is defined as $[x \mapsto
x]_{x \in \varset}$, the subscript $x \in \varset$ denotes that the MIR is
constructed by collecting for each program variable $x$, the mapping shown in
the bracket, \ie~the paring of the variable $x$ with the expression $x$.  The
subscript $x \in \varset$ means that for each $x$, the MIR pairs it with $x \in
\sexprset$, which is a semantic expression containing just the variable $x$
itself.  We will repeatedly make use of the notation of constructing an MIR,
$[x \mapsto e(x)]_{p(x)}$ in the following sections, where $x$ is a variable
and $p(x)$ is a predicate on $x$, the MIR is constructed by collecting for all
$x$ the mapping from $x$ to the expression $e(x)$, when $p(x)$ is true.

\subsection{Assignment}

An assignment statement is in the form of ``$x \assign e$'', where $x \in Var$
is a program variable and $e \in \aexprset$ is an arithmetic expression.  The
metasemantic analysis of it produces an MIR as follows, where the operator
$\join$ merges the two MIRs together:
\begin{equation}
    \left[
        y \mapsto \left\{
            \begin{aligned}
                & e && \text{if~} y = x \\
                & y && \text{if~} y \neq x
            \end{aligned}
        \right.
    \right]_{y \in \varset}
    \label{po:eq:mir_assign}
\end{equation}
The MIR in~\eqref{po:eq:mir_assign} signifies for a variable $y \in \varset$,
if $y$ is $x$, then we assign the expression $e$ to the variable $x$, where
$e$ is a semantic expression represented with a DAG\@.  The DAG shares all
common subexpressions in $e$.  For instance, an expression written as $(x + 1)
\times (x + 1)$ shares the subexpression node $x + 1$ by reusing the node in
the DAG\@.  For each other program variable $y \in \varset$, where $y \neq x$,
$y$ is associated with a semantic expression $y \in \sexprset$, representing
that the MIR does not the value of all program variables except $x$, because
only $x$ is updated in the statement.

For example, for a program with two variables $x$ and $y$, analyzing the
statement ``$y \assign x \times 2$'' produces the following MIR\@, note the
variable $x$ is shared between two semantic expressions:
\mirfig{mir_assign_2}

\subsection{Sequential statements}
\label{po:sub:sequential_statements}

A sequential statement, ``$s_1 \semicolon s_2$'' is formed by joining together
$s_1$ and $s_2$, where $s_1, s_2 \in \stmtset$ are statements.  It signifies
that $s_1$ and $s_2$ are executed in sequence.  Therefore, it is necessary to
\emph{append} the effect of executing $s_1$ to that of $s_2$, to arrive at
the full MIR of ``$s_1 \semicolon s_2$''.  This concept can be realized by
defining a new operator $\expand$, the substitution operator, such that the MIR
of ``$s_1 \semicolon s_2$'' is equal to $\mu_2 \expand \mu_1$, where $\mu_1$
and $\mu_2$ are the MIRs of $s_1$ and $s_2$ respectively.  The resulting MIR
of $\mu_2 \expand \mu_1$ is constructed by substituting, for every expression
$e \in \sexprset$ in $\mu_2$, each variable $x$ in $e$ with $\mu_1(x)$, which
is the associated expression of $x$ in $\mu_1$.  Furthermore, the operator
allows the format $e \expand \mu$, where $e \in \sexprset$ is called the target
expression and $\mu \in \mirset$ is the source MIR, to mean the variables in
$e$ is substituted with $\mu$ using the substitution strategy above.

We illustrate this by using a simple example program $p = {}$``$x
\assign x + 1 \semicolon y \assign x \times 2$''.  Using the MIR of
assignments, the MIR of $p$ can be derived, as shown in the right-hand
side of equation~\eqref{po:eq:mir_seq_1}.  By substituting the variables
with corresponding expressions, we arrive at the left-hand side of
\eqref{po:eq:mir_seq_1}:
\mirfig{mir_seq_1}

\subsection{Conditional Branches}

Conditional branches, or \iflit~statements, are represented with ``$\iflit~
b~ \thenlit~ s_1~ \elselit~ s_2$''.  Here $b \in \bexprset$ is a Boolean
expression, and $s_1, s_2 \in \stmtset$ are respectively the true- and
false-branches.  Our analysis of \iflit~statements is slightly more complex,
as we start to consider control flows.  The analysis is carried out in two
steps.  The first step is to compute recursively, the MIRs $\mu_1, \mu_2
\in \mirset$ of the respective true- and false-branches, namely, $s_1$ and
$s_2$.  We introduce the conditional node ``$?$'', which is derived from
C syntax, to signify conditional branches in expressions.  The left-most,
middle and right-most children of this node are respectively the Boolean
expression, the true- and false-expressions.  Then the second step is to
compute a new MIR, where each program variable $x \in \varset$ is associated
with a conditional node with three children, the Boolean expression $b$,
$\mu_1(x)$ and $\mu_2(x)$, that is, $[x \mapsto \select{b}{\mu_1}{\mu_2}]_{x
\in \varset}$.  As an example we consider the program ``$\iflit~(x < 0)$
$\thenlit~(y \assign x \times 2)$ $\elselit~\skipstmt$'', where the set of
program variables is $\{x, y\}$.  Its MIR is shown in the left-hand side
of~\eqref{po:eq:mir_if_2}.  Because both true- and false-expressions of $x$
are the same, regardless of the truth value of $x < 0$, the two expressions
will evaluated to the same value.  In our analysis we further simplify
the expression of $x$, and the resulting MIR is in the right-hand side
of~\eqref{po:eq:mir_if_2}.
\mirfig{mir_if_2}

The traditional approach of program abstraction uses control and data flow
graphs (CDFGs)~\cite{namballa04}, which preserves the ordering of sequential
statements, uses a one-to-one mapping from assignment statements to assignment
nodes, storage nodes are used to store the result of assignments, \ie~it allows
nodes to act as a memory to store values, and finally, uses cycles in graphs
to represent program loops.  In contrast, our MIRs, from our analysis point of
view, use no local storage, discard unnecessary intermediate statements, and
most importantly, we treat control structures as operators in expressions, in
the same way as arithmetic computations.  In comparison with CDFGs, these above
facts make MIRs a more suitable candidate for program transformations.

\subsection{``While'' Loops}

For \whilelit~loops, ``$\whilelit\, b\, \dolit\, s$'', where $b \in \bexprset$
and $s \in \stmtset$, we begin by proposing a new operator for semantic
expressions, ``$\fix$'', which we call the \emph{fixpoint} operator.  It is
used to stand for \whilelit~loop structures in semantic expressions.  It has
three child nodes, the Boolean expression $b$, the loop body represented by an
MIR, and the loop exit variable.  The loop body MIR can be obtained with our MA
of the loop body, and the loop exit variable denotes which variable we use on
loop exit as the evaluated value of the fixpoint expression.  We let $\mu_s$ to
be the MIR of the loop body $s$, and derive the MIR of ``$\whilelit\, b\, s$'',
by computing the fixpoint expression for each variable:
\begin{equation}
    \left[
        x \mapsto \left\{
            \begin{aligned}
                & \begin{aligned}
                    \includegraphics[scale=\mirfigscale]{fix_expr}
                \end{aligned} && \text{if~} x \in \varfunc{\mu_s} \\
                & \quad ~ ~ x && \text{otherwise}
            \end{aligned}
        \right.
    \right]_{x \in \varset}
    \label{po:eq:mir_while}
\end{equation}
Here, $\varfunc{\mu_s}$ computes the set of variables that is assigned in the
loop body $\mu_s$.  If a program variable $x$ is in the set $\varfunc{\mu_s}$,
it is paired with its fixpoint expression $\fixpoint{b, \mu_s, x}$; otherwise
the variable $x$ is not updated in the loop, and the loop has no effect on its
value, therefore it is paired with an expression $x$.
