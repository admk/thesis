\section{Metasemantic Intermediate Representation}
\label{po:sec:program_to_mir}

There are infinite number of ways to rewrite numerical C programs, and many of
these rewrites produce programs that have the same resource usage, accuracy and
latency characteristics.  For instance, consider the following two pairs of
programs, where each pair are equivalent, but syntactically different, as they
carry out the same (and potentially redundant) computations.
\begin{figure}[ht]
    \centering
    \newsavebox{\mirlsta}
    \begin{lrbox}{\mirlsta}
        \begin{lstlisting}
$ $
x = x + 1;
y = 2 * x;
x = x + 3;
        \end{lstlisting}
    \end{lrbox}
    \newsavebox{\mirlstb}
    \begin{lrbox}{\mirlstb}
        \begin{lstlisting}
y = x + 1;
x = y;
y = y * 2;
x = x + 3;
        \end{lstlisting}
    \end{lrbox}
    \newsavebox{\mirlstc}
    \begin{lrbox}{\mirlstc}
        \begin{lstlisting}
$ $
x = x + 1;
if ($b$)
  x = 2 * x;
        \end{lstlisting}
    \end{lrbox}
    \newsavebox{\mirlstd}
    \begin{lrbox}{\mirlstd}
        \begin{lstlisting}
if ($b$)
  x = 2*(x+1);
else
  x++;
        \end{lstlisting}
    \end{lrbox}
    \newcommand{\lstwidth}{0.25\textwidth}
    \subfloat[$P_1$]{%
        \begin{minipage}{\lstwidth}
            \usebox{\mirlsta}
        \end{minipage}
    }
    \subfloat[$P^\prime_1$]{%
        \begin{minipage}{\lstwidth}
            \usebox{\mirlstb}
        \end{minipage}
    }
    \subfloat[$P_2$]{%
        \begin{minipage}{\lstwidth}
            \usebox{\mirlstc}
        \end{minipage}
    }
    \subfloat[$P^\prime_2$]{%
        \begin{minipage}{\lstwidth}
            \usebox{\mirlstd}
        \end{minipage}
    }
    \caption{%
        Two pairs of programs that are equivalent but syntactically different.
    }\label{po:fig:equiv_progs}
\end{figure}

In practice, it is desirable to eliminate as much as possible the need for
these syntactic rewrites that do not affect our performance metrics, \eg~the
numerical accuracy and resource usage of synthesized circuits.  It is therefore
desirable to perform transformations on a \gls{dag} representation of the
program, rather than on the program text directly.  This section introduces a
new \gls{ir}, which we call the \acrfull{mir}.  It expresses how each program
variable is updated, but abstracts away the order in which the updates occur,
and ignores any temporary variables that are not marked as program outputs.

As an example, the two equivalent programs $P_1$ and $P^\prime_1$ in
Figure~\ref{po:fig:equiv_progs} can be automatically translated into an
identical \gls{mir}\@:
\begin{equation}
    \begin{tikzpicture}[mir]
        \node[mirnode] (var_x) at (0,0) {\texttt{x}};
        \node[mirnode] (plus1) [right=of var_x] {$+$};
        \node[mirnode] (n3)    [below left=of plus1] {$3$};
        \node[mirnode] (plus2) [below right=of plus1] {$+$};
        \node[mirnode] (var_y) [right=of plus1] {\texttt{y}};
        \node[mirnode] (times) [right=of var_y] {$\times$};
        \node[mirnode] (n2)    [below right=of times] {$2$};
        \node[mirnode] (x)     [below left=of plus2] {\texttt{x}};
        \node[mirnode] (n1)    [below right=of plus2] {$1$};
        \draw[|->] (var_x) -- (plus1);
        \draw[<-] (plus1) -- (n3);
        \draw[<-] (plus1) -- (plus2);
        \draw[<-] (plus2) -- (x);
        \draw[<-] (plus2) -- (n1);
        \draw[|->] (var_y) -- (times);
        \draw[<-] (times) -- (plus2);
        \draw[<-] (times) -- (n2);
        \brackets{(current bounding box)}
    \end{tikzpicture}\,.
\end{equation}

\Glspl{mir} also abstract the control structure (\ie~\iflit~branches and
\whilelit~loops) of a program, preserving only the computations that lead
to the outputs. For instance, by using the ternary conditional operator
``$\qop$'' from C, programs with conditionals such as $P_2$ and $P^\prime_2$ in
Figure~\ref{po:fig:equiv_progs} has the following \gls{mir} form:
\begin{equation}
    \begin{tikzpicture}[mir]
        \node[mirnode] (var_x) at (0,0) {\texttt{x}};
        \node[mirnode] (qop)   [right=8mm of var_x] {$\qop$};
        \node[mirnode] (b)     [below left=of qop] {$b$};
        \node[mirnode] (times) [below=6mm of qop] {$\times$};
        \node[mirnode] (n2)    [below left=of times] {$2$};
        \node[mirnode] (plus)  [below right=of times] {$+$};
        \node[mirnode] (x)     [below left=of plus] {\texttt{x}};
        \node[mirnode] (n1)    [below right=of plus] {$1$};

        \draw[|->] (var_x) -- (qop);
        \draw[<-] (qop) -- (b);
        \draw[<-] (qop) -- (times);
        \draw[<-] (times) -- (n2);
        \draw[<-] (times) -- (plus);
        \draw[<-] (plus) -- (x);
        \draw[<-] (plus) -- (n1);
        \draw[<-] (qop) to[bend left] (plus);
        \brackets{(current bounding box)}
    \end{tikzpicture}\,.
\end{equation}

This representation is useful to us, because a single \gls{mir} is able to
capture a class of syntactically-distinct programs, all of which have the
same resource usage, accuracy, and latency characteristics.  By searching for
transformations on \glspl{mir}, we drastically reduce the size of our search
space.  Note that expressions in the \gls{mir} can share common structures;
this is useful for modeling the sharing of common subexpressions and makes the
search for optimizations much more efficient.

The first step of our approach is to analyze the program return value into
a \gls{mir}\@.  This procedure is called \gls{ma}.  The \gls{ma} abstracts
away irrelevant information, and preserves the essence of program execution.
Details such as temporary variables and the ordering of program statements are
discarded, whereas the abstraction still retains data-flow dependencies and
keeps only computations that contribute to the final results.

We work with the \gls{mir} as an abstraction of the program because the
discovery of equivalent structures can be much simplified.  For instance,
the program ``\verb|x = 1; y = 2;|'' is the same as ``\verb|y = 2; x = 1;|''
because interleaving of non-dependent statements does not change program
semantics.  If we were to base our transformations on the program syntax, we
will need to enable this kind of equivalence relation even though it has zero
impact on our optimization with respect to resource usage and accuracy.  A
simpler \gls{ir} means that we can explore a much smaller search space.

Our method analyzes a program by recursively dividing the program into smaller
parts, where each part can be separately analyzed into a \gls{mir} and
composed together to form a single \gls{mir}\@.  A \gls{mir} is a mathematical
object that associates each program variable with a semantic expression.  A
semantic expression is an arithmetic expression, but with additional syntactic
features to support \iflit{} statements and \whilelit{} loops.  We represent
semantic expressions with \glspl{dag} that share common structures and define
$\sexprset$ as the set of semantic expressions, and $\mirset$ as the set of
\glspl{mir}.  Because a \gls{mir} pairs a variable with an expression, we
can view it as a function $\varset \to \sexprset$ that maps a variable into
a semantic expression.  For instance, $\mu(\varx)$ returns the associated
expression of the variable $\varx \in \varset$ for the \gls{mir} $\mu \in
\mirset$.  For each variable, its semantic expression in itself provides a
complete picture of how computations can lead to the resulting value of the
variable.  In the rest of this section, we progressively explain how each type
of program statement defined in~\eqref{po:eq:program_syntax} is analyzed into a
\gls{mir}\@.

Similar to arithmetic expressions, which can be written in a linear form
(\eg~$a + b$), or in a tree structure (\eg~\binarymir{$+$}{$a$}{$b$}),
\glspl{mir} can also be expressed in both.  The former is more concise, whereas
the latter explicitly shares common structures.

\subsection{Assignment statements}

An assignment statement is in the form of ``\texttt{x = $e$;}'', where
$\varx \in \varset$ is a program variable and $e \in \aexprset$ is an
arithmetic expression.  The metasemantic analysis of it produces a \gls{mir}
as follows:
\begin{equation}
    {\left[
        \vary \mapsto \left\{
            \begin{aligned}
                & e && \text{if~} \vary = \varx \\
                & \vary && \text{otherwise}
            \end{aligned}
        \right.
    \right]}_{\vary \in \varset}.
    \label{po:eq:mir_assign}
\end{equation}

The \gls{mir} in~\eqref{po:eq:mir_assign} signifies for a variable $\vary \in
\varset$, if $\vary$ is $\varx$, then we assign the expression $e$ to the
variable $\varx$.  In graph form, $e$ is a semantic expression represented with
a \gls{dag}\@.  The \gls{dag} shares all common subexpressions in $e$.  For
instance, an expression written as $(\varx + 1) \times (\varx + 1)$ shares the
subexpression node $\varx + 1$ by reusing the node in the \gls{dag}\@.  For
each other program variable $\vary \in \varset$, where $\vary \neq \varx$,
$\vary$ is associated with a semantic expression $\vary \in \sexprset$,
representing that the \gls{mir} does not the value of all program variables
except $\varx$, because only $\varx$ is updated in the statement.

For example, we consider a program with two variables $\varx$ and $\vary$.
Analyzing the statement ``\verb|y = x * 2;|'' produces the following \gls{mir}
graph, it is notable that the variable $\varx$ is shared between two semantic
expressions:
\begin{equation}
    \begin{tikzpicture}[mir]
        \node[mirnode] (x) at (0, 0) {\varx};
        \node[mirnode] (xexpr) [right=of x] {\varx};
        \node[coordinate] (mid) [right=of xexpr] {};
        \node[coordinate] (amid) [above=of mid] {};
        \node[mirnode] (y) [right=of mid] {\vary};
        \node[mirnode] (times) [right=of y] {$\times$};
        \node[mirnode] (two) [below right=of times] {2};

        \draw[|->] (x) -- (xexpr);
        \draw[|->] (y) -- (times);
        \draw[<-] (times) -- (two);
        \node[coordinate] (left) [below=of y] {};
        \draw[<-] (times) to[out=-135, in=0] (left);
        \draw[-] (left) to[out=180, in=-45] (amid);
        \draw[-] (amid) to[out=135, in=60] (xexpr);

        \brackets{(current bounding box)}
    \end{tikzpicture}\,.
    \label{po:eq:mir_assign_2}
\end{equation}

\subsection{Sequential statements}
\label{po:sub:sequential_statements}

A sequential statement, ``$s_1 s_2$'' is formed by joining together $s_1$
and $s_2$, where $s_1, s_2 \in \stmtset$ are statements.  It signifies that
$s_1$ and $s_2$ are executed in sequence.  Therefore, it is necessary to
\emph{append} the effect of executing $s_2$ to that of $s_1$, to arrive at the
full \gls{mir} of ``$s_1 s_2$''.  This concept can be realized by defining
a new operator $\expand$, the \emph{composition operator}, such that the
\gls{mir} of ``$s_1 s_2$'' is equal to $\mu_2 \expand \mu_1$, where $\mu_1$ and
$\mu_2$ are the \glspl{mir} of $s_1$ and $s_2$ respectively.  The resulting
\gls{mir} of $\mu_2 \expand \mu_1$ is constructed by substituting, for every
expression $e \in \sexprset$ in $\mu_2$, each variable $\varx$ in $e$ with
$\mu_1(\varx)$, which is the associated expression of $\varx$ in $\mu_1$.
Furthermore, the operator allows the format $e \expand \mu$, where $e \in
\sexprset$ is called the target expression and $\mu \in \mirset$ is the source
\gls{mir}, to mean the variables in $e$ is substituted with $\mu$ using the
composition strategy above.

We illustrate this by finding the \gls{mir} of a simple example program,
``\verb|x=x+1; y=x*2;|''.  Using the \gls{mir} of assignments, the
\glspl{mir} of the respective assignment statements can be derived, as shown
in~\eqref{po:eq:mir_seq_1}.
\begin{equation}
    \begin{tikzpicture}[mir]
        \node[mirnode] (x) at (0, 0) {\varx};
        \node[mirnode] (xexpr) [right=of x] {\varx};
        \node[coordinate] (mid) [right=of xexpr] {};
        \node[coordinate] (amid) [above=of mid] {};
        \node[mirnode] (y) [right=of mid] {\vary};
        \node[mirnode] (times) [right=of y] {$\times$};
        \node[mirnode] (two) [below right=of times] {2};

        \draw[|->] (x) -- (xexpr);
        \draw[|->] (y) -- (times);
        \draw[<-] (times) -- (two);
        \node[coordinate] (left) [below=of y] {};
        \draw[<-] (times) to[out=-135, in=0] (left);
        \draw[-] (left) to[out=180, in=-45] (amid);
        \draw[-] (amid) to[out=135, in=60] (xexpr);

        \node[coordinate] (seqleft) [right=of times, xshift=5mm] {};
        \node[mirnode] (seq) [above right=of seqleft] {$\expand$};
        \node[coordinate] (seqright) [below right=of seq] {};

        \draw[<-] (seq) -- (seqleft);
        \draw[<-] (seq) -- (seqright);

        \node[mirnode] (x2) [right=of seqright] {\varx};
        \node[mirnode] (plus2) [right=of x2] {+};
        \node[mirnode] (xto2) [below left=of plus2] {\varx};
        \node[mirnode] (one2) [below right=of plus2] {1};
        \node[coordinate] (mid2) [right=of plus2] {};
        \node[mirnode] (y2) [right=of mid2] {\vary};
        \node[mirnode] (yexpr2) [right=of y2] {\vary};

        \draw[|->] (x2) -- (plus2);
        \draw[<-] (plus2) -- (xto2);
        \draw[<-] (plus2) -- (one2);
        \draw[|->] (y2) -- (yexpr2);

        \brackets{(x) (two)};
        \brackets{(x2) (yexpr2) (one2)};
    \end{tikzpicture}\,.
    \label{po:eq:mir_seq_1}
\end{equation}
By substituting the variables with corresponding expressions, we arrive at
the \glspl{mir} as shown in~\eqref{po:eq:mir_seq_2} below.  This \gls{mir}
simplification step is always carried out when possible.
\begin{equation}
    \begin{tikzpicture}[mir]
        \node[mirnode] (x) at (0, 0) {\varx};
        \node[mirnode] (plus) [right=of x] {$+$};
        \node[mirnode] (xto) [below left=of plus] {\varx};
        \node[mirnode] (one) [below right=of plus] {1};
        \node[coordinate] (mid) [right=of plus] {};
        \node[coordinate] (amid) [above=of mid] {};
        \node[mirnode] (y) [right=of mid] {\vary};
        \node[mirnode] (times) [right=of y] {$\times$};
        \node[mirnode] (two) [below right=of times] {2};

        \draw[|->] (x) -- (plus);
        \draw[<-] (plus) -- (xto);
        \draw[<-] (plus) -- (one);
        \draw[|->] (y) -- (times);
        \draw[<-] (times) -- (two);

        \node[coordinate] (left) [below=of y] {};
        \draw[<-] (times) to[out=-135, in=0] (left);
        \draw[-] (left) to[out=180, in=-45] (amid);
        \draw[-] (amid) to[out=135, in=60] (plus);

        \brackets{(x) (two)};
    \end{tikzpicture}\,.
    \label{po:eq:mir_seq_2}
\end{equation}

\subsection{Conditional Branches}

Conditional branches, or \iflit~statements, are represented with
``\lstinline[basicstyle=\tt]|if ($b$) {$s_1$} else {$s_2$}|''.  Here $b
\in \bexprset$ is a Boolean expression, and $s_1, s_2 \in \stmtset$ are
respectively the true- and false-branches.  Our analysis of \iflit~statements
is slightly more complex, as we start to consider control flows.  The analysis
is carried out in two steps.  The first step is to compute recursively,
the \glspl{mir} $\mu_1, \mu_2 \in \mirset$ of the respective true- and
false-branches, namely, $s_1$ and $s_2$.  We introduce the conditional node
``$\qop$'' which is derived from C syntax, to signify conditional branches in
expressions.  The left-most, middle and right-most children of this node are
respectively the Boolean expression, the true- and false-expressions.  Then the
second step is to compute a new \gls{mir}, where each program variable $\varx
\in \varset$ is associated with a conditional node with three children, the
Boolean expression $b$, $\mu_1(\varx)$ and $\mu_2(\varx)$.  The final \gls{mir}
is therefore:
\begin{equation}
    {
        \left[
            \varx \mapsto \select{b}{\mu_1(\varx)}{\mu_2(\varx)}
        \right]
    }_{\varx \in \varset}.
\end{equation}

As an example, we consider the program
``\lstinline[basicstyle=\ttfamily]{if (x < 0) y = x * 2;}'', where the set
of program variables is $\{\varx, \vary\}$.  Its \gls{mir} in graph form is
therefore:
\begin{equation}
    \begin{tikzpicture}[mir]
        \node[mirnode] (x) at (0, 0) {\varx};
        \node[mirnode] (xcond) [right=of x] {$\qop$};
        \node[mirnode] (xin) [below=of xcond, yshift=-7mm] {\varx};
        \node[mirnode] (less) [below right=of xcond, xshift=5mm, yshift=-1mm]
            {$<$};
        \node[mirnode] (zero) [below right=of less, yshift=-2mm] {0};

        \node[mirnode] (y) at (25mm, 0) {\vary};
        \node[mirnode] (ycond) [right=of y] {$\qop$};
        \node[mirnode] (times) [below=of ycond] {$\times$};
        \node[mirnode] (yin) [below right=of ycond] {\vary};
        \node[mirnode] (two) [below right=of times] {2};

        \draw[|->] (x) -- (xcond);
        \draw[channel, <-] (xcond) to[out=-90, in=135] (xin);
        \draw[channel, <-] (xcond) to[out=-45, in=90] (xin);
        \draw[channel, <-] (xcond) to[out=-135, in=135] (less);
        \draw[<-] (less) -- (xin);
        \draw[channel, <-] (less) -- (zero);

        \draw[|->] (y) -- (ycond);
        \draw[<-] (ycond) to[out=-135, in=45] (less);
        \draw[<-] (ycond) -- (times);
        \draw[<-] (ycond) -- (yin);
        \draw[channel, <-] (times) -- (xin);
        \draw[channel, <-] (times) -- (two);

        \brackets{(current bounding box)};
    \end{tikzpicture}\,.
\end{equation}
Because both true- and false-expressions of $\varx$ are the same, regardless of
the truth value of $\varx < 0$, the two expressions evaluate to the same value.
In our analysis we can immediately simplify the expression of $\varx$, and the
resulting \gls{mir} is semantically equivalent to the original:
\begin{equation}
    \begin{tikzpicture}[mir]
        \node[mirnode] (x) at (0, 0) {\varx};
        \node[mirnode] (xin) [right=of x] {\varx};
        \node[mirnode] (less) [below right=of xin, xshift=3mm, yshift=-1mm]
            {$<$};
        \node[mirnode] (zero) [below right=of less, yshift=-2mm] {0};

        \node[mirnode] (y) at (25mm, 0) {\vary};
        \node[mirnode] (ycond) [right=of y] {$\qop$};
        \node[mirnode] (times) [below=of ycond] {$\times$};
        \node[mirnode] (yin) [below right=of ycond] {\vary};
        \node[mirnode] (two) [below right=of times] {2};

        \draw[|->] (x) -- (xin);
        \draw[<-] (less) to[out=-135, in=45] (xin);
        \draw[channel, <-] (less) -- (zero);

        \draw[|->] (y) -- (ycond);
        \draw[<-] (ycond) to[out=-135, in=45] (less);
        \draw[<-] (ycond) -- (times);
        \draw[<-] (ycond) -- (yin);
\begin{pgfinterruptboundingbox}
        \draw[channel, <-] (times) to[out=-135, in=60] (xin);
\end{pgfinterruptboundingbox}
        \draw[channel, <-] (times) -- (two);

        \brackets{(x) (zero) (two)};
    \end{tikzpicture}\,.
\end{equation}

The traditional approach of program abstraction uses
\glspl{cdfg}~\cite{namballa04}, which preserves the ordering of sequential
statements, uses a one-to-one mapping from assignment statements to assignment
nodes, storage nodes are used to store the result of assignments, \ie~it allows
nodes to act as a memory to store values, and finally, uses cycles in graphs
to represent program loops.  In contrast, our \glspl{mir}, from our analysis
point of view, use no local storage of temporary values, discard unnecessary
intermediate statements.  Most importantly, we treat control structures as
operators in expressions, in the same way as arithmetic computations.  In
comparison with \glspl{cdfg}, these above facts make \glspl{mir} a more
suitable candidate for exploring the search space of program transformations.

\subsection{Loops}

A possible way to represent a \whilelit~loop,
``\lstinline[basicstyle=\tt]|while ($b$) {$s$}|'', where $b \in \bexprset$
and $s \in \stmtset$, is to effectively view it as the semantic expression of
a fully unrolled loop.  Unfortunately this expression has an infinite depth,
which cannot be represented fully in a data structure.  We therefore introduce
a new operator for semantic expressions, ``$\fix$'', which we call the
\emph{fixpoint} operator.  We further use the fixpoint expression $\fix(f)$ to
represent the \whilelit~loop, where the function $f: \sexprset \to \sexprset$,
defined as $f(e_\varx) = \select{b}{e_\varx \expand \mu_s}{\varx}$, represents
the computation of one iteration of the loop, and $\mu_s$ is the \gls{mir}
of the loop body.  Finally, the fixpoint expression of a \whilelit~loop can
be written more succinctly using $lambda$-expression~\cite{barendregt13},
\ie~$\fix\left(\lambda e_\varx \dotop \select{b}{e_\varx \expand
\mu_s}{\varx}\right)$.

In graph form, for simplicity the fixpoint node admits three child nodes,
namely, the Boolean expression $b$, the loop body represented by a \gls{mir},
and the loop exit variable.  The loop body \gls{mir} can be obtained with our
\gls{ma} of the loop body, and the loop exit variable denotes which variable we
use on loop exit as the evaluated result of the fixpoint expression.  We let
$\mu_s$ to be the \gls{mir} of the loop body $s$, and derive the \gls{mir} of
the \whilelit~loop, by computing the fixpoint expression for each variable:
\begin{equation}
    {\left[
        \varx \mapsto \left\{
            \begin{aligned}
                & \fixexprmir
                    && \text{if~} \varx \in \varfunc{\mu_s} \\
                & \quad ~ ~ \varx && \text{otherwise}
            \end{aligned}
        \right.
    \right]}_{\varx \in \varset}
    \label{po:eq:mir_while}
\end{equation}
Here, $\varfunc{\mu_s}$ computes the set of variables that is assigned
in the loop body $\mu_s$.  If a program variable $\varx$ is in the set
$\varfunc{\mu_s}$, it is paired with its fixpoint expression \fixexprmir;
otherwise $\varx$ is not updated in the loop, and the loop has no effect
on its value, therefore it is paired with an expression $\varx$.  Finally,
the constructed \gls{mir} maximally shares common expressions across nested
\glspl{mir}.

As a simple example, consider the program in
Figure~\ref{po:lst:loop_share_example}.  The \gls{mir} of this example program
allows a nested \gls{mir} to reference and reuse an expression from the outer
\gls{mir} as demonstrated in Figure~\ref{po:fig:mir_fix_external_2}, where:
\begin{equation}
    \mu = [\varx \mapsto \varx + 1],
    \text{~and} \quad
    b = \varx < \vary.
\end{equation}

\begin{figure}[ht]
    \centering
    \newsavebox{\loopsharelst}
    \begin{lrbox}{\loopsharelst}
    \begin{lstlisting}
y = y + 1;
while (x < y) {
    x = x + 1;
}
    \end{lstlisting}
    \end{lrbox}
    \subfloat[The program.]{%
        \begin{minipage}{0.3\textwidth}
            \usebox{\loopsharelst}
        \end{minipage}\label{po:lst:loop_share_example}
    }
    \subfloat[The \gls{mir}.]{%
        \begin{tikzpicture}[mir]
            \node[mirnode] (x) at (0, 0) {\varx};
            \node[mirnode] (*) [right=of x] {$\expand$};
            \draw[|->] (x) -- (*);

            \node[mirnode] (fix) [below left=of *] {$\fix$};
            \node[mirnode] (b) [below left=of fix] {$b$};
            \node[mirnode] (mu) [below=of fix] {$\mu$};
            \node[mirnode] (xin) [below right=of fix] {\varx};
            \draw[<-] (*) -- (fix);
            \draw[<-] (fix) -- (b);
            \draw[<-] (fix) -- (mu);
            \draw[<-] (fix) -- (xin);

            \node[coordinate] (mir) [below right=of *] {};
            \node[mirnode] (mirx) [right=of mir, xshift=-1mm] {\varx};
            \node[mirnode] (mirxin) [right=of mirx] {\varx};
            \node[mirnode] (miry) [right=of mirxin] {\vary};
            \node[coordinate] (mirymid) [right=of miry] {};
            \node[coordinate] (miryout) [right=of mirymid] {};
            \node[coordinate] (miryedge) [right=of miryout] {};
            \draw[<-] (*) -- (mir);
            \draw[|->] (mirx) -- (mirxin);
            \draw[|->] (miry) -- (mirymid);
            \brackets{(mirx) (mirxin) (miryedge)};

            \node[mirnode] (y) [right=of *, xshift=35mm] {\vary};
            \node[mirnode] (+) [right=of y] {$+$};
            \node[mirnode] (yin) [below left=of +] {\vary};
            \node[mirnode] (1) [below right=of +] {$1$};
            \draw[|->] (y) -- (+);
            \draw[<-] (+) -- (yin);
            \draw[<-] (+) -- (1);

            \draw[<-] (miryout)
                to[out=-90, in=-180] (38mm, -7mm)
                to[out=0, in=180] (48mm, 5mm)
                to[out=0, in=135] (+);

            \brackets{(x) (b) (mu) (1)};
        \end{tikzpicture}\label{po:fig:mir_fix_external_2}
    }
    \caption{%
        A simple program which exhibits common subexpressions reuse across
        nested \acrshortpl{mir}.
    }
\end{figure}


\subsection{Example analysis}

To illustrate all these above translation process in conjunction,
we can perform the \gls{ma} on the program \verb|basel| in
Figure~\ref{po:lst:syntax_example}.  Here, we translate it into the following
\gls{mir}, and the semantic expression for $\varx$ is omitted for simplicity:
\begin{equation}
\begin{tikzpicture}[mir]
    \node[mirnode] (x) at (0, 0) {\varx};
    \node[mirnode] (dots) [right=of x] {\textellipsis};
    \node[mirnode] (y) [right=of dots, xshift=5mm] {\vary};
    \node[mirnode] (expand) [right=of y] {$\expand$};
    \node[mirnode] (fix) [below left=of expand] {$\fix$};
    \node[mirnode] (less) [below left=of fix, xshift=-5mm] {$<$};
    \node[mirnode] (xin) [below left=of less] {\varx};
    \node[mirnode] (nin) [below right=of less] {\varn};
    \node[coordinate] (fixmir) [below=of fix, xshift=2mm, yshift=-7mm] {};
    \node[coordinate] (initmir) [right=of expand, xshift=4mm] {};
    \node[mirnode] (yin) [below right=of fix] {\vary};
    \node[mirnode] (a) [right=of expand, xshift=40mm] {\vara};
    \node[mirnode] (ain) [right=of a] {\vara};
    \node[mirnode] (n) [right=of ain] {\varn};
    \node[mirnode] (nin2) [right=of n] {\varn};
    \draw[|->] (x) -- (dots);
    \draw[|->] (y) -- (expand);
    \draw[|->] (a) -- (ain);
    \draw[|->] (n) -- (nin2);
    \draw[<-] (expand) -- (fix);
    \draw[<-] (fix) -- (less);
    \draw[<-] (less) -- (xin);
    \draw[<-] (less) -- (nin);
    \draw[<-] (fix) -- (fixmir);
    \draw[<-] (fix) -- (yin);
    \draw[<-] (expand) to[out=-45, in=135] (initmir);

    \node[mirnode] (fixx) [below=of fixmir, xshift=1mm, yshift=2mm] {\varx};
    \node[mirnode] (fixadd1) [right=of fixx] {$+$};
    \node[mirnode] (fixxin) [below left=of fixadd1] {\varx};
    \node[mirnode] (fixone) [below right=of fixadd1] {1};
    \node[mirnode] (fixy) [right=of fixadd1, xshift=5mm] {\vary};
    \node[mirnode] (fixadd2) [right=of fixy] {$+$};
    \node[mirnode] (fixyin) [below left=of fixadd2] {\vary};
    \node[mirnode] (fixdiv) [below right=of fixadd2] {$/$};
    \node[mirnode] (fixmul) [below right=of fixdiv] {$\times$};
    \node[coordinate] (fixmulin1) [left=of fixmul, xshift=-13mm] {};
    \node[coordinate] (fixmulin2) [right=of fixadd2, xshift=5mm] {};
    \node[coordinate] (fixmuledge) [right=of fixmul, xshift=-2.1mm] {};
    \node[mirnode] (fixa) [below=of fixx, yshift=-10mm] {\vara};
    \node[mirnode] (fixain) [right=of fixa] {\vara};
    \draw[|->] (fixx) -- (fixadd1);
    \draw[<-] (fixadd1) -- (fixxin);
    \draw[<-] (fixadd1) -- (fixone);
    \draw[|->] (fixy) -- (fixadd2);
    \draw[<-] (fixadd2) -- (fixyin);
    \draw[<-] (fixadd2) -- (fixdiv);
    \draw[<-] (fixdiv) -- (fixmul);
    \draw[<-] (fixmul) to[out=-135, in=-45] (fixmulin1);
    \draw[-] (fixmulin1) to[out=135, in=45] (fixadd1);
    \draw[<-] (fixmul) to[out=-45, in=-45] (fixmulin2);
    \draw[-] (fixmulin2) to[out=135, in=55] (fixadd1);
    \draw[|->] (fixa) -- (fixain);
    \draw[channel, <-] (fixdiv) to[out=-135, in=45] (fixain);
    \brackets{(fixx) (fixa) (fixmuledge)};

    \node[mirnode] (initx) [below=of initmir, xshift=2mm, yshift=2mm] {\varx};
    \node[mirnode] (initzero1) [right=of initx] {$0$};
    \node[mirnode] (inity) [below=of initx] {\vary};
    \node[mirnode] (initzero2) [right=of inity] {$0.0$};
    \node[mirnode] (inita) [right=of initzero1, xshift=5mm] {\vara};
    \node[mirnode] (initain) [right=of inita] {\vara};
    \node[mirnode] (initn) [below=of inita] {\varn};
    \node[mirnode] (initnin) [right=of initn] {\varn};
    \draw[|->] (initx) -- (initzero1);
    \draw[|->] (inity) -- (initzero2);
    \draw[|->] (inita) -- (initain);
    \draw[|->] (initn) -- (initnin);
    \brackets{(initx) (initnin)};

    \brackets{(current bounding box)};
\end{tikzpicture}\,.
\end{equation}
