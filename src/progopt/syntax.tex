\section{Syntax Definition}
\label{po:sec:syntax_definition}

Before we discuss program transform, we first look at the syntax definition
used to write numerical programs.  Our program transformation optimizes
\numimp{} programs.  In this section, we formally introduce \numimp, a simple
imperative language which is a subset of C that supports arithmetic and Boolean
expressions, conditional branches, as well as \texttt{while} loops.  Our
language allows numerical data types $\inttype$ and $\floattype$, respectively
stand for integer and floating-point types.

We define $\aexprset, \bexprset$ as the set of arithmetic and Boolean
expressions respectively, and $\stmtset$ denotes the set of program statements.
We then have following syntax definition for expressions and \numimp{}
programs, written in the Backus-Naur Form~\cite{knuth64}:
\newcommand{\syndef}{\ensuremath\mathbin{::=}}%
\newcommand{\synor}{\ensuremath\mathbin{\mid}}%
\begin{equation}
    \begin{aligned}
        a \syndef {} &
            n \synor
            x \synor
            a_1 \odot a_2,
        \quad b \syndef x < a, \\
        s \syndef {} &
            x~\texttt{=}~a \synor
            s_1 \semicolon s_2 \synor
            \mathtt{if}~(b)~\{ s_1 \}~\mathtt{else}~\{ s_2 \} \synor
            \whilelit~(b)~\{s\}
    \end{aligned}
    \label{po:eq:program_syntax}
\end{equation}
We define $\odot \in \left\{ +, -, \times, / \right\}$ to be the arithmetic
operators, $n$ is a numerical constant of type either \inttype{} or \floattype;
$x \in \varset$ is a variable; $a, a_1, a_2 \in \aexprset$ are arithmetic
expressions; $b$ ranges over Boolean expressions, $\bexprset$; and similarly,
$s, s_1, s_2 \in \stmtset$ are program statements.  In our formal definition,
for the purpose of simplicity, we restrict the Boolean expressions to those
of the form $x < a$, where $x$ is a variable and $a$ is an expression;
more complex Boolean expressions are included trivially in our actual
implementation.  Although \forlit~loop is not explicitly defined in the above
syntax definition, it can be trivially derived from a \whilelit~loop.

Furthermore, we introduce the ``\verb|#pragma| \verb|soap begin|'' and
``\verb|#pragma| \verb|soap end|'' directives to delimit the code fragment
to be optimized.  We can also use ``\verb|#pragma| \verb|soap in|'' and
``\verb|#pragma| \verb|soap out|'' to provide input ranges and to declare
output variables, respectively.

As a simple example, the program in Figure~\ref{po:fig:syntax_example} computes
an approximate value of ${\pi^2 a}/6$.  It has two inputs $a$, a floating point
value between 0 and 1, and $n$, an integer value between 10 and 20, which
determines the number of iterations for the loop, and a return variable $y$.

\begin{figure}[ht]
    \begin{lstlisting}
    #pragma soap in \
        float a = [0.0, 1.0], int n = [10, 20]
    #pragma soap out float y
    x = 0;
    y = 0.0;
    while (x < n) {
        x = x + 1;
        y = y + a / (x * x);
    }
    \end{lstlisting}
    \caption{A simple program written with our syntax definition.}
    \label{po:fig:syntax_example}
\end{figure}

Despite the simplicity of our syntax, it includes all the features of
a full programming language rather than an expression language used in
Chapter~\ref{chp:stropt}.  We will add support for arrays and matrices in
Chapter~\ref{chp:latopt}, and show that this can be added with little changes
to our method.
