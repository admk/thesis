\section{Syntax Definition}
\label{po:sec:syntax_definition}

Before we discuss program transform, we first look at the syntax definition
used to write numerical programs.  Our program transformation optimizes
programs written in a subset of C99.  In this section, we formally introduce
the syntax that constitutes a subset of C that supports arithmetic and Boolean
expressions, conditional branches, as well as \whilelit~and \forlit~loops.  Our
language allows numerical data types $\inttype$ and $\floattype$, respectively
stand for integer and floating-point types.

We define $\aexprset, \bexprset$ as the set of arithmetic and Boolean
expressions respectively, and $\stmtset$ denotes the set of program statements.
We then have following simplified syntax definition for expressions and
numerical programs, written in the Backus-Naur form~\cite{knuth64}:
\begin{equation}
    \begin{aligned}
        a \syndef {} &
            n \synor
            x \synor
            \texttt{-}a_1 \synor
            a_1 \opsymbol a_2, \qquad
        b \syndef
            \texttt{!}b_1 \synor
            a_1 \oplus a_2 \synor
            b_1 \odot b_2, \\
        s \syndef {} &
            \left[ t \right] x~\left[ \texttt{=}~a \right] \semicolon \synor
            s_1~s_2 \synor
            \iflit~(b)~\{ s_1 \}~\left[\elselit~\{ s_2 \}\right] \synor
            \whilelit~(b)~\{s\}, \\
        x \syndef {} & v, \qquad
        v \syndef \varx \synor \vary \synor \varz \synor \mathellipsis, \qquad
        t \syndef \inttype \synor \floattype.
    \end{aligned}
    \label{po:eq:program_syntax}
\end{equation}

We define $\opsymbol \in \left\{ \texttt{+}, \texttt{-}, \texttt{*}, \texttt{/}
\right\}$, $\oplus \in \{ \texttt{<}, \texttt{<=}, \texttt{>}, \texttt{>=},
\texttt{==}, \texttt{!=} \}$, and $\odot \in \{ \texttt{||}, \texttt{\&\&}
\}$ respectively to be the arithmetic, comparison and boolean operators,
$n$ is a numerical constant of type either \inttype{} or \floattype.  The
$v$ terms $\varx, \vary, \varz \in \varset$ are variables.  In the next
chapter, the definition of $x$ is further extended to arrays.  Additionally,
$a, a_1, a_2 \in \aexprset$ are arithmetic expressions, and $b, b_1, b_2$
ranges over Boolean expressions, $\bexprset$.  Program statements $s, s_1,
s_2 \in \stmtset$ comprises assignment statements, sequential statements,
\iflit~branches and \whilelit~loops.  Although \forlit~loop is not explicitly
defined in the above syntax definition, it can be trivially expressed using a
\whilelit~loop.  Finally, $t$ is an optional keyword that specifies the type of
$x$, if $x$ is not previously declared.  A term of the form $[d]$ indicates $d$
is optional.

Furthermore, we introduce the ``\verb|#pragma| \verb|soap begin|'' and
``\verb|#pragma| \verb|soap end|'' directives to delimit the code fragment
to be optimized.  We can also use ``\verb|#pragma| \verb|soap in|'' and
``\verb|#pragma| \verb|soap out|'' to provide input ranges and to declare
output variables, respectively.

As a simple example, the program in Figure~\ref{po:lst:syntax_example} computes
an approximate value of ${\pi^2 a}/6$.  It has two inputs $a$, a floating point
value between 0 and 1, and $n$, an integer value between 10 and 20, which
determines the number of iterations for the loop, and a return variable $y$.

\begin{figure}[ht]
    \centering
    \begin{minipage}{0.8\textwidth}
    \begin{lstlisting}
    #pragma soap begin
    #pragma soap in \
        float a = [0.0, 1.0], int n = [10, 20]
    #pragma soap out float y
    x = 0;
    y = 0.0f;
    while (x < n) {
        x = x + 1;
        y = y + a / (x * x);
    }
    #pragma soap end
    \end{lstlisting}
    \end{minipage}
    \caption{%
        A simple program, \texttt{basel}, written with our syntax definition.
    }\label{po:lst:syntax_example}
\end{figure}

Despite the simplicity of our syntax, it includes all the features of a full
programming language rather than an arithmetic expression language used in
Chapter~\ref{chp:stropt}.  We will add support for arrays and matrices in
Chapter~\ref{chp:latopt}, and show that this can be added with few changes to
our method.
