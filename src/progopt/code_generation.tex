\section{Code Generation}
\label{po:sec:code_generation}

The final stage is to translate the optimized MIR back to a program in its
original syntax.  As discussed earlier, the MA produces an abstraction of the
program, which means there are generally many ways of generating different
programs from the same MIR\@.  For this reason, certain heuristic optimizations
are performed before or during code generation, such as branch and loop fusion
transformations explained in our resource usage analysis to produce a unique
and deterministic translation from the MIR\@.

Our code generation is carried out in three stages.  The first stage applies
the transformations outlined in Section~\ref{po:sub:resource_usage_analysis}
to perform loop and branch fusion, and to allow sharing of expressions
across nested MIRs. After the first stage, we perform a simple one-to-one
mapping from MIRs to program code.  This process is a breadth-first traversal
of the MIR, it keeps removing nodes that has no data flows to other nodes
in the MIR and translate each node into one statement, and repeat this
process until the MIR is empty.  The final step of our code generation, is
to perform code sinking, which moves parts of the code so that when their
results are not needed, they are not executed~\cite{llvm}.  For instance,
the result of the statement ``$\assignstmt{x}{x + 1}$'' is only used in the
true-branch of the program ``$\assignstmt{y}{x + 1} \semicolon$ $\iflit~(x <
1)$ $\thenlit~(\assignstmt{y}{y + 1})$ $\elselit~(\assignstmt{y}{x})$''.  In
this final step of code generation, we move this statement into the true-branch
of the \iflit~statement, so it may be evaluated only when needed.
