\section{Code Generation}
\label{po:sec:code_generation}

The final stage is to translate the optimized \gls{mir} back to a program
in its original syntax.  As discussed earlier, the \gls{ma} produces an
abstraction of the program, which means there are generally many ways of
generating different programs from the same \gls{mir}\@.  For this reason,
certain heuristic optimizations are performed before or during code generation,
such as branch- and loop-fusion transformations explained in our resource
usage analysis to produce a unique and deterministic translation from the
\gls{mir}\@.

Our code generation is carried out in three stages.  The first stage applies
the transformations outlined in Section~\ref{po:sec:resource} to perform
loop- and branch-fusion, and to allow sharing of expressions across nested
\glspl{mir}.

After the first stage, we create a topological sort of all nodes, which
produces a linear ordering of all nodes such that the control- and
data-dependences are preserved.  After this, we perform a simple one-to-one
mapping from the list of nodes to program code.  An arithmetic node is
translated into an assignment statement which assigns a temporary variable with
the result of the arithmetic computation.  A ternary conditional and a fixpoint
node respectively generate an \iflit~statement and a \whilelit~loop.  Finally,
a composition node \binarymir{$\expand$}{$e$}{$\mu$} ensures that $\mu$ is
generated before $e$.

The final and optional step of our code generation, is to perform code sinking,
which moves parts of the code so that when their results are not needed, they
are not executed~\cite{llvm}.  For instance, the result of the statement
``\lstinline[basicstyle=\tt]{y = x + 1;}''
is only used in the true-branch of the program:
\begin{lstlisting}
    y = x + 1;
    if (x < 1)
        y = y + 1;
    else
        y = x;
\end{lstlisting}
This statement can therefore be moved into the true-branch of the
\iflit~statement, so it may be evaluated only when needed.
