\chapter{Benchmark Source Code}
\label{app:source}

In Section~\ref{po:sec:results} of Chapter~\ref{chp:progopt} we explored the
experimental results of several numerical programs as our benchmark examples.
This appendix contains the source code of the benchmark suite used.

\begin{figure}[ht]
\begin{lstlisting}
#pragma soap in float x=[0, 20]
#pragma soap out x

while (x > 1.0) {
    x = 0.9 * x;
}
\end{lstlisting}
\caption{\texttt{simple}}
\end{figure}

\begin{figure}[ht]
\begin{lstlisting}
#pragma soap in \
    int n=[10, 20], float x=[-0.1, 0.1], float y=[0, 1]
#pragma soap out z

float a = 1;
int b = 1;
float p = 1;
float z = 0.0f;
for (int i = 0; i < n; i++) {
    a = -a;
    b *= (2 * i + 1) * (2 * i);
    p *= (x + y) * (x + y);
    z += (a / b) * p;
}
\end{lstlisting}
\caption{\texttt{taylor}}
\end{figure}

\begin{figure}[ht]
\begin{lstlisting}
#pragma soap in \
    float a0=[0, 0.2], float a1=[0, 0.2], \
    float a2=[0.0, 0.2], float b0=[0, 0.2], \
    float b1=[0, 0.2], float b2=[0.0, 0.2], \
    float x=[0, 1], int n=20
#pragma soap out y

float x1 = 0.0f, x2 = 0.0f;
float y1 = 0.0f, y2 = 0.0f;
float y = x;
for (int i = 0; i < n; i++) {
    float yt = y;
    y = b0 * x + b1 * x1 + b2 * x2 +
        a0 * y + a1 * y1 + a2 * y2;
    x2 = x1;
    x1 = x;
    y2 = y1;
    y1 = yt;
}
\end{lstlisting}
\caption{\texttt{filter}}
\end{figure}

\begin{figure}[ht]
\begin{lstlisting}
#pragma soap in \
    float u=[0.0, 1.0], float w=[0.0, 1.0], \
    int n=[0, 20], float dt=[0.1, 0.1]
#pragma soap out u, v

float u;
float v = 0.0f;
for (int i = 0; i < n; i++) {
    float u0 = u + v * dt;
    v -= w * u * dt;
    u = u0;
}
\end{lstlisting}
\caption{\texttt{euler}}
\end{figure}

\begin{figure}[ht]
\begin{lstlisting}
#define n 20
#pragma soap in \
    float kp=[9, 10], float ki=[0.5, 0.7], float kd=[0, 3], \
    float dt=[0.2, 0.2], float m=8.0, float c=5.0,
#pragma soap out m

float i = 0.0f, e0 = 0.0f;
float m, e, d, r;
for (int j = 0; j < n; j++) {
    e = c - m;
    i += ki * dt * e;
    d = kd * (e - e0) / dt;
    r = kp * e + i + d;
    e0 = e;
    m += 0.01f * r;
}
\end{lstlisting}
\caption{\texttt{pid}}
\end{figure}
