\section{Introduction}

% floating-point arithmetic has round-off errors, exploit equivalence

As we have explained earlier, due to a finite number of values that
can be represented in floating-point arithmetic, numerical algorithms
generally always have round-off errors.  Therefore, equivalence rules
such as \emph{associativity} $(a + b) + c \equiv a + (b + c)$, and
\emph{distributivity} $(a + b) \times c \equiv a \times c + b \times c$
for real arithmetic no longer hold under floating-point arithmetic.  This
allows us to exploit these equivalence relations, to automatically generate
different equivalent expressions from the same arithmetic expression.  For the
optimization of FPGA implementations, these equivalent expressions can then
be selected for the optimal trade-off between resource usage when synthesized
into circuits, that is, the number of look-up tables (LUTs) and digital
signal processing (DSP) elements utilized, and accuracy when evaluated using
floating-point computations.  For example, with single precision floating-point
format, our tool found that give an input $x \in [0, 100]$ and $y \in [0, 2]$,
then the program:
\begin{equation}
    \begin{aligned}
        & \iflit~(x < 1)~\thenlit~( \\
        & \quad \assignstmt{x}{(x + y) + 0.1} \\
        & )~\elselit~( \\
        & \quad \assignstmt{x}{x + (y + 0.1)} \\
        & )
    \end{aligned}
\end{equation}
is most accurate when the subexpression $(x + y) + 0.1$ is written as
$(x + 0.1) + y$, because the subexpression is only evaluated when $x <
1$, our tool infers a tighter bound on $x$, $[0, 1]$, to optimize it;
and the original program uses fewest resources when subexpressions are
shared and the \if~statement is eliminated, \ie~``$\assignstmt{x}{x + (y
+ 0.1)}$''.  This kind of optimization generates a Pareto optimal set of
implementations.  A na{\"\i}ve strategy to search for the Pareto optimal
implementations is to discover all possible equivalent expressions.  However,
this would result in combinatorial explosion and become intractable even
for very small expressions~\cite{soap,ioualalen,mouilleron}.  To remedy
this, Gao~\etal~\cite{soap} proposed a novel approach, known as \soap, to
significantly reduce the space and time complexity to produce a subset of the
Pareto frontier.

% program transformation why?

In this chapter, we propose a new general \emph{program} optimization technique
for numerical algorithms, which allows \texttt{if} statements as well as
\texttt{while} loops, and developed its accompanied tool, \newsoap, to enable
the joint optimization of accuracy and resource usage, as well as the trade-off
between these performance metrics.  We develop a tool, which we call \newsoap,
to perform source-to-source optimization of numerical programs targeting FPGAs,
and generate implementations that trade off resource usage and numerical
accuracy.

Our program optimization flow is \emph{safe}, \emph{semantics-directed} and
\emph{flexible}. \emph{Safety} means that because we make use of formal
mathematics to optimize programs, our approach can be proved correct, in
the sense that when executed using exact real arithmetic, the transformed
version produces exactly the same output values as the original program.
\emph{Semantics-directed} transformation means that not only do we use
program syntax, but also the semantics to guide optimization and guarantee
safety properties of the optimized program.  Our technique obtains when
necessary, by analyzing the program, a bound and a round-off error bound on
each variable in every program location.  These information are then used
to guide program optimization.  By analyzing and manipulating not only the
syntax, but also the semantics of programs.  The meaning of a \emph{flexible}
program transformation is three-fold.  First, arithmetic computations can be
optimized across assignments, \iflit~statements and \whilelit~loops.  Secondly,
we automatically explore the numerical implications of partial loop unrolling
and loop splitting, which can create more opportunity for minimizing round-off
errors, hence further increases range of options in the Pareto frontier of
trade-offs.  Finally, our method naturally subsumes constant propagation,
redundant code elimination, and also branch and loop fusions.

% Our tool fits in the familiar setting of the high-level synthesis tool flow, as a front-end that performs the source-to-source optimization of 

Our main contributions in this chapter are as follows:
\begin{enumerate}
    \vspace{-6pt}
    \item
        A new intermediate representation of the behaviour of numerical
        programs, its structure is designed to be manipulated and analyzed
        with ease.  A new framework of numerical program transformations is
        developed to enable the back and forth translation between the program
        and a new intermediate representation (IR), which preserves the
        semantics of the original program.
    \vspace{-6pt}
    \item
        Semantics-based analyses that reason about not only the resource
        utilization (number of LUTs and DSP elements), and safe ranges of
        values and errors for programs, but also potential errors such as
        overflows and non-termination.
    \vspace{-6pt}
    \item
        A new tool, \newsoap, which trades off resource usage and accuracy
        by providing a safe, semantics-directed and flexible optimization
        targeting numerical programs for high-level synthesis.
    \vspace{-6pt}
\end{enumerate}

Section~\ref{sec:related_work} provides an overview of the existing work
related in both the software and high-level synthesis community.  We then
proceed to define our program syntax in Section~\ref{sec:syntax_definition}.
Using the syntax definition, we provide a detailed formal explanation
of our numerical program transformation, which consists of three
stages.  Section~\ref{sec:program_to_mir}, {}~\ref{sec:transformations},
{}~\ref{sec:code_generation} describe in sequence how numerical programs
can be translated into MIRs, how we infer bounds and error bounds on
variables and analyze resource usage estimates for the efficient discovery
of equivalent structures in the analyzed MIR, and finally, we explain how
a chosen MIR can be translated into an optimized numerical program.  Then
we present the optimization results in Section~\ref{sec:results} and
Section~\ref{sec:conclusion} concludes this chapter.
