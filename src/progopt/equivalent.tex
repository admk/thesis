\section{Equivalent Structure Analysis}
\label{po:sec:equivalence_analysis}

The next step is to use the analyses of accuracy and resource usage of
equivalent structures in \glspl{mir} to efficiently discover optimized
equivalent \glspl{mir}.  In this section, equivalent relations from
Section~\ref{so:sec:equivalent} of Chapter~\ref{chp:stropt} are extended
for control-flow structures, and we guide this process efficiently with our
analyses of accuracy and resource usage.

We start by taking one step further from simple arithmetic equivalence
relations such as associativity, commutativity and distributivity described
in Section~\ref{so:sec:equivalent} of Chapter~\ref{chp:stropt}.  We define
additional equivalent relations for composition, conditional and fixpoint
expressions to fully enable the equivalence transformations of \glspl{mir}.
Then we go on to improve the methods described in \soap~to explore more
equivalent structures, but still in an efficient and scalable way.


\subsection{Equivalence Relations}

In Chapter~\ref{chp:stropt}, the \soap~framework formally defines a set of
equivalent relations:
\begin{equation}
    {{}\eqrel{}} \subset \aexprset \times \aexprset,
\end{equation}
for discovering equivalent arithmetic expressions.  This equivalent relation
$\eqrel$ consists of arithmetic equivalence rules such as associativity,
commutativity and distributivity, as well as reduction rules that propagate
constants and simplify expressions.  We now expand it by making $\eqrel$ a
subset of $M \times M$:
\begin{equation}
    {{}\eqrel{}} \subset M \times M,
\end{equation}
where $M = \sexprset \cup \mirset$, by defining additional rules for $\eqrel$,
which relates equivalent semantic expressions that make use of the additional
operators introduced in this chapter.

For the following rules, we assume $b, b_1, b_2 \in \bexprset$, $e, e_1, e_2
\in \sexprset$, $\mu \in \mirset$, and $\opsymbol \in \{+, -, \times, /\}$, and
each rule has its inversed version.

\subsubsection{Ternary Conditional Operator}

The ternary conditional operator has the following distributivity
rules.  Firstly, for unary and binary arithmetic operators:
\begin{equation}
    \begin{aligned}
        -\left( \select{b}{e_1}{e_2} \right)
        & \quad \eqrel \quad
        \select{b}{-e_1}{-e_2}, \\
        e \opsymbol \left( \select{b}{e_1}{e_2} \right)
        & \quad \eqrel \quad
        \select{b}{%
            \left( e \opsymbol e_1 \right)
        }{%
            \left( e \opsymbol e_2 \right)
        }, \\
        \left( \select{b}{e_1}{e_2} \right) \opsymbol e
        & \quad \eqrel \quad
        \select{b}{%
            \left( e_1 \opsymbol e \right)
        }{%
            \left( e_2 \opsymbol e \right)
        }.
    \end{aligned}
\end{equation}
Secondly, we can distribute over the conditional operators:
\begin{equation}
    \begin{aligned}
        \select{b_1}{\left(\select{b_2}{e_1}{e_2}\right)}{e}
        & \quad \eqrel \quad
        \select{b_2}{%
            \left(\select{b_1}{e_1}{e}\right)}{%
            \left(\select{b_1}{e_2}{e}\right)}, \\
        \select{b_1}{e}{\left(\select{b_2}{e_1}{e_2}\right)}
        & \quad \eqrel \quad
        \select{b_2}{%
            \left(\select{b_1}{e}{e_1}\right)}{%
            \left(\select{b_1}{e}{e_2}\right)}.
    \end{aligned}
\end{equation}

\subsubsection{Composition Operator}

Similarly, the composition operator can also be distributed across arithmetic
and conditional operators:
\begin{equation}
    \begin{aligned}
        \left( -e \right) \expand \mu
        & \quad \eqrel \quad
        -\left( e \expand \mu \right), \\
        \left( e_1 \opsymbol e_2 \right) \expand \mu
        & \quad \eqrel \quad
        \left( e_1 \expand \mu \right) \opsymbol
        \left( e_2 \expand \mu \right), \\
        \left( \select{b}{e_1}{e_2} \right) \expand \mu
        & \quad \eqrel \quad
        \select{%
            \left( b \expand \mu \right)}{%
            \left( e_1 \expand \mu \right)}{%
            \left( e_2 \expand \mu \right)}.
    \end{aligned}
\end{equation}

\subsubsection{Fixpoint Operator}

The fixpoint operator represents loops, and loops can be partially unrolled,
similarly, we can define a set of equivalence relations to perform partial
unrolling on fixpoint expressions:
\begin{equation}
    \fix\left(b, \mu, \varx\right)
    \quad \eqrel \quad
    \fix\left(b, p_\mu^k \left(\mu\right), \varx\right)
    \quad \text{for $k \in \naturalset$}.
    \label{po:eq:unroll}
\end{equation}
Here $\varx \in \varset$ and $p_\mu$ is a function that computes the unrolling
of the loop \gls{mir} $\mu$, where
\begin{equation}
    p_\mu(\mu^\prime) = {\left[
        \varx \mapsto ( b \expand \mu ) \qop
            ( \mu^\prime(\varx) \expand \mu ) \colonop \mu(\varx)
    \right]}_{\varx \in \varset}.
\end{equation}
This set of rules formally defines the partial unrolling of loops with
a certain number of steps $k$.  Using the rules to unroll the loop
\lstinline[basicstyle=\tt]|while ($b$) {$s$}| where $b \in \bexprset$,
$s \in \stmtset$ is equivalent to unrolling the loop syntactically
as follows, as an example, we consider cases when $k = 0, 1, 2$ in
Figure~\ref{po:lst:loop_unroll}.
\begin{figure}[ht]
    \newsavebox{\whilefirstlstbox}
    \begin{lrbox}{\whilefirstlstbox}
        \begin{lstlisting}
$ $





while ($b$) {
    $s$
}
        \end{lstlisting}
    \end{lrbox}

    \newsavebox{\whilesecondlstbox}
    \begin{lrbox}{\whilesecondlstbox}
        \begin{lstlisting}
$ $


while ($b$) {
    $s$
    if ($b$) {
        $s$
    }
}
        \end{lstlisting}
    \end{lrbox}

    \newsavebox{\whilethirdlstbox}
    \begin{lrbox}{\whilethirdlstbox}
        \begin{lstlisting}
while ($b$) {
    $s$
    if ($b$) {
        $s$
        if ($b$) {
            $s$
        }
    }
}
        \end{lstlisting}
    \end{lrbox}
    \centering
    \subfloat[$k = 0$.]{%
        \begin{minipage}{0.3\textwidth}
            \usebox{\whilefirstlstbox}
        \end{minipage}
    }
    \subfloat[$k = 1$.]{%
        \begin{minipage}{0.3\textwidth}
            \usebox{\whilesecondlstbox}
        \end{minipage}
    }
    \subfloat[$k = 2$.]{%
        \begin{minipage}{0.3\textwidth}
            \usebox{\whilethirdlstbox}
        \end{minipage}
    }
    \caption{%
        A transformed \whilelit~loop with different partial loop unroll depths.
    }\label{po:lst:loop_unroll}
\end{figure}

The next set of rules in~\eqref{po:eq:tree} allows us to inductively discover
equivalent structures in child expressions.  In these rules, the formulae above
the line are the premises, whereas the ones below the line are conclusions,
this means that in such a rule, if the premise is true, then the conclusion
must be true.  For instance, in a conditional expression $\select{\varx <
\vary}{(\varx + \vary) + \varz}{\varx}$, because the subexpression $(\varx
+ \vary) + \varz$ is equivalent to $\varx + (\vary + \varz)$, it also
has an equivalent expression $\select{\varx < \vary}{\varx + (\vary +
\varz)}{\varx}$.  We define similar rules respectively for the conditional,
fixpoint and composition operators, and finally we also have an analogous rule
for \glspl{mir}, which formalizes that if any subexpressions $\mu(\varx)$ of
an \gls{mir} $\mu$ has an equivalent expression $e_\varx$, then the \gls{mir}
also has an equivalent \gls{mir} by replacing the expression $\mu(\varx)$ with
$e_\varx$.
\begin{equation}
    \begin{aligned}
        \inference{%
            b \eqrel b^\prime \quad
            e_1 \eqrel e^\prime_1 \quad
            e_2 \eqrel e^\prime_2
        }{%
            \select{b}{e_1}{e_2} \eqrel
            \select{b^\prime}{e^\prime_1}{e^\prime_2}
        }
        & \quad
        \inference{%
            b \eqrel b^\prime \quad
            \mu \eqrel \mu^\prime
        }{%
            \fix\left(b, \mu, \varx\right) \eqrel
            \fix\left(b^\prime, \mu^\prime, \varx\right)
        }
    \end{aligned}
    \nonumber
\end{equation}
\vspace{-10pt}
\begin{equation}
    \begin{aligned}
        \inference{%
            e \eqrel e^\prime \quad
            \mu \eqrel \mu^\prime
        }{%
            e \expand \mu \eqrel e^\prime \expand \mu^\prime
        }
        & \quad
        \inference{%
            \forall \varx \in \varset:
            \exists e_x \in \sexprset:
            \left( \mu\left(\varx\right) \eqrel e_x \right)
        }{%
            \mu \quad \eqrel \quad [\varx \mapsto e_\varx]_{\varx \in \varset}
        }
    \end{aligned}
    \label{po:eq:tree}
\end{equation}


\subsection{Discovering Equivalent Structures Efficiently}
\label{po:sub:discovering}

As we discussed earlier, discovering the full set of equivalent expressions
by finding the transitive closure of the relations is infeasible because of
combinatorial explosion.  Chapter~\ref{chp:stropt} proposed a new method to
drastically reduce the space and time complexity of discovering equivalent
expressions, while achieving high quality optimizations.

We base our equivalent expression discovery on this method, but extend it to
support not only simple arithmetic expressions but also additional program
transform features proposed in this chapter, \ie~conditional, composition
and fixpoint expressions, and \glspl{mir}.  This enables full program
transformations.

This section provides an informal overview of the equivalent discovery
procedure.  In Appendix~\ref{app:formal}, we discuss the formal definition
of this procedure for all semantic expressions and \glspl{mir} by
extending the optimization function $\optfunc{e}\sigma^\sharp$ proposed in
Section~\ref{so:sec:equivalent} of Chapter~\ref{chp:stropt}.

\subsubsection{Ternary Conditional Operator}

For conditional expressions of the form $\select{b}{e_1}{e_2}$, we first
optimize the Boolean expression $b$.  Then we use each equivalent expression
of $b$ to constraint the program state for the optimization of $e_1$ and
$e_2$.  Finally, after optimizing child nodes, we combine the equivalent
expressions of these three child nodes to form a set of equivalent
conditional expressions.

\subsubsection{Composition Operator}

Expressions of the form $e \expand \mu$ are optimized by first finding
equivalences of $\mu$, and then for each of them discovered, we compute a new
program state $\sigma^\sharp_0 = \mirerrorfunc{\mu} \sigma^\sharp$, and use it
to optimize $e$.

\subsubsection{Fixpoint Operator}

Fixpoint expressions are optimized in three steps.  Initially,
partially unrolled versions are discovered using the rules defined
in~\eqref{po:eq:unroll}, the expressions are partially unrolled by a factor up
to $3$.  Then for each unrolled version, we obtain its loop invariant using
the algorithm illustrated in Figure~\ref{po:alg:fix}, and the loop invariant
is then used to optimize its child nodes, which are the Boolean expression and
the loop \gls{mir}\@.  Finally, a set of equivalent fixpoint expressions can be
derived by combining equivalent child nodes together, and $\frontier$ is then
used on this set to eliminate Pareto-suboptimal fixpoint expressions.

\subsubsection{MIR}

\Glspl{mir} can be optimized by first optimizing their expressions
individually, then constructing a set of \glspl{mir} by enumerating all
combinations of optimized expressions for each variable.  The $\frontier$
function is additionally applied to reduce the size of the final set.

However, because a \gls{mir} consists of multiple expressions and each has its
own accuracy, we need a strategy to quantify the accuracy of a \gls{mir}\@.
In Section~\ref{so:sub:multiple_expressions} of Chapter~\ref{chp:stropt}, we
discuss the reasons for this requirement, and various strategies to optimize
multiple expressions, which are readily extendible to \glspl{mir}.
