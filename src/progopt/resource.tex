\section{Resource Usage Analysis}
\label{po:sec:resource}

In this section we give a detailed explanation of how resources in \glspl{mir}
can be shared and how to analyze the resource utilization of \glspl{mir}.

Expressions can have common subexpressions, and eliminating them reduces
resource usage.  We identify and eliminate common subexpressions when we
construct \glspl{dag} from programs.  Resource statistics can be estimated
by accumulating \gls{lut} and \gls{dsp} counts of each operator in the
\gls{dag}\@.  However, we can further merge multiple nodes into one to reduce
the estimated resource usage of generated code.  Conventional compiler
optimizations~\cite{kuck81} such as branch and loop fusion, as well as
redundant code elimination can be applied in an equivalent fashion to our
\glspl{mir} to share conditional and fixpoint operators.  This process further
reintroduces control structures to \glspl{mir}, so that the code generation
stage can make use of the result to synthesize code without redundant control
structures.


\subsection{Sharing conditional expressions}

In its essence, the sharing of the conditional operator is equivalent to
branch fusion.  For example, we consider the \gls{mir} of the program in
Figure~\ref{po:lst:branch_example}.  The \gls{ma} of it produces the \gls{mir}
in Figure~\ref{po:fig:mir_cond_fusion_1}.

\begin{figure}[ht]
    \centering
    \newsavebox{\branchfusionlst}
    \begin{lrbox}{\branchfusionlst}
    \begin{lstlisting}
if (x < 0) {
    x = 1;
    y = 2;
} else {
    x = 3;
    y = 4;
}
    \end{lstlisting}
    \end{lrbox}
    \subfloat[The program.]{%
        \begin{minipage}{0.25\textwidth}
            \usebox{\branchfusionlst}
        \end{minipage}\label{po:lst:branch_example}
    }
    \subfloat[Before fusion.]{%
        % \includegraphics[scale=\mirfigscale]{mir_cond_fusion_1}
        \begin{tikzpicture}[mir]
            \node[mirnode] (x) at (0, 0) {\varx};
            \node[mirnode] (xq) [right=of x] {$\qop$};
            \node[mirnode] (1) [below=of xq, yshift=-2mm, inner sep=2mm] {$1$};
            \node[mirnode] (3)
                [below right=of xq, yshift=-2mm, inner sep=1mm] {$3$};

            \node[coordinate] (mid) [right=of xq, xshift=13mm] {};
            \node[mirnode] (y) [right=of mid, xshift=5mm] {\vary};
            \node[mirnode] (yq) [right=of y] {$\qop$};
            \node[mirnode] (2) [below=of yq, inner sep=2mm] {$2$};
            \node[mirnode] (4) [below right=of yq, inner sep=1mm] {$4$};

            \node[mirnode] (<) [below=of mid, yshift=-2mm] {$<$};
            \node[mirnode] (xin) [below left=of <] {\varx};
            \node[mirnode] (0) [below right=of <] {$0$};

            \draw[|->] (x) -- (xq);
            \draw[|->] (y) -- (yq);
            \draw[<-] (xq) to[out=-135, in=135] (<);
            \draw[<-] (yq) to[out=-135, in=45] (<);
            \draw[channel,-] (xq) -- (1);
            \draw[channel,-] (xq) -- (3);
            \draw[<-] (yq) -- (2);
            \draw[<-] (yq) -- (4);
            \draw[<-] (<) -- (xin);
            \draw[<-] (<) -- (0);

            \brackets{(current bounding box)};
        \end{tikzpicture}\label{po:fig:mir_cond_fusion_1}
    }\quad
    \subfloat[After fusion.]{%
        % \includegraphics[scale=\mirfigscale]{mir_cond_fusion_2}
        \begin{tikzpicture}[mir]
            \node[mirnode] (x) at (0, 0) {\varx};
            \node[coordinate] (xmid) [right=of x] {};
            \node[coordinate] (xout) [right=of xmid] {};
            \node[coordinate] (mid) [right=of xout] {};
            \node[mirnode] (y) [right=of mid] {\vary};
            \node[coordinate] (ymid) [right=of y] {};
            \node[coordinate] (yout) [right=of ymid] {};
            %\node[coordinate] (yedge) [right=of yout] {};

            \node[coordinate] (out) [below=of mid, yshift=-2mm] {};
            \node[mirnode, inner sep=2mm] (?) [below=of out] {$\qop$};
            \node[mirnode] (<) [below left=of ?, xshift=-10mm, yshift=2mm] {$<$};
            \node[coordinate] (in1) [below=of ?, yshift=-3mm] {};
            \node[coordinate] (in2) [below right=of ?, xshift=10mm] {};

            \node[mirnode] (xin) [below left=of <] {\varx};
            \node[mirnode] (0) [below right=of <] {0};
            \node[mirnode] (1) [below left=of in1] {1};
            \node[mirnode] (2) [below right=of in1] {2};
            \node[mirnode] (3) [below left=of in2] {3};
            \node[mirnode] (4) [below right=of in2] {4};

            \draw[|->] (x) -- (xmid);
            \draw[|->] (y) -- (ymid);
            \draw[<-] (yout) to[out=-90, in=45] (out);
            \draw[<-] (xout) to[out=-90, in=135] (out);
            \draw[o] (?) -- (out);

            \draw[<-] (?) -- (<);
            \draw[<-] (<) -- (xin);
            \draw[<-] (<) -- (0);
            \draw[<-] (?) -- (in1);
            \draw[<-] (?) -- (in2);
            \draw[-] (1) -- (in1);
            \draw[o] (2) -- (in1);
            \draw[-] (3) -- (in2);
            \draw[o] (4) -- (in2);

            \brackets{(current bounding box)};
        \end{tikzpicture}\label{po:fig:mir_cond_fusion_2}
    }
    \caption{The sharing of conditional expressions in a simple program.}
\end{figure}

Because we compute an abstraction of the program, the \gls{mir} does not
keep the structure of the \iflit~statement to allow them to be optimized
separately, as doing this would allow our optimization to produce more accurate
implementations.  The resulting \gls{mir} of the program consists of two
conditional expressions as shown in Figure~\ref{po:fig:mir_cond_fusion_1}.
Because of this, after optimization, the \gls{mir} may has duplicate control
paths.  To resolve this, we introduce new kinds of nodes, as shown in
Figure~\ref{po:fig:mir_cond_fusion_2}, to ``bundle up'' more than one
conditional expressions, when they all have the same Boolean expression.


\subsection{Resource sharing in composition expressions}

Composition expressions can be fused in an analogous manner.  Any \glspl{mir}
$\mu_1$ and $\mu_2$ can be merged to form a new \gls{mir} if there are no
conflicts in the variable-expression pairing, \ie~for any variable $x$ that
is assigned an expression in both $\mu_1$ and $\mu_2$, $\mu_1(x)$ is equal
to $\mu_2(x)$.  A ``bundle'' of expressions can also be created to denote
the sharing of the composition operator.  For example, a simple \gls{mir}
in Figure~\ref{po:fig:mir_sub_fusion_1} can be restructured by fusing both
variables $x$ and $y$.
\begin{figure}[ht]
    \centering
    \subfloat[Before fusion.]{%
        % \includegraphics[scale=\mirfigscale]{mir_sub_fusion_1}
        \begin{tikzpicture}[mir]
            \node[mirnode] (x) at (0, 0) {\varx};
            \node[mirnode] (x*) [right=of x] {$\expand$};
            \node[coordinate] (mid) [right=of x*, xshift=10mm] {};
            \node[mirnode] (y) [right=of mid] {\vary};
            \node[mirnode] (y*) [right=of y] {$\expand$};

            \node[mirnode] (xin) [below left=of x*] {\varx};
            \node[mirnode] (xmir) [below right=of x*] {};
            \node[mirnode] (zin) [below left=of y*] {\varz};
            \node[mirnode] (zmir) [below right=of y*] {};

            \draw[<-] (x*) -- (xin);
            \draw[<-] (x*) -- (xmir);
            \draw[<-] (y*) -- (zin);
            \draw[<-] (y*) -- (zmir);

            \node[mirnode] (xmirx) [below=of x*] {\varx};
            \node[mirnode] (xmir+) [right=of xmirx] {$+$};
            \node[mirnode] (xmirxin) [below left=of xmir+] {\varx};
            \node[mirnode] (xmir1) [below right=of xmir+] {$1$};
            \draw[|->] (xmirx) -- (xmir+);
            \draw[<-] (xmir+) -- (xmirxin);
            \draw[<-] (xmir+) -- (xmir1);
            \brackets{(xmirx) (xmir1)};

            \node[mirnode] (zmirz) [below=of y*] {\varz};
            \node[mirnode] (zmir*) [right=of zmirz] {$\times$};
            \node[mirnode] (zmirxin) [below right=of zmir*] {\varx};
            \node[mirnode] (zmirzin) [below left=of zmir*] {\varz};
            \draw[|->] (zmirz) -- (zmir*);
            \draw[<-] (zmir*) -- (zmirzin);
            \draw[<-] (zmir*) -- (zmirxin);
            \brackets{(zmirz) (zmirxin)};

            \draw[|->] (x) -- (x*);
            \draw[|->] (y) -- (y*);

            \brackets{(current bounding box)};
        \end{tikzpicture}\label{po:fig:mir_sub_fusion_1}
    }\quad
    \subfloat[After fusion.]{%
        % \includegraphics[scale=\mirfigscale]{mir_sub_fusion_2}
        \begin{tikzpicture}[mir]
            \node[mirnode] (x) at (0, 0) {\varx};
            \node[coordinate] (xmid) [right=of x] {};
            \node[coordinate] (xout) [right=of xmid] {};
            \node[coordinate] (mid) [right=of xout] {};
            \node[mirnode] (y) [right=of mid] {\vary};
            \node[coordinate] (ymid) [right=of y] {};
            \node[coordinate] (yout) [right=of ymid] {};
            \draw[|->] (x) -- (xmid);
            \draw[|->] (y) -- (ymid);

            \node[coordinate] (out) [below right=of xout, yshift=-2mm] {};
            \node[mirnode] (*) [below=of out] {$\expand$};
            \draw[<-] (xout) -- (out);
            \draw[<-] (yout) to[out=-90, in=45] (out);
            \draw[o] (*) -- (out);

            \node[coordinate] (lin) [below left=of *] {};
            \node[coordinate] (mir) [below right=of *] {};
            \node[mirnode] (xin) [below left=of lin] {\varx};
            \node[mirnode] (zin) [below right=of lin] {\varz};
            \draw[<-] (*) -- (lin);
            \draw[-] (xin) -- (lin);
            \draw[o] (zin) -- (lin);
            \draw[<-] (*) -- (mir);

            \node[mirnode] (mirz) [right=of *] {\varz};
            \node[mirnode] (mir*) [right=of mirz] {$\times$};
            \node[mirnode] (mirx) [right=of mir*] {\varx};
            \node[mirnode] (mir+) [right=of mirx] {$+$};
            \node[mirnode] (mirzin) [below left=of mir*] {\varz};
            \node[mirnode] (mirxin) [below left=of mir+] {\varx};
            \node[mirnode] (mir1) [below right=of mir+] {$1$};
            \draw[|->] (mirz) -- (mir*);
            \draw[|->] (mirx) -- (mir+);
            \draw[<-] (mir*) -- (mirzin);
            \draw[<-] (mir*) -- (mirxin);
            \draw[<-] (mir+) -- (mirxin);
            \draw[<-] (mir+) -- (mir1);
            \brackets{(mirz) (mir1)};

            \brackets{(current bounding box)};
        \end{tikzpicture}\label{po:fig:mir_sub_fusion_2}
    }
    \caption{The sharing of composition expressions.}
\end{figure}


\subsubsection{Resource sharing in fixpoint expressions}

For fixpoint expressions that represent \whilelit~loops, discovering resource
sharing opportunities is a little more complex.  We fuse fixpoint expressions
in an analogous fashion as described above for conditional expressions, and
it is related to the conventional loop fusion compiler optimization.  For
instance, the program in Figure~\ref{po:lst:loop_example} has the \gls{mir}
in Figure~\ref{po:fig:mir_fix_fusion_1}, where:
\begin{equation}
    b = \varx > \vary, \quad
    \mu_1 = [\varx \mapsto \varx / 2], \text{~and} \quad
    \mu_2 = [\varx \mapsto \varx / 2, \varz \mapsto \varz + \varx].
\end{equation}

During program optimization, the two fixpoint expressions are optimized
individually, because loop splitting could enhance the accuracy of both $x$ and
$z$.  After this, in the resulting \gls{mir}, fixpoint expressions may or may
not share computations.  The fixpoint expressions can often be fused together
to save resources in the conditional expressions and the control logic that
are formerly duplicated, if they share common computations that can be merged
without conflict.  For instance, Figure~\ref{po:fig:mir_fix_fusion_2} shows how
two expressions can be fused into one fixpoint expression.
\begin{figure}[ht]
    \centering
    \newsavebox{\loopfusionlst}
    \begin{lrbox}{\loopfusionlst}
    \begin{lstlisting}
while (x > y) {
    z = z + x;
    x = x / 2;
}
    \end{lstlisting}
    \end{lrbox}
    \subfloat[The program.]{%
        \begin{minipage}{0.26\textwidth}
            \usebox{\loopfusionlst}
        \end{minipage}\label{po:lst:loop_example}
    }
    \subfloat[Before fusion.]{%
        \begin{tikzpicture}[mir]
            \node[mirnode] (x) at (0, 0) {\varx};
            \node[mirnode] (xfix) [right=of x] {$\fix$};
            \node[mirnode] (b1) [below left=of xfix] {$b$};
            \node[mirnode] (mu1) [below=of xfix] {$\mu_1$};
            \node[mirnode] (xin) [below right=of xfix] {\varx};

            \node[mirnode] (z) [right=of xfix, xshift=5mm] {\varz};
            \node[mirnode] (zfix) [right=of z] {$\fix$};
            \node[mirnode] (b2) [below left=of zfix] {$b$};
            \node[mirnode] (mu2) [below=of zfix] {$\mu_2$};
            \node[mirnode] (zin) [below right=of zfix] {\varz};

            \node[mirnode] (y) [right=of zfix, xshift=5mm] {\vary};
            \node[mirnode] (yin) [right=of y] {\vary};

            \draw[|->] (x) -- (xfix);
            \draw[|->] (z) -- (zfix);
            \draw[|->] (y) -- (yin);
            \draw[<-] (xfix) -- (b1);
            \draw[<-] (xfix) -- (mu1);
            \draw[<-] (xfix) -- (xin);
            \draw[<-] (zfix) -- (b2);
            \draw[<-] (zfix) -- (mu2);
            \draw[<-] (zfix) -- (zin);

            \brackets{(current bounding box)};
        \end{tikzpicture}\label{po:fig:mir_fix_fusion_1}
    } \quad
    \subfloat[After fusion.]{%
        \begin{tikzpicture}[mir]
            \node[mirnode] (x) at (0, 0) {\varx};
            \node[coordinate] (xmid) [right=of x] {};
            \node[coordinate] (xout) [right=of xmid] {};

            \node[mirnode] (z) [right=of xout, xshift=5mm] {\varz};
            \node[coordinate] (zmid) [right=of z] {};
            \node[coordinate] (zout) [right=of zmid] {};

            \node[mirnode] (y) [right=of zout, xshift=5mm] {\vary};
            \node[mirnode] (yin) [right=of y] {\vary};

            \node[coordinate] (out) [below right=of xout, yshift=-2mm] {};
            \node[mirnode] (fix) [below=of out] {$\fix$};
            \draw[<-] (xout) -- (out);
            \draw[<-] (zout) to[out=-90, in=45] (out);
            \draw[o] (fix) -- (out);

            \node[mirnode] (b) [below left=of fix] {$b$};
            \node[mirnode] (mu) [below=of fix] {$\mu_2$};
            \node[coordinate] (in) [below right=of fix, xshift=5mm] {};
            \node[mirnode] (xin) [below left=of in] {\varx};
            \node[mirnode] (zin) [below right=of in] {\varz};
            \draw[<-] (fix) -- (b);
            \draw[<-] (fix) -- (mu);
            \draw[<-] (fix) -- (in);
            \draw[-] (xin) -- (in);
            \draw[o] (zin) -- (in);

            \draw[|->] (x) -- (xmid);
            \draw[|->] (z) -- (zmid);
            \draw[|->] (y) -- (yin);

            \brackets{(current bounding box)};
        \end{tikzpicture}\label{po:fig:mir_fix_fusion_2}
    }
    \caption{%
        The sharing of fixpoint expressions in a simple example program.
    }
\end{figure}

However in some cases the above transformations could create cyclic graphs.  To
illustrate this, consider the following conditional expression fusion example
in Figure~\ref{po:fig:mir_cond_fusion_3}.  By fusing both conditionals $a$ and
$b$, a cycle is created as in Figure~\ref{po:fig:mir_cond_fusion_4}.
\begin{figure}[ht]
    \centering
    \subfloat[Before fusion, the \gls{mir} is acyclic.]{%
        % \includegraphics[scale=\mirfigscale]{mir_cond_fusion_3}
        \begin{minipage}{0.35\textwidth}
        \centering
        \begin{tikzpicture}[mir]
            \node[mirnode] (x) at (0, 0) {\varx};
            \node[mirnode] (x?) [right=of x] {$\qop$};
            \node[mirnodealt] (x?2) [below=of x?] {$\qop$};
            \node[mirnode] (xin) [below right=of x?] {\varx};
            \node[mirnodealt] (1)
                [below right=of x?2, xshift=-3mm, yshift=-1mm] {$1$};
            \node[mirnodealt] (2) [below right=of x?2] {$2$};
            \node[mirnodealt] (a) [below=of x?2, yshift=-8mm] {$a$};

            \node[mirnode] (y) [right=of x?] {\vary};
            \node[mirnode] (y?) [right=of y] {$\qop$};
            \node[mirnodealt] (y?2) [below=of y?] {$\qop$};
            \node[mirnode] (yin) [below right=of y?] {\vary};
            \node[mirnodealt] (3)
                [below right=of y?2, xshift=-3mm, yshift=-1mm] {$3$};
            \node[mirnodealt] (4) [below right=of y?2] {$4$};
            \node[mirnodealt] (b) [below=of y?2, yshift=-7mm] {$b$};
            \node[coordinate] (x?2-b) [above left=of b, xshift=-4mm] {};

            \draw[|->] (x) -- (x?);
            \draw[<-] (x?) -- (x?2);
            \draw[<-] (x?) -- (xin);
            \draw[<-] (x?2) -- (1);
            \draw[<-] (x?2) -- (2);

            \draw[|->] (y) -- (y?);
            \draw[<-] (y?) -- (y?2);
            \draw[<-] (y?) -- (yin);
            \draw[<-] (y?2) -- (3);
            \draw[<-] (y?2) -- (4);

            \draw[<-] (x?) to[out=-135, in=135] (a);
            \draw[<-] (x?2) to[out=-135, in=-180] (x?2-b);
            \draw[-] (x?2-b) to[out=0, in=135] (b);
            \draw[<-] (y?) to[out=-135, in=120] (b);
            \draw[channel,<-] (y?2) to[out=-120, in=45] (a);

            \brackets{(current bounding box)};
        \end{tikzpicture}
        \end{minipage}\label{po:fig:mir_cond_fusion_3}
    }\quad
    \subfloat[%
        Fusing the two conditional operators $a$ and $b$ creates a cycle.
    ]{%
        % \includegraphics[scale=\mirfigscale]{mir_cond_fusion_4}
        \begin{tikzpicture}[mir]
            \node[mirnode] (x) at (0, 0) {\varx};
            \node[coordinate] (xmid) [right=of x] {};
            \node[coordinate] (xout) [right=of xmid] {};
            \node[mirnode] (y) [right=of xout, xshift=15mm] {\vary};
            \node[coordinate] (ymid) [right=of y] {};
            \node[coordinate] (yout) [right=of ymid] {};
            \draw[|->] (x) -- (xmid);
            \draw[|->] (y) -- (ymid);

            \node[coordinate] (o1) [below right=of xout] {};
            \node[mirnodealt] (?1) [below left=of o1] {$\qop$};
            \node[mirnode] (a) [below left=of ?1] {$a$};
            \node[coordinate] (i1) [below=of ?1, yshift=-3mm, xshift=-3mm] {};
            \node[coordinate] (i2) [below right=of ?1, xshift=2mm] {};
            \node[mirnode] (3) [below right=of i1] {$3$};
            \node[mirnode] (xin) [below left=of i2] {\varx};
            \node[mirnode] (4) [below right=of i2] {$4$};

            \node[coordinate] (o2) [below left=of y] {};
            \node[mirnodealt] (?2) [below right=of o2] {$\qop$};
            \node[mirnode] (b) [below left=of ?2] {$b$};
            \node[coordinate] (i3) [below=of ?2, xshift=5mm] {};
            \node[coordinate] (i4) [below right=of ?2, xshift=10mm] {};
            \node[mirnode] (1) [below left=of i3] {$1$};
            \node[mirnode] (2) [below left=of i4] {$2$};
            \node[mirnode] (yin) [below right=of i4] {\vary};

            \draw[-, red] (o1) -- (?1);
            \draw[<-, red] (?1) -- (i1);
            \node[coordinate] (xl) [left=of x, yshift=-7mm] {};
            \draw[-, red] (i1) to[out=-135, in=-90] (xl);
            \node[coordinate] (xa) [above=of xout, yshift=1mm] {};
            \draw[-, red] (xl) to[out=90, in=180] (xa);
            \draw[-, red] (xa) to[out=0, in=135] (o2);
            \draw[<-, red] (?2) -- (i3);
            \node[coordinate] (1b) [below=of 1] {};
            \draw[-, red] (i3) to[out=-45, in=0] (1b);
            \node[coordinate] (bb) [below=of b, yshift=1mm, xshift=-1mm] {};
            \draw[-, red] (1b) to[out=180, in=-60] (bb);
            \draw[-, red] (bb) to[out=120, in=45] (o1);

            \draw[<-] (?1) -- (a);
            \draw[<-] (?1) -- (i2);
            \draw[o] (3) -- (i1);
            \draw[-] (xin) -- (i2);
            \draw[o] (4) -- (i2);
            \draw[<-,o] (xout) -- (o1);

            \draw[<-] (?2) -- (b);
            \draw[o] (1) -- (i3);
            \draw[-] (2) -- (i4);
            \draw[-] (yin) -- (i4);
            \draw[<-,o] (?2) -- (i4);
            \draw[<-] (yout) to[out=-90, in=45] (o2);

            \draw[o, red] (?2) -- (o2);

            \brackets{(x) (3) (yin) (xl)};
        \end{tikzpicture}
        {}\label{po:fig:mir_cond_fusion_4}
    }
    \caption{Example branch fusion creates a cycle.}
\end{figure}

Not only does conditional expression fusion create dependency cycles, all
of these above transformations when performed in conjunction also has the
potential to generate cycles and there are potentially multiple ways of
restructuring and combining fixpoint expressions, such cycles in \glspl{mir}
cannot be directly translated back into the source syntax.  In our example,
only one of either $a$ or $b$ can be fused without generating a dependency
cycle in the \gls{mir}\@.  There is more than one way of restructuring an
\gls{mir}, and each way could produce a different \gls{mir} that generates a
different program.  It is therefore necessary to be specific about the strategy
to make this procedure deterministic.  Our method explores fusion opportunities
with a depth-first traversal of the \gls{mir}\@.  While traversing nodes,
we attempt to form the maximal sharing for the current node; when a cycle
is created, the algorithm backtracks to the previous acyclic version.  This
process is deterministic and ensures the order of resource sharing.


\subsubsection{Resource counting}

Finally, after applying the resource sharing transformations outlined above, we
count the number of \glspl{lut} and \glspl{dsp}, by accumulating the statistics
for each operator instance in the \gls{mir}, in the same way described in
Section~\ref{so:sec:resource} of Chapter~\ref{chp:stropt}.  For this resource
estimation, we gather statistics of individual operators by synthesizing each
arithmetic operator (addition, subtraction, multiplication, division and
comparison), for single- and double-precision floating-point, using Altera's
floating-point megafunctions~\cite{altfp}.  Similarly, the resource usage of
integer arithmetic, conditional and fixpoint operators are synthesized and
estimated from multiplexers, and the composition operator uses no resources
because it does not perform computations.
