\subsection{Resource Usage Analysis}
\label{sub:resource_usage_analysis}

Expressions can have common subexpressions, and eliminating them reduces
resource usage.  We identify and eliminate common subexpressions when we
construct DAGs from programs.  Resource statistics can be estimated by
accumulating LUT and DSP counts of each operator in the DAG\@.  However, we can
further merge multiple nodes into one to reduce resource usage.  Conventional
compiler optimizations~\cite{kuck81} such as branch and loop fusion, as well as
redundant code elimination can be applied in an equivalent fashion to our MIRs
to share conditional and fixpoint operators.

\subsubsection{Sharing conditional expressions}

In its essence, the sharing of the conditional operator is equivalent to branch
fusion.  For example, the metasemantic analysis of the program below produces
the MIR in Figure~\ref{fig:mir_cond_fusion_1}:
\begin{equation}
    \begin{aligned}
        & \iflit~(x < 0)~\thenlit~( \\
        & \quad \assignstmt{x}{1} \semicolon
                \assignstmt{y}{2} \\
        & )~\elselit~( \\
        & \quad \assignstmt{x}{3} \semicolon
                \assignstmt{y}{4} \\
        & )
    \end{aligned}
\end{equation}

\begin{figure}[ht]
    \centering
    \subfloat[Before fusion]{%
        \includegraphics[scale=\mirfigscale]{mir_cond_fusion_1}
        {}\label{fig:mir_cond_fusion_1}
    }\quad
    \subfloat[After fusion]{%
        \includegraphics[scale=\mirfigscale]{mir_cond_fusion_2}
        {}\label{fig:mir_cond_fusion_2}
    }
    \caption{The sharing of conditional expressions.}
\end{figure}

Because we compute an abstraction of the program, the MIR does not keep
the structure of the \iflit~statement to allow them to be optimized
separately, as doing this would allow our optimization to produce more
accurate implementations.  The resulting MIR of the program consists of two
conditional expressions as shown in Figure~\ref{fig:mir_cond_fusion_1}.
Because of this, after optimization, the MIR may has duplicate control
paths.  To resolve this, we introduce new kinds of nodes, as shown in
Figure~\ref{fig:mir_cond_fusion_2}, to ``bundle up'' more than one conditional
expressions, when they all have the same Boolean expression.


\subsubsection{Resource sharing in substitution expressions}

Substitution expressions can be fused in an analogous manner.  Any MIRs
$\mu_1$ and $\mu_2$ can be merged to form a new MIR if there are no conflicts
in the variable-expression pairing, \ie~for any variable $x$ that is
assigned an expression in both $\mu_1$ and $\mu_2$, $\mu_1(x)$ is equal
to $\mu_2(x)$.  A ``bundle'' of expressions can also be created to denote
the sharing of the substitution operator.  For example, a simple MIR in
Figure~\ref{fig:mir_sub_fusion_1} can be restructured by fusing both variables
$x$ and $y$.
\begin{figure}[ht]
    \centering
    \subfloat[Before fusion]{%
        \includegraphics[scale=\mirfigscale]{mir_sub_fusion_1}
        {}\label{fig:mir_sub_fusion_1}
    }
    \subfloat[After fusion]{%
        \includegraphics[scale=\mirfigscale]{mir_sub_fusion_2}
        {}\label{fig:mir_sub_fusion_2}
    }
    \caption{The sharing of substitution expressions.}
\end{figure}

\subsubsection{Resource sharing in fixpoint expressions}

For fixpoint expressions that represent \whilelit~loops, discovering
resource sharing opportunities is a little more complex.  We fuse
fixpoint expressions in an analogous fashion as described above for
conditional expressions, and it is related to the conventional loop fusion
compiler optimization.  For instance, the program $\whilelit~(x > y)$
$\dolit~(\assignstmt{z}{z + x} \semicolon$ $\assignstmt{x}{x / 2})$ has the
MIR in Figure~\ref{fig:mir_fix_fusion_1}, where $b = x > y$, $\mu_1 = [x
\mapsto x / 2]$ and $\mu_2 = [x \mapsto x / 2, z \mapsto z + x]$.  During
our optimizations, we may wish to evaluate both $x$ and $z$ as accurate as
possible, therefore the two fixpoint expressions are optimized individually,
because loop splitting could enhance the accuracy of both $x$ and $z$.
After optimization, in the resulting MIR, fixpoint expressions may share
computations.  For instance, by fusing these two expressions into one fixpoint
expression, as illustrated in Figure~\ref{fig:mir_fix_fusion_2}, we save
resource in the conditional expressions and the control logic that are formerly
duplicated.
\begin{figure}[ht]
    \centering
    \subfloat[Before fusion]{%
        \includegraphics[scale=\mirfigscale]{mir_fix_fusion_1}
        {}\label{fig:mir_fix_fusion_1}
    }
    \subfloat[After fusion]{%
        \includegraphics[scale=\mirfigscale]{mir_fix_fusion_2}
        {}\label{fig:mir_fix_fusion_2}
    }
    \caption{The sharing of fixpoint expressions.}
\end{figure}

Another transformation is to allow sharing of expressions across nested
MIRs.  As a simple example, the program ``$\assignstmt{y}{y + 1}
\semicolon \whilestmt{x < y}{\assignstmt{x}{x + 1}}$'' has the MIR in
Figure~\ref{fig:mir_fix_external_1}, where $\mu = [x \mapsto x + 1]$ and $b =
x < y$, which also has duplicate computations (shaded region).  An additional
structure is required to share them across expressions.  We restructure it
by introducing a special \emph{external} operator, to allow a nested MIR to
reference and reuse an expression from the outer MIR as demonstrated below in
Figure~\ref{fig:mir_fix_external_2}.
\begin{figure}[ht]
    \centering
    \subfloat[Before]{%
        \includegraphics[scale=\mirfigscale]{mir_fix_external_1}
        {}\label{fig:mir_fix_external_1}
    }
    \subfloat[After]{%
        \includegraphics[scale=\mirfigscale]{mir_fix_external_2}
        {}\label{fig:mir_fix_external_2}
    }
    \caption{Restructuring nested MIR to reuse expression.}
    \centering
\end{figure}

However in some cases the above transformations could create cyclic graphs.
To illustrate this, consider the following conditional expression fusion
example in Figure~\ref{fig:mir_cond_fusion_3}.  By fusing both conditionals
$a$ and $b$, a cycle is created as in Figure~\ref{fig:mir_cond_fusion_4}.
\begin{figure}[ht]
    \centering
    \subfloat[Before fusion, the MIR is acyclic]{%
        \includegraphics[scale=\mirfigscale]{mir_cond_fusion_3}
        {}\label{fig:mir_cond_fusion_3}
    }\quad
    \subfloat[Fusion of $a$ and $b$ creates a cycle]{%
        \includegraphics[scale=\mirfigscale]{mir_cond_fusion_4}
        {}\label{fig:mir_cond_fusion_4}
    }
    \caption{Example branch fusion creates a cycle.}
\end{figure}

Not only does conditional expression fusion create dependency cycles, all
of these above transformations when performed in conjunction also has the
potential to generate cycles and there are potentially multiple ways of
restructuring and combining fixpoint expressions, such cycles in MIRs cannot
be directly translated back into the source syntax.  In our example, only
one of either $a$ or $b$ can be fused without generating a dependency cycle
in the MIR\@.  There are more than one way of restructuring an MIR, and
each way could produce a different MIR that generates a different program.
Therefore it is necessary to be specific about the strategy to make this
procedure deterministic.  Our method explores fusion opportunities with a
depth-first traversal of the MIR\@.  While traversing nodes, we attempt
to explore possibilities of sharing for the current node; when a cycle is
created, we simply revert to the previous acyclic version.  This process is
deterministic and ensures the order of resource sharing.

\subsubsection{Resource counting}

Finally, after applying the resource sharing transformations outlined
above, we count the number of LUTs and DSPs, by accumulating the statistics
for each operator instance in the MIR, in the same way we described
in Chapter~\ref{chp:stropt}.  For this resource estimation, we gather
statistics of individual operators by synthesizing each arithmetic operator
(addition, subtraction, multiplication, division and comparison), for
single and double precision floating-point, using Altera's floating-point
megafunctions~\cite{altfp}.  Similarly, the resource usage of integer
arithmetic, conditional and fixpoint operators are synthesized and estimated
from multiplexers, and the substitution operator uses no resources because it
does not perform computations.
