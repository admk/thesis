\section{Intermediate Representations}
\label{bg:sec:intermediate}

In this section, we delve deeper into intermediate representations that are
designed for program optimization.

The structural optimization of general numerical programs is much more complex
than that of arithmetic expressions.  The reasons are two-fold.  First,
during program execution, variables are often updated with new values.  Our
optimization should therefore perform static analysis on the values of
variables, and use the result to optimize specifically for the trade-off
between accuracy and resource usage.  Second, it is much more difficult to
formally define program equivalence, and subsequently, to search efficiently
for optimized equivalent programs.  We proceed to explain further why this is
the case.

For instance, even without the introduction of branches and loops, the
following two programs in Figure~\ref{bg:fig:equiv_progs} are equivalent, but
syntactically, they are very different.
\begin{figure}[ht]
    \centering
    \subfloat[$P_1$]{%
        \shortstack[l]{%
            \texttt{x = x + 1;} \\
            \texttt{y = 2 * x;} \\
            \texttt{x = x + 3;}
        }
    } \qquad \qquad
    \subfloat[$P_2$]{%
        \shortstack[l]{%
            \texttt{y = x + 1;} \\
            \texttt{x = y;} \\
            \texttt{y = y * 2;} \\
            \texttt{x = x + 3;}
        }
    }
    \caption{%
        Two programs that are equivalent but syntactically different.
    }\label{bg:fig:equiv_progs}
\end{figure}

In fact, one can easily imagine infinitely many ways to rewrite numerical
programs, and often these equivalent programs have identical resource usage,
latency and accuracy characteristics.  In practice, it is desirable to
eliminate as much as possible the need for these syntactic rewrites that do not
affect our performance metrics, so that the search space of equivalent programs
is greatly reduced.

For this, we typically use an alternative representation of the original
program, which are called \emph{intermediate representations} (IRs).  These
representations include \emph{static} and \emph{dynamic single assignment
forms} (SSA, DSA)~\cite{rau92, cytron91}, and \emph{control and data flow
graphs} (CDFG)~\cite{gajski94}.  These IRs are however less suitable for
our work because they are all statement-based and do not identify as many
equivalent programs as MIRs do.  Dependence graphs~\cite{rau94}, on the other
hand, are designed for the purpose of capturing data-flow dependences in
polyhedral methods, but they generally do not preserve enough information for
us to reconstruct a program from the graph itself.

In software, \emph{equality saturation} is proposed in~\cite{tate09}.  It
creates a new intermediate representation, PEG, to encode the effects of
executing the program, and used to discover equivalent structures.  The
differences between PEG and what is required by us is that, instead of using
graphs with cycles in them to represent \whilelit~loops, we must use a
directed acyclic graph to encode programs and use a \emph{fixpoint} operator
to define recursive functions which can be used to represent loops.  Cycles in
graphs require reanalyzing a large proportion of the IR whenever a structural
modification is made, whereas a tree structure can be reasoned in a bottom-up
hierarchy.  This has significant implications on program transformation.
First, for the above reason, their approach does not use semantics to optimize
for numerical accuracy, while we want our method to reason about semantics
and utilize them to steer optimization.  Moreover, tree structure allows us
to easily support partial loop unrolling by simply extending the equivalence
relations while avoiding re-evaluation.  In contrast, like~\cite{martel09}
and~\cite{damouche15}, equality saturation is unable to perform partial loop
unrolling.
