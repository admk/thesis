\section{Abstract Interpretation}
\label{bg:sec:abstract_interpretation}

This section starts by introducing the basic concepts of program semantics
and the abstract interpretation framework, a static analysis technique.  This
is then extended to define a scalable analysis capturing accuracy.  Later in
Chapter~\ref{chp:progopt} we accommodate sequential statements, \iflit~branches
and \whilelit~loops in the accuracy analysis, and in Chapter~\ref{chp:latopt},
we further improve our analysis by supporting multi-dimensional arrays.


\subsection{Intervals}
\label{bg:sub:intervals}

We illustrate these concepts by putting the familiar idea of \emph{interval
arithmetic}~\cite{moore} in the framework of abstract interpretation. As
an illustration, consider the following expression and its DFG in
Figure~\ref{bg:fig:sample_tree}\@:
\begin{equation}
    (a + b) \times (a + b)
    \label{bg:eq:absint_sample}
\end{equation}
\begin{figure}[ht]
    \centering
    \includegraphics[scale=0.6]{sample_tree}
    \caption{The DFG for the sample expression.}\label{bg:fig:sample_tree}
\end{figure}

We may wish to ask: if initially $a$ and $b$ are real numbers in the range of
$[0.2, 0.3]$ and $[2, 3]$ respectively, what would be the outcome of evaluating
this expression with real arithmetic? A straightforward approach is simulation.
Evaluating the expression for a large quantity of inputs will produce a set
of possible outputs of the expression. However the simulation approach is
unsafe, since there are infinite number of real-valued inputs possible and it
is infeasible to simulate for all.

A better method might be to represent the possible values of $a$ and $b$ using
ranges. To compute the ranges of its output values, we could operate on ranges
rather than values (note that the superscript $\sharp$ denotes ranges). Assume
that $a^\sharp_{init} = [0.2, 0.3]$, $b^\sharp_{init} = [2, 3]$, which are the
input ranges of $a$ and $b$, and $\enter{l}$ where $l \in \{1, 2, 3, 4\}$ are
the intervals of the outputs of the boxes labelled with $l$ in the DFG\@. We
extract the data flow from the DFG to produce the following set of equations:
\begin{equation}
    \begin{aligned}
        \enter{1} &= a^\sharp_{init} \\
        \enter{2} &= b^\sharp_{init} \\
        \enter{3} &= \enter{1} + \enter{2} \\
        \enter{4} &= \enter{3} \times \enter{3}
    \end{aligned}
    \label{bg:eq:absint_sample_analysis}
\end{equation}
For the equations above to make sense, addition and multiplication need to be
defined on intervals. We may define the following interval operations:
\begin{equation}
    \begin{aligned}
        \interval{a}{b} + \interval{c}{d} &= \interval{a + c}{b + d} \\
        \interval{a}{b} - \interval{c}{d} &=  \interval{a - d}{b - c} \\
        \interval{a}{b} \times \interval{c}{d} &=
            \interval{\min(s)}{\max(s)} \\
        \text{where~} s &= \{ a \times c, a \times d, b \times c, b \times d \}
    \end{aligned}
    \label{bg:eq:interval_operations}
\end{equation}
The solution to the set of~\eqref{bg:eq:absint_sample_analysis} for $\enter{4}$
is $[4.84, 10.89]$, which represents a safe bound on the output at the end
of program execution. Note that in actual execution of the program, the
semantics represent the values of intermediate variables, which are real
values. In our case, a set of real values forms the set of all possible
values produced by our code. However computing this set precisely is not,
in general, a possible task. Instead, we use abstract interpretation based
on intervals, which gives the abstract semantics of this program. Here, we
have achieved a classical interval analysis by \emph{defining} the meaning of
addition and multiplication on abstract mathematical structures (in this case
intervals) which capture a safe approximation of the original semantics of the
program. Later in Section~\ref{so:sec:semantics} of Chapter~\ref{chp:stropt},
we further generalize the idea by defining the meaning of these operations on
more complex abstract structures which allow us to scalably reason about the
area of FPGA implementations and equivalent program structures.


\subsection{Accuracy Analysis}
\label{bg:sub:accuracy}

Because we optimize numerical programs in a way that may have significant
impact on accuracy, and one of our objectives is to minimize round-off error
in the process, it is necessary to perform accuracy analysis on optimized
candidates.

Since our numerical programs make use of floating-point arithmetic, we first
introduce the concepts of the floating-point representation~\cite{ieee754}. Any
values $v$ representable in floating-point with standard exponent offset can be
expressed with the format given by the following equation:
\begin{equation}
    v = s \times 2^{e + 2^{k - 1} - 1} \times 1.{m_1 m_2 m_3 \ldots m_p}
    \label{bg:eq:floating_point}
\end{equation}
In~\eqref{bg:eq:floating_point}, the bit $s$ is the sign bit, the $k$-bit
unsigned integer $e$ is known as the exponent bits, and the $p$-bits $m_1 m_2
m_3 \ldots m_p$ are the mantissa bits, here we use $1.{m_1 m_2 m_3 \ldots m_p}$
to indicate a fixed-point number represented in unsigned binary format.

Because of the finite characteristic of IEEE 754 floating-point format, it
is not always possible to represent exact values with it. Computations in
floating-point arithmetic often induces roundoff errors. Therefore, following
Martel~\cite{martel07}, we bound with ranges the values of floating-point
calculations, as well as their roundoff errors. Our accuracy analysis
determines the bounds of all possible outputs and their associated range
of roundoff errors for expressions. For example, assume that $a \in [0.2,
0.3]$, $b \in [2.3, 2.4]$, it is possible to derive that in single precision
floating-point computation with rounding to the nearest, ${(a + b)}^2 \in
[6.24999857, 7.29000187]$ and the error caused by this computation is bounded
by $[-1.60634534\times10^{-6}, 1.60634534\times10^{-6}]$.

We employ abstract error semantics for the calculation of errors described
in~\cite{ioualalen, martel07}. First we define the domain $\errorset
= \floatintervalset\times\intervalset$, where $\intervalset$ and
$\floatintervalset$ respectively represent the set of real intervals, and
the set of floating-point intervals (intervals exactly representable in
floating-point arithmetic). The value $(x^\sharp, \mu^\sharp) \in \errorset$
represents a safe bound on floating-point values and the accumulated error
represented as a range of real values. Then addition and multiplication can be
defined for the semantics as in~\eqref{bg:eq:error_semantics}:
\begin{equation}
    \begin{aligned}
        \left( x^\sharp_1, \mu^\sharp_1 \right) +
        \left( x^\sharp_2, \mu^\sharp_2 \right)
    &=  \left(
            \roundup{x^\sharp_1 + x^\sharp_2},
            \mu^\sharp_1 + \mu^\sharp_2 +
            \rounddown{x^\sharp_1 + x^\sharp_2}
        \right) \\
        \left( x^\sharp_1, \mu^\sharp_1 \right) -
        \left( x^\sharp_2, \mu^\sharp_2 \right)
    &=  \left(
            \roundup{x^\sharp_1 - x^\sharp_2},
            \mu^\sharp_1 - \mu^\sharp_2 +
            \rounddown{x^\sharp_1 - x^\sharp_2}
        \right) \\
        \left( x^\sharp_1, \mu^\sharp_1 \right) \times
        \left( x^\sharp_2, \mu^\sharp_2 \right)
    &=  \left(
            \roundup{x^\sharp_1 \times x^\sharp_2},
            x^\sharp_1 \times \mu^\sharp_2 + x^\sharp_2 \times \mu^\sharp_1 +
            \mu^\sharp_1 \times \mu^\sharp_2 +
            \rounddown{x^\sharp_1 \times x^\sharp_2}
        \right) \\
    &\qquad\qquad\qquad\qquad\qquad\qquad\text{~for~}
        \left( x^\sharp_1, \mu^\sharp_1 \right) \in \errorset,
        \left( x^\sharp_2, \mu^\sharp_2 \right) \in \errorset
    \end{aligned}
    \label{bg:eq:error_semantics}
\end{equation}

The addition, subtraction and multiplication of intervals follow the standard
rules of interval arithmetic defined earlier in~\eqref{bg:eq:interval_operations}.
In~\eqref{bg:eq:error_semantics}, the function $\rounddownop: \intervalset \to
\intervalset$ determines the range of roundoff error due to the floating-point
computation under one of the rounding modes $\circ \in \{ -\infty, \infty, 0,
\neg0, \sim \}$ which are round towards negative infinity, towards infinity,
towards zero, away from zero and towards nearest floating-point value
respectively. It is defined as:
\begin{equation}
    \begin{aligned}
        & \downarrow^\sharp_\circ([a, b]) = \left\{
            \begin{aligned}
                & \left[ -\frac{z}{2}, \frac{z}{2}\right]
                    & \quad \circ & \text{~is~}\sim \\
                & \left[ -z, z\right]
                    & \quad \circ & \in \{ -\infty, \infty, 0, \neg0 \}
            \end{aligned}
        \right. \\
        & \qquad\qquad\qquad\qquad \text{where~} z = \max(ulp(a), ulp(b))
    \end{aligned}
\end{equation}
Here $z$ denotes the maximum rounding error that can occur for values
within the range $[a, b]$, and the unit of the last place (ulp) function
$ulp(x)$~\cite{muller} characterizes the distance between two adjacent
floating-point values $f_1$ and $f_2$ satisfying $f_1 \leq x \leq
f_2$~\cite{goldberg}. In our analysis, the function $ulp$ is defined as:
\begin{equation}
    ulp(x) = 2^{e(x) + 2^{k - 1} - 1} \times 2^{-p}
\end{equation}
where $e(x)$ is the exponent of $x$, $k$ and $p$ are the parameters of the
floating-point format as defined in~\eqref{bg:eq:floating_point}. The function
$\roundupop: \intervalset \to \floatintervalset$ computes the floating-point
bound from a real bound, by rounding the infimum $a$ and supremum $b$ of the
input interval $[a, b]$.

Expressions can be evaluated for their accuracy by the method as follows.
Initially the expression is parsed into a data flow graph (DFG). By way of
illustration, the sample expression ${(a + b)}^2$ has the tree structure
in Figure~\ref{bg:fig:sample_tree}. Then the exact ranges of values of $a$ and
$b$ are converted into the abstract semantics using a cast operation as in
\eqref{bg:eq:cast}:
\begin{equation}
    cast(x^\sharp) = (\roundup{x^\sharp}, \rounddown{x^\sharp})
    \label{bg:eq:cast}
\end{equation}
For example, for the variable $a \in [0.2, 0.3]$ under single precision
with rounding to nearest,
\begin{equation}
    cast([0.2, 0.3]) = ([0.200000003, 0.300000012], [-1/67108864, 1/67108864])
\end{equation}
After this, the propagation of bounds in the data flow graph is carried out as
described in Section~\ref{bg:sub:intervals}, where the difference is the abstract
error semantics defined in~\eqref{bg:eq:error_semantics} is used in lieu of the
interval semantics. At the root of the tree (\ie~the exit of the DFG) we find
the value of the accuracy analysis result for the expression.
