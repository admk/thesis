\section{High-Level Synthesis}
\label{bg:sec:high_level_synthesis}

\subsection{Overview}
\label{bg:sub:hls_overview}

High-level synthesis (HLS) is the process of compiling a high-level
representation of an application (usually in C, C++ or MATLAB) into
register-transfer-level (RTL) implementation for digital circuit
designs~\cite{coussy, gajski}.  HLS tools enable us to work in a high-level
language, as opposed to facing labor-intensive tasks such as optimizing timing,
and designing control logic in the RTL implementation. This allows application
designers to focus instead on the algorithmic and functional aspects of their
implementation~\cite{coussy}, without concerning themselves with the above
intricate details of manual RTL designs.

Another advantage of using HLS tools is that they are in general more
productive and less error-prone to work with, when compared with traditional
RTL tools.  The reasons are two-fold.  First, a C description is smaller than a
traditional RTL description by a factor of 10~\cite{coussy, bdti}.  Second, C
code can be easily tested on an ordinary microprocessor, debug and verification
tools for C are relatively more abundant when compared to RTL counterparts.

HLS tools benefit us in their ability to automatically search the design
space with a reasonable design cost~\cite{bdti}, explore a large number of
trade-offs between performance, cost and power~\cite{mcfarland}, which is
generally much more difficult to achieve in RTL tools.  Our thesis proposes
a natural extension to HLS tools by automatically exploring the space of
rewrites of floating-point numerical C programs, which are equivalent in real
arithmetic, but could trade off accuracy, throughput and resource utilization
when synthesized into circuits.

With recent advancements in this area, HLS tools has received a resurgence of
interest, particularly in the FPGA community, and circuits synthesized with
HLS tools are now with similar performance when compared with hand-crafted RTL
implementations~\cite{bdti}.  Xilinx now incorporates a sophisticated HLS flow
into its Vivado design suite~\cite{vivado_hls} and the open-source HLS tool,
LegUp~\cite{legup}, is gaining significant traction in the research community.


\subsection{Compilation Stages}
\label{bg:sub:compilation_stages}


