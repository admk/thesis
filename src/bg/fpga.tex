\section{Field-Programmable Gate Arrays}
\label{bg:sec:fpga}

\subsection{Computing Architectures}
\label{bg:sub:computing_architectures}

When it comes to implementing computations, we often choose from a spectrum of
computing machines.  These choices range from those with fixed architectures
that compute by executing \emph{software} designs such as \glspl{cpu} and
\glspl{gpu}, those that implement custom \emph{hardware} architectures,
such as \glspl{fpga}, to those with custom integrated circuits to carry out
computations, \ie~\glspl{asic}.

There has been a great amount of effort in recent decades to make
fixed-architecture machines run as fast as possible; many novel and intricate
ideas were proposed and we now have a variety of general circuits in
\glspl{cpu} to improve their performance~\cite{comparch}.  For instance,
\glspl{cpu} could have a pipeline that spans several clock cycles to fetch and
decode instructions, and access data from memory or registers to carry out
computations.  At the same time, they could make predictions about the branches
taken in control-flows, as the pipeline must be flushed if an incorrect
instruction is fetched, incurring a penalty in speed.  Superscalar architecture
and out-of-order execution are used to increase instruction-level parallelism.
They may also exploit data-level and thread-level parallelism, in order to
maximize opportunities to parallelize computations.  These are just the tip of
the iceberg, as many other architectural advancements exist.

The great majority of fixed-architecture computing machines are based on the
\emph{von Neumann architecture}, which consists of three parts: a computation
unit, a memory and a bus between them to move data back and forth.  Often
applications running on these machines could spend a majority of their time
and energy to move data and instructions in the memory from/to the right
location in the processor as fast as possible, in order to carry out arithmetic
computations.  Because computational tasks frequently reuse input data and
intermediate results, a hierarchy of caches, in tandem with cache-aware
compiler optimizations~\cite{kowarschik03}, are often used to mitigate the
costs of exchanging data between the processor and the memory.  Despite
these optimization efforts to run software code as fast as possible, the
processor-memory bus, which is often referred to as the \emph{von Neumann
bottleneck}~\cite{backus78}, inherently exists and remains the limiting factor
of performance in the architecture.  This phenomenon is known to many as
\emph{hitting the memory wall}~\cite{bacon13, wulf94}.

Custom architectures in general achieve much higher performance, thanks to
their ability to implement arbitrary digital circuits specifically designed
for the application under consideration, and spatially distribute memory
bandwidth and computations.  This is in stark contrast to microprocessors
such as \glspl{cpu} and \glspl{gpu}, which utilize general-purpose circuits
to cope with a wide-range of applications. \glspl{asic} provide the best
power efficiency and performance among all above architectures, however
they are often associated with long development cycles and high costs; any
updates to the design would require a complete and expensive re-spinning
of the circuits~\cite{bacon13}, as they are inherently non-programmable.
\Glspl{fpga} provide a good trade-off between processors and \glspl{asic}.
Not only do \glspl{fpga} have better performance and power characteristics
than fixed architectures, they also offer high programmability which makes
\glspl{fpga} cost-effective low-volume \gls{asic} replacements~\cite{karen04,
bacon13}.  At the same time, with a much shorter development period than
\glspl{asic}, a hardware design on \glspl{fpga} can be implemented with a much
lower cost.  With a shorter time to market, it further enables a substantially
larger profit by a competitively early market entry~\cite{semico12}.

For the above reasons, being able to leverage parallelism from bit-level all
the way to the loop- and task-level, \glspl{fpga} have been increasingly used
as high-performance and low-power alternatives to \glspl{cpu} and \glspl{gpu}
for many classes of applications~\cite{bacon13, brodtkorb10, sirowy08}.  For
example, Thomas~\etal~\cite{thomas09} reported a \gls{fpga}-based random number
generator can obtain a $260\times$ speed-up, while costing less than 1\% of
energy to produce each random sample, when compared to its software counterpart
running on a \gls{cpu}\@.  Microsoft initiated a mid-scale deployment of
Stratix V \glspl{fpga} in their data center, improving the throughput of their
Bing web search engine by a factor of 95\%~\cite{catapult}.


\subsection{FPGA Architecture}
\label{bg:sub:fpga_architecture}

\Glspl{fpga} owe their high performance and power efficiency to the design
of the architecture, we thus use Altera Stratix V~\cite{stratix5} as an
example to explain the architecture.  The Stratix V fabric contains a
two-dimensional array of \glspl{lab}.  Each \gls{lab} in turn consists of an
array of 10 \glspl{alm}.  Figure~\ref{bg:fig:alm} shows a high-level block
diagram of an \gls{alm} in Stratix V.  In an \gls{alm}, multiplexers can be
configured to choose whether full adders and registers are used.  Dedicated
full adders enable more complex Boolean functions to be implemented in a
single \gls{alm}, whereas the use of registers, which store intermediate
values, determines whether the circuit is combinational or sequential.  The
two \glspl{lut} in an \gls{alm} can be configured to compute a combination
of two arbitrary Boolean functions, each with up to 5 inputs from 8 inputs
in total.  Stratix 10, slated to be released in the next couple of years,
has up to 1.87 million \glspl{alm} and 7.47 registers in total for the most
demanding applications~\cite{stratix10stat}.  Interconnects, another class of
key configurable resources on \glspl{fpga}, enable the inputs and outputs of
\glspl{alm} to be wired, in order to form larger and complete circuits from the
two-dimensional array of \glspl{alm}.
\begin{figure}[ht]
    \centering
    \singlespacing%
    \begin{tikzpicture}[line width=0.8pt, node distance=2em]
        \node (alut) [rect, minimum height=60mm] at (0, 0) {Adaptive \\ LUT};
        \matrix (in)
        [matrix of nodes, text height=2.5mm, row sep=2mm, left=of alut] {
            1 \\ 2 \\ 3 \\ 4 \\ 5 \\ 6 \\ 7 \\ 8 \\
        };
        \foreach \i in {1,...,8} {
          \draw[->] (in-\i-1) -- +(11.5mm, 0);
        }

        \node (fa1) [rect, right=5mm of alut, yshift=10mm, minimum width=5em]
            {Full \\ Adder};
        \node (fa1l) [coordinate, left=5mm of fa1] {};
        \node (fa1a) [coordinate, above=20mm of fa1] {};
        \node (fa2) [rect, right=5mm of alut, yshift=-10mm, minimum width=5em]
            {Full \\ Adder};
        \node (fa2l) [coordinate, left=5mm of fa2] {};
        \node (fa2b) [coordinate, below=20mm of fa2] {};
        \draw[->] (fa1l) -- (fa1);
        \draw[->] (fa2l) -- (fa2);
        \draw[->] (fa1a) -- (fa1);
        \draw[->] (fa1) -- (fa2);
        \draw[->] (fa2) -- (fa2b);

        \path (alut.70) -- +(35mm, 0) node (mux1) [mux, rotate=-90] {};
        \node (mux2) [mux, below=12mm of mux1.north, rotate=-90] {};
        \node (mux3) [mux, below=12mm of mux2.north, rotate=-90] {};
        \node (mux4) [mux, below=12mm of mux3.north, rotate=-90] {};
        \draw[<-] (mux1.south west)
            -- +(-5mm, 0) node (mid1) [coordinate] {} -- +(-33mm, 0);
        \draw[->] (mid1) |- (mux2.-30);
        \draw[->] (fa1.19) -- node (mid2) [coordinate, pos=0.6]{} (mux2.-120);
        \draw[->] (mid2) |- (mux1.-40);
        \draw[<-] (mux3.south west)
            -- +(-5mm, 0) node (mid3) [coordinate] {} -- +(-33mm, 0);
        \draw[->] (mid3) |- (mux4.-30);
        \draw[->] (fa2.-4) -- node (mid4) [coordinate, pos=0.6]{} (mux4.-120);
        \draw[->] (mid4) |- (mux3.-40);

        \foreach \i in {1,...,4} {
            \node (reg\i) [rect, right=of mux\i, yshift=4.5mm] {Register};
            \draw[->] (mux\i) -- node (rmid\i) [coordinate, pos=0.6]{} (reg\i);
            \draw[->] (reg\i) -- +(20mm, 0);
            \draw[->] (rmid\i) |- +(30.5mm, -5mm);
        }
    \end{tikzpicture}
    \caption{%
        A high-level block diagram of an \gls{alm} in Stratix V, from Stratix V
        Device Handbook~\cite{stratix5}.
    }\label{bg:fig:alm}
\end{figure}

% distributed memory and dsp elements

\Glspl{fpga} with enough \glspl{alm} and interconnects can implement arbitrary
digital designs.  This versatile architecture therefore overcomes the memory
wall problem by not restricting itself to the von Neumann architecture.  As
we have mentioned earlier, \glspl{fpga} can implement a circuit that is
individually tailored for the application, in contrast, \glspl{cpu} have
general-purpose circuits designed for a wide range of applications, which may
therefore have lower power-efficiency and performance.  Moreover, unlike the
\gls{cpu} which only has a small set of registers, the \gls{fpga} with its
flexibility and abundant registers, allows designs to distribute memory blocks
and computation units and place them in close proximity.

Traditionally, multipliers, when implemented as soft-logic in \glspl{fpga},
cost a large number of \glspl{alm}.  Stratix devices thus further include an
array of hardened components to carry out arithmetic operations distributed
on the \gls{fpga} fabric, known as \glspl{dsp} blocks, or simply \glspl{dsp}.
Because of the dedicated hardened circuits, \glspl{dsp} compute faster
than arithmetic operators formed by \glspl{alm} only, meanwhile they free
\gls{alm} resources to perform more non-arithmetic computations.  In Stratix
V, each variable-precision \gls{dsp} is paired with a \gls{lab}\@.  These
\glspl{dsp}, can be configured in combinations to perform a wide variety
of arithmetic operations, ranging from those using a single \gls{dsp}
element to synthesize three multipliers each with two 9-bit inputs, up
to those combining four \glspl{dsp} to form a complex-number multiplier
with two 27-bit inputs~\cite{stratix5}.  Computations with larger inputs
can also be implemented by using \glspl{alm} and \glspl{dsp} to form
larger arithmetic circuits.  Finally, Stratix 10 will introduce hardened
floating-point \glspl{dsp}, enabling IEEE 754~\cite{ieee754} single-precision
floating-point additions and multiplications, achieving a performance of up
to 10 \glspl{tflops}~\cite{stratix10fp}.  These \gls{dsp} blocks can also be
adapted to multiply fixed-point inputs.

\Glspl{dsp} accelerate arithmetic computations, however they need to be
supplied with inputs as fast as they can process to fully utilize them.
In general, in most applications, data are frequently reused by the same
computation unit.  Stratix V therefore includes dedicated embedded memory
called M20K blocks (20 Kb storage) to be arranged and combined into dual-port
\glspl{ram}.  Half of the \glspl{lab} on the device, called \glspl{mlab} can
also be configured to become a 640-bit \glspl{ram}.  These memory blocks are
distributed across the \gls{fpga} fabric, so that \glspl{dsp} may find them in
proximity.


\subsection{RTL Design Flow}
\label{bg:sub:rtl_design}

Modern \glspl{fpga}---with up to several million \glspl{lut}, and thousands
of embedded memory and \gls{dsp} blocks, wired through a programmable fabric
of interconnects---are humanly intractable to program at the granularity of
these individual components~\cite{kapre08}.  \Gls{fpga} applications are
thus commonly written in \gls{rtl} \gls{hdl}, such as Verilog~\cite{verilog}
and VHDL~\cite{vhdl}.  These \gls{hdl} source programs implement the desired
hardware by describing the logic between registers.  \Gls{eda} tools can
then automatically translate these descriptions into hardware circuits in
\glspl{fpga}.

\Gls{eda} tools, go through several stages to synthesize \gls{hdl} source
code into circuits, To explain these stages in depth, we take Altera Quartus
II~\cite{quartus} as an example design flow, shown in Figure~\ref{fig:quartus}.
\begin{figure}[ht]
    \centering
    \singlespacing%
    \begin{tikzpicture}[line width=0.8pt, node distance=1.2em]
        \node (synth) [block] at (0, 0) {Synthesis};
        \node (pnr) [block, right=of synth, align=center] {Place \& \\ Route};
        \node (ta) [block, right=of pnr, align=center] {Timing \\ Analysis};
        \node (sim) [block, right=of ta] {Simulation};
        \node (prog) [block, right=of sim] {Programming};
        \draw[->] (synth) -- (pnr);
        \draw[->] (pnr) -- (ta);
        \draw[->] (ta) -- (sim);
        \draw[->] (sim) -- (prog);
    \end{tikzpicture}
    % \includegraphics[width=\textwidth]{bg/fig/quartus}
    \caption{Quartus II design flow.}\label{fig:quartus}
\end{figure}

Quartus II starts its compilation of the \gls{rtl} program by verifying
source code for syntax and semantic errors and design specification for
inconsistencies, then applies a methodology, called \emph{technology mapping},
which maps a graph of device-independent logic gates in logic expressions onto
a network of functional blocks (such as \glspl{lut}, \glspl{dsp} and memory
blocks) in the target \gls{fpga} device~\cite{cong08}; this generated network
is known as a \emph{technology-mapped netlist}.  In this process, synthesis
tools may optimize the circuit by performing additional transformations such as
redundant logic removal~\cite{quartus}.

The following stage, \emph{place \& route}, utilizes a heuristic placement
algorithm, which takes as it inputs the netlist, together with a device map
showing the location of each of its functional units, in order to select a
legal location on the \gls{fpga} for each functional block in the netlist,
such that the routing of these blocks is optimized~\cite{betz08}.  In general,
synthesis tools allow some freedom in the user's preference of the placement
of circuit.  Additional automated optimizations may be applied to improve
performance.  For example, Quartus II has the option to enable \emph{register
retiming}~\cite{quartus}, which allows registers to move across combinational
logic to reduce \emph{critical path} delay, \ie~the longest delay required
for an output of any source register to propagate to the input of any target
register in the circuit.  The end result of this step is a circuit fully mapped
on the target \gls{fpga}\@.

In the following step, \emph{timing analysis}, the tool computes the longest
delay of all critical paths, which determines the maximum frequency at which
the application can run.  Users can also inspect the list of critical paths and
their delay statistics, so that one may focus their effort on optimizing the
timing of these critical paths by, for instance, splitting them up by adding
registers.

In the \emph{simulation} stage, the resulting design is simulated using
\gls{eda} simulation tools.

The final step, \emph{programming} is to translate the circuit generated by the
tool into a \emph{bitstream}, which is a binary data file used to program the
\gls{fpga}\@.  Similar to processors, which can be programmed by incrementally
reading and executing instructions from an executable program file, a bitstream
is used to program the individual components such as \glspl{lut}, \gls{dsp}
blocks, dedicated memory blocks and interconnects on an \gls{fpga}, so the
circuit is formed on the device.  The difference between them is that while
processors continuously read instructions from memory, \glspl{fpga} are
typically programed only once during the initial setup, and the bitstream
data are used spatially to infer a circuit rather than sequentially as
instructions~\cite{guccione08}.
