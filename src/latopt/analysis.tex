\section{Performance Analysis}
\label{sec:performance_analysis}

This section explains how we analyze MIRs for our three performance metrics:
latency, resource usage, and accuracy. %  Later we use these analyses
% to evaluate equivalent numerical programs, and also guide our optimization
% procedure.

\subsection{Latency Analysis}
\label{sub:latency_analysis}

The purpose of our latency analysis is not to create a complete scheduling
of numerical programs, as this would be computationally expensive, and
would need to be repeated for tens of thousands of equivalent programs.
Instead, our latency analysis computes a lower bound of the loop's initiation
interval ($\II$), \ie~the minimum initiation interval ($\MII$).  (Recall from
Sec.~\ref{sec:introduction} that the initiation interval is the number of clock
cycles that must elapse between the starts of two consecutive loop iterations,
and is determined by data dependences and resource constraints.)  We then
compute the overall latency of the loop, and subsequently, the total latency of
the program.

Following LegUp~\cite{legup} and Vivado HLS~\cite{vivado_hls}, we compute
$\MII$ values using \emph{iterative modulo scheduling}~\cite{rau94}. For our
work, we have adapted this analysis to apply directly to MIRs. The structure of
MIRs already captures intra-iteration data dependences; to this, we add extra
latency information as attributes on the edges of MIRs, plus new edges to form
cycles that capture inter-iteration data dependences.  The analysis is carried
out in three stages.

The analysis starts with the MIR of the loop under analysis. Each edge in the
MIR, say $s\rightarrow t$, represents a data dependence: the operation at
node $s$ must be evaluated fully before the operation at $t$ can begin. The
first step is to add a 2-tuple $\langle l, d \rangle$ to each edge of the
MIR\@. Here, $l$ is the \emph{latency} of the edge (the number of \emph{clock
cycles} that must elapse between the start of $s$ and the start of $t$) and
$d$ is the \emph{dependence distance} (the number of \emph{loop iterations}
that must elapse between the start of $s$ and the start of $t$). Because all
operations in the MIR are performed in a single iteration, all edges have
$d=0$. The value of $l$ is given by the latency of the operation at node $s$;
if $s$ corresponds to an input variable or a numerical constant, then $l=0$.

The second stage is to add edges to form a cyclic dependence graph that
captures \emph{read after write} (RAW) dependences across loop iterations.
This step involves checking whether each pair of ``$\accessop$'' and
``$\updateop$'' nodes has a dependence, and if so, adding a new edge between
them with latency and dependence distance attributes.  As an example, consider
the MIR in~\eqref{eq:array_example} and assume each iteration increments
\verb|i| by 1.  Because in the original program, \verb|A[i]| and \verb|A[i+1]|
are respectively reading from and writing to the same array \verb|A|, we need
to check if these accesses could touch the same memory location in different
iterations.  For this, our analysis formulates an integer linear programming
problem for the dependence distance, and solves it using the integer set
library (isl)~\cite{isl}.  In this example, the dependence distance is 1
because the value written to \verb|A[i+1]| in the current iteration \verb|i|
is immediately used in the next iteration \verb|i+1|.  Similarly, we also add
new edges for reads and writes to the same variable, which can be treated as a
special array with only one element. Our analysis yields the following MIR\@:
\begin{equation}
\newcommand\pair[2]{\footnotesize $#1,#2$}
    \label{eq:array_example_latency}
    \begin{tikzpicture}[mir, inner sep=0pt]
        \node[mirnode] (var_A) at (0,0) {{\tt A}};
        \node[mirnode] (update)[right=of var_A, xshift=5mm] {$\updateop$};
        \node[mirnode] (A1)    [below left=of update, xshift=-1mm, yshift=-2mm] {{\tt A}};
        \node[mirnode] (plus)  [below=of update, yshift=-3mm] {$+$};
        \node[mirnode] (i1)    [below left=of plus] {{\tt i}};
        \node[mirnode] (n1)    [below right=of plus] {$1$};
        \node[mirnode] (times) [below right=of update, xshift=11mm] {$\times$};
        \node[mirnode] (n2)    [below left=of times, xshift=-1mm, yshift=-1mm] {$2$};
        \node[mirnode] (access)[below right=of times] {$\accessop$};
        \node[mirnode] (A2)    [below left=of access] {{\tt A}};
        \node[mirnode] (i2)    [below right=of access] {{\tt i}};

        \draw[|->] (var_A) -- (update);
        \draw[<-] (update) to[auto, swap, pos=0.8]
        node{\pair00} (A1);
        \draw[<-] (update) to[auto]
        node{\pair{10}0} (plus);
        \draw[<-] (update) to[auto]
        node{\pair70} (times);
        \draw[<-] (plus) to[auto, swap]
        node{\pair00} (i1);
        \draw[<-] (plus) to[auto]
        node{\pair00} (n1);
        \draw[<-] (times) to[auto, swap]
        node{\pair00} (n2);
        \draw[<-] (times) to[auto]
        node{\pair20} (access);
        \draw[<-] (access) to[auto, swap]
        node{\pair00} (A2);
        \draw[<-] (access) to[auto]
        node{\pair00} (i2);
        \brackets{(current bounding box)}
\begin{pgfinterruptboundingbox}
        \draw[<-,dashed] (access) to[auto, swap, pos=0.2, out=45, in=30]
        node{\pair{-2}1} (update);
\end{pgfinterruptboundingbox}
    \end{tikzpicture}
\end{equation}

Note the new dashed edge from the $\updateop$ node to the $\accessop$ node,
which is labeled $\langle -2, 1 \rangle$.  The first value, $-2$, signifies
that the latency of the edge between $\times$ and $\accessop$, which is 2
cycles, is canceled out because the multiplier can reuse its output from the
previous iteration as the input for the current iteration. The second value,
$1$, indicates that there is a data flow dependence from iteration $i$ to
iteration $i+1$.

We assume no limit on the number of operators we can allocate, so operators
do not constraint $\II$.  However, in Vivado HLS, each array is usually
translated into a dual-port RAM, which allows only two accesses per clock
cycle~\cite{vivado_hls}, and thus constrains $\MII$.  For instance, for a
loop to perform 3 accesses to a single array in each iteration, $\II$ must be
greater than 1.  This lower bound on $\II$ is known as \emph{resource-based
minimum initiation interval}, $\ResMII$~\cite{rau94}. It is defined as $\max_A
\lceil {n_A}/{r_A} \rceil$, where $A$ ranges over all arrays in the loop body,
$n_A$ is the number of accesses to the array $A$, and $r_A$ is the maximum
number of accesses allowed per cycle, which is $2$ in our case.

The final step is to calculate an integer $\RecMII$ which is defined as
$\max_c \lceil {l_c}/{d_c} \rceil$, where $c$ ranges over all cycles in
the graph, and $l_c$ and $d_c$ are respectively the sums of all latencies
and dependence distances of the edges in the cycle. This value is known
as the \emph{recurrence-based minimum initiation interval}~\cite{rau94}.
Because a typical MIR with array accesses could have a very large number of
cycles, we efficiently search for an $\MII$ using a modified Floyd--Warshall
algorithm~\cite{rau94}.  Finally, we estimate the total latency $L$ of the loop
with:
\begin{equation}
    L = (N - 1) \MII + D,
    \text{~where~}
        \MII = \max\left(\RecMII, \ResMII\right)
    \nonumber
\end{equation}
where, recalling from Sec.~\ref{sec:introduction}, $N$ is the maximum
\emph{trip count}, \ie~the loop's total number of iterations, and $D$ is the
loop's depth, \ie~the total number of cycles per iteration.

Because we optimize programs in a bottom-up hierarchy, as described in
Sec.~\ref{sec:structural_optimization}, when an expression in a loop is
optimized, its latency is estimated by scheduling its operations by using an
As-Late-As-Possible (ALAP)~\cite{wang_hls} scheduling algorithm, where each
operation is scheduled to the latest opportunity, while respecting the order of
data dependences.  Because the expression is eventually used in a loop, and the
$\II$ of the loop is critical to how fast the loop can execute, it is necessary
to start optimizing for $\II$ as soon as possible. Therefore, in our latency
analysis of an MIR that is a fragment of a loop, our algorithm automatically
shortens any paths between any pair of dependent accesses in the MIR\@, as we
use the latency analysis as a component to manoeuvre our optimization on the
Pareto frontier.  Moreover, we place greater weights on dependent accesses with
smaller dependence distances, because these impact the resulting loop $\II$
more significantly than larger distances.  We use the following formula as the
analyzed latency value to guide the optimization for $\II$ for subexpressions
in a loop, where $\mathrm{Deps}$ is a set of 2-tuple attributes of all
inter-iteration dependences:
\begin{equation}
    L = \max_{\langle l, d \rangle \in \mathrm{Deps}} \frac{l}{d}
\end{equation}

\subsection{Resource Utilization Analysis}
\label{sub:resource_utilization_analysis}

The hardware resource usage analysis of Gao~\etal~\cite{soap2} captures the
sharing of common subexpressions, but cannot analyze resource binding, which
allows common operations to be shared across clock cycles. For instance, in
the floating-point expression $a + (b + c)$, the two additions can be computed
using one addition operator only. In this paper, we develop a new resource
usage analysis that fully understands how resources are shared in an FPGA
implementation of numerical programs.

We rely on the foundation of \SOAP{}, which counts the number $n_\opsymbol$
of each type of operation $\opsymbol$, while maximally sharing common
subexpressions.  In a pipelined loop, we compute a lower bound $a_\opsymbol$
on the number of instances of $\opsymbol$ that must be allocated, using
the equation $a_\opsymbol = \lceil n_\opsymbol/\MII\rceil$.  For instance,
if we know that a pipelined loop has $\MII = 3$, and each iteration uses 6
multiplications, then we can compute that we need to synthesize at least 2
multipliers.  Integer operators are typically not shared~\cite{cong15}, so the
number of operations is the number of allocated instances.

For straight-line code, non-pipelined loops, and different loops, we use a
simple ALAP scheduling~\cite{wang_hls} to estimate resource utilization.

Finally, we accumulate the number of LUTs and DSP elements for all allocated
operators, which is the estimated resource utilization for the full program.
% \todo{JW: Still need to reconcile this with the fact that the SOAP tool
% only seems to output the LUT count.}


\subsection{Accuracy Analysis}
\label{sub:accuracy_analysis}

We extend the accuracy analysis of Gao~\etal~\cite{soap2} to support
arrays. Because our benchmark suite consists of programs with large arrays,
we keep the analysis efficient by treating an \emph{entire} array as a pair
of a floating-point interval and an interval of accumulated round-off errors.
These intervals accumulate all values that are assigned to the array, and
never shrink the range bounded by these intervals when we assign new values to
an array location.  Additionally, because most of the loops in our benchmark
programs consist of nested loops and have large iterations, we modified the
analysis routine in \SOAP{} to analyze only a small fraction of loop execution,
and use our dependence analysis to detect whether errors are accumulated
across iterations, in order to extrapolate the total round-off errors from the
results.

% Because the accuracy analysis not our contribution here, please refer to
% Gao~\etal~\cite{soap2} for a detailed explanation of their accuracy
% analysis.

%%% Local Variables:
%%% mode: latex
%%% TeX-master: "paper"
%%% End:
