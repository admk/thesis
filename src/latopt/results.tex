\section{Evaluation}
\label{lo:sec:results}

\newcommand\redcircle{%
\begin{tikzpicture}\draw[red] (0,0) circle (1mm);\end{tikzpicture}}


\subsection{Method}
\label{lo:sub:method}

We have evaluated \soap~on a suite of benchmark examples which consists of
several applications that have recurring inter-iteration dependences:
\begin{itemize}

\item A simple loop, \verb|sum|, that sums the elements in an array.

\item Two kernels from Livermore Loops~\cite{livermore}: \verb|dotprod|, which
computes the dot product of two vectors, and \verb|tridiag|, which solves a
tridiagonal linear system of equations.

\item Nine kernels from PolyBench~\cite{polybench}, which calculate
matrix/vector transpositions, additions and multiplications (\verb|2mm|,
\verb|3mm|, \verb|atax|, \verb|gemm|, \verb|gemver|, \verb|mvt|), the
bi-conjugate gradient stabilized method (\verb|bicg|), the Seidel stencil
computation (\verb|seidel|), and symmetric rank-2k operations
(\verb|syr2k|).

\end{itemize}

All elements of input arrays and matrices are set to be single-precision
floating-point values between $0$ and $1$.  We optimized all of these
benchmark examples using \soap, specifically targeting the Xilinx Virtex7
device running at $333~\mathrm{MHz}$, for the three objectives of accuracy,
resource utilization and latency simultaneously.  We then used \gls{vhls}
2015.2~\cite{vivado_hls} to synthesize the resulting optimized programs
into \gls{rtl} implementations for exact latency information, and performed
place-and-route using Vivado Design Suite 2015.2~\cite{vivado_ds}, to
obtain exact resource utilization statistics.  Finally, \soap~produces a
four-dimensional Pareto frontier for each program, where the dimensions are
accuracy, latency, the number of \glspl{lut} and \gls{dsp} element count, for
clarity we visualize the results in three dimensions only.

\subsection{Results}
\label{lo:sub:results}

Table~\ref{lo:tab:results} compares, for each benchmark in our evaluation
set, the performance metrics of the original program against those of the
program with the smallest latency discovered by \soap.  We synthesized
each program to a circuit to obtain exact statistics, which are shown in
Table~\ref{lo:tab:results}.
\begin{table}[t]
    \centering
    % \renewcommand\arraycolsep{0.45mm}
    \newcommand\unitk{\,\text{k}}
    \newcommand\unitm{\,\text{m}}
    \newcommand\unitmu{\,\text{\textmu}}
    \newcommand\unitM{\,\text{M}}
    \newcommand\unitG{\,\text{G}}
    \newcommand\Mid[1]{\multirow{2}*{$#1$}}
    \newcommand\Shade{\cellcolor{black!10}}
    \newcommand\name[1]{\texttt{#1}}
    \caption{%
        Comparisons of the original (non-shaded rows) and the optimized
        program with lowest latency (shaded rows), for each benchmark. Values
        in parentheses are obtained after slightly tweaking our experimental
        set-up; see Section~\ref{lo:sub:discussion}.  We performed
        place-and-route for exact statistics.
    }\label{lo:tab:results}
    $\begin{array}{l||r|r|r|r|r|r|r|r|r}
        \hline
        \multicolumn{1}{c||}{\text{Name}} &
        \multicolumn{1}{c|}{\text{\glspl{dsp}}} &
        \multicolumn{2}{c|}{\text{\glspl{lut}}} &
        \multicolumn{2}{c|}{\text{Error}} &
        \multicolumn{1}{c|}{\text{Clock}} &
        \multicolumn{3}{c}{\text{Latency}}
        \\
        & & &
        \multicolumn{1}{c|}{\text{ratio}} & &
        \multicolumn{1}{c|}{\text{ratio}} &
        \multicolumn{1}{c|}{\text{(ns)}} &
        \multicolumn{1}{c|}{\text{(cycles)}} &
        \multicolumn{1}{c|}{\text{(s)}} &
        \multicolumn{1}{c}{\text{ratio}}
        \\ \hline\hline
        \Mid{\name{sum}} &
        2 & 303 & \Mid{0.257} & 914\unitmu & \Mid{7.93} &
        2.54 & 41.0\unitk & 104\unitmu & \Mid{12.8} \\ &
        \Shade 4 & \Shade 1181 & & \Shade 1.15\unitmu & &
        \Shade 2.54 & \Shade 3.21\unitk & \Shade 8.17\unitmu &
        \\ \hline
        \Mid{\name{dotprod}} &
        5 & 411 & \Mid{0.231} & 926\unitmu & \Mid{7.29} &
        2.54 & 41.0\unitk & 104\unitmu & \Mid{12.4} \\ &
        \Shade 10 & \Shade 1781 & & \Shade 127\unitmu & &
        \Shade 2.62 & \Shade 3.23\unitk & \Shade 8.44\unitmu &
        \\ \hline
        \Mid{\name{tridiag}} &
        5 & 470 & \Mid{0.288} & 63.1\unitmu & \Mid{1.06} &
        2.54 & 17.8\unitM & 45.3\unitm & \Mid{3.41} \\ &
        \Shade 8 & \Shade 1631 & & \Shade 59.4\unitmu & &
        \Shade 2.69 & \Shade 4.93\unitM & \Shade 13.3\unitm &
        \\ \hline
        \Mid{\name{2mm}} &
        5 & 781 & \Mid{0.385} & 209 & \Mid{3.40} &
        2.79 & 20.4\unitG & 57.0 & \Mid{7.46} \\ &
        \Shade 8 & \Shade 2029 & & \Shade 61.4 & &
        \Shade 2.92 & \Shade 2.62\unitG & \Shade 7.64 &
        \\ \hline
        \Mid{\name{3mm}} &
        5 & 760 & \Mid{0.207} & 114 & \Mid{6.76} &
        2.55 & 32.3\unitG & 82.3 & \Mid{9.13} \\ &
        \Shade 10 & \Shade 3677 & & \Shade 16.9 & &
        \Shade 2.82 & \Shade 3.19\unitG & \Shade 9.01 &
        \\ \hline
        \Mid{\name{atax}} &
        5 & 627 & \Mid{0.507} & 353\unitm & \Mid{1.54} &
        2.60 & 176\unitM & 457\unitm & \Mid{5.42} \\ &
        \Shade 5 & \Shade 1237 & & \Shade 230\unitm & &
        \Shade 2.61 & \Shade 32.4\unitM & \Shade 84.3\unitm &
        \\ \hline
        \Mid{\name{bicg}} &
        5 & 427 & \Mid{0.304} & 887\unitmu & \Mid{6.72} &
        2.54 & 160\unitM & 407\unitm & \Mid{8.98} \\ &
        \Shade 5 & \Shade 1406 & & \Shade 132\unitmu & &
        \Shade 2.78 & \Shade 16.3\unitM & \Shade 45.3\unitm &
        \\ \hline
        \Mid{\name{gemm}} &
        5 & 524 & \Mid{0.234} & 1.99 & \Mid{2.97} &
        2.54 & 10.8\unitG & 27.4 & \Mid{9.13} \\ &
        \Shade 10 & \Shade 2240 & & \Shade 0.67 & &
        \Shade 2.69 & \Shade 1.12\unitG & \Shade 3.00 &
        \\ \hline
        \Mid{\name{seidel}} &
        5 & 620 & \Mid{0.349} & 10.7\unitmu & \Mid{2.46} &
        2.60 & {960}\unitM & 2.50 & \Mid{7.16} \\ &
        \Shade 8 & \Shade 1778 & & \Shade 4.31\unitmu & &
        \Shade 2.66 & \Shade {131}\unitM & \Shade 0.349 &
        \\ \hline
        \Mid{\name{gemver}} &
        5 & 809 & \Mid{0.382} & 7.28\unitM & \Mid{4.46} &
        2.87 & {23.1}\unitM & 66.2\unitm & 3.15 \\ &
        \Shade 5 & \Shade 2120 & & \Shade 1.63\unitM & &
        \Shade 2.77 & \Shade {7.60}\unitM & \Shade 2.10\unitm & (8.29)
        \\ \hline
        \Mid{\name{mvt}} &
        5 & 701 & \Mid{0.251} & 91.0\unitmu & \Mid{3.32} &
        2.56 & {23.1}\unitM & 59.1\unitm & 7.49 \\ &
        \Shade 10 & \Shade 2793 & & \Shade 27.4\unitmu & &
        \Shade 2.80 & \Shade {2.82}\unitM & \Shade 7.89\unitm & (9.30)
        \\ \hline
        \Mid{\name{syr2k}} &
        5 & 709 & \Mid{0.259} & 250\unitmu & \Mid{4.07} &
        2.89 & {14.0}\unitG & 40.3 & 6.95 \\ &
        \Shade 10 & \Shade 2740 & & \Shade 61.4\unitmu & &
        \Shade 2.71 & \Shade {2.14}\unitG & \Shade 5.80 & (7.62)
        \\ \hline
        \multicolumn{3}{r|}{\Mid{\text{Geomean}}} &
        \Mid{0.289} & & \Mid{3.69} & \multicolumn{3}{r|}{} & 7.19 \\
        \multicolumn{3}{r|}{} & & & & \multicolumn{3}{r|}{} & (8.01)
        \\ \hline
    \end{array}$
\end{table}

\begin{figure}[ht]
    \centering
    \begin{tikzpicture}
        \begin{axis}[
                ylabel={\small Estimated}, xlabel={\small Actual},
                height=100mm, width=100mm, ylabel near ticks,
                scaled ticks=base 10:-3,
                ymax=4000, xmax=4000, ymin=0, xmin=0]
            \pgfplotstableread[col sep = comma]{latopt/csv/lut.csv}\lutcsv
            \addplot[draw=none, mark=*, draw opacity=0, fill opacity=0.5]
                table [y=Estimate, x=Actual] \lutcsv;
            \addplot[mark=none, draw=none]
                table
                [x=Actual, y={create col/linear regression={y=Estimate}}]
                \lutcsv;
            \xdef\slope{\pgfplotstableregressiona} % save the slope parameter
            \xdef\intercept{\pgfplotstableregressionb} % save the intercept parameter
            \addplot [no markers, domain=0:4000] {\intercept+\slope*x};
        \end{axis}
    \end{tikzpicture}
    \caption{%
        Comparisons of our estimated \gls{lut} counts against actual \gls{lut}
        counts from \gls{vhls}\@.
    }\label{lo:fig:lut}
\end{figure}

\begin{figure}[ht]
    \centering
    \begin{tikzpicture}
        \begin{loglogaxis}[
                ylabel={\small Estimated}, xlabel={\small Actual},
                height=100mm, width=100mm, ylabel near ticks,
                %scaled ticks=base 10:-9,
                xmin=1000, ymin=1000, xmax=100000000000, ymax=100000000000]
            \pgfplotstableread[col sep = comma]{latopt/csv/lat.csv}\latcsv
            \addplot[draw=none, mark=*, draw opacity=0, fill opacity=0.5]
                table [y=Estimate, x=Actual] \latcsv;
            \addplot[mark=none, draw=none]
                table
                [x=Actual, y={create col/linear regression={y=Estimate}}]
                \latcsv;
            \xdef\slope{\pgfplotstableregressiona} % save the slope parameter
            \xdef\intercept{\pgfplotstableregressionb} % save the intercept parameter
            \addplot
                [no markers, domain=1000:100000000000] {\intercept+\slope*x};
        \end{loglogaxis}
    \end{tikzpicture}
    \caption{%
        Comparisons of our estimated latency statistics against actual latency
        from \gls{vhls}\@.}
    \label{lo:fig:lat}
\end{figure}

\begin{figure*}[t]
    \centering
    \tikzset{%
        every axis/.style={%
            height=70mm, width=70mm, ylabel near ticks, try min ticks=5},
        every plot/.style={draw=none},
        normal points/.style={mark=*, draw opacity=0, fill opacity=0.5},
        origin point/.style={mark=x, mark size=3},
        special point/.style={mark=o, color=red, mark size=3},
    }
    \pgfplotstableread[col sep=comma, row sep=newline]
        {latopt/csv/seidel.csv}\seidel

    \begin{tikzpicture}
        \begin{axis}[
                ylabel={\small LUT count},
                xlabel={\small Latency (cycles)},
                scaled y ticks=base 10:-3,
            ]
            \addplot[normal points, select coords between index={1}{24}]
                table [y=lut, x=latency]\seidel;
            \addplot[normal points,  select coords between index={26}{28}]
                table [y=lut, x=latency]\seidel;
            \addplot[origin point, select coords between index={0}{0}]
                table [y=lut, x=latency]\seidel;
            \addplot[special point, select coords between index={25}{25}]
                table [y=lut, x=latency]\seidel;
        \end{axis}
    \end{tikzpicture}

    \begin{tikzpicture}
        \begin{axis}[
                ylabel={\small LUT count},
                xlabel={\small Error},
                scaled y ticks=base 10:-3,
            ]
            \addplot[normal points, select coords between index={1}{24}]
                table [y=lut, x=error]\seidel;
            \addplot[normal points,  select coords between index={26}{28}]
                table [y=lut, x=error]\seidel;
            \addplot[origin point, select coords between index={0}{0}]
                table [y=lut, x=error]\seidel;
            \addplot[special point, select coords between index={25}{25}]
                table [y=lut, x=error]\seidel;
        \end{axis}
    \end{tikzpicture}

    \begin{tikzpicture}
        \begin{axis}[
                ylabel={\small Latency (cycles)},
                xlabel={\small Error},
            ]
            \addplot[normal points, select coords between index={1}{24}]
                table [y=latency, x=error]\seidel;
            \addplot[normal points, select coords between index={26}{28}]
                table [y=latency, x=error]\seidel;
            \addplot[origin point, select coords between index={0}{0}]
                table [y=latency, x=error]\seidel;
            \addplot[special point, select coords between index={25}{25}]
                table [y=latency, x=error]\seidel;
        \end{axis}
    \end{tikzpicture}
    \caption{%
        Pareto-optimal variants of the Seidel stencil program from
        Figure~\ref{lo:fig:seidel_prog}. Each graph shows a 2D projection
        of the 3D Pareto frontier. In each graph, the original program is
        marked $\times$, and the lowest-latency variant obtained by arithmetic
        transformations alone is marked by the red circle.}
    \label{lo:fig:seidel}
\end{figure*}

Figure~\ref{lo:fig:lut} compares our estimated \gls{lut} counts (vertical axis)
against the exact \gls{lut} counts (horizontal axis) obtained by synthesizing
\gls{rtl} implementations of each program in Table~\ref{lo:tab:results}.
Although our estimates deviate from the exact values, because we compute
lower bounds on resource utilizations, and finite state machines synthesized
and address calculation are not taken into account, our estimate can still
accurately predict the general trend---a linear regression of all scatter
points finds $R^2 = 0.9344$.

Figure~\ref{lo:fig:lat} compares our latency estimates (vertical axis)
against the actual latency values (horizontal axis). The solid line
represents the linear regression of data points that we have gathered in
Table~\ref{lo:tab:results}. This line is a tight fit with our data, with $R^2 =
0.9959$, which indicates that our latency estimation can accurately predict the
exact latency of synthesized implementations.

Returning to our motivating example from Section~\ref{lo:sec:motivation},
Figure~\ref{lo:fig:seidel} demonstrates the range of optimized programs
discovered by \soap{} when applied to the Seidel stencil loop kernel. In the
figure, $\times$-points indicate the original program. By using only the rules
of real arithmetic, \soap{} finds a more efficient program that can improve run
time by 2.5$\times$, as shown by the \redcircle-points. However, by enabling
partial loop unrolling and our dependence elimination rules, the performance
is further improved, resulting in a 6.7$\times$ reduction of total run time.
Furthermore, we have found that numerical accuracy can often be optimized at
the same time as we optimize the initiation intervals of loops. Because by
partially unrolling loops, the sizes of the expressions in loop grow, which
provides \soap{} a greater freedom in terms of discovering more accurate
expressions. In this example, the most efficient program is also the most
accurate one: it minimizes round-off errors by approximately 2.5$\times$. It is
worth noting that \soap{} can detect that as it explores deep levels of partial
loop unrolling, we start to see a diminishing return in performance as it hits
a bottleneck in memory bandwidth.  This is due to the fact that \gls{vhls}
synthesizes dual port \glspl{ram} for arrays, and in one clock cycle we can
only read from the memory allocating array twice.  Our optimization flow is
aware of this bottleneck and stops exploring further loop unrolling.

Similar graphs for the other benchmarks can be viewed
online,\footnote{\url{https://admk.github.io/soap/plot.html}} each showing
three projections from different axes of the 3D Pareto frontier. Our web page
can be used to interactively explore the positions of each data point on the
three projections simultaneously, and view the corresponding generated C
programs.

\subsection{Discussion}
\label{lo:sub:discussion}

As demonstrated by Figure~\ref{lo:fig:lat}, \soap{} generally produces
accurate latency estimates.  However, we have discovered a few notable
discrepancies.  For instance, \verb|gemver|, \verb|mvt| and \verb|syr2k| all
have significant differences between our estimated latency and the actual
latency from synthesized \gls{rtl} implementations.  An inspection of these
programs reveals that they all share a common programming idiom:
%
\begin{lstlisting}
  for (int i=0; i<N; i++)
    for (int j=0; j<N; j++)
      x[i] += ...;
\end{lstlisting}
%
We found that \gls{vhls} occasionally fails to find the optimal schedule,
predicted by \soap, that could pipeline this loop as tightly as possible.  By
rewriting the above code into the following:
%
\begin{lstlisting}
  for (int i=0; i<N; i++) {
    float sum = x[i];
    for (int j=0; j<N; j++)
      sum += ...;
    x[i] += sum;
  }
\end{lstlisting}
%
fixes this problem, and enables \gls{vhls} to generate a hardware
implementation with the expected \gls{ii}. The ratios in parentheses in
Table~\ref{lo:tab:results} reflect the speedup by performing this simple fix.
