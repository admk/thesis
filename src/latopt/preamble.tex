\documentclass{sig-alternate}
% \renewcommand\floatpagefraction{.9}
% \renewcommand\dblfloatpagefraction{.9}
% \renewcommand\topfraction{.9}
% \renewcommand\dbltopfraction{.9}
% \setcounter{totalnumber}{50}
% \setcounter{topnumber}{50}
% \setcounter{bottomnumber}{50}

\usepackage{amsmath}
\usepackage{amssymb}
\usepackage{algorithmicx}
\usepackage{booktabs}   % for \toprule and \bottomrule
\usepackage{color}
\usepackage{colortbl}   % for \rowcolor, \cellcolor, etc.
\usepackage{fixltx2e}   % for \textsubscript and \textsuperscript
\usepackage{graphicx}
\usepackage{courier}    % makes \tt font same size as \rm font
\usepackage{stmaryrd}   % double brackets
\usepackage{listings}   % source code
\usepackage{multirow}   % for \multirow
\usepackage{textgreek}  % for \textmu
\usepackage{url}        % url stuff
% \usepackage{flushend}   % balance end of page columns
\usepackage{algpseudocode}
\usepackage{subfig}
% \usepackage[inference]{semantic}
\usepackage{verbatim}
\usepackage{tikz}       % graphs
\usetikzlibrary{positioning,fit,decorations.pathmorphing,shapes}
\usepackage{pgfplots}
\usepackage{pgfplotstable}

% From http://tex.stackexchange.com/a/199396:
% {
\pgfplotsset{select coords between index/.style 2 args={
    x filter/.code={
        \ifnum\coordindex<#1\def\pgfmathresult{}\fi
        \ifnum\coordindex>#2\def\pgfmathresult{}\fi
    },
    filter discard warning=false
}}
% }

\lstset{
  language=C,
  basicstyle=\tt,
  columns=fixed,
  mathescape=true,
  keywordstyle=\bfseries,
  otherkeywords={\#pragma,\#define},
}

% \usepackage{tikz-qtree} % tree macros
% \tikzset{level distance=20pt}
% \usepackage{galois}
% \usepackage{algpseudocode}
% \usepackage{graphicx}
% \usepackage{verbatimbox}
% \usepackage{hyperref}
% \hypersetup{%
%     pdftitle=\mytitle,
%     pdfauthor=\myauthor,
% }

\graphicspath{{fig/}}
\DeclareGraphicsExtensions{.pdf}

\title{\mytitle}
\numberofauthors{1}
\ifblind%
\author{%
    \alignauthor~\\
    \affaddr{~\\~\\~} \\
    \email{~}
}
\else%
\author{%
    \alignauthor\myauthor{} \\
    \myaddress{}
}
\fi
\date{\today}

\newcommand{\todo}[1]{\iftodos{\{\{\textbf{TODO\@:} {#1}\}\}\typeout{TODO\@: {#1}}}\fi}

% latin stuff
\newcommand{\eg}{\textit{e.g.}}
\newcommand{\etc}{\textit{etc.}}
\newcommand{\etal}{\textit{et~al.}}
\newcommand{\ie}{\textit{i.e.}}

% names of stuff
%\newcommand{\newsoap}{\textbf{[Tool Name]}}
\newcommand\SOAP{SOAP}

% numerical domains
\newcommand{\integerset}{\ensuremath\mathbb{Z}}
\newcommand{\naturalset}{\ensuremath\mathbb{N}}
\newcommand{\realset}{\ensuremath\mathbb{R}}
\newcommand{\floatset}{\ensuremath\mathbb{F}}
\newcommand{\intervalset}{\ensuremath\mathbf{Interval}}
\newcommand{\floatintervalset}{\ensuremath\intervalset_\floatset}
\newcommand{\errorset}{\ensuremath\mathbb{E}^\sharp}
\newcommand{\errordom}{\ensuremath\mathbf{\Sigma}_{\errorset}}

% numerical operators
\newcommand{\join}{\ensuremath\sqcup}
\newcommand{\meet}{\ensuremath\sqcap}

% program objects
\newcommand{\inttype}{\texttt{int}}
\newcommand{\floattype}{\texttt{float}}
\newcommand{\varset}{\ensuremath\mathbf{Var}}
\newcommand{\aexprset}{\ensuremath\mathbf{ArithExpr}}
\newcommand{\bexprset}{\ensuremath\mathbf{BoolExpr}}
\newcommand{\stmtset}{\ensuremath\mathbf{Stmt}}

% program syntax
\DeclareMathOperator{\notop}{\mathtt{not}}
\newcommand{\andop}{\ensuremath\mathbin\mathtt{and}}
\newcommand{\orop}{\ensuremath\mathbin\mathtt{or}}
\newcommand{\semicolon}{\ensuremath\mathbin{;}}
\newcommand{\assign}{\ensuremath\mathbin{:=}}
\newcommand{\iflit}{\texttt{if}}
\newcommand{\thenlit}{\texttt{then}}
\newcommand{\elselit}{\texttt{else}}
\newcommand{\whilelit}{\texttt{while}}
\newcommand{\forlit}{\texttt{for}}
\newcommand{\dolit}{\texttt{do}}
\newcommand{\tolit}{\texttt{to}}
\newcommand{\assignstmt}[2]{\ensuremath{#1}\assign{#2}}
\newcommand{\skipstmt}{\ensuremath\mathtt{skip}}
\newcommand{\ifstmt}[2]{%
\ensuremath\iflit \left(#1\right) \thenlit \left(#2\right)}
\newcommand{\ifelsestmt}[3]{%
    \ensuremath\iflit \left(#1\right) \thenlit \left(#2\right)
    \elselit \left(#3\right)}
\newcommand{\whilestmt}[2]{%
    \ensuremath\whilelit \left(#1\right) \dolit \left(#2\right)}
\newcommand{\forstmt}[2]{%
    \ensuremath\forlit \left(#1\right) \dolit \left(#2\right)}

% metasemantic syntax
\newcommand{\qop}{\ensuremath\mathbin{?}}
\newcommand{\colonop}{\ensuremath\mathbin{:}}
\newcommand{\select}[3]{\ensuremath#1\qop#2\colonop#3}
\newcommand{\expand}{\ensuremath\mathbin{\star}}
\DeclareMathOperator{\fixop}{\mathrm{fix}}
\newcommand{\fixpoint}[1]{\ensuremath\fixop\left({#1}\right)}
\DeclareMathOperator{\accessop}{\mathrm{access}}
\newcommand{\access}[2]{\ensuremath\accessop\left({#1}, {#2}\right)}
\DeclareMathOperator{\updateop}{\mathrm{update}}
\newcommand{\update}[3]{\ensuremath\updateop\left({#1}, {#2}, {#3}\right)}

% metasemantic objects
\newcommand{\sexprset}{\ensuremath\mathbf{SemExpr}}
\newcommand{\mirset}{\ensuremath\mathbf{MIR}}

% metasemantic functions
\DeclareMathOperator{\metasemop}{\mathsf{M}}
\DeclareMathOperator{\mirerrorop}{\mathsf{E}_\mathsf{m}}
\DeclareMathOperator{\exprerrorop}{\mathsf{E}_\mathsf{s}}
\DeclareMathOperator{\ulp}{\mathrm{ulp}}
\DeclareMathOperator{\roundupop}{\uparrow^\sharp_\circ}
\DeclareMathOperator{\rounddownop}{\downarrow^\sharp_\circ}
\newcommand{\varfunc}[1]{%
    \ensuremath\mathrm{var}\left(#1\right)
}
\newcommand{\metasemfunc}[1]{%
    \ensuremath\metasemop\left\llbracket{#1}\right\rrbracket}
\newcommand{\exprerrorfunc}[1]{\ensuremath\exprerrorop\left[{#1}\right]}
\newcommand{\mirerrorfunc}[1]{\ensuremath\mirerrorop\left[{#1}\right]}
\newcommand{\roundup}[1]{\ensuremath\roundupop\left(#1\right)}
\newcommand{\rounddown}[1]{\ensuremath\rounddownop\left(#1\right)}

% equivalent relations
\newcommand{\eqrel}{\ensuremath\mathbin{\rhd}}

% optimization
\DeclareMathOperator{\frontier}{\textsc{Frontier}}
\DeclareMathOperator{\optimize}{\textsc{O}}
\newcommand{\optfunc}[1]{\ensuremath\optimize\left[#1\right]}
\DeclareMathOperator{\closure}{\textsc{C}}

% misc
\newcommand{\identity}{\mathrm{id}}
\newcommand{\powersetof}[1]{\ensuremath\mathcal{P}\left({#1}\right)}
\newcommand{\labelset}{\ensuremath\mathbf{Label}}
\newcommand{\env}[1]{\ensuremath\mathbf{Env}_{#1}}
\DeclareMathOperator{\labfunc}{\mathrm{Label}}
\DeclareMathOperator{\fresh}{\mathit{fresh}}
\DeclareMathOperator{\eqstep}{\blacktriangleright}
\DeclareMathOperator{\dom}{\mathrm{Dom}}
\DeclareMathOperator{\area}{\mathrm{Area}}
% \DeclareMathOperator{\error}{\mathrm{Error}}
\DeclareMathOperator{\abserr}{\mathrm{AbsError}}
\DeclareMathOperator{\abs}{\mathrm{abs}}
\newcommand\cycle[1]{\scriptstyle{\mathbf{#1}}}

\newcommand\II{\mathit{II}}
\newcommand\MII{\mathit{II}_{\rm min}}
\newcommand\RecMII{\mathit{II}_{\rm min}^{\rm rec}}
\newcommand\ResMII{\mathit{II}_{\rm min}^{\rm res}}

\newcommand\opsymbol{\otimes}

% mir figures
\newcommand{\mirfigscale}{0.20}
\newcommand{\mirfig}[1]{%
    \begin{equation}
        \begin{aligned}
            \includegraphics[scale=\mirfigscale]{#1}
            \label{eq:#1}
        \end{aligned}
    \end{equation}
}

% \xrightsquigarrow
\newcounter{sarrow}
\newcommand\xrightsquigarrow[1]{%
\stepcounter{sarrow}%
\begin{tikzpicture}[decoration=snake]
\node (\thesarrow) {\strut#1};
\draw[->,decorate] (\thesarrow.south west) -- (\thesarrow.south east);
\end{tikzpicture}%
}

\newcommand{\compresslist}{
  \vspace{-1em}
  \setlength{\itemsep}{1pt}
  \setlength{\parskip}{0pt}
  \setlength{\parsep}{0pt}
}

% vim: nospell
