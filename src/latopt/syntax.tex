\section{Syntax Definition}
\label{lo:sec:syntax_definition}

Because we incorporate multi-dimensional arrays in our program
optimization, this section further extends the syntax definition in
Section~\ref{po:sec:syntax_definition} of Chapter~\ref{chp:progopt} to support
array data types.  This can be achieved by modifying the definition of $x$
in~\eqref{po:eq:program_syntax}:
\begin{equation}
    x \syndef v \zeroormore{\texttt{[$a$]}}.
\end{equation}
Here $\zeroormore{\texttt{[$a$]}}$ indicates that the term $\texttt{[} a
\texttt{]}$ can be repeated zero or more than zero times, and $v$ is a variable
and $a$ is an arithmetic expression.  The number of repetition $N$ indicates
the variable $v$ refers to an $N$-dimensional array, and the absence ($N = 0$)
denotes a scalar.  When $x$ is used to declare an array, $a$ must be a positive
integer to conform to the C syntax.  To reflect this restriction and all above
changes, the new syntax definition for numerical programs is therefore:
\begin{equation}
    \begin{aligned}
        a \syndef {} &
            n \synor
            v \zeroormore{\texttt{[$a$]}} \synor
            \texttt{-}a_1 \synor
            a_1 \opsymbol a_2, \qquad
        b \syndef
            \texttt{!}b_1 \synor
            a_1 \oplus a_2 \synor
            b_1 \odot b_2, \\
        s \syndef {} &
            \left[ t \right] v \zeroormore{\texttt{[$c$]}}
                \left[ \texttt{=}~a \right] \semicolon \synor
            s_1~s_2 \synor
            \iflit~(b)~\{ s_1 \}~\left[\elselit~\{ s_2 \}\right] \synor
            \whilelit~(b)~\{s\}, \\
        v \syndef {} & \varx \synor \vary \synor \varz \synor \mathellipsis,
            \qquad
        t \syndef \inttype \synor \floattype.
    \end{aligned}
    \label{lo:eq:program_syntax}
\end{equation}
Here, $c \in \naturalset$ is the set of positive integers.  Recall that
$a, a_1, a_2 \in \aexprset$, $b, b_1, b_2 \in \bexprset$, $s, s_1, s_2 \in
\stmtset$, $v \in \varset$ are respectively arithmetic expressions, Boolean
expressions, program statements, and variables of type $t$ which is either
$\inttype$ or $\floattype$.
