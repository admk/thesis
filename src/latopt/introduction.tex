\section{Introduction}
\label{lo:sec:introduction}

% There are many reasons why FPGA implementations of numerical algorithms
% are best obtained via high-level synthesis (HLS) from C\@: less
% development effort, the abundance of software engineers compared to
% hardware designers, the relative ease of testing C code on an ordinary
% microprocessor, the opportunities for rapid design space exploration, and
% so on~\cite{meeus12}. Great advances have been made in this area recently,
% and the output from HLS tools is nowadays competitive with hand-crafted
% designs~\cite{bdti_xilinx}.

Numerical C programs typically spend most of their time in loops.  For this
reason, \gls{hls} tools adopt state-of-the-art \emph{polyhedral compilation}
techniques~\cite{canis14} to synthesize loops to run as fast as possible.  This
is achieved by pipelining them to maximally exploit parallelism across loop
iterations.  Certain program transformations, such as conventional program
equivalences (\eg~partial loop unrolling and array access pattern changes) are
highly ubiquitous in their compilation process.

However, their ability to perform pipelining, even with any combinations of
these equivalences, is fundamentally constrained by data dependences that
are carried across iterations, \ie~\emph{inter-iteration dependences}.  To
relax these constraints, we must use equivalence rules in real arithmetic
(\eg~associativity and distributivity), in tandem with the conventional rules
above to enable much more efficiently pipelined \gls{rtl} designs.  A simple
example of this is the summation of all elements in an array:
\begin{lstlisting}
  float sum = 0;
  for (int i = 0; i < N; i++)
    sum += a[i];
\end{lstlisting}
This code can be partially unrolled and the sequence of additions can be
rewritten using tree adders to reduce its latency, but we will see later in
Section~\ref{lo:sec:results} that more efficient implementations are possible.

% Unfortunately, in the presence of floating-point arithmetic, these program
% transformations could have an impact on numerical accuracy.  For instance,
% under single-precision floating-point arithmetic with rounding to the nearest,
% the result of $(2^{-24} + 2^{-24}) + 1 = 1.00000012\textellipsis$ is exact,
% but $(1 + 2^{-24}) + 2^{-24}$ is rounded to $1$.  The difference between
% the actual result in real arithmetic and the rounded result is known as the
% \emph{round-off error}.  Round-off errors, when accumulated, can have a
% devastating effect on numerical accuracy~\cite{higham02}.  Round-off errors in
% a numerical program are dependent on every arithmetic operation and every input
% value, and with the impact on floating-point accuracy being so esoteric, it is
% challenging for engineers to understand the repercussions of switching between
% ``\verb|(a + b)| \verb|* c|'' and ``\verb|a * c +| \verb|b * c|'' in their
% programs.

% Experienced engineers apply expression rewriting intuitions in numerical
% programs.  For instance, when summing a sequence of floating-point values, one
% can sometimes reduce round-off error in the result by summing the inputs in
% ascending order.  On the other hand, one can often reduce latency by applying
% \emph{expression balancing}, \ie~rearranging operators in an expression to
% construct a balanced tree, so that more operators can work in parallel.  These
% heuristics cover a very limited number of possible transformations and may not
% always improve the original code.  Therefore a trivial process does not exist
% to apply steps of transformations using equivalence rules to \emph{optimally}
% trade off latency, resources and numerical accuracy.

% Existing HLS tools consider these rewrites to be unsafe, and thus make
% very limited use of them when restructuring floating-point data-paths.
% For instance, \gls{vhls}~\cite{vivado_hls} has only a very simple
% \emph{expression balancing} feature that uses associativity to improve
% latency, and only expressions with either additions or multiplications are
% optimized.  Moreover, it does not produce optimal loop pipelining, because
% it does not take into account the implications of these transformations on
% inter-iteration dependences and does not explore partial loop unrolling.  In
% addition, \gls{vhls} cannot reason about how this feature affects numerical
% accuracy; there is no guarantee that this transformation will not result in a
% catastrophically inaccurate implementation.

% In response, we have developed a tool---a fully automatic source-to-source
% optimizer---that augments \gls{vhls}, by optimizing a given program using these
% transformations.  Our optimizer discovers not only one, but a wide spectrum of
% program candidates.  When synthesized in \gls{vhls}, these candidates trade
% off three performance metrics of great importance to engineers: run time,
% resource usage and round-off error.  Here, run time refers to the latency in
% clock cycles, resource usage refers to the number of look-up tables (LUTs)
% and digital signal processing (DSP) elements.  Some of these performance
% metrics could be in conflict.  For example, higher performance tends to
% require more circuitry, and how to resolve this trade-off depends on the
% user's requirements.  As a result, our tool produces a \emph{set} of optimized
% programs, known as the \emph{Pareto frontier}: those programs $P$ for which the
% tool has found no $P'$ that improves on $P$ in all three metrics.

In contrast to the expression balancing optimization pass in \gls{vhls},
our tool \emph{automatically} produces results that are significantly
better than \emph{manually} tuning partial unrolling factors and expression
balancing \verb|#pragma|s in \gls{vhls}, because it is fully aware of how
data dependences are carried across iterations, and uses this to steer the
optimization process.  Our tool is also fully aware of the impact these
transformations could have on round-off errors, and minimizes them in the
optimization process, as we treat numerical accuracy as one of the three
simultaneous objectives.  Furthermore, \gls{vhls} only generates one result
which does not necessarily improve over the original code.

% Generating candidate optimizations na{\"\i}vely would produce a combinatorial
% explosion, even for small input programs.  For instance, a simple summation of
% $n$ variables could have $(2n - 1)!! = 1 \times 3 \times 5 \times \textellipsis
% \times (2n - 1)$ equivalent expressions~\cite{mouilleron}.  Therefore, we base
% our optimizer on the open-source \soap{} framework that we have developed in
% Chapters~\ref{chp:stropt}~\&~\ref{chp:progopt}, which specifically tackles
% the efficient discovery of equivalent structures in numerical programs,
% by intelligently pruning the set of candidates as it progresses up the
% input program's abstract syntax tree.

We also exploit \soap{}'s ability to analyze the numerical accuracy of
a given program.  To analyze the run time and resource utilization of a
given program, we use a variant of the \emph{iterative modulo scheduling}
algorithm~\cite{rau94} that computes fundamental lower bounds of these metrics.

We evaluated our tool on a suite of 11 programs from the Livermore
Loops~\cite{livermore} and PolyBench~\cite{polybench} benchmark suites.  Our
tool obtained a wide selection of Pareto-optimized programs.  Programs with
the best latency obtained speedups of up to $12\times$ ($7\times$ on average
across the suite), and increases in accuracy of up to $7\times$ ($2.7\times$
on average), while using up to $4\times$ ($2.5\times$ on average) more
\glspl{lut}.  We were unable to decrease the resource utilization in any of the
benchmarks, as they have no redundant computations.

\begin{trivlist}\item{\textbf{Our contributions}}

\begin{itemize}

    \item We are the first to describe how standard program equivalence
    that do not affect program behavior, \eg~partial loop unrolling and
    rules that removes extraneous array accesses, can give rise to the
    freedom for non-standard transformation rules, \eg~arithmetic rules,
    to significantly impact latency, resource usage and accuracy in a loop
    (Section~\ref{lo:sub:transformation_rules}).

    \item We significantly improve the performance of the
    efficient discovery of equivalent through improved accuracy
    analysis (Section~\ref{lo:sub:accuracy_analysis}), graph
    partitioning and intelligent pruning of optimization candidates
    (Section~\ref{lo:sub:algorithm}).

    \item We designed a new scheduling analysis that estimates
    the latency and resource usage of a given optimized candidate
    (Section~\ref{lo:sec:performance_analysis}).

    \item Incorporating the above-mentioned techniques, we have developed the
    first optimizer to \emph{automatically} and \emph{safely} produce optimized
    programs (and subsequent \gls{rtl} implementations with \gls{vhls}) on
    the three-dimensional Pareto frontier of options that trade off run time,
    accuracy, and area.  Our improvements on latency are notably better than
    the only ones produced by \gls{vhls}'s \emph{unsafe} optimizations. We have
    evaluated our tool on a suite of Livermore Loops and PolyBench benchmarks
    (Section~\ref{lo:sec:results}).

\end{itemize}

\end{trivlist}

\todo{Chapter outline.}
