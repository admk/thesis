\section{Extending MIRs with Arrays}
\label{lo:sec:intermediate}

% MIRs are also capable of representing loops~\cite{soap2}, but we do not
% exploit that in this paper, despite the centrality of loops to our work.
% When we optimize loop nests, we are specifically applying transformations
% to the kernels of the flattened loop nests.  Therefore, we find that when
% analyzing the latency and resource usage of a loop, we need only have the
% \emph{body} of the loop as a MIR\@.

In Section~\ref{po:sec:program_to_mir} of Chapter~\ref{chp:progopt} we
discussed a new \gls{ir} for program optimization.  However, it did not include
support for arrays in the original description of the \gls{mir} format.  All
examples that motivate us in extending \soap{} include arrays, so in this
chapter, we extend \glspl{mir} to be able to represent programs that not only
use scalar values, but also single- or multi-dimensional arrays.

In many imperative languages such as C, arrays are stateful objects, \ie~they
are used to store information, and changes to them are reflected to concurrent
parts of the program that may be oblivious to the changes.  This characteristic
is known as the lack of \emph{referential transparency}~\cite{sondergaard90}.
Such behavior is not present in arithmetic expressions, many functional
programming languages, \gls{ssa} forms, as well as \glspl{mir}.  This proves
to be a challenge to us, because our efficient program optimization relies on
recursively dividing the program into smaller subprograms that can be optimized
independently, without affecting other subprograms.

To remedy this, we treat arrays as immutable.  We use a function
$\updateop(A,\bar{x}, e)$ to return a new array that is the same as $A$ but
with (multi-dimensional) index $\bar{x}$ now containing $e$.  Similarly, the
function $\access{A}{\bar{x}}$ returns the element of $A$ at index $\bar{x}$.
As a simple example, a loop body
``\lstinline[basicstyle=\tt]{A[i + 1] = 2 * A[i];}''
can be translated into the following \gls{mir}\@:
\begin{equation}
    \label{lo:eq:array_example}
    \begin{tikzpicture}[mir]
        \node[mirnode] (var_A) at (0,0) {\texttt{A}};
        \node[mirnode] (update)[right=of var_A] {$\updateop$};
        \node[mirnode] (x1)    [below left=of update] {\texttt{A}};
        \node[mirnode] (plus)  [below=of update] {$+$};
        \node[mirnode] (i1)    [below left=of plus] {\texttt{i}};
        \node[mirnode] (n1)    [below right=of plus] {$1$};
        \node[mirnode] (times) [below right=of update, xshift=7mm] {$\times$};
        \node[mirnode] (n2)    [below left=of times] {$2$};
        \node[mirnode] (access)[below right=of times] {$\accessop$};
        \node[mirnode] (x2)    [below left=of access] {\texttt{A}};
        \node[mirnode] (i2)    [below right=of access] {\texttt{i}};

        \draw[|->] (var_A) -- (update);
        \draw[<-] (update) -- (x1);
        \draw[<-] (update) -- (plus);
        \draw[<-] (update) -- (times);
        \draw[<-] (plus) -- (i1);
        \draw[<-] (plus) -- (n1);
        \draw[<-] (times) -- (n2);
        \draw[<-] (times) -- (access);
        \draw[<-] (access) -- (x2);
        \draw[<-] (access) -- (i2);
        \brackets{(current bounding box)}
    \end{tikzpicture}\,.
\end{equation}

The implication of making arrays immutable is two-fold.  Firstly, we disallow
pointer aliasing, \ie~\verb|float *b = a;| is not allowed in the C code, to
keep the translation simple.  However this is not a problem for us because the
programs that can benefit from our optimizations usually do not manipulate
pointers.  This issue can also be addressed in the future by performing
pointer analysis. Secondly, diverged paths in array updates could occur if
we na{\"\i}vely optimize \glspl{mir}.  For instance, if $A$ is an input
array, consider the two expressions in a \gls{mir}, $\update{A}{\bar{x}}{e}$
and $\update{A}{\bar{x}}{e^\prime}$, where $e, e^\prime$ are equivalent.
They respectively update the $x$-th element of $A$ with $e$ and $e^\prime$
and return different arrays.  A C program cannot be generated from this
\gls{mir} without duplicating $A$.  We solve this problem by partitioning
the \gls{mir} at ``$\updateop$'' nodes using the method described in
Section~\ref{lo:sec:structural_optimization}.

To give an example, consider the program which computes the Fibonacci sequence
in Figure~\ref{lo:fig:fib}, and it can be represented by the \gls{mir} shown in
Figure~\ref{lo:fig:fib_mir}.
\begin{figure}[ht]
    \centering
    \newsavebox{\fiblst}
    \begin{lrbox}{\fiblst}
        \begin{lstlisting}
for (int i=2; i<1023; i++)
{
    A[i] = A[i-1] + A[i-2];
}
        \end{lstlisting}
    \end{lrbox}
    \subfloat[The program.]{%
        \begin{minipage}{0.42\textwidth}
            \usebox{\fiblst}
        \end{minipage}\label{lo:fig:fib}
    } \quad
    \subfloat[The \gls{mir}.]{%
        \begin{tikzpicture}[mir]
            \node[mirnode] (A) at (0, 0) {\texttt{A}};
            \node[mirnode] (expand) [right=10mm of A] {$\expand$};

            \node[coordinate] (initmir) [below right=of expand, xshift=5mm] {};
            \node[mirnode] (initi) [below right=1mm of initmir] {\texttt{i}};
            \node[mirnode] (init2) [right=of initi] {$2$};
            \node[mirnode] (inita) [right=of init2] {\texttt{A}};
            \node[mirnode] (initain) [right=of inita] {\texttt{A}};
            \draw[|->] (initi) -- (init2);
            \draw[|->] (inita) -- (initain);
            \brackets{(initi) (initain)};
            \draw[<-] (expand) -- (initmir);

            \node[mirnode] (fix) [below left=of expand] {$\fix$};
            \node[mirnode] (fixA) [below right=of fix, xshift=5mm]
                {\texttt{A}};
            \node[mirnode] (b) [below left=10mm of fix] {$<$};
            \node[mirnode] (bi) [below left=of b] {\texttt{i}};
            \node[mirnodealt] (bn) [below right=of b, xshift=-2mm] {$1023$};

            \node[coordinate] (mir) [below=7mm of fix] {};
            \node[mirnode] (mA) [below right=1mm of mir] {\texttt{A}};
            \node[mirnode] (upd) [right=of mA] {$\updateop$};
            \node[mirnode] (updi) [below=of upd] {\texttt{i}};
            \node[mirnode] (+) [below right=of upd] {$+$};
            \node[mirnode] (acc1) [below left=of +] {$\accessop$};
            \node[mirnode] (acc2) [below right=of +] {$\accessop$};
            \node[mirnode] (-1) [below right=6mm of acc1] {$-$};
            \node[mirnode] (-2) [below right=6mm of acc2] {$-$};
            \node[mirnode] (1) [below right=of -1] {$1$};
            \node[mirnode] (2) [below right=of -2] {$2$};
            \node[mirnode] (-1i) [below left=of -1] {\texttt{i}};
            \node[mirnode] (-2i) [below left=of -2] {\texttt{i}};
            \node[mirnode] (Ain) [below left=8mm of acc1] {\texttt{A}};
            \node[mirnode] (mi) [right=15mm of upd] {\texttt{i}};
            \node[mirnode] (i+1) [right=of mi] {$+$};
            \node[mirnode] (i+1 i) [below left=of i+1] {\texttt{i}};
            \node[mirnode] (i+1 1) [below right=of i+1] {$1$};
            \draw[|->] (mA) -- (upd);
            \draw[<-] (upd) -- (+);
            \draw[<-] (+) -- (acc1);
            \draw[<-] (+) -- (acc2);
            \draw[<-] (acc1) -- (-1);
            \draw[<-] (acc2) -- (-2);
            \draw[<-] (-1) -- (1);
            \draw[<-] (-2) -- (2);
            \draw[<-] (upd) -- (Ain);
            \draw[<-] (upd) -- (updi);
            \draw[<-] (acc1) -- (Ain);
            \draw[<-,channel] (acc2) -- (Ain);
            \draw[<-] (-1) -- (-1i);
            \draw[<-] (-2) -- (-2i);
            \draw[|->] (mi) -- (i+1);
            \draw[<-] (i+1) -- (i+1 i);
            \draw[<-] (i+1) -- (i+1 1);
            \brackets{(mA) (Ain) (2)};

            \draw[|->] (A) -- (expand);
            \draw[<-] (expand) -- (fix);
            \draw[<-] (fix) -- (b);
            \draw[<-] (fix) -- (fixA);
            \draw[<-] (b) -- (bi);
            \draw[<-] (b) -- (bn);
            \draw[<-] (fix) -- (mir);
            \brackets{(current bounding box)};
        \end{tikzpicture}
        {}\label{lo:fig:fib_mir}
    }
    \caption{%
        A simple example program and its corresponding \acrshort{mir}\@.
    }
\end{figure}
