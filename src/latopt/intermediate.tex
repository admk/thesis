\section{Intermediate Representation}
\label{lo:sec:intermediate}

%\subsection{Background: Translating Programs to Graphs}
%\label{lo:sub:background}

There are infinite number of ways to rewrite numerical C programs, and many of
these rewrites produce programs that have the same resource usage, accuracy
and latency characteristics.  For instance, the following two programs
are equivalent, but syntactically different, as they carry out the same
computations.
\begin{figure}[ht]
    \centering
    \subfloat[$P_1$]{%
        \shortstack[l]{%
            \texttt{x = x + 1;} \\
            \texttt{y = 2 * x;} \\
            \texttt{x = x + 3;}
        }
    } \qquad \qquad
    \subfloat[$P_2$]{%
        \shortstack[l]{%
            \texttt{y = x + 1;} \\
            \texttt{x = y;} \\
            \texttt{y = y * 2;} \\
            \texttt{x = x + 3;}
        }
    }
    \caption{Two programs that are equivalent but syntactically different.}
    \label{lo:fig:equiv_progs}
\end{figure}

In practice, it is desirable to eliminate as much as possible the need
for these syntactic rewrites that do not affect our performance metrics.
Therefore, following Chapter~\ref{chp:progopt}, we perform transformations not
on the program text directly, but on a \gls{dag} representation of the program
called a \gls{mir}\@.  It expresses how each program variable is updated, but
abstracts away the order in which the updates occur, and ignores any temporary
variables that are not marked as program outputs.

% As an example, the two equivalent programs above can be automatically
% translated into an identical MIR\@:
% \begin{equation}
    % \label{lo:eq:assign_example}
    % \begin{tikzpicture}[mir]
        % \node[mirnode] (var_x) at (0,0) {\texttt{x}};
        % \node[mirnode] (plus1) [right=of var_x] {$+$};
        % \node[mirnode] (n3)    [below left=of plus1] {$3$};
        % \node[mirnode] (plus2) [below right=of plus1] {$+$};
        % \node[mirnode] (var_y) [right=of plus1] {\texttt{y}};
        % \node[mirnode] (times) [right=of var_y] {$\times$};
        % \node[mirnode] (n2)    [below right=of times] {$2$};
        % \node[mirnode] (x)     [below left=of plus2] {\texttt{x}};
        % \node[mirnode] (n1)    [below right=of plus2] {$1$};
        % \draw[|->] (var_x) -- (plus1);
        % \draw[<-] (plus1) -- (n3);
        % \draw[<-] (plus1) -- (plus2);
        % \draw[<-] (plus2) -- (x);
        % \draw[<-] (plus2) -- (n1);
        % \draw[|->] (var_y) -- (times);
        % \draw[<-] (times) -- (plus2);
        % \draw[<-] (times) -- (n2);
        % \brackets{(current bounding box)}
    % \end{tikzpicture}
% \end{equation}

This representation is useful to us, because a single \gls{mir} is able to
capture a class of syntactically-distinct programs, all of which have the
same resource usage, accuracy, and latency characteristics.  By searching for
transformations on \glspl{mir}, we drastically reduce the size of our search
space.  Note that expressions in the \gls{mir} can share common structures;
this is useful for modeling the sharing of common subexpressions and makes the
search for optimizations much more efficient.

% MIRs also abstract the control structure of a program, preserving only the
% computations that lead to the outputs. For instance, by using the ternary
% conditional operator ``$\qop$'' from C, programs with conditionals such as:
% \begin{lstlisting}
  % x = x + 1;
  % if (b)
      % x = 2 * x;
% \end{lstlisting}
% can be represented in MIR form as follows:
% \begin{equation}
    % \label{lo:eq:if_example}
    % \begin{tikzpicture}[mir]
        % \node[mirnode] (var_x) at (0,0) {\texttt{x}};
        % \node[mirnode] (qop)   [right=8mm of var_x] {$\qop$};
        % \node[mirnode] (b)     [below left=of qop] {\texttt{b}};
        % \node[mirnode] (times) [below right=of qop, yshift=2mm] {$\times$};
        % \node[mirnode] (n2)    [below left=of times] {$2$};
        % \node[mirnode] (plus)  [below right=of times, yshift=2mm] {$+$};
        % \node[mirnode] (x)     [below left=of plus, yshift=2mm] {\texttt{x}};
        % \node[mirnode] (n1)    [below right=of plus, yshift=2mm] {$1$};

        % \draw[|->] (var_x) -- (qop);
        % \draw[<-] (qop) -- (b);
        % \draw[<-] (qop) -- (times);
        % \draw[<-] (times) -- (n2);
        % \draw[<-] (times) -- (plus);
        % \draw[<-] (plus) -- (x);
        % \draw[<-] (plus) -- (n1);
        % \draw[<-] (qop) to[bend left] (plus);
        % \brackets{(current bounding box)}
    % \end{tikzpicture}
% \end{equation}

% MIRs are also capable of representing loops~\cite{soap2}, but we do not
% exploit that in this paper, despite the centrality of loops to our work.
% When we optimize loop nests, we are specifically applying transformations
% to the kernels of the flattened loop nests.  Therefore, we find that when
% analyzing the latency and resource usage of a loop, we need only have the
% \emph{body} of the loop as a MIR\@.

\subsection{Representing arrays}
\label{lo:sub:extending_the_translation_to_handle_arrays}

Chapter~\ref{chp:progopt} did not include support for arrays in their original
description of the \gls{mir} format. However, the examples that motivate our
work all include arrays, so in this chapter, we extend \glspl{mir} to be able
to represent programs that use single- or multi-dimensional arrays.

In many imperative languages such as C, arrays are stateful objects, \ie~they
are used to store information, and changes to them are reflected to concurrent
parts of the program that may be oblivious to the changes.  This characteristic
is known as the lack of \emph{referential transparency}.  Such behavior is
not present in arithmetic expressions, many functional programming languages,
\gls{ssa}, as well as \glspl{mir}.  This proves to be a challenge to us,
because our efficient program optimization relies on recursively dividing the
program into smaller subprograms that can be optimized independently, without
affecting other subprograms.

To remedy this, we treat arrays as immutable.  We use a function
$\updateop(A,\bar{x}, e)$ to return a new array that is the same as $A$ but
with (multi-dimensional) index $\bar{x}$ now containing $e$.  Similarly, the
function $\access{A}{\bar{x}}$ returns the element of $A$ at index $\bar{x}$.
As a simple example, a loop body:
\begin{lstlisting}
    A[i + 1] = 2 * A[i];
\end{lstlisting}
can be translated into the following \gls{mir}\@:
\begin{equation}
    \label{lo:eq:array_example}
    \begin{tikzpicture}[mir]
        \node[mirnode] (var_A) at (0,0) {\texttt{A}};
        \node[mirnode] (update)[right=of var_A] {$\updateop$};
        \node[mirnode] (x1)    [below left=of update] {\texttt{A}};
        \node[mirnode] (plus)  [below=of update] {$+$};
        \node[mirnode] (i1)    [below left=of plus] {\texttt{i}};
        \node[mirnode] (n1)    [below right=of plus] {$1$};
        \node[mirnode] (times) [below right=of update, xshift=7mm, yshift=2mm] {$\times$};
        \node[mirnode] (n2)    [below left=of times] {$2$};
        \node[mirnode] (access)[below right=of times, yshift=2mm] {$\accessop$};
        \node[mirnode] (x2)    [below left=of access, yshift=1mm] {\texttt{A}};
        \node[mirnode] (i2)    [below right=of access, yshift=1mm] {\texttt{i}};

        \draw[|->] (var_A) -- (update);
        \draw[<-] (update) -- (x1);
        \draw[<-] (update) -- (plus);
        \draw[<-] (update) -- (times);
        \draw[<-] (plus) -- (i1);
        \draw[<-] (plus) -- (n1);
        \draw[<-] (times) -- (n2);
        \draw[<-] (times) -- (access);
        \draw[<-] (access) -- (x2);
        \draw[<-] (access) -- (i2);
        \brackets{(current bounding box)}
    \end{tikzpicture}
\end{equation}

The implication of making arrays immutable is two-fold.  Firstly, we disallow
pointer aliasing, \ie~\verb|float *b = a;| is not allowed in the C code, to
keep the translation simple.  However this is not a problem for us because the
programs that can benefit from our optimizations usually do not manipulate
pointers.  This issue can also be addressed in the future by performing
pointer analysis. Secondly, diverged paths in array updates could occur if
we na{\"\i}vely optimize \glspl{mir}.  For instance, if $A$ is an input
array, consider the two expressions in a \gls{mir}, $\update{A}{\bar{x}}{e}$
and $\update{A}{\bar{x}}{e^\prime}$, where $e, e^\prime$ are equivalent.
They respectively update the $x$-th element of $A$ with $e$ and $e^\prime$
and return different arrays.  A C program cannot be generated from this
\gls{mir} without duplicating $A$.  We solve this problem by partitioning
the \gls{mir} at ``$\updateop$'' nodes using the method described in
Section~\ref{lo:sec:structural_optimization}.
