\section{Conclusion}
\label{sec:conclusion}

Minimizing the latency of loops is a central task for HLS tools that obtain
FPGA implementations from numerical C programs. Loop latency can often be
reduced by performing simple rewrites to minimize inter-iteration data
dependences, but HLS tools cannot enable such rewrites by default because
they may spoil the accuracy of floating-point computations. This paper has
presented the first tool that is able to automatically rewrite a given program
to optimize latency, while controlling for accuracy and resource usage. Our
experimental results suggest that, in fact, latency and accuracy are often
\emph{not} in conflict: that programs aggressively optimized for latency can
also have minimal round-off errors, albeit greater resource usage. We have
demonstrated that our tool can optimize commonly used code fragments from
PolyBench~\cite{polybench} and Livermore Loops~\cite{livermore} to have up to a
$12\times$ increase in performance, and up to $7\times$ reduction of round-off
errors, at the cost of up to $3.8\times$ more resource utilization.  Our tool
is open-source and can be freely downloaded.\footnote{The URL will be available
in the final version.} % JW: Not sure that's worth saying?

Currently, our tool sees a diminishing performance return when loops are deeply
unrolled, because of a memory bottleneck. We are exploring an extension to
our tool that enables it to automatically partition arrays upon hitting such
a memory bottleneck. Also, our tool currently supports only single-precision
floating-point data types; we intend to extend this to multiple-precision
types, and explore the impact on our three performance metrics: latency,
resource utilization and numerical accuracy.

%%% Local Variables:
%%% mode: latex
%%% TeX-master: "paper"
%%% End:
