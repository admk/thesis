% !TEX root = ../thesis-example.tex
%
\pdfbookmark[0]{Abstract}{Abstract}
\chapter*{Abstract}
\label{sec:abstract}
\vspace*{-10mm}

% This paper introduces SOAP, a new tool to automatically optimize the
% structure of arithmetic expressions for FPGA implementation as part of a high
% level synthesis flow, taking into account axiomatic rules derived from real
% arithmetic, such as distributivity, associativity and others. We explicitly
% target an optimized area/accuracy trade-off, allowing arithmetic expressions
% to be automatically re-written for this purpose. For the first time, we
% bring rigorous approaches from software static analysis, specifically
% formal semantics and abstract interpretation, to bear on source-to-source
% transformation for high-level synthesis. New abstract semantics are developed
% to generate a computable subset of equivalent expressions from an original
% expression. Using formal semantics, we calculate two objectives, the accuracy
% of computation and an estimate of resource utilization in FPGA\@. The
% optimization of these objectives produces a Pareto frontier consisting of a
% set of expressions. This gives the synthesis tool the flexibility to choose
% an implementation satisfying constraints on both accuracy and resource
% usage. We thus go beyond existing literature by not only optimizing the
% precision requirements of an implementation, but changing the structure
% of the implementation itself. Using our tool to optimize the structure
% of a variety of real world and artificially generated examples in single
% precision, we improve either their accuracy or the resource utilization by up
% to 60\%.

% This paper introduces a new technique, and its associated open source tool to
% automatically perform source-to-source optimization of numerical programs,
% specifically targeting the trade-off between numerical accuracy and resource
% usage as a high-level synthesis flow for FPGA implementations.  We introduce
% a new intermediate representation, which we call metasemantic intermediate
% representation (MIR), to empower the abstraction and optimization of
% numerical programs.  We efficiently discover equivalent structures in
% MIRs by exploiting the rules of real arithmetic, such as associativity
% and distributivity, and rules that enable control flow restructuring, and
% produce Pareto frontiers of equivalent programs that trades off LUTs, DSPs
% and accuracy.  Additionally, we further broaden the Pareto frontier in our
% optimization flow to automatically explore the numerical implications of
% partial loop unrolling and loop splitting.  In real applications, our tool
% discovers a wide range of Pareto optimal options, and the most accurate one
% improves the accuracy of numerical programs by up to 65\%.

% Loops are pervasive in numerical programs, so state-of-the-art high-level
% synthesis (HLS) tools use pipelining to schedule them efficiently. Still, the
% run time performance of the resultant FPGA implementation is limited by data
% dependences between loop iterations. Some of these dependence constraints can
% be alleviated by rewriting the program according to arithmetic identities
% (\eg~associativity and distributivity), memory access reductions, and
% control flow optimizations (\eg~partial loop unrolling). HLS tools cannot
% safely enable such rewrites by default because they may spoil the accuracy
% of floating-point computations and increase area usage. In this paper,
% we introduce the first open-source program optimizer for automatically
% rewriting a given program to optimize latency while controlling for accuracy
% and area. Our tool reports a three-dimensional Pareto frontier that the
% programmer can use to resolve the trade-off according to their needs. When
% applied to a suite of PolyBench and Livermore Loops benchmarks, our tool has
% generated programs that enjoy up to a 12$\times$ speedup, with a simultaneous
% 7$\times$ increase in accuracy, at a cost of up to 4$\times$ more LUTs.
